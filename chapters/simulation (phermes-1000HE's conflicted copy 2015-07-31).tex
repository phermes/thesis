\chapter{Simulation Tools}\label{chap:simulation_tools}
\section*{Introduction}
Several tools are used to simulate the particle beams in the LHC, and to give predicitions on the feasibility of certain machine configurations, in terms of optics and collimator settings. Besides a precise tracking routine, a code for the simulation of a multistage collimation system requires the simulation of the particle-matter interactions while passing the collimator material. Depending on the beam parameters and the particle type, different types of interaction must be respected. 

\section{Tracking Tools}

\subsection{Symplectic Integration}

The Jacobian for a transformation 

\begin{align}
(x_1, p_1, x_2, p_2, ..., x_N, p_N) \rightarrow (X_1,P_1, X_2, P_2,...,X_N,P_N) \, ,
\end{align}

is defined as 

\begin{align}
\mathrm{J} = 

\begin{pmatrix}
\frac{\partial X_1}{\partial x_1} & 
\end{pmatrix}

\end{align}



\section{Simulation of Transverse Beam Dynamics}
\subsection{MAD-X}
MAD-X\footnote{Methodical Accelerator Design} is the standard simulation tool to evaluate the betatronic and dispersion functions in modern particle accelerators. The code reads in the lattice (e.g. the sequence of magnets and drift spaces) and the respective magnetic strengths of the machine and calculates the beam parameters at every element defined in the sequence file. 

Besides simple calculation of the Twiss parameters, the code provides algorithms for the matching of several parameters, such as the optical or chromatic functions under respect of user-defined constraints and variables. 

\section{Simulation of Collimation Efficiency}
\subsection{SixTrack}
SixTrack is the standard simulation tool to evaluate the efficiency of the LHC collimation system in proton operation. The initial SixTrack code was able to track particles, including synchrotron motion, chromatic effects to high order through a particle accelerator lattice. The extended version of SixTrack ('Collimation-SixTrack'), containing collimator information, and scattering routines, was developped in the XXXXs. 

\subsubsection{K2 Scattering Routine}
In order to simulate the interaction of the protons with the collimator material, the subroutine \textit{K2} was developped. Formerly developped for the COLLTRACK program, the K2 scattering routine was implemented in the collimation version of SixTrack. K2 respects the physics processes which are dominating in the concerning energy range (see chapter xxx).


% ----------------------------------- PARTICLE MATTER INTERACTION ----------------------------------

\section{Simulation of Particle-Matter Interaction}
\subsection{FLUKA}



\subsection{K2 Proton Scattering Routine}

Integration in SixTrack \\
Update by Claudia