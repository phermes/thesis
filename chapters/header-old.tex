\documentclass [11pt,twoside,a4paper,titlepage,bigheadings,fleqn,openright]{report}
%\usepackage{ngerman}
\usepackage[english]{babel}
\usepackage{amsmath}
\usepackage{endnotes}
\usepackage[ansinew]{inputenc}
\usepackage{amssymb}
\usepackage{stmaryrd}
 \usepackage[numbers]{natbib}
\usepackage{graphics}
\usepackage{url}


\usepackage{wrapfig}
\usepackage{color}
\usepackage{nicefrac} 
%\definecolor{myColor}{rgb}{0.9,0.9,0.6}%
\usepackage{inputenc}
\usepackage{cancel}
\usepackage{caption}
\usepackage{bbold}
\usepackage{ifthen}



% Line Numbers:
%
\usepackage{lineno}
%
% enable the line numbers:
%\linenumbers
% disable the line numbers:
% \nolinenumbers



% fancyhdr for own definition of headings
%
\usepackage{fancyhdr}
%\pagestyle{fancy}
\fancyhead{}
\fancyfoot{}
\renewcommand{\chaptermark}[1]{\markboth{\slshape \thechapter. \ #1}{}}
%\fancyhead[LO,LE]{\leftmark}
%\fancyhead[RO,RE]{\pagemark}
%\renewcommand{\headrulewidth}{0.4pt}
%\renewcommand{\footrulewidth}{0pt}
%\fancypagestyle{plain}{%
%\fancyhf{}%
%\fancyhead[RO,RE]{\thepage}
%\renewcommand{\headrulewidth}{0pt}
%\renewcommand{\footrulewidth}{0pt}}





%%%%%%%%%%%%% Glossaries
\usepackage[nonumberlist]{glossaries}     % lets latex use the package glossaries
%\makeglossaries             % allows latex to create the glossary files
% Documentation:
% usage: build and read in file with the glossary entries in the following scheme:
%
% \newglossaryentry{electrolyte}{name=electrolyte,description={solution able to conduct electric current}}
%
% read in the glossary file
% % glossary

\newglossary[slg]{symbol}{sot}{stn}{Symbols}
\makeglossaries

%\newglossaryentry{alpha}{
%  type=symbol,
%  name={Alphanumeric},
%  description={}}

\newglossaryentry{area}{
  type=symbol,
  name={\ensuremath{S}},
  description={reference area},
  sort={S}%,
 % parent=alpha
}

\newglossaryentry{emittance}{
  type=symbol,
  name={$\epsilon_N$},
  sort={e},
  description={Normalized transverse r.m.s. emittance}%,
  %parent=alpha
}

\newglossaryentry{relgamma}{
  type=symbol,
  name={$\gamma$},
  description={Relativistic Lorentz factor},
  sort={g}%,
%  parent=alpha
}










\newacronym[sort={adt}]{ADT}{ADT}{Transverse damper}
\newacronym[sort={blm}]{BLM}{BLM}{Beam loss monitor}
\newacronym[sort={b1}]{B1}{B1}{LHC Beam 1, rotating in clockwise direction}
\newacronym[sort={b2}]{B2}{B2}{LHC Beam 2, rotating in counter-clockwise direction}
\newacronym[sort={tcp}]{TCP}{TCP}{Target collimator primary (primary collimator)}
\newacronym[sort={tcsg}]{TCSG}{TCSG}{Target collimator secondary (secondary collimator)}
\newacronym[sort={tcT}]{TCT}{TCT}{Target collimator tertiary (tertiary collimator)}
\newacronym[sort={tcld}]{TCLD}{TCLD}{Target collimator long dispersion suppressor}
\newacronym[sort={lhc}]{LHC}{LHC}{Large Hadron Collider}
\newacronym[sort={ds}]{DS}{DS}{Dispersion suppressor}



% to place the glossary in the document, use \printglossary
%Once you have defined a term, you can use it in the document. The glossaries package provides a number of commands that are described in the section "Links to Glossary Entries" in the main glossaries user manual. Here, I shall just cover the main commands:

%\gls{<label>}

%This prints the term associated with <label>. Using the example in the previous section, \gls{oesophagus} will display �sophagus. If the hyperref package has also been loaded, the term will also be hyperlinked to the relevant entry in the glossary.

%\glspl{<label>}

%This prints the plural of the term associated with <label>, so \glspl{oesophagus} will display �sophagi. Aga%in, the text will be a hyperlink to the relevant entry in the glossary, if hyperlinks are defined.

%There are also versions that convert the first character to upper case if you need to start a sentence with %a term:

%\Gls{<label>}

%and

%\Glspl{<label>}

%For example:

%\Glspl{electrolyte} usually exist as solutions of salts, 
%bases or acids.
%Note that these commands all have two optional arguments that aren't discussed here, but they are described %in the section "Links to Glossary Entries" in the main glossaries user manual.


%Caveat: don't use these commands in moving arguments, such as those used by \chapter, \section and \caption.
% You can, instead, use commands such as

%\glsentrytext{<label>}


\usepackage{hyperref}

%\usepackage{MnSymbol}
%tikz
\usepackage{graphicx}
\usepackage{graphics}



%_______Packages for development

% show keys in development mode
%\usepackage{showkeys}


%%%% Packages for the use of Tables

\usepackage{multirow}  
% extend elements over some rows. 
% usage: \multirow{Zeilen}*{Text} 
% then: let & in other rows existant. replace \hline by \cline{2-5}
\usepackage{makecell}  
\usepackage{rotating}  
% rotate elements of tables
% usage: \rotcell{Text}
% Description: Rotates automatically by 90 deg
\usepackage{pdflscape}
% puts selected pages in landscape mode
% usage: \begin{landscape} ... \end{landscape}

%\usepackage{caption}
%\DeclareCaptionFormat{myformat}{#1#2#3\hrulefill}
%\DeclareCaptionFormat{myformat}{#1#2#3\vspace{-6pt}\par\rule{0.15\linewidth}{0.1pt}}
%\captionsetup[figure]{format=myformat}


%\usepackage{slashbox}
\textwidth16.0cm
\oddsidemargin0.3cm
\evensidemargin-0.3cm
\setlength{\parindent}{0em}
\setlength{\parskip}{0.4ex plus0.2ex minus0.2ex}
\newcommand{\bst}{$\beta^*$}
\renewcommand{\baselinestretch}{1.2}
\renewcommand{\arraystretch}{1.2}
%\frenchspacing
\sloppy \hfuzz 2pt \vfuzz 2pt
\renewcommand{\topfraction}{0.9}
\renewcommand{\bottomfraction}{0.9}
\renewcommand{\textfraction}{0.1}

\usepackage{listings}%__________allow writing source code in the latex document

%%%% Design of the report

\addto\captionsenglish{%
 \renewcommand\chaptername{}} %don't write chapter at beginning of chapters

\usepackage{titlesec}

\titleformat{\chapter}[hang] 
{\normalfont\Huge\bfseries}{\chaptertitlename\ \thechapter.}{1em}{} 




%%%% Definition of own symbols

\usepackage{tikz}
\newcommand*\circled[1]{\tikz[baseline=(char.base)]{
  \node[shape=circle,draw,inner sep=1pt] (char) {#1};}}




%%%% Table design

\usepackage{booktabs}%__________ use toprule midrule and bottomrule
\usepackage{xcolor}%____________ define plots of percentages and so on
\newlength{\basis}\setlength{\basis}{1cm }
\newlength{\balkenlaenge}
\newcommand*{\Balken}[1]{\setlength{\balkenlaenge}{0.01\basis}\setlength{\balkenlaenge}
 {#1\balkenlaenge} \textcolor{black!40}{\rule{\balkenlaenge}{1.5ex}} & #1\,\%}

%\renewcommand\thempfootnote{\arabic{mpfootnote}}
\renewcommand{\eqref}[1]{Eq.~\ref{#1}}
\newcommand{\phase}{\phi}
\newcommand{\bone}{\mbox{Beam 1 }}
\newcommand{\btwo}{\mbox{Beam 2 }}
\newcommand{\figref}[1]{\mbox{Fig. \ref{#1}}}
\newcommand{\citedr}{\cite{CERN-2004-003-V1}}
\newcommand{\chapref}[1]{\mbox{Chap. \ref{#1}}}
\newcommand{\tabref}[1]{\mbox{Table \ref{#1}}}

\newcommand{\minipages}[4]{
\begin{figure}[t]
\begin{minipage}[t]{0.475\textwidth}
%\centering
\includegraphics[width=\textwidth]{pictures/#1}
\end{minipage}
\hfill
\begin{minipage}[t]{0.475\textwidth}
\centering
\includegraphics[width=\textwidth]{pictures/#2}
\end{minipage}
\caption{ #3 }
\label{#4}
\end{figure}
}

\newcommand{\minipageswidth}[5]{
\begin{figure}[t]
\begin{minipage}[t]{#5\textwidth}
%\centering
\includegraphics[width=\textwidth]{pictures/#1}
\end{minipage}
\hfill
\begin{minipage}[t]{#5\textwidth}
\centering
\includegraphics[width=\textwidth]{pictures/#2}
\end{minipage}
\caption{ #3 }
\label{#4}
\end{figure}
}

\newcommand{\minipageswidthtb}[6]{
\begin{figure}[#6]
\begin{minipage}[t]{#5\textwidth}
%\centering
\includegraphics[width=\textwidth]{pictures/#1}
\end{minipage}
\hfill
\begin{minipage}[t]{#5\textwidth}
\centering
\includegraphics[width=\textwidth]{pictures/#2}
\end{minipage}
\caption{ #3 }
\label{#4}
\end{figure}
}

\newcommand{\fourpictures}[7]{
\begin{figure}[#7]
\begin{minipage}[t]{0.5\textwidth}
%\centering
\includegraphics[width=\textwidth]{pictures/#1}
\includegraphics[width=\textwidth]{pictures/#3}
\end{minipage}
\hfill
\begin{minipage}[t]{0.5\textwidth}
\centering
\includegraphics[width=\textwidth]{pictures/#2}
\includegraphics[width=\textwidth]{pictures/#4}
\end{minipage}
\caption{ #5 }
\label{#6}
\end{figure}
}




%\renewcommand\thechapter{\Roman{chapter}}
\pagestyle{headings}

%\usepackage[a4paper,margin=1cm,landscape]{geometry}
%\usepackage{color}
%\usepackage{ifpdf}
