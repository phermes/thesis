\chapter[Applications of Heavy-Ion Collimation Simulations]{Applications of Heavy-Ion Collimation \\ Simulations} \label{chap:sim_meas}

%\chapter[toc version]{doc version}
\chaptermark{Applications of Heavy-Ion Collimation Simulations}

In this chapter, the new simulation tools are applied to different LHC and HL-LHC configurations. STIER\footnote{At the time the studies were conducted, the hiSixTrack-FLUKA coupling was not yet available.} simulations are presented, which were used to validate the collimator settings for the 2015 heavy-ion run. Furthermore, STIER is used to probe alternative optics with additional orbit bumps in IR7 for their potential of loss reduction. Finally, it is presented how STIER simulations were used in the 2015 heavy-ion run to reduce the loss rate at the IR2 TCT.

Another application presented in this chapter is the study of the cleaning performance in \mbox{HL-LHC}. The simulations are carried out with the hiSixTrack-FLUKA coupling. Prospective collimation system upgrades are simulated, to estimate their potential for loss reduction. Finally, the cleaning performance in the HL-LHC configuration is studied with heavy ions different \mbox{than \lead} which could possibly be stored in the LHC.



\section{Preparation of the LHC Heavy-Ion Operation in 2015} \label{chap:prep2015}

\begin{figure}[t]  
    \centering
    \includegraphics[width=0.6\textwidth]{pictures/16020502.pdf}
    \caption{Stored heavy-ion beam energy in the past LHC heavy-ion runs. The error bars are derived from the uncertainty of the measured bunch intensity.}  
    \label{pic:16020502}
    %/home/phermes/Dropbox/PhD/notebooks/plots/stored_beam_energy.pdf
\end{figure}

During the operation with \lead beams in late 2015, the LHC reached particle momenta of 6.37$\,Z\,$TeV and stored a beam energy of approximately 9.5~MJ~\cite{IPAC16:TUPMW027}. The latter corresponds to more than twice the design value~\citedr. While the number of injected bunches \mbox{($n_B=426$)} was still below the design value of $n_B^{d} = 518$, the excellent injector performance led to a significant increase of the bunch intensity to $(2.2 \pm 0.3)$ ions per bunch~\cite{IPAC16:TUPMW027}. In combination with the small emittance of $\epsilon_N=$(1.50$\pm 0.15)$\mum\,rad achieved, these improvements enabled luminosities surpassing the design value by a \mbox{factor of 3.6}~\cite{IPAC16:TUPMW027}. The $\beta^*$-value of 0.8~m applied in ATLAS, CMS and ALICE was slightly above the design value of 0.5~m~\citedr.

The stored beam energy was highly challenging for the LHC collimation system. Occasionally, the amount of collimation debris measured with the BLMs in the IR7 DS exceeded the allowed thresholds and triggered protection dumps. Furthermore, losses at the tertiary collimators produced a perturbing background in the experiments, especially in ALICE (see \chapref{chap:ir2loss}). 

\newpage
Alongside the physics program, which was accomplished with great success and unprecedented integrated luminosities collected~\cite{IPAC16:TUPMW027}, several machine experiments were carried out with the LHC. These included two quench tests. In one of these experiments, a main dipole was quenched with the secondary BFPP beams generated in IP5~\cite{accnote_bfpp_quench}. At the last day of heavy-ion operation, a collimation quench test was carried out with collimation debris generated at the primary collimator in IR7. This experiment is described in detail in \chapref{chapter:quenchtest}.

In this section, cleaning simulations carried out for the preparation and optimization of the collimation system in the 2015 configuration are presented.

\subsection{Validation of Collimator Settings} \label{chap:STIER:validation}

In preparation of the 2015 heavy-ion run, STIER was used to estimate the expected cleaning performance and to study possible strategies to reduce the cleaning inefficiency in the IR7 DS magnets. This includes the study of different collimator settings, in particular different retractions of the TCSG collimators. The study serves the purpose of validating the collimation system and give estimates about the expected loss reduction with tighter settings. 

With the geometrical collimator settings taken from the precedent proton operation (see \tabref{tab:14070901}), the TCSG collimators are retracted by $\Delta N_{S,P}=2.5\,\sigma$ with respect to the primary collimators. STIER is employed to simulate the cleaning inefficiency for the reference settings and in addition with smaller retractions of $ 2.0\,\sigma, 1.5\,\sigma$ and $1.0\,\sigma$. 

\newpage
In the fragmentation simulation, the \lead beam is simulated to impact the carbon target in the same geometry already used for the simulations presented in \chapref{chap:stier}. The impact parameter in these simulations is set to $b=2\,\mu$m. With an initial sample of $6\times10^6$ heavy ions impacting the TCP jaw, the fragment distribution obtained from FLUKA is processed into initial coordinates in B1H for STIER. The simulation is carried out for both jaws individually to disentangle the losses arising from particles starting at each jaw. This approach turned out to be very useful for the analysis and mitigation of the losses at the TCT in IR2, as discussed in \chapref{chap:ir2loss}. For the four simulations with different TCSG retractions, the same sample of initial coordinates is used, so the obtained loss patterns are quantitatively comparable.

\begin{figure}[t]
  \centering
  \begin{tikzpicture}
    \footnotesize
    \node[anchor=south west,inner sep=0] (image) at (0,0) {\includegraphics[width=1.0\linewidth]{pictures/16061502.pdf}};
      \node [rotate=0 , fill=white, x={(image.south east)},y={(image.north west)},anchor=east]       at (0.995,0.92)  {$2.5\,\sigma$ retraction (nominal)};
      \node [rotate=0 ,fill=white, x={(image.south east)},y={(image.north west)},anchor=east]       at (0.995,0.70)  {$2.0\,\sigma$ retraction };
      \node [rotate=0 ,fill=white, x={(image.south east)},y={(image.north west)},anchor=east]       at (0.995,0.49)  {$1.5\,\sigma$ retraction };
      \node [rotate=0 ,fill=white, x={(image.south east)},y={(image.north west)},anchor=east]       at (0.995,0.27)  {$1.0\,\sigma$ retraction };
      \tiny
      \node [rotate=90 ,fill=white, x={(image.south east)},y={(image.north west)}]                   at (0.155,0.17)  {TCSG.A6L7.B1};
  \end{tikzpicture}
  \caption{IR7 view of the STIER simulated loss maps for \lead ions at 6.37$\,$TeV in the configuration of the 2015 heavy-ion run for different retractions of the TCSG collimators with respect to the primary collimators. }  
  \label{pic:16061502}
  %/media/phermes/local/160614_STIER_validation/analysis/examples/validation_comparison.pdf
  \end{figure}


\begin{figure}[htbp]
	% minipage mit (Blind-)Text
	\begin{minipage}{0.5\textwidth} 
  \centering
  \begin{tikzpicture}
    \footnotesize
    \node[anchor=south west,inner sep=0] (image) at (0,0)      {\includegraphics[width=1.0\linewidth]{pictures/16061604.pdf}};
    %
    %  \node [draw,rotate=90,x={(image.south east)},y={(image.north west)}]  at (0.50,0.50)  {text0};
    %  pure text 
        \node [draw,rotate=0 ,fill=white, x={(image.south east)},y={(image.north west)}]                   at (0.36,0.6)  {DS1};
        \node [draw,rotate=0, fill=white ,x={(image.south east)},y={(image.north west)}]       at (0.75,0.6)  {DS2};
        \end{tikzpicture}
	\end{minipage}
	\hfill
	\begin{minipage}{0.5\textwidth}
  \centering
  \begin{tikzpicture}
    \node[anchor=south west,inner sep=0] (image) at (0,0) {\includegraphics[width=1.0\linewidth]{pictures/16082301.pdf}};
  \end{tikzpicture}
	\end{minipage}
	\caption{Left: Cleaning inefficiency in the dispersion suppressor loss clusters with nominal TCSG retraction (top panel) and with a TCSG retraction of 1.0$\,\sigma$ (middle panel). Left bottom panel: loss reduction with 1.0$\,\sigma$ retraction compared to the nominal retraction. Right: Evolution of the highest cleaning inefficiency and the integrated losses in DS1 and DS2 as a function of the applied TCSG retraction. The error bars correspond to the statistical errors.}
	\label{pic:quantiative_TCSG_reduction} 
 %/media/phermes/local/160614_STIER_validation/analysis/examples/cleaning_evolution.pdf
\end{figure}


The loss maps simulated with STIER are shown zoomed to IR7 in \figref{pic:16061502} and for the full ring in \figref{pic:16061503}. With the smaller TCSG half gap, additional losses occur only at the first secondary collimator (TCSG.A6L7.B1) downstream of the TCP. For the tightest setting studied, the losses at this collimator are increased by 60\% with \mbox{respect to the nominal case}. 


The loss pattern simulated with STIER is similar in all studied cases. For the losses in the IR7 DS, slight differences in the loss pattern are visible for the different scenarios. The simulated cleaning performance with nominal retraction is quantitatively compared to the simulation result with the tightest settings in \figref{pic:quantiative_TCSG_reduction}. Also, selected numeric key quantities as a measure of the cleaning performance evolution with decreasing TCSG retraction are listed in \tabref{tab:2015_performance_param}. 

\begin{figure}[p]
  \centering
  \begin{tikzpicture}
    \footnotesize
    \node[anchor=south west,inner sep=0] (image) at (0,0) {\includegraphics[width=1.0\linewidth]{pictures/16061505.pdf}};

      \node [rotate=0 , fill=white, align=right, x={(image.south east)},y={(image.north west)},anchor=east]       at (0.995,0.95)  {$2.5\,\sigma$ retraction \\ (nominal)};
      \node [rotate=0 ,x={(image.south east)},y={(image.north west)},anchor=east]       at (0.995,0.73)  {$2.0\,\sigma$ retraction };
      \node [rotate=0 ,x={(image.south east)},y={(image.north west)},anchor=east]       at (0.995,0.49)  {$1.5\,\sigma$ retraction };
      \node [rotate=0 ,x={(image.south east)},y={(image.north west)},anchor=east]       at (0.995,0.25)  {$1.0\,\sigma$ retraction };
      \footnotesize
      \node [rotate=0 ,x={(image.south east)},y={(image.north west)},anchor=east]       at (0.8,1.02)  {STIER B1H loss map, $^{208}$Pb$^{82+}$, $E = 6.37\,Z\,$TeV, $N_P=5.5\,\sigma$};
  \end{tikzpicture}
  \caption{STIER simulated loss maps in the 2015 configuration with \lead beams at $6.37\,Z\,$TeV for different retractions of the TCSG collimators.}  
  \label{pic:16061503}
  %/media/phermes/local/160614_STIER_validation/analysis/examples/validation_comparison_LHC.pdf
  \end{figure}


\begin{table}[htbp]
\centering
\caption{STIER simulation results quantifying the cleaning performance as a function of the applied TCSG retraction $\Delta N_{S,P}$ for the 2015 heavy-ion run. The quantity $\eta_{cold}^{max}$ is the highest cleaning inefficiency in cold LHC regions. The uncertainties correspond to the statistical errors.}
\label{tab:2015_performance_param}
\begin{tabular}{cccc}
\toprule
\begin{tabular}[c]{@{}c@{}}$\Delta N_{S,P}$ \\ {[}$\sigma${]}\end{tabular} & 
\begin{tabular}[c]{@{}c@{}}$\eta^{max}_{cold}$\\ {[$10^{-2}/$m]}\end{tabular} &  
\begin{tabular}[c]{@{}c@{}}$ \eta^{int}_{DS1}$\\ {[$10^{-2}$]}\end{tabular} &  
\begin{tabular}[c]{@{}c@{}}$ \eta^{int}_{DS2}$\\ {[$10^{-2}$]}\end{tabular} %&  
%\begin{tabular}[c]{@{}c@{}}$ \eta^{int}_{cold}$\\ {[$10^{-2}$]}\end{tabular}     
\\ \midrule
    2.5      &      $1.44\pm 0.01$  &  $8.1\pm 0.1$   &  $5.5\pm 0.1$       
%$14.0\pm 0.1$   
\\
    2.0      &      $1.41\pm 0.01$  &  $6.8\pm 0.1$   &  $5.5\pm 0.1$       
%$12.7\pm 0.1$   
\\
    1.5      &      $1.31\pm 0.01$  &  $5.6 \pm 0.1$   &  $5.4\pm 0.1$      
%$11.3\pm 0.1$
   \\
    1.0      &      $1.07\pm 0.01$  &  $4.4 \pm 0.1$   &  $5.3\pm 0.1$      
%$10.0\pm 0.1$   
\\ \bottomrule
\end{tabular}
\end{table}
\newpage

The comparison shows that, with tighter setting, the TCSG captures particularly ion fragments which are lost in the first DS loss cluster otherwise. In terms of integrated losses $\eta^{int}$, a reduction of almost 50\% is simulated in the DS1, if the smallest retraction is applied. For the DS2, the integrated losses are only reduced by 4\%. The peak loss in the superconducting LHC regions is found in the DS1 for all simulations, and is predicted to be reduced by approximately $25\%$ with the tightest setting. The loss distribution in the remaining LHC ring is not changed significantly. An analysis of the loss pattern in IR2 is presented in \chapref{chap:ir2loss}.

In conclusion, the analysis shows that a small reduction of the DS losses could be achieved by varying the TCSG settings. One remaining uncertainty are particles generated in secondary collimator interactions, which could still be lost in the DS1. Given the low potential for loss reduction, the nominal retraction was maintained for the 2015 heavy-ion run.



\subsection{Orbit Bumps in the IR7 Dispersion Suppressor} \label{chap:orbump}

%\subsubsection{Introduction}

The loss location of off-rigid particles which are lost due to magnetic dispersion can be modified by means of dedicated orbit changes. In the 2015 heavy-ion run, the BFPP losses in IR1 and IR5 were shifted into an empty connection cryostat between two superconducting magnets. This reduced the amount of energy lost in the coils of the superconducting magnets and hence also the risk of magnet quench~\cite{PRSTAB:12:071002,IPAC16:TUPMW028}. 

\begin{figure}[b]  
    \centering
    \includegraphics[width=0.6\textwidth]{pictures/16052601.pdf}
    \caption{Projected penetration depth from the extrapolated particle track in the aperture.}  
    \label{pic:16052601}
    %/home/phermes/Dropbox/PhD/pictures/160526_penetration_depth/annotated/drawing_annotated-crop.pdf
\end{figure}

Such shifts of the loss position can be achieved if the theoretical particle trajectory (simulated without aperture restrictions) is beating with the dispersion function and the projected penetration depth (PPD) $d_p$ in the aperture is not too large. The latter is the difference between the dimension of the aperture and the maximum amplitude a particle would reach at the dispersive peak downstream of the loss location (which is at a quadrupole center), as illustrated in \figref{pic:16052601}. 

\newpage
As shown in \figref{pic:16052602}, the loss location can be altered towards a dispersive peak further downstream, if $d_p$ is small enough to be compensated by a moderate orbit bump in the opposite direction. Additional bumps in the machine reduce the normalized aperture at the bump location and introduce additional dispersion in the machine, which must be taken into account and limits the achievable bump amplitude. The peak amplitude of the 2015 BFPP bump is $3\,$mm. 

The potential reduction of collimation losses by compensating bumps is limited by the asymmetry of the loss distribution in $x$. The maximum  relative reduction of the cleaning inefficiency $\Delta \bar{\eta}_{max}$ as a function of the cleaning inefficiency from particles impacting on the right side of the beam pipe $\eta_{R}$ and those impacting at the left side $\eta_L$ is given by
\begin{align}
\Delta \bar{\eta}_{max} = \frac{\eta_R}{\eta_L + \eta_R} \, ,
\end{align}
assuming that more particles impact the right side of the beam pipe.  In case of a full asymmetry of the losses, they can in theory be fully shifted by a dedicated bump if this does not increase the loss rate at the other side of the beam pipe.

\begin{figure}[t]  
    \centering
    \includegraphics[width=0.8\textwidth]{pictures/16052603.pdf}
    \caption{Loss location shift of an off-rigid particle by a compensating orbit bump.}  
    \label{pic:16052602}
    %/afs/cern.ch/work/p/phermes/private/150629_coupling_ions/hiSix/160526_tracks/loss_shift_with_bump.pdf
\end{figure}


\subsubsection{PPD Estimates}

The possibility of collimation loss mitigation with orbit bumps in the IR7 DS can be explored from the STIER simulation data already produced for the study presented in the previous section. The analysis of this data shows that approximately 75\% of the collimation debris lost in the DS1 impacts the beam pipe aperture on the left hand side ($x>0$). In the DS2, this fraction is even 97\%. This finding is in line with the asymmetry in $\chi$ observed for the fragments scattered out of the primary collimator (see \figref{pic:16061608}).

% loss mitigation with orbit bumps in the IR7 DS is explored in a dedicated STIER simulation for B1H in the 2015 heavy-ion run configuration. The fragmentation simulation setup and the reference simulation are taken over from the settings validation simulation presented in the previous chapter. 

 %These numbers also describe the maximum potential for reduction, which is fully exploited if all losses at the dominating side of impact are alleviated (assuming that the number of lost particles at the other side of the beam pipe does not increase).


\begin{figure}[htbp]
  \centering
  \begin{tikzpicture}
    \node[anchor=south west,inner sep=0] (image) at (0,0) {\includegraphics[width=0.5\linewidth]{pictures/16061608.pdf}};
    %
    %  \node [draw,rotate=90,x={(image.south east)},y={(image.north west)}]  at (0.50,0.50)  {text0};
    %  pure text 
    %  \node [draw,rotate=0 ,x={(image.south east)},y={(image.north west)}]       at (0.22,0.965)  {text1};
    %  \node [draw,rotate=0 ,x={(image.south east)},y={(image.north west)},anchor=west]       at (0.22,0.8)    {text2};
    %
  \end{tikzpicture}
  \caption{Distribution in $\chi$ of the particles scattered out of the primary collimator, simulated with FLUKA for a primary beam of  \lead ions at 6.37$\,Z$TeV.}  
  \label{pic:16061608}
  %/media/phermes/ph3tboffice/ph1tbwd/FLUKA_results/151105_HeavIon_6370GeV_2um_runII_2015/raw_data/chi_distribution_2015.pdf
  \end{figure}

\begin{figure}[htbp]
  \centering
  \begin{tikzpicture}
    \node[anchor=south west,inner sep=0] (image) at (0,0) {\includegraphics[width=1.0\linewidth]{pictures/16061703.pdf}};
    %
      \node [draw,rotate=0,x={(image.south east)},y={(image.north west)}]  at (0.45,0.925)  {DS1};
      \node [draw,rotate=0,x={(image.south east)},y={(image.north west)}]  at (0.95,0.925)  {DS2};

    %  pure text 
    %  \node [draw,rotate=0 ,x={(image.south east)},y={(image.north west)}]       at (0.22,0.965)  {text1};
    %  \node [draw,rotate=0 ,x={(image.south east)},y={(image.north west)},anchor=west]       at (0.22,0.8)    {text2};
    %
  \end{tikzpicture}
  \caption{Top row: Histogram of the projected penetration depth of the different isotopes impacting the left side of the beam pipe in DS1 and DS2, weighted with the ion momentum. Both histograms are normalized by the total momentum of the impacting ions. Bottom row: Potential loss reduction as a function of the bump amplitude applied at each loss region. }  
  \label{pic:16061701}
  %/home/phermes/Dropbox/codes/python/160309_STIER_runII_bumps/amplitude_depth.pdf
  \end{figure}



The possible loss reduction by an orbit bump can be studied by means of the PPD of the isotopes impacting at the location of the two DS clusters. The quantity $d_p$ is derived from the STIER simulation data by a linear extrapolation of the particle trajectory. The extrapolation is based on information on the longitudinal distance between impact location and quadrupole center (in which the betatron function and dispersion are maximum) and the impacting angle at the aperture. This method allows to relate a target loss reduction to the required bump amplitude at the loss location. 

The distribution of the projected penetration depth of particles impacting the left hand side of the DS magnets is shown for the DS loss clusters in \figref{pic:16061701}. Both clusters show a broad distribution of $d_p$ reaching to values up to more than 25$\,$mm. In the bottom panel of \figref{pic:16061701}, the integrated losses are shown as a function of the projected penetration depth. This can be interpreted as the reachable loss reduction with respect to a given bump amplitude. Both integrated loss curves are calculated with respect to the highest achievable loss reduction, determined from the loss asymmetry. The contribution of each impacting particle is weighted with its momentum.





\begin{figure}[t]
  \centering
  \begin{tikzpicture}
    \footnotesize
    \node[anchor=south west,inner sep=0] (image) at (0,0) {\includegraphics[width=0.95\linewidth]{pictures/16061706.pdf}};
    %
      \node [fill=white,x={(image.south east)},y={(image.north west)}]  at (0.30,0.91)  {Nominal 2015};
      \node [fill=white,x={(image.south east)},y={(image.north west)}]  at (0.30,0.62)  {2015 with bump};

      \node [fill=white,x={(image.south east)},y={(image.north west)}]  at (0.50,0.87)  {DS1};
      \node [fill=white,x={(image.south east)},y={(image.north west)}]  at (0.565,0.87)  {DS2};

      \node [fill=white,x={(image.south east)},y={(image.north west)}]  at (0.65,0.55)  {A1};
      \node [fill=white,x={(image.south east)},y={(image.north west)}]  at (0.875,0.55)  {A2};


%$      \node [fill=white,x={(image.south east)},y={(image.north west)}]  at (0.30,0.91)  {};
    %  pure text 
    %  \node [draw,rotate=0 ,x={(image.south east)},y={(image.north west)}]       at (0.22,0.965)  {text1};
    %  \node [draw,rotate=0 ,x={(image.south east)},y={(image.north west)},anchor=west]       at (0.22,0.8)    {text2};
    %
  \end{tikzpicture}
  \caption{Top row: nominal STIER simulated cleaning inefficiency in the 2015 heavy-ion configuration. Middle row: STIER simulation with an additional orbit bump at a maximum amplitude of $x_b=+3\,$mm at the DS2. Bottom row: beam orbit and periodic dispersion function (the local dispersion function from the TCP is similar) with the applied bump.}  
  \label{pic:16052701}
  %/home/phermes/Dropbox/codes/python/160309_STIER_runII_bumps/lossmap_bump_comparison.pdf
  \end{figure}

The bump amplitude required to alleviate 50\% of the losses yields approximately 17$\,$mm in the DS1 and $14\,$mm in the DS2. Both of them are significantly too large to be applied in operation (the horizontal beam pipe aperture is only 22~mm). Applicable bump amplitudes are rather in the order of a few mm. Assuming a maximum bump amplitude of 3\,mm, the reachable loss reduction yields only 6\% in the DS1 and 14\% in the DS2. 

\subsubsection{STIER Simulation with Orbit Bump}

The conclusion of the previous sub-section is supported by the outcome of a dedicated STIER simulation in which an additional horizontal bump in positive direction is applied.  The bump has a maximum amplitude of \mbox{$3\,$mm} in the DS2 cluster (see \figref{pic:16052701}). Apart from the bump, the configuration is identical to the reference STIER simulation of the 2015 heavy-ion run. Also the initial distribution of fragments is identical to that used for the previous studies.

The simulated loss map is compared to the STIER simulation with nominal optics settings in \figref{pic:16052701}.
The bump amplitude in the region of the DS1 cluster is small, so the loss pattern is very similar to the nominal one. The shape of the loss pattern in the DS2 cluster is changed, but the peak amplitude is not significantly reduced. The change of the loss distribution can be explained by the fact that the loss location of the individual isotopes is shifted to the right. This is especially true for the losses at the right bound of the DS2, which are shifted into the first loss peak of the LHC arc region (A1) which results in an increased loss signal at this location. The integrated losses in the DS2 are reduced by approximately 13\%. Note that the bump is not at its maximum amplitude over the full length of the DS2 cluster.


\newpage
The loss reduction derived from the STIER simulation with altered reference orbit is in excellent agreement with that derived from the PPD distribution. This demonstrates that the latter can be used to accurately predict the reduction of losses from orbit or aperture shifts. This method is used in \chapref{chap:impr} to study discrepancies between measured and simulated losses.

\subsubsection{Conclusions}

In conclusion, the analysis of the projected penetration depth shows that the alleviation of the DS losses would require very large bump amplitudes, which are not compatible with safe operation of the LHC. Also the shift of losses towards the connection cryostat, as it is done for the BFPP losses, is not possible within the available margins of bump amplitudes, because the rigidity offset of the impacting particles is too large. On the contrary, the loss reduction that is achievable with reasonable bump amplitudes is not sufficiently beneficial to justify the additional effort of integrating it into the operational configuration. 

% This simulation has furthermore proven the method of loss prediction by means of the projected penetration depth to be accurate. In the analysis of loss patterns, this method could also be useful to explain discrepancies between measured and simulated losses, because of orbit offsets or aperture displacements (see \chapref{}).



% My hypothesis was that in the DS there are particles ending up at both
% sides of the beam pipe (x<0, x>0), because particles with both signs of
% delta_eff are generated. Now Roderik had the idea that I could check if
% the same amount of energy deposited on both sides is identical. As a
% matter of fact the energy deposited on x>0 is significantly larger than
% on the other side (~factor 20).

% The reason why the bumps are not effective could be that the particles
% are too much off momentum and the bump amplitude is not sufficient to
% avoid them being lost.

% Find attached the comparison of the loss maps for three cases:

%   * nominal without bump
%   * bump of +3mm at MQ.9R7
%   * bump of -3mm at MQ.9R7

% The simulation is done for Run II conditions. 

  % [(3, 1),    0.15161517193976892],
  % [(203, 82), 0.059987912744568654],
  % [(202, 82), 0.054357628332861764],
  % [(199, 81), 0.04103390268372144],
  % [(200, 81), 0.0406380961080765],
  % [(201, 82), 0.035369827352218386],
  % [(201, 81), 0.03490685772452313],
  % [(198, 81), 0.028612559490733408],
  % [(7, 3),    0.028273891240351855],
  % [(200, 82), 0.026975152549905734],
  % [(198, 80), 0.026866058395211372],
  % [(196, 80), 0.02676397893184573],
  % [(197, 80), 0.02287859898815695],
  % [(197, 81), 0.020408720936471763],
  % [(8, 3),    0.017968917065147375],
  % [(10, 4),   0.01708417603420238],
  % [(2, 1),    0.01683096661661283],
  % [(195, 80), 0.016261258514147035],
  % [(195, 79), 0.015158800309798085],
  % [(6, 2),    0.01413517886023306], 
%
%
%
%

\newpage

\section{Suppression of Losses at the IR2 TCT}\label{chap:ir2loss}

% \begin{figure}[bthp]  
%     \centering
%     \begin{tikzpicture}
%       \footnotesize
%       \node[anchor=south west,inner sep=0] (image) at (0,0) {\includegraphics[width=1.0\linewidth]{pictures/16020101.pdf}};
%       \node [rotate=0,x={(image.south east)},y={(image.north west)}]  at (0.24,0.85)  {TCT2};
%       \node [rotate=0,x={(image.south east)},y={(image.north west)}]  at (0.5,1.02)  {Betatron loss map, B1H, 07.12.2015, 17:17:33h};
%       \draw [>=,x={(image.south east)},y={(image.north west)}] (0.24,0.80) -- (0.19,0.7);
%     \end{tikzpicture}
%       \caption{Beam 1 horizontal qualification loss map measured in the 2015 heavy-ion run.}  
%     \label{fig:2015_lossmap}
%     %/afs/cern.ch/work/p/phermes/private/151124_ion_lossmaps_runII/raw/IR2_TCT_loss_uncorrected.pdf
% \end{figure}

In the commissioning phase of the 2015 heavy-ion run, also measured qualification loss maps became available. The global B1H betatron qualification loss map with squeezed beams is shown in the top panel of \figref{pic:16090502}. The relative BLM signal in the IR7 DS reaches values of $\mathcal{B}_\text{DS} \approx 10^{-2}$, comparable to the 2011 heavy-ion run. Again, individual loss spikes are located in the arc regions over the complete ring. Compared to the operation in 2011, some of them are at different locations. A very high loss signal was measured at the horizontal tertiary collimator for B1 in IR2 (TCTHP.4L2.B1). In the following, this collimator is referred to as TCT2. The normalized BLM signal\footnote{Measured with the BLMTI.04L2.B1E10\_TCTPH.4L2.B1.} at this location yields 
%
\begin{align}
  \mathcal{B}_\text{TCT2} = (6.6 \pm 0.5) \times 10^{-2} \, .
\end{align}
%
The uncertainty is estimated from variations of the measured $\mathcal{B}_\text{TCT2}$ in different measurement campaigns. It corresponds to the difference between the largest and the smallest $\mathcal{B}_\text{TCT2}$ measured.%
%
The high collimation losses at the TCT2 were present during physics operation as a regular background also without additional beam excitation. The secondary showers created from the particles impacting the TCT2 caused a radiation background and hence distorted the operation of the ALICE experiment. The STIER simulation output was consulted to understand these losses and work out possible strategies to reduce them.


\subsection{Situation and Analysis}


\begin{figure}[htbp]
  \centering
  \begin{tikzpicture}
    \footnotesize
    \node[anchor=south west,inner sep=0] (image) at (0,0) {\includegraphics[width=1.0\linewidth]{pictures/16090502.pdf}};
  % \node [draw,rotate=90,x={(image.south east)},y={(image.north west)}]                   at (0.50,0.50)    {text0};
  % \node [draw,rotate=0 ,x={(image.south east)},y={(image.north west)}]                   at (0.22,0.96)    {text1};
  % \node [draw,rotate=0 ,x={(image.south east)},y={(image.north west)},anchor=west]       at (0.22,0.80)    {text2};
  % \draw[->,color=black,thick,x={(image.south east)},y={(image.north west)}]             (0.42,0.22) -- (0.37,0.23);

%  \draw[help lines,step=.05,x={(image.south east)},y={(image.north west)}] (0,0) grid (1,1);
%  \draw[help lines,line width=.6pt,step=0.1,x={(image.south east)},y={(image.north west)}] (0,0) grid (1,1);
%  \foreach \x in {0,0.1,0.2,0.3,0.4,0.5,0.6,0.7,0.8,0.9,1.0}
%       \node[anchor=north,x={(image.south east)},y={(image.north west)}] at (\x,-0.01) {\x};
%  \foreach \y in {0.0,0.1,0.2,0.3,0.4,0.5,0.6,0.7,0.8,0.9,1.0}
%      \node[anchor=east,x={(image.south east)},y={(image.north west)}] at (-0.01,\y) {\y};

    \node [rotate=90,fill=white,rotate=0,x={(image.south east)},y={(image.north west)},align=left]                   at (0.17,0.85)    {TCT2};


    \node [fill=white,rotate=0,x={(image.south east)},y={(image.north west)},align=left]                   at (0.20,0.97)    {Qualification loss map \\ 24/11/2015 10:44:06h};


    \node [fill=white,rotate=0,x={(image.south east)},y={(image.north west)},align=center]                   at (0.50,0.63)    {STIER: Fragments from \\ left TCP jaw};

    \node [fill=white,rotate=0,x={(image.south east)},y={(image.north west)},align=center]                   at (0.50,0.31)    {STIER: Fragments from \\ right TCP jaw};

  \end{tikzpicture}
  \caption{Top panel: B1H qualification loss map measured in the 2015 heavy-ion run. Middle and bottom panel: STIER simulation result of the \lead cleaning inefficiency for B1H in the configuration of the 2015 heavy-ion run at 6.37$\,Z\,$TeV. The simulation result is shown for particles starting at the left and at the right horizontal TCP jaw separately. Both simulations are carried out with $6 \times 10^6$ initial \lead particles.}  
  \label{pic:16090502}
  %/home/phermes/Dropbox/PhD/pictures/160421_compare_STIER/LHC_left_right_BLM.pdf
  \end{figure}




% \begin{figure}[b]
%   \centering
%   \begin{tikzpicture}
%     \footnotesize
%     \node[anchor=south west,inner sep=0] (image) at (0,0) {\includegraphics[width=1.0\linewidth]{pictures/16062901.pdf}};


%     \node [fill=white,rotate=90,x={(image.south east)},y={(image.north west)}]                   at (0.165,0.85)    {TCT2};
%   \end{tikzpicture}
%   \caption{}  
%   \label{pic:16062901}
%   %/home/phermes/Dropbox/PhD/pictures/160421_compare_STIER/LHC_left_right.pdf
% \end{figure}

Considering that the losses occur at the horizontal TCT for B1, the STIER simulation presented in \chapref{chap:STIER:validation} can be taken as a reference for the analysis. The global loss map simulated with STIER is shown, disentangled for the particles starting at the left and the right TCP jaw, in the middle and bottom panel of \figref{pic:16090502}. With the TCT2 losses clearly visible in both simulations, the comparison unveils that the larger fraction originates from the left TCP jaw. The fragments starting from the left jaw cause 20 times more losses than those starting from the right jaw. 

\begin{table}[t]
	\centering
	\caption{Loss composition at the TCT2 simulated with STIER for B1H in the 2015 configuration using a TCP impact parameter of 2$\,\mu$m. The energy fraction is calculated from the number of particles of each isotope, weighted with the particle momenta.}
    \label{tab:2015_ionrun}
	\begin{tabular}{ccc}
		\toprule
		\begin{tabular}[c]{@{}c@{}}Isotope\\ (A,Z)\end{tabular} & \begin{tabular}[c]{@{}c@{}}TCP\\ jaw\end{tabular} & \begin{tabular}[c]{@{}c@{}}Fraction\\ (\%)\end{tabular} \\ \midrule
		\iso{207}{Pb}{82+} & left  & 92.5 \\
                \iso{204}{Tl}{81+} & right & 3.6  \\
		\iso{202}{Hg}{80+} & left  & 2.2  \\
		\iso{199}{Au}{79+} & right & 0.3  \\  \bottomrule
	\end{tabular}
\end{table}

\begin{figure}[b]  
    \centering
    \includegraphics[width=1.0\textwidth]{pictures/16020406.pdf}
    \caption{Simulated track of the secondary $^{207}$Pb$^{82+}$ beam starting at the left jaw of the horizontal TCP in IR7, which is lost at the TCT2. }  
    \label{pic:16020405}
    %/media/phermes/ph3tboffice/ph1tbwd/160112_IR2_loss_mitigation/output_82_207_7/plots/pb207_tracks.pdf
\end{figure}



The quantitative analysis summarized in \tabref{tab:2015_ionrun} shows that the isotope $^{207}$Pb$^{82+}$ starting at the left TCP jaw is clearly dominating over all other isotopes lost at the TCT2. Approximately 92.5\% of the total simulated TCT2 signal (by combining the simulations for the individual jaws) is caused by this single isotope. Note, however,  that the production rate of this isotope and therefore also the loss composition at the TCT2 depends crucially on the impact parameter at the TCP, which is 2\,$\mu$m in this simulation. The presented quantitative comparison is therefore only valid for this impact parameter.

\newpage

A better understanding of the situation can be obtained from the horizontal trajectory of the secondary $^{207}$Pb$^{82+}$ beam which is generated by EMD in the TCP. The simulated horizontal track of this isotope starting from the left TCP jaw is shown with the machine aperture and the collimators in \figref{pic:16020405}. The secondary beam of this isotope is not intercepted by the TCSG collimators in IR7 or the IR7 DS magnet aperture. It passes the edge of the right TCTH.4L8.B1 (TCT8) jaw at a small distance. Finally, it impacts the left jaw of the TCT2 with an impact parameter of several mm. STIER predicts the secondary  $^{207}$Pb$^{82+}$ beam starting from the other TCP jaw to be intercepted by the momentum collimators in IR3. 


\subsection{Mitigation Strategies and their Application in the LHC}

Based on the findings presented above, two different mitigation strategies have been worked out and tested with a \lead beam in the LHC during the 2015 heavy-ion run.

\subsubsection{Retraction of the left TCP jaw}


\begin{figure}[b]
  \centering
  \begin{tikzpicture}
    \footnotesize
    \node[anchor=south west,inner sep=0] (image) at (0,0) {\includegraphics[width=0.70\linewidth]{pictures/16090505.pdf}};
  % \node [draw,rotate=90,x={(image.south east)},y={(image.north west)}]                   at (0.50,0.50)    {text0};
  % \node [draw,rotate=0 ,x={(image.south east)},y={(image.north west)}]                   at (0.22,0.96)    {text1};
  \node [fill=white,x={(image.south east)},y={(image.north west)},anchor=west,align=center]       at (0.63,0.29)    {STIER \\ Prediction};
  \draw[->,color=black,thick,x={(image.south east)},y={(image.north west)}]             (0.76,0.20) -- (0.80,0.165);
  % \node [rotate=90,fill=white,x={(image.south east)},y={(image.north west)},anchor=west]       at (0.01,0.42)    {$\mathcal{B}_\text{TCT2}$ [$10^{-2}$]};
  % \footnotesize
  % \node [fill=white,x={(image.south east)},y={(image.north west)},anchor=west]       at (0.73,0.915)    {Measured $\mathcal{B}_\text{TCT2}$};
    \draw [white,fill=white,x={(image.south east)},y={(image.north west)}] (0.525,0.1) rectangle (0.575,1.01);
    \draw [white,fill=white,x={(image.south east)},y={(image.north west)}] (0.6,0.07) rectangle (1.0,0.12);

    \draw[color=black,x={(image.south east)},y={(image.north west)}]   (0.52,0.10) -- (0.53,0.15);
    \draw[color=black,x={(image.south east)},y={(image.north west)}]   (0.57,0.10) -- (0.58,0.15);

    \draw[color=black,x={(image.south east)},y={(image.north west)}]   (0.52,0.96) -- (0.53,1.01);
    \draw[color=black,x={(image.south east)},y={(image.north west)}]   (0.57,0.96) -- (0.58,1.01);
    \scriptsize
  \node [fill=white,x={(image.south east)},y={(image.north west)},anchor=west,align=center]       at (0.72,0.087)    {Full Retraction};

 % \draw[help lines,step=.05,x={(image.south east)},y={(image.north west)}] (0,0) grid (1,1);
 % \draw[help lines,line width=.6pt,step=0.1,x={(image.south east)},y={(image.north west)}] (0,0) grid (1,1);
 % \foreach \x in {0,0.1,0.2,0.3,0.4,0.5,0.6,0.7,0.8,0.9,1.0}
 %      \node[anchor=north,x={(image.south east)},y={(image.north west)}] at (\x,-0.01) {\x};
 % \foreach \y in {0.0,0.1,0.2,0.3,0.4,0.5,0.6,0.7,0.8,0.9,1.0}
 %     \node[anchor=east,x={(image.south east)},y={(image.north west)}] at (-0.01,\y) {\y};
  \end{tikzpicture}
  \caption{Measured BLM Signal (integration time 1.3\,s) at the TCT2 for different settings of the left TCP jaw, compared to the prediction by the STIER simulation.}  
  \label{pic:16020307}
  %/afs/cern.ch/work/p/phermes/private/151124_ion_lossmaps_runII/raw/tct2_mitigation_tcp.pdf
  \end{figure}


The asymmetry in the origin of the TCT2 losses can be exploited if asymmetric collimator settings are applied for the TCP. Given that the majority of losses is caused by particles starting from the left TCP jaw, a significant reduction of these losses can be expected if the latter is retracted. The primary losses in this case are shifted to the right TCP jaw. According to STIER, the TCT2 losses should be reduced by approximately 94.7\% at full retraction of the left TCP jaw. This corresponds to a reduction of the measured BLM signal from \mbox{$\mathcal{B}_{\text{TCT2}} = (6.6\pm 0.5)\times 10^{-2}$} to \mbox{$\mathcal{B}_{\text{TCT2}} = (3.5 \pm 0.3) \times 10^{-3}$}. Note that in this study it is only possible to make quantitative predictions because the initial BLM signal is known and scaled with the reduction factor calculated with STIER. Also, the predicted reduction can only be accurate if the impact parameter in the real machine is similar to that used for the STIER simulation.


% \begin{figure}[htbp]  
%     \centering
%     \includegraphics[width=0.7\textwidth]{pictures/16080602.pdf}
%     \caption{}  
%     \label{pic:16020307}
%     %/afs/cern.ch/work/p/phermes/private/151124_ion_lossmaps_runII/raw/tct2_mitigation_tcp.pdf
% \end{figure}











The experimental validation was conducted the 07.12.2015 with low intensity beams circulating in the LHC at 6.37\,$Z$\,TeV with squeezed optics. To probe the hypothesized loss reduction, the left TCP jaw was retracted in steps and the ADT was used to induce the primary losses at the TCP at every step. The individual loss maps measured at every step are shown in \figref{fig:retractionLM}. The measured signal at the TCT2 is shown as a function of the left TCP jaw position in \figref{pic:16020307}.

\newpage

During the experiment, the loss signal indeed decreased with increasing retraction of the left TCP jaw and yielded $\mathcal{B}_\text{TCT2}=(6 \pm  5 )\times 10^{-3}$ at the most extreme scenario of a full retraction. This result is in excellent agreement with the prediction made by STIER. As a consistency check, the same experiment was repeated with with the left TCP jaw in place and the right TCP jaw retracted. As expected, no loss reduction was achieved in this configuration.

\subsubsection{Tightening the TCT in IR8}

\begin{figure}[b]
  \centering
  \begin{tikzpicture}
    \node[anchor=south west,inner sep=0] (image) at (0,0) {\includegraphics[width=0.7\linewidth]{pictures/16090506.pdf}};
    %\node [draw,rotate=90,x={(image.south east)},y={(image.north west)}]                   at (0.50,0.50)    {text0};
    %\node [draw,rotate=0 ,x={(image.south east)},y={(image.north west)}]                   at (0.22,0.96)    {text1};
    %\node [draw,rotate=0 ,x={(image.south east)},y={(image.north west)},anchor=west]       at (0.22,0.80)    {text2};
    %\draw[->,color=black,thick,x={(image.south east)},y={(image.north west)}]             (0.42,0.22) -- (0.37,0.23);
  \end{tikzpicture}
  \caption{Normalized BLM signal (RS09) at the TCT2 as a function of the TCT8 half gap. The blue line shows the predicted upper boundary of the cleaning inefficiency simulated with STIER. This is achieved by taking the BLM signal at the nominal TCT8 setting as a reference and re-scaling it with the relative loss reduction predicted by STIER.}  
  \label{pic:16062101}
  %/afs/cern.ch/work/p/phermes/private/151124_ion_lossmaps_runII/raw/tct2_mitigation_tct8.pdf
  \end{figure}

The horizontal track shows that the secondary ion beam passes the horizontal TCT8 at a very small distance in $x$. In a second experiment, this prediction was studied by closing the TCT8 in steps. The measured loss maps for the individual steps are shown in \figref{pic:16060908} in the Appendix. 

STIER can be used to derive the expected loss signal at the TCT2 as a function of the TCT8 half gap. This requires information about the horizontal distribution of the $^{207}$Pb$^{82+}$ particles at the TCT8, to determine which fraction is intercepted with different TCT8 settings. 

The horizontal betatron function of the secondary $^{207}$Pb$^{82+}$ beam is calculated taking into account the offset in rigidity with respect to the main beam\footnote{The betatron function is rigidity-dependent.}. Assuming that the emittance is not significantly different from that of the main beam, the RMS size of the secondary beam is derived. The particle distribution is then modeled as a Gaussian, horizontally centered at the simulated $^{207}$Pb$^{82+}$ particle track predicted by STIER. 

With this model, the fraction of the secondary $^{207}$Pb$^{82+}$ beam that is intercepted by the TCT8 is determined and converted into the expected loss reduction at the TCT2. The studied isotope is not the only one lost at the TCT2, hence the predicted reduction of the secondary $^{207}$Pb$^{82+}$ beam intensity must be scaled with its contribution to the TCT2 losses (which is 92.5\%). Furthermore, the remaining isotopes may also be intercepted by the TCT8 which would also reduce the TCT2 loss signal. This is not taken into account in the presented approach. The result therefore indicates only an upper boundary of the expected TCT2 signal. 


In \figref{pic:16062101}, the measured TCT2 signals are shown as a function of the TCT8 half gap, together with the upper boundary predicted by STIER. The measured and predicted loss reduction is again in an excellent agreement. This result demonstrates that STIER can indeed be used to accurately predict particle tracks of heavy ions in the LHC. 



\subsection{Conclusions}

Both mitigation strategies derived from the STIER simulation data have been proven to be effective and the predicted loss levels are quantitatively supported by the measured BLM signals. This result underlines the importance of heavy-ion collimation simulation tools, which are essential to understand loss mechanisms and measured loss patterns. They provide indispensable input to optimize the collimation system if the stored beam energy shall be further increased. 

\vspace{0.2cm}

After the successful test, the application of asymmetric TCP settings in regular heavy-ion operation was discussed, to reduce the background at the ALICE experiment. The decision was taken to accept the TCT2 losses and maintain the nominal collimator configuration because of the time which would have been required for a re-commissioning of the new configuration. Also, the beam lifetime improved later in the run, so that the amount of losses at the TCT2 was reduced. However, the strategy of asymmetric TCP settings could be used in future heavy-ion runs to mitigate losses in similar situations.



\newpage

\section{High Luminosity LHC} \label{chap:hllhc}
%

With the High Luminosity LHC (HL-LHC) upgrade, a significant increase of the stored heavy-ion beam energy is envisaged, as explained in \chapref{chap:introhl}. Already in the 2015 heavy-ion run, the target bunch intensity for LIU has been exceeded. Upcoming upgrades of the LHC injectors will possibly lead to a further increase.

An inevitable consequence of the higher stored beam energy is the larger amount of collimation debris which is lost in the DS magnets.  Also for proton beams, the collimation losses in the DS become a serious issue with the envisaged intensities. Therefore, the installation of additional collimators (TCLD) in the IR7 DS is foreseen for HL-LHC. Simulations for proton beams indicate a significant improvement of the cleaning inefficiency with this upgrade~\cite{Bruce2014}. 

This chapter is dedicated to simulations of the cleaning inefficiency of heavy-ion beams in the HL-LHC configuration with and without TCLD collimators. The outcome of the presented studies is going to be used to define the requirements on the collimation system upgrades for future heavy-ion operation.

\subsection{Collimation System Upgrades for HL-LHC} \label{hl:coll}


The safe and uninterrupted operation with the proton intensities foreseen for HL-LHC requires several upgrades of the LHC collimation system. The upgrade schedule includes the installation of new collimators and the replacement of the active material of some existing collimators. Here, only the most essential modifications are summarized. A detailed overview is given in \cite{IPAC16:WEPMW030}. 


The increased luminosities in IR1 and IR5 are going to increase the amount of collision debris. To protect the surrounding magnets from this debris, these IRs are going to be equipped with additional TCLX collimators.  Although Q4 and Q5 magnets in IR1/IR5 are envisaged to be replaced by quadrupoles with larger geometric aperture, they might still intercept high beam losses~\cite{IPAC15:TUPTY066}. To protect their aperture, another set of two tertiary collimators per beam is envisaged to be installed in cell 5, in addition to those currently installed in cell 4. This could also reduce the background to the experiment. It is also under study to change the TCT material from IT180 to a composite material, copper-diamond~\cite{IPAC16:WEPMW031}, for better robustness. 

In IR7, the changes include new active materials for the TCSGs with lower impedance but same robustness as the CFC presently used. To mitigate the potentially critical losses in the DS region of IR7 (for both protons and heavy ions), the installation of additional local collimators is planned. The DS1 and DS2 loss clusters are located in the cells 9 and 11 downstream of IR7. Therefore, the proposal foresees two new horizontal collimators to be installed  upstream of them in the cells 8 and 10~\cite{hb08:wgd08,Bruce2014}. 

The required space would be made available by replacing a nominal main dipole by two stronger and shorter dipoles than can provide a maximum magnetic field strength of 11~T. 

The new collimator, called TCLD\footnote{Target Collimator Long Dispersion Suppressor.}, could be installed in between the new dipoles. This is illustrated in \figref{pic:16042602}. To provide a large stopping power, the present layout foresees tungsten as the active material of the TCLD collimators. SixTrack simulations predict a significant improvement of the proton cleaning inefficiency when the TCLD collimators are installed~\cite{Bruce2014}.


\begin{figure}[t]  
    \centering
%    \includegraphics[width=1.0\textwidth]{pictures/16042602.pdf}
    \includegraphics[width=1.0\textwidth]{pictures/16092103.pdf}
    \caption{LHC beam line sequence in IR7 for B1. Top panel: Nominal sequence. Middle panel: sequence with the main dipoles in cell 8 and cell 10 replaced by two 11~T dipoles and the TCLD collimators in between. Bottom panel: local hor. dispersion function from the TCP.}  
    \label{pic:16042602}
    %/home/phermes/Dropbox/codes/madx/160426_blg_hllhc/tcld.pdf
\end{figure}

Especially the construction of the stronger dipole magnets requires significant financial resources. Present cost estimates assume approximately 10 million CHF for the installation of one TCLD~\cite{Redaelli:private}. It is also not yet clear how many of them can be manufactured by the time their installation is foreseen~\cite{Redaelli:private}. For these reasons, a solution is preferred in which only one TCLD per beam is required (installed in cell 8). 


The heavy-ion cleaning inefficiency in the HL-LHC configuration with and without TCLD collimators is studied by means of the hiSixTrack-FLUKA coupling. The presented simulations are therefore carried out for three cases to compare the limiting cleaning inefficiency and evaluate the risk of quenches for the three following scenarios:
%
\begin{itemize}
  \item 0 TCLDs: reference simulation for the present layout without TCLD collimators,
  \item 1 TCLD:  layout with an additional TCLD collimator in cell 8 (TCLD8), 
  \item 2 TCLDs: TCLD collimators are installed in cell 8 and cell 10 (TCLD10). 
\end{itemize}
%

It is also envisaged to install TCLD collimators in the DS of IR2, to intercept the secondary beams generated from BFPP~\cite{IPAC16:WEPMW030}. In this IR, they can be installed in the empty connection cryostat in cell 10, which makes the installation of 11~T dipoles unnecessary. In the following simulations the IR2 TCLDs are not included, because their half gap is going to be in the order of $30\,\sigma$. The influence on collimation cleaning simulations should therefore be small~\cite{mertens:private}.


\subsection{Cleaning Simulations without TCLD Collimators in IR7}

\begin{table}[htbp]
\centering
\caption{Collimator half gaps in $\sigma$ envisaged for HL-LHC.}
\label{tab:sets_hl}
\begin{tabular}{cccccc} 
\toprule
 \multicolumn{2}{c}{IR7}           &   \multicolumn{2}{c}{IR3}             &    \multicolumn{2}{c}{IR1/2/5/8/6} \\ \midrule
TCP    &  5.7  &   TCP   &  15.0    &    TCT  &   10.9 \\ 
TCSG   &  7.7  &   TCSG  &  18.0    &    TCL  &   12.0 \\
TCLA   & 10.0  &   TCLA  &  20.0    &   TCDQ  &   9.0  \\ 
       &       &         &          &   TCSP  &   8.5  \\ \bottomrule
\end{tabular}
\end{table}

The cleaning simulations with the hiSixTrack-FLUKA coupling are carried out for the different planes (B1H, B2H, B1V, B2V) with different impact parameters to find the scenario with the largest cleaning inefficiency. For this scenario the cleaning inefficiency is then studied for different possible half gaps of the TCLD collimators. 

All simulations are carried out with the HL-LHC heavy-ion optics V.1.2 with $\beta^*=0.48\,$m in IP1/IP5, $\beta^*=0.5\,$m in IP2 and $\beta^*=3.0\,$m in IP8. The collimator settings are summarized in \tabref{tab:sets_hl}. The upgrades of the collimator material are not incorporated in the simulation. The materials used in the presented studies are hence the standard materials stored in the FEDB. However, studies with protons show that the influence of the new materials on the cleaning inefficiency is small~\cite{IPAC16:WEPMW031}. Future simulations are envisaged to also study the effect of the changed materials on the heavy-ion cleaning inefficiency.


\subsubsection{Quench Risk and Cleaning Inefficiency}

The main motivation for the validation studies for the 2015 heavy-ion operation was to reduce the amount of losses in the DS clusters to reduce the risk of protection dumps by the BLM system. The cleaning performance of the LHC collimation system was quantified mostly by means of the peak cleaning inefficiency in the superconducting LHC magnets and by the integrated cleaning inefficiency in the DS loss clusters. For HL-LHC, the stored beam energies are significantly larger and the heavy-ion program is seriously endangered by the risk of beam-induced quenches. 

To quantify the risk of quench, the maximum average cleaning inefficiency is considered. It takes into account, that in reality also showers from particles lost upstream contribute to the energy deposition in the magnet coils. The average cleaning inefficiency $\eta^{avg}$ is defined as the average $\eta(s)$ in a given interval $[s_0,s_1]$. For the presented analyses, the average cleaning inefficiency is calculated in the DS loss clusters in intervals of 7.2$\,$m length, roughly half the length of a LHC main dipole. Since the loss distribution in the DS clusters is inhomogeneous and  $\eta^{avg}$ depends on the starting point $s_0$, all possible intervals are scanned in steps of 0.1\,m and the respective maximum average cleaning inefficiency in the two DS clusters is defined as $\eta^{avg}_{DS1}$ and $\eta^{avg}_{DS2}$. In the following, the global maximum average cleaning inefficiency in the superconducting LHC regions $\eta^{avg}_{max}$ is considered to quantify the risk of quench. 

Evidently, this is an approximation and a better modeling of the quench risk could be obtained if the energy deposition in the superconducting coils would be simulated. This requires additional studies with FLUKA, and is planned for the future.




\subsubsection{Selection of Beam and Plane}

In a first set of simulations, the most critical loss plane is identified. For this purpose simulations are carried out with identical settings without TCLDs for both beams of both planes. All simulations are carried out with 6$\times 10^6$ particles of \lead. The resulting loss maps are shown for IR7 in \figref{pic:16081101} in the Appendix. The most important key parameters $\eta_{cold}^{max}$ and $\eta_{max}^{avg}$ deduced from the simulations of the cleaning inefficiency are summarized in \figref{pic:16080901}. The analysis shows that the largest cleaning inefficiency is simulated in B1H in terms of both key parameters. For the following studies the simulations are therefore carried out for B1H. 

\begin{figure}[htbp]
  \centering
  \begin{tikzpicture}
    \node[anchor=south west,inner sep=0] (image) at (0,0) {\includegraphics[width=0.6\linewidth]{pictures/16090903.pdf}};
    %\node [draw,rotate=90,x={(image.south east)},y={(image.north west)}]                   at (0.50,0.50)    {text0};
    %\node [draw,rotate=0 ,x={(image.south east)},y={(image.north west)}]                   at (0.22,0.96)    {text1};
    %\node [draw,rotate=0 ,x={(image.south east)},y={(image.north west)},anchor=west]       at (0.22,0.80)    {text2};
    %\draw[->,color=black,thick,x={(image.south east)},y={(image.north west)}]             (0.42,0.22) -- (0.37,0.23);
  \end{tikzpicture}
  \caption{Maximum peak cleaning inefficiency $\eta_{cold}^{max}$, and maximum average cleaning inefficiency $\eta^{avg}_{max}$ in the different planes simulated with the hiSixTrack-FLUKA coupling in the HL-LHC configuration at $7\,Z\,$TeV for \lead beams. }  
  \label{pic:16080901}
  %/media/phermes/local/hisix_results/HLLHC/B1H/analysis/postprocessing/plane_comparison.pdf
  \end{figure}




\subsubsection{Impact Parameters}

In another set of cleaning simulations, the impact parameter associated to the largest cleaning inefficiency for B1H shall be identified. Seven different simulations are carried out using 6$\times 10^6$ initial \lead ions with impact parameters of $0.1\,\mu$m, $0.33\,\mu$m, $1.0\,\mu$m, $3.3\,\mu$m, $10.0\,\mu$m, $33.3\,\mu$m and $100\,\mu$m. The key parameters derived from these simulations are summarized in \figref{pic:16062203}. The figure also shows the distance $d$, which the particles traverse in the collimator material as a function of the impact parameter, assuming an impact angle of $2.1\times 10^{-2}\,$mrad. 


% \begin{figure}[t]
%   \centering
%   \begin{tikzpicture}
%     \node[anchor=south west,inner sep=0] (image) at (0,0) {\includegraphics[width=1.0\linewidth]{pictures/16072501.pdf}};
%     %\node [draw,rotate=90,x={(image.south east)},y={(image.north west)}]                   at (0.50,0.50)    {text0};
%     %\node [draw,rotate=0 ,x={(image.south east)},y={(image.north west)}]                   at (0.22,0.96)    {text1};
%     %\node [draw,rotate=0 ,x={(image.south east)},y={(image.north west)},anchor=west]       at (0.22,0.80)    {text2};
%     %\draw[->,color=black,thick,x={(image.south east)},y={(image.north west)}]             (0.42,0.22) -- (0.37,0.23);
%   \end{tikzpicture}
%   \caption{Comparison of the cleaning inefficiency simulated with the hiSixTrack-FLUKA coupling for four different impact parameters \mbox{$b=0.1\mu$m, $1.0\,\mu$m, 10.0$\,\mu$m, 100.0$\,\mu$m}. }  
%   \label{pic:16062201}
%   %/media/phermes/local/hisix_results/HLLHC/B1H/analysis/postprocessing/output/comparison_impact_IR7.pdf
%   \end{figure}






\begin{figure}[b]
  \centering
  \begin{tikzpicture}
    \footnotesize
    \node[anchor=south west,inner sep=0] (image) at (0,0) {\includegraphics[width=0.6\linewidth]{pictures/16090906.pdf}};
    % \node [fill=white,x={(image.south east)},y={(image.north west)}]                   at (0.77,0.82)    {$d$};
    % \node [rotate=90,x={(image.south east)},y={(image.north west)}]                   at (1.03,0.55)    {Traversed distance $d$ [m]};
    %\node [draw,rotate=0 ,x={(image.south east)},y={(image.north west)}]                   at (0.22,0.96)    {text1};
    %\node [draw,rotate=0 ,x={(image.south east)},y={(image.north west)},anchor=west]       at (0.22,0.80)    {text2};
    %\draw[->,color=black,thick,x={(image.south east)},y={(image.north west)}]             (0.42,0.22) -- (0.37,0.23);
  \end{tikzpicture}
  \caption{Average cleaning inefficiency in the DS1 and DS2 cluster for B1H with different impact parameters, derived from cleaning simulations with the hiSixTrack-FLUKA coupling in the HL-LHC configuration with \lead ions at $7\,Z\,$TeV.}  
  \label{pic:16062203}
  %/media/phermes/local/hisix_results/HLLHC/B1H/analysis/postprocessing/output/eta_vs_b.pdf
  \end{figure}



The comparison shows that the average cleaning inefficiency in the two IR7 DS clusters increases from $b=0.1$\mum\, to 3.3\mum\, and then decreases by two orders of magnitude for when the impact parameter is increased to 10.0\mum. From 10\mum\, to 33\mum, the cleaning inefficiency decreases slightly and remains constant when increasing the impact parameter to 100\mum.

\newpage
As shown in \figref{pic:16062203}, the particles traverse the full length of the collimator jaw if the impact parameter is larger than 12\mum. Therefore, the fragmentation spectrum and hence the cleaning inefficiency does not change significantly from 33.3\mum\, to 100.0\mum. 

The average cleaning inefficiency simulated for the DS1 is similar to that of the DS2 for impact parameters $b\leq 1$\mum. For larger impact parameters, the losses in the DS1 become significantly larger than in the DS2. As discussed in \chapref{chap:STIERfrag}, heavy nuclei with mass and charge close to the main beam that could be lost in the DS2, are mostly fragmented into other isotopes when larger impact parameters are applied. Hence, with larger impact parameters the rigidity offset of the fragments increases. In consequence the dispersion in the DS1 is sufficient to guide them into the magnet aperture and remove them from the beam before they reach the DS2.  



% A further observation is the diminished density in the $\chi$ histogram with increasing impact parameter. The losses for the larger impact parameters are caused by less different isotopes which is also visible in the distribution of nuclear mass numbers.  For the largest impact parameter, the largest fradction of the losses arises from tritium ions with the spectrum of isotopes lost in the DS ending at $A=85$. Along with this observation comes the larger amount of energy lost in the TCP, leading to a reduced cleaning inefficiency in the DS clusters. The cleaning inefficiency does not change significantly from $b=10$\mum to 100\mum, because at 10\mum the primary heavy-ions traverse almost the full length of the collimator (for the presented setup they do so at $b=13$\mum). 

% \begin{figure}[htbp]
%   \centering
%   \begin{tikzpicture}
%     \footnotesize
%     \node[anchor=south west,inner sep=0] (image) at (0,0) {\includegraphics[width=1.0\linewidth]{pictures/16062701.pdf}};
%     \node [x={(image.south east)},y={(image.north west)},anchor=west]       at (0.25,0.92)    {$^{3}$H$^{1+}$};
%     \draw[->,color=black,thick,x={(image.south east)},y={(image.north west)}]             (0.28,0.91) -- (0.315,0.90);
%     \node [fill=white,x={(image.south east)},y={(image.north west)},anchor=west]       at (0.37,0.93)    {$b=0.1\,\mu$m};
%     \node [fill=white,x={(image.south east)},y={(image.north west)},anchor=west]       at (0.37,0.71)    {$b=1.0\,\mu$m};
%     \node [fill=white,x={(image.south east)},y={(image.north west)},anchor=west]       at (0.37,0.47)    {$b=10\,\mu$m};
%     \node [fill=white,x={(image.south east)},y={(image.north west)},anchor=west]       at (0.37,0.25)    {$b=100\,\mu$m};
%     %\draw[->,color=black,thick,x={(image.south east)},y={(image.north west)}]             (0.42,0.22) -- (0.37,0.23);
%   \end{tikzpicture}
%   \caption{Distribution of $\chi$ and $A$ in the DS1 (left) and DS2 (right) clusters with different impact parameters from $b=0.1\,\mu$m to $100\,\mu$m.}  
%   \label{pic:16062204}
%   %/media/phermes/local/hisix_results/HLLHC/B1H/analysis/postprocessing/chi_A_DS1_DS2.pdf
%   \end{figure}

To understand why the cleaning performance for an impact parameter of 0.1\mum\, is better than for 1.0\mum, a dedicated simulation was conducted. Possible effects of the phase space space shift with the associated increase of the impact parameter (this was discussed in \chapref{chap:pha_shift}) shall be ruled out by switching off the sextupole magnets. The simulation unveils that the cleaning inefficiency for $b=1.0$\mum\, is still worse than for 0.1\mum, also without sextupoles. 


\begin{figure}[b]
  \centering
  \begin{tikzpicture}
    \node[anchor=south west,inner sep=0] (image) at (0,0) {\includegraphics[width=0.6\linewidth]{pictures/16081005.pdf}};
    %\node [draw,rotate=90,x={(image.south east)},y={(image.north west)}]                   at (0.50,0.50)    {text0};
    %\node [draw,rotate=0 ,x={(image.south east)},y={(image.north west)}]                   at (0.22,0.96)    {text1};
    %\node [draw,rotate=0 ,x={(image.south east)},y={(image.north west)},anchor=west]       at (0.22,0.80)    {text2};
    %\draw[->,color=black,thick,x={(image.south east)},y={(image.north west)}]             (0.42,0.22) -- (0.37,0.23);
  \end{tikzpicture}
  \caption{Impact parameters at the TCP at turns subsequent to the first interaction with the TCP for $b=0.1$\mum\, and 1\mum. All particles arriving at the TCP in a subsequent turn are \lead ions. Both distributions are weighted with the particle momenta and normalized with respect to the total momentum of all particles.}  
  \label{pic:16081005}
  %/media/phermes/local/hisix_results/HLLHC/B1H/analysis/postprocessing/secondary_impact.pdf
  \end{figure}



The distribution of impact parameters at the TCP for impacts at turns subsequent to the first interaction with the TCP is shown in \figref{pic:16081005}. This means that only ions are studied which have already passed through the TCP material at least once, but have been scattered out without fragmenting. In the following description, the impact parameter at the first TCP interaction is referred to as primary impact parameter. For impacts at subsequent turns, the denotation secondary impact parameter is used. 


For secondary impact parameters smaller than 120$\,\mu$m, higher abundances are visible for the simulation with a primary impact parameter of 0.1$\,\mu$m. For those larger than 120$\,\mu$m, the opposite applies. This might come from the lower momentum offset particles receive from ionization losses during the first passage with a primary impact parameter of $0.1\,\mu$m, because the distance traversed in the collimator is smaller. These particles arrive at a subsequent turn with a smaller dispersive offset, than those with a primary impact parameter of 1.0$\,\mu$m. However, the integral is quantitatively similar for both study cases. The worse cleaning inefficiency for 1\mum\, must therefore be related to the interaction with the TCP at the primary passage. 

\begin{figure}[t]
  \centering
  \begin{tikzpicture}
    \footnotesize
    \node[anchor=south west,inner sep=0] (image) at (0,0) {\includegraphics[width=0.6\linewidth]{pictures/16081004.pdf}};
    \draw [white,fill=white,x={(image.south east)},y={(image.north west)}] (0.0,0.4) rectangle (0.07,0.7);

    %\node [draw,rotate=90,x={(image.south east)},y={(image.north west)}]                   at (0.50,0.50)    {text0};
    %\node [draw,rotate=0 ,x={(image.south east)},y={(image.north west)}]                   at (0.22,0.96)    {text1};
    \node [rotate=90 ,x={(image.south east)},y={(image.north west)},anchor=west]     at (0.05,0.4)    {Abundance};
    %\draw[->,color=black,thick,x={(image.south east)},y={(image.north west)}]             (0.42,0.22) -- (0.37,0.23);

 % \draw[help lines,step=.05,x={(image.south east)},y={(image.north west)}] (0,0) grid (1,1);
 % \draw[help lines,line width=.6pt,step=0.1,x={(image.south east)},y={(image.north west)}] (0,0) grid (1,1);
 % \foreach \x in {0,0.1,0.2,0.3,0.4,0.5,0.6,0.7,0.8,0.9,1.0}
 %      \node[anchor=north,x={(image.south east)},y={(image.north west)}] at (\x,-0.01) {\x};
 % \foreach \y in {0.0,0.1,0.2,0.3,0.4,0.5,0.6,0.7,0.8,0.9,1.0}
 %     \node[anchor=east,x={(image.south east)},y={(image.north west)}] at (-0.01,\y) {\y};
  \end{tikzpicture}
  \caption{Abundance of the nuclear mass number for fragments created at the first passage of one million \lead ions through the TCP with $b=1\,\mu$m and $b=0.1\,\mu$m. Both distributions are normalized by the total number of particles.}  
  \label{pic:16081003}
  %/media/phermes/local/hisix_results/HLLHC/B1H/analysis/postprocessing/A_comparison_b.pdf
  \end{figure}


The distribution of nuclear mass numbers for the heavy-ion fragments generated at the first TCP passage is shown in \figref{pic:16081003}. The comparison shows that the number of fragments with intermediate mass numbers is significantly higher for 1\mum\, than for 0.1\mum. The origin of this difference is the distance the particles traverse in the TCP material, which is ten times larger for an impact parameter of 1\mum\, than for 0.1\mum. In total, the production rate for particles with rigidity offsets beyond 5\% is higher by a factor of approximately two for the simulation with $b=$1\mum. This corresponds to the difference of the average cleaning inefficiency in the DS which is observed between the simulations with 0.1\mum\, and 1\mum. It is hence the difference in fragmentation at the first TCP passage which leads to the larger cleaning inefficiency at 1$\,\mu$m.






% \subsubsection{Comparison to previous Heavy-Ion Operation}

% The simulated global loss map for HL-LHC with \lead ions at 7$\,Z\,$TeV is compared to the simulated loss maps for the 2015 and 2011 heavy-ion runs in \figref{pic:16082306} (full LHC ring) and \figref{pic:16082305} (zoomed to IR2). The applied impact parameter is 1\mum for all simulation cases. The comparison shows that the simulated loss patterns for the 2011 and the 2015 heavy-ion run is qualitatively very similar, e.g. most of the loss peaks simulated for the two configurations appear at the same locations. For the HL-LHC simulation, additional loss peaks appear in regions in which small or no losses appear in the other simulations. This applies in particular for the TCT in IR1 in which the simulated cleaning inefficiency reaches almost $10^{-1}\,$m$^{-1}$, compared to $5 \times 10^{-6}$ simulated in the 2015 configuration. 

% The opposite applies to the losses at the TCT2, which are simulated to be higher by almost one order of magnitude in the 2015 configuration compared to the 2011 configuration. In the HL-LHC configuration, no losses at the TCT2 are simulated, but a intense loss peak in the superconduting magnets upstream of the TCT2 is visible. Further investigation of the particles lost in this region shows that the consist of \iso{207}{Pb}{82+} particles starting from the left TCP jaw. The same origin was found for the particles lost at the TCT2 in 2015, as discussed in \chapref{chap:ir2loss} and for the simulated TCT2 losses in the 2011. The origin of the shift in loss position of these particles are the different particle energies, starting conditions in IR7 and the difference in dispersion induced by the different optics between IR7 and IR2. 




% \begin{figure}[htbp]
%   \centering
%   \begin{tikzpicture}
%     \footnotesize
%     \node[anchor=south west,inner sep=0] (image) at (0,0) {\includegraphics[width=1.0\linewidth]{pictures/16082306.pdf}};
%   \node [fill=white,x={(image.south east)},y={(image.north west)}]                   at (0.20,0.97)    {HL-LHC};
%   \node [fill=white,x={(image.south east)},y={(image.north west)}]                   at (0.20,0.66)    {2015 heavy-ion run};
%   \node [fill=white,x={(image.south east)},y={(image.north west)}]                   at (0.20,0.34)    {2011 heavy-ion run};
%   \end{tikzpicture}
%   \caption{B1H betatron loss maps simulated with the hiSixTrack-FLUKA coupling for three different configurations: HL-LHC (top panel), the 2015 heavy-ion run (middle panel) and the 2011 heavy-ion run (bottom panel).}  
%   \label{pic:16082306}
%   %/media/phermes/local/hisix_results/HLLHC/B1H/analysis/postprocessing/output/HL_2011_2015_LHC.pdf
%   \end{figure}


% \begin{figure}[htbp]
%   \centering
%   \begin{tikzpicture}
%     \node[anchor=south west,inner sep=0] (image) at (0,0) {\includegraphics[width=1.0\linewidth]{pictures/16082305.pdf}};
%   % \node [draw,rotate=90,x={(image.south east)},y={(image.north west)}]                   at (0.50,0.50)    {text0};
%   % \node [draw,rotate=0 ,x={(image.south east)},y={(image.north west)}]                   at (0.22,0.96)    {text1};
%   % \node [draw,rotate=0 ,x={(image.south east)},y={(image.north west)},anchor=west]       at (0.22,0.80)    {text2};
%   % \draw[->,color=black,thick,x={(image.south east)},y={(image.north west)}]             (0.42,0.22) -- (0.37,0.23);

%   \end{tikzpicture}
%   \caption{B1H betatron loss maps (zoomed to IR2) simulated with the hiSixTrack-FLUKA coupling for three different configurations: HL-LHC (top panel), the 2015 heavy-ion run (middle panel) and the 2011 heavy-ion run (bottom panel).}  
%   \label{pic:16082305}
%   %/media/phermes/local/hisix_results/HLLHC/B1H/analysis/postprocessing/output/HL_2011_2015_IR2.pdf
%   \end{figure}





\subsection{Cleaning Simulations with TCLD Collimators in IR7}

In this section, the improvement of the cleaning inefficiency from TCLD collimators in IR7 is studied. Particular emphasis is given to the question if a significant improvement can be obtained from the installation of only one TCLD per beam in IR7. If the TCLD was to be installed in cell 8 (TCLD8), the main loss reduction is expected in the DS1 loss cluster, where the dispersion is comparable to that at the TCLD8. The locally generated dispersion from the TCP is smaller at the DS1 than at the DS2 (see \figref{pic:16042602}). Particles lost in the DS1 are those with larger rigidity offsets compared to those lost in the DS2 (this was discussed in \chapref{fig:15032102}). Therefore, it is possible that the particles lost in the DS2 loss cluster are not intercepted by the TCLD8. The worst case scenario in terms of the maximum average cleaning inefficiency at the DS2 is the simulation with $b=1$\mum. This is hence the baseline for the following simulations.



\subsubsection{Cleaning Performance with one TCLD}


Present estimates assume TCLD half gaps between $10\,\sigma$ and $15\,\sigma$. The following studies are carried out with half gaps of 10$\,\sigma$, 12$\,\sigma$ and 15$\,\sigma$. These should be regarded as tentative. The applicability of the settings also depends on constraints imposed by the risk of asynchronous beam dump and the settings of the momentum collimators in IR3. An initial distribution of $6.0 \times 10^6$ ions of \lead is used in every simulation.  The loss maps simulated with the hiSixTrack-FLUKA coupling with the TCLD8 collimator set to these half gaps are shown in \figref{pic:16091420} and \figref{pic:16081104}. The key parameters quantifying the cleaning inefficiency in the DS1 and DS2 are compared in \figref{pic:16081102}, \figref{pic:16091422} and \tabref{tab:peakred}.


\begin{figure}[p]
  \centering
  \begin{tikzpicture}
    \footnotesize
    \node[anchor=south west,inner sep=0] (image) at (0,0) {\includegraphics[width=1.0\linewidth]{pictures/16091421.pdf}};
    \node [fill=white,x={(image.south east)},y={(image.north west)}]                   at (0.9,0.97)    {TCLD at $10\,\sigma$};
    \node [fill=white,x={(image.south east)},y={(image.north west)}]                   at (0.9,0.73)    {TCLD at $12\,\sigma$};
    \node [fill=white,x={(image.south east)},y={(image.north west)}]                   at (0.9,0.49)    {TCLD at $15\,\sigma$};
    \node [fill=white,x={(image.south east)},y={(image.north west)}]                   at (0.9,0.25)    {Without TCLD};

    \node [fill=white,x={(image.south east)},y={(image.north west)}]                   at (0.18,0.22)    {C2};

%  \draw[help lines,step=.05,x={(image.south east)},y={(image.north west)}] (0,0) grid (1,1);
%  \draw[help lines,line width=.6pt,step=0.1,x={(image.south east)},y={(image.north west)}] (0,0) grid (1,1);
%  \foreach \x in {0,0.1,0.2,0.3,0.4,0.5,0.6,0.7,0.8,0.9,1.0}
%       \node[anchor=north,x={(image.south east)},y={(image.north west)}] at (\x,-0.01) {\x};
%  \foreach \y in {0.0,0.1,0.2,0.3,0.4,0.5,0.6,0.7,0.8,0.9,1.0}
%      \node[anchor=east,x={(image.south east)},y={(image.north west)}] at (-0.01,\y) {\y};

  \end{tikzpicture}
  \caption{B1H loss maps simulated for the HL-LHC configuration with \lead ions at 7$\,Z\,$TeV and one TCLD collimator in cell 8 of IR7. The four panels show the loss maps simulated with three different settings of the TCLD collimators (three top panels) and without TCLD collimator (bottom panel).}  
  \label{pic:16091420}
  %/media/phermes/local/hisix_results/HLLHC/B1H/analysis/postprocessing/output/comparison_1TCLD_LHC.pdf
  \end{figure}


\begin{figure}[p]
  \centering
  \begin{tikzpicture}
%    \node[anchor=south west,inner sep=0] (image) at (0,0) {\includegraphics[width=1.0\linewidth]{pictures/16090301.pdf}};
    \node[anchor=south west,inner sep=0] (image) at (0,0) {\includegraphics[width=1.0\linewidth]{pictures/16090701.pdf}};

    \footnotesize
    \node [rotate=90,fill=white,x={(image.south east)},y={(image.north west)}]                   at (0.55,0.85)    {TCLD8};
    \node [fill=white,x={(image.south east)},y={(image.north west)}]                   at (0.605,0.90)    {DS1};
    \node [fill=white,x={(image.south east)},y={(image.north west)}]                   at (0.685,0.90)    {DS2};
    \node [fill=white,x={(image.south east)},y={(image.north west)}]                   at (0.9,0.92)    {TCLD at $10\,\sigma$};
    \node [fill=white,x={(image.south east)},y={(image.north west)}]                   at (0.9,0.70)    {TCLD at $12\,\sigma$};
    \node [fill=white,x={(image.south east)},y={(image.north west)}]                   at (0.9,0.48)    {TCLD at $15\,\sigma$};
    \node [fill=white,x={(image.south east)},y={(image.north west)}]                   at (0.9,0.25)    {Without TCLD};
  \end{tikzpicture}
  \caption{B1H loss maps zoomed to IR7, simulated for the HL-LHC configuration with \lead ions at 7$\,Z\,$TeV and one TCLD collimator in cell 8 of IR7. The four panels show the loss maps simulated with three different settings of the TCLD collimators (three top panels) and without TCLD collimator (bottom panel).}  
  \label{pic:16081104}
  %/media/phermes/local/hisix_results/HLLHC/B1H/analysis/postprocessing/output/comparison_1TCLD.pdf
  \end{figure}






\mbox{} \\ 
\textit{Global Loss Distribution}\\ 
The global loss distribution, shown in \figref{pic:16091420}, is very similar for all study cases. For all TCLD settings, the loss peaks are at the same location. Some loss spikes, which are visible in the simulation without TCLDs, are suppressed when the TCLD is set to 12$\,\sigma$ or 15$\,\sigma$. However, the global cleaning inefficiency is not significantly improved with one TCLD collimator. 

In neither of the study cases, particles are lost at the TCT2. Instead, a loss spike (C2) in the superconducting magnets upstream of IP2 is simulated. Further analysis shows that these losses are mainly caused by \iso{207}{Pb}{82} starting from the left TCP jaw. The different optics and collimator settings compared to the 2015 heavy-ion run have shifted the losses from the TCT2 into the arc region upstream. According to the results of the study presented in \chapref{chap:ir2loss}, this loss peak could be suppressed by applying asymmetric TCP settings. 

\mbox{} \\
\textit{Overview of IR7}\\
In regions upstream of the TCLD8, the loss maps with either TCLD setting are nearly identical. Downstream of the TCLD8, the loss maps show significant differences compared to the nominal simulation. The DS1 loss cluster is not completely removed, but secondary fragments generated in the TCLD8 are still lost in this region. This finding is new compared to  studies conducted with STIER without secondary fragmentation~\cite{phermes_ipac2015_1}. 

\mbox{} \\
\textit{Losses in the DS1}\\ 
With the most relaxed setting of 15$\,\sigma$, the peak cleaning inefficiency $\eta_{DS1}^{max}$ in the DS1 is reduced by approximately 73\% compared to the simulation without TCLD. For the tighter settings, $\eta_{DS1}^{max}$ is even reduced by more than two orders of magnitude. The integrated losses $\eta_{DS1}^{int}$ in the DS1 are reduced by approximately 95\% with the most relaxed TCLD setting and by more than two orders of magnitude for a half gap of $10\,\sigma$.
%
%
\begin{figure}[t]
  \centering
  \begin{tikzpicture}
    \node[anchor=south west,inner sep=0] (image) at (0,0) {\includegraphics[width=0.6\linewidth]{pictures/16091424.pdf}};
 
    \draw [white,fill=white,x={(image.south east)},y={(image.north west)}] (0.71,0.1) rectangle (0.75,1.01);
    \draw[color=black,x={(image.south east)},y={(image.north west)}]   (0.705,0.095) -- (0.715,0.145);
    \draw[color=black,x={(image.south east)},y={(image.north west)}]   (0.745,0.095) -- (0.755,0.145);

    \draw[color=black,x={(image.south east)},y={(image.north west)}]   (0.705,0.95) -- (0.715,1.0);
    \draw[color=black,x={(image.south east)},y={(image.north west)}]   (0.745,0.95) -- (0.755,1.0);
  \end{tikzpicture}
  \caption{Peak and integrated cleaning inefficiency in the DS1 and DS2 with one TCLD in cell 8 for \lead ions at 7$\,Z\,$TeV in the HL-LHC configuration (B1H). Parking corresponds to a fully retracted TCLD.}  
  \label{pic:16081102}
  %/media/phermes/local/hisix_results/HLLHC/B1H/analysis/postprocessing/output/comparison_1TCLD_eta.pdf
  \end{figure}
%
%
%
% \begin{figure}[htbp]
%   \centering
%   \begin{tikzpicture}
%     \node[anchor=south west,inner sep=0] (image) at (0,0) {\includegraphics[width=0.6\linewidth]{pictures/16081205.pdf}};
%     %\node [draw,rotate=90,x={(image.south east)},y={(image.north west)}]                   at (0.50,0.50)    {text0};
%     %\node [draw,rotate=0 ,x={(image.south east)},y={(image.north west)}]                   at (0.22,0.96)    {text1};
%     %\node [draw,rotate=0 ,x={(image.south east)},y={(image.north west)},anchor=west]       at (0.22,0.80)    {text2};
%     %\draw[->,color=black,thick,x={(image.south east)},y={(image.north west)}]             (0.42,0.22) -- (0.37,0.23);
%   \end{tikzpicture}
%   \caption{Maximum cleaning inefficiency $\eta_{cold}^{max}$ normalized by the maximum cleaning inefficiency without TCLD $\eta_\text{cold}^{max,nominal}$ and the resulting maximum stored beam energy for differen TCLD8 settings as simulated with the hiSixTrack-FLUKA coupling. Simulations carried out with TCLD in cell 8 only. }  
%   \label{pic:16081203}
%   %/media/phermes/local/hisix_results/HLLHC/B1H/analysis/postprocessing/output/maxEnergy.pdf
%   \end{figure}
%

\mbox{} \\ 
\textit{Losses in the DS2}\\ 
Shape and amplitude of the DS2 loss cluster are not significantly changed by the TCLD8 collimator, regardless of the half gap applied. Compared to the simulation without TCLD, the peak loss $\eta_{DS2}^{max}$ in the DS2 is only reduced by approximately 12\% for the most relaxed TCLD8 setting. With the tightest half gaps applied, the maximum cleaning inefficiency in the DS2 can be reduced by approximately 77\%. The integrated losses in the DS2 are reduced by approximately 35\% with the most relaxed setting and by approximately 63\% with the tightest setting. 

 \mbox{} \\ 
\textit{Maximum average cleaning inefficiency}\\ 
In the simulation of the nominal HL-LHC without TCLD, the maximum average cleaning inefficiency $\eta^{avg}_{max}$ is found in the DS1. With the TCLD8 installed, the losses in the DS1 are reduced and the highest average cleaning inefficiency is simulated to be in the DS2 cluster. The possible improvement in terms of quench risk is the ratio of the maximum average cleaning inefficiency $\eta^{avg}_{max}$ with and $\eta^{avg}_{max,nominal}$ in the nominal HL-LHC configuration without TCLD collimator. As shown in \figref{pic:16091422}, the reduction achieved yields approximately 11\% for the most relaxed setting and approximately 45\% for the tightest setting. 






\subsubsection{Cleaning Performance with two TCLDs}



In this section, the previous study is repeated with a second TCLD installed in cell 10 (TCLD10). Also here, the cleaning inefficiency is studied for the three different TCLD settings. 

The evolution of the peak, integrated and maximum average cleaning inefficiency as a function of the TCLD setting is shown in \figref{pic:16081603}. Furthermore, the numeric values are given in \tabref{tab:peakred}. The simulated loss maps are shown in \figref{pic:16081604} and \figref{pic:16081602}. 



\begin{figure}[b]
  \centering
  \begin{tikzpicture}
    \node[anchor=south west,inner sep=0] (image) at (0,0) {\includegraphics[width=0.60\linewidth]{pictures/16091425.pdf}};

    % \draw[color=black,x={(image.south east)},y={(image.north west)}]             (0.755,0.12) -- (0.765,0.17);
    % \draw[color=black,x={(image.south east)},y={(image.north west)}]             (0.795,0.12) -- (0.805,0.17);

    % \draw[color=black,x={(image.south east)},y={(image.north west)}]             (0.755,0.92) -- (0.765,0.97);
    % \draw[color=black,x={(image.south east)},y={(image.north west)}]             (0.795,0.92) -- (0.805,0.97);

    \draw [white,fill=white,x={(image.south east)},y={(image.north west)}] (0.71,0.1) rectangle (0.75,1.01);
    \draw[color=black,x={(image.south east)},y={(image.north west)}]   (0.705,0.095) -- (0.715,0.145);
    \draw[color=black,x={(image.south east)},y={(image.north west)}]   (0.745,0.095) -- (0.755,0.145);

    \draw[color=black,x={(image.south east)},y={(image.north west)}]   (0.705,0.95) -- (0.715,1.0);
    \draw[color=black,x={(image.south east)},y={(image.north west)}]   (0.745,0.95) -- (0.755,1.0);
    %\node [draw,rotate=90,x={(image.south east)},y={(image.north west)}]                   at (0.50,0.50)    {text0};
    %\node [draw,rotate=0 ,x={(image.south east)},y={(image.north west)}]                   at (0.22,0.96)    {text1};
    %\node [draw,rotate=0 ,x={(image.south east)},y={(image.north west)},anchor=west]       at (0.22,0.80)    {text2};
    %\draw[->,color=black,thick,x={(image.south east)},y={(image.north west)}]             (0.42,0.22) -- (0.37,0.23);
  \end{tikzpicture}
  \caption{Peak and integrated cleaning inefficiency in the DS1 and DS2 with TCLDs in cell 8 and cell 10 for \lead ions at 7$\,Z\,$TeV in the HL-LHC configuration (B1H). Parking corresponds to a fully retracted TCLD.}  
  \label{pic:16081603}
  %/media/phermes/local/hisix_results/HLLHC/B1H/analysis/postprocessing/output/comparison_2TCLD_eta.pdf
  \end{figure}



\begin{figure}[p]
  \centering
  \begin{tikzpicture}
    \footnotesize
    \node[anchor=south west,inner sep=0] (image) at (0,0) {\includegraphics[width=1.0\linewidth]{pictures/16090308.pdf}};
    %\node [draw,rotate=90,x={(image.south east)},y={(image.north west)}]                   at (0.50,0.50)    {text0};
    \node [fill=white,x={(image.south east)},y={(image.north west)}]                   at (0.9,0.97)    {TCLD at $10\,\sigma$};
    \node [fill=white,x={(image.south east)},y={(image.north west)}]                   at (0.9,0.73)    {TCLD at $12\,\sigma$};
    \node [fill=white,x={(image.south east)},y={(image.north west)}]                   at (0.9,0.50)    {TCLD at $15\,\sigma$};
    \node [fill=white,x={(image.south east)},y={(image.north west)}]                   at (0.9,0.26)    {Without TCLD};

  \end{tikzpicture}
  \caption{B1H loss maps (full LHC ring) simulated for the HL-LHC configuration with \lead ions at 7$\,Z\,$TeV and two TCLD collimators in cell 8 and cell 10 of IR7. The four panels show the loss maps simulated with three different settings of the TCLD collimators (three top panels) and without TCLD collimator (bottom panel).}  
  \label{pic:16081604}
  %/media/phermes/local/hisix_results/HLLHC/B1H/analysis/postprocessing/output/comparison_2TCLD_LHC.pdf
  \end{figure}



\begin{figure}[t]
  \centering
  \begin{tikzpicture}
    \footnotesize
    \node[anchor=south west,inner sep=0] (image) at (0,0) {\includegraphics[width=1.0\linewidth]{pictures/16090702.pdf}};

    \node [fill=white,x={(image.south east)},y={(image.north west)}]                   at (0.605,0.250)    {DS1};
    \node [fill=white,x={(image.south east)},y={(image.north west)}]                   at (0.685,0.250)    {DS2};

    \node [rotate=90,fill=white,x={(image.south east)},y={(image.north west)}]                   at (0.55,0.88)    {TCLD8};
    \node [rotate=90,fill=white,x={(image.south east)},y={(image.north west)}]                   at (0.63,0.88)    {TCLD10};

    \node [fill=white,x={(image.south east)},y={(image.north west)}]                   at (0.9,0.92)    {TCLD at $10\,\sigma$};
    \node [fill=white,x={(image.south east)},y={(image.north west)}]                   at (0.9,0.70)    {TCLD at $12\,\sigma$};
    \node [fill=white,x={(image.south east)},y={(image.north west)}]                   at (0.9,0.48)    {TCLD at $15\,\sigma$};
    \node [fill=white,x={(image.south east)},y={(image.north west)}]                   at (0.9,0.25)    {Without TCLD};
  \end{tikzpicture}
  \caption{B1H loss maps (zoomed to IR7) simulated for the HL-LHC configuration with \lead ions at 7$\,Z\,$TeV and two TCLD collimators (installed in cell 8 and cell 10) per beam in IR7. The four panels show the loss maps simulated with three different settings of the TCLD collimators (three top panels) and without TCLD collimator (bottom).}  
  \label{pic:16081602}
  %/media/phermes/local/hisix_results/HLLHC/B1H/analysis/postprocessing/output/comparison_1TCLD.pdf
  \end{figure}





%
%

%\newpage

\mbox{} \\ 
\textit{Global Loss Distribution} \\
The comparison illustrated in \figref{pic:16081604} shows that on a global scale, the losses can be significantly reduced by the installation of the second TCLD collimator. For the tightest TCLD setting applied, the peak cleaning inefficiency in cold LHC regions is in the order of $\eta^{max}\approx 3 \times 10^{-5}$\,m$^{-1}$. The location exposed to the highest amount of losses is the DS1 cluster in the IR7 DS. Apparently, the installation of the second TCLD collimator removes a large fraction of the off-rigid ion fragments from the beam, which liberates the complete ring from most of the collimation losses. 




\newpage
%\mbox{} \\ 
\textit{Losses in the DS1} \\
Only slight differences arise compared to the simulation with one TCLD. In the latter, very few \lead particles re-impact at the TCP in a subsequent turn, which slightly changes the loss pattern in IR7, including the DS1. These out-scattered particles are now intercepted by the TCLD10. Nevertheless, their contribution is very small and the DS1 losses remain almost unchanged with respect to the previous simulation. In all studied scenarios, the globally largest amount of collimation losses in superconducting magnets is simulated in the DS1.

\begin{table}[b]
\centering
\caption{Cleaning inefficiency in the DS1 and DS2 clusters with different TCLD half gaps. The quantities $\eta_{DS1}^{max}$ and $\eta_{DS2}^{max}$ correspond to the maximum average cleaning inefficiency simulated in the DS1 and DS2 respectively.}
\label{tab:peakred}
\footnotesize

\begin{tabular}{cccccccc}
\toprule

Study   & Setting    & $\eta_{DS1}^{max}$           & $\eta_{DS2}^{max}$           & $\eta_{DS1}^{int}$  & $\eta_{DS2}^{int}$ & $\eta_{DS1}^{avg}$  & $\eta_{DS2}^{avg}$    \\
        & [$\sigma$] &  [m$^{-1}$]       & [m$^{-1}$]        &                  &                      &  [m$^{-1}$]       & [m$^{-1}$]                                      \\\midrule
No TCLD & -          &  $2.7 \times 10^{-2}$       &    $2.5 \times 10^{-2}$ &     \tm{9.5}{2}             &         \tm{1.1}{1}  &  \tm{4.0}{3}    &     \tm{4.7}{3}                                         \\ \midrule

1 TCLD  & 15         &  $7.2 \times 10^{-3}$       &    $2.2 \times 10^{-2}$      &    \tm{3.9}{3}              &        \tm{7.2}{2}   &  \tm{1.0}{3}   &     \tm{4.2}{3}                                          \\
1 TCLD  & 12         &  $7.5 \times 10^{-5}$       &    $1.0 \times 10^{-2}$      &    \tm{7.0}{4}              &        \tm{4.5}{2}   &   \tm{3.6}{5}    &     \tm{2.7}{3}                                          \\
1 TCLD  & 10         &  $4.2 \times 10^{-5}$       &    $5.7 \times 10^{-3}$      &    \tm{2.7}{4}              &        \tm{4.0}{2}   &   \tm{1.5}{5}    &      \tm{2.6}{3}                                         \\ \midrule
2 TCLDs & 15         &  $7.2 \times 10^{-3}$       &    $8.0 \times 10^{-4}$      &    \tm{3.9}{3}              &        \tm{2.5}{3}   &    \tm{1.0}{3}  &     \tm{1.4}{4}                                         \\
2 TCLDs & 12         &  $7.5 \times 10^{-5}$       &    $8.4 \times 10^{-6}$      &    \tm{7.0}{4}              &        \tm{2.7}{5}   &   \tm{3.6}{5}   &     \tm{1.5}{6}                                         \\
2 TCLDs & 10         &  $4.2 \times 10^{-5}$       &    $1.2 \times 10^{-5}$      &    \tm{2.7}{4}              &        \tm{3.2}{5}   &    \tm{1.5}{5}  &     \tm{2.5}{6}                                         \\ \bottomrule
\end{tabular}
\end{table}

\newpage
%\mbox{} \\ 
\textit{Losses in the DS2} \\
The losses in the DS2 are significantly reduced when the second TCLD collimator is installed. Already with the most relaxed TCLD setting of 15\,$\sigma$, the maximum cleaning inefficiency in the DS2 is reduced by almost three orders of magnitude. For tighter TCLD settings, the integrated and peak losses in the DS2 cluster increase, because more particles which would be lost in other regions are intercepted by the TCLD10 and hence more debris is produced in it. This finding is in line with the fact that the global losses are significantly reduced when TCLDs are installed in IR7. In the STIER simulations of the same scenario, presented in \cite{phermes_ipac2015_1}, this information was inaccessible because secondary fragmentation is not included in STIER.

\mbox{} \\ 
%\newpage
\textit{Maximum average cleaning inefficiency} \\
With two TCLDs at the most relaxed setting of 15$\,\sigma$, the maximum average cleaning inefficiency is reduced by approximately 79\% compared to the nominal HL-LHC simulation without TCLDs. In this scenario, the limiting losses are measured at the DS1. With tighter settings, the quench risk can be reduced by more than two orders of magnitude. The maximum average cleaning inefficiency with a TCLD half gap of 10$\,\sigma$ is only $3 \times 10^{-3}$ of the nominal value without TCLDs. 

The direct comparison to the study case with one TCLD in \figref{pic:16091422} shows that the maximum average cleaning inefficiency can be significantly reduced by the installation of a second TCLD per beam. This reflects directly in the achievable stored beam energy, as it shall be discussed in \chapref{chap:perflimi}. With two TCLD collimators per beam, intensity limitations by the risk of quenches from collimation losses can be abolished.



\begin{figure}[t]
  \centering
  \begin{tikzpicture}
    \node[anchor=south west,inner sep=0] (image) at (0,0) {\includegraphics[width=0.6\linewidth]{pictures/16091423.pdf}};
  % \node [draw,rotate=90,x={(image.south east)},y={(image.north west)}]                   at (0.50,0.50)    {text0};
  % \node [draw,rotate=0 ,x={(image.south east)},y={(image.north west)}]                   at (0.22,0.96)    {text1};
  % \node [draw,rotate=0 ,x={(image.south east)},y={(image.north west)},anchor=west]       at (0.22,0.80)    {text2};
  % \draw[->,color=black,thick,x={(image.south east)},y={(image.north west)}]             (0.42,0.22) -- (0.37,0.23);
    % \draw [white,fill=white,x={(image.south east)},y={(image.north west)}] (0.65,0.1) rectangle (0.69,1.01);
    % \draw[color=black,x={(image.south east)},y={(image.north west)}]   (0.645,0.095) -- (0.655,0.145);
    % \draw[color=black,x={(image.south east)},y={(image.north west)}]   (0.685,0.095) -- (0.695,0.145);

    % \draw[color=black,x={(image.south east)},y={(image.north west)}]   (0.645,0.95) -- (0.655,1.0);
    % \draw[color=black,x={(image.south east)},y={(image.north west)}]   (0.685,0.95) -- (0.695,1.0);
%  \draw[help lines,step=.05,x={(image.south east)},y={(image.north west)}] (0,0) grid (1,1);
%  \draw[help lines,line width=.6pt,step=0.1,x={(image.south east)},y={(image.north west)}] (0,0) grid (1,1);
%  \foreach \x in {0,0.1,0.2,0.3,0.4,0.5,0.6,0.7,0.8,0.9,1.0}
%       \node[anchor=north,x={(image.south east)},y={(image.north west)}] at (\x,-0.01) {\x};
%  \foreach \y in {0.0,0.1,0.2,0.3,0.4,0.5,0.6,0.7,0.8,0.9,1.0}
%      \node[anchor=east,x={(image.south east)},y={(image.north west)}] at (-0.01,\y) {\y};

  \end{tikzpicture}
  \caption{Maximum average cleaning inefficiency in the DS1 and DS2 with one and two TCLDs for \lead ions at 7$\,Z\,$TeV in the HL-LHC configuration (B1H). Parking corresponds to a fully retracted TCLD.}  
  \label{pic:16091422}
  %/media/phermes/local/hisix_results/HLLHC/B1H/analysis/postprocessing/output/comparison_etaavg_eta.pdf
  \end{figure}



\newpage
\subsection{Conclusions}




The simulations with the hiSixTrack-FLUKA coupling for the HL-LHC configuration with \lead beams at 7$\,Z\,$TeV have shown that the cleaning performance can be improved if one or two TCLDs are installed in IR7. A complete overview of the cleaning inefficiency in the superconducting magnets is given for the different settings of the TCLDs in \tabref{tab:peakred}. 
\vspace{0.2cm}

% Overall, the installation of one TCLD in cell 8 leads to a significant reduction of the DS1 losses for all study cases. However, the most crucial parameter to quantify the risk of quench is the peak cleaning inefficiency $\eta_{cold}^{max}$ in the superconducting LHC magnets. In the nominal simulation, the highest loss peak in the cold regions is simulated in the DS1. With the TCLD8 installed the highest loss peak in superconducting magnets is found in the DS2. The possible improvement of the cleaning inefficency is hence the ratio of the highest cleaning inefficiency in cold regions with and without TCLD collimator. It yields approximately 18\% for the most relaxed setting and approximately 78\% for the tightest setting. The associated increase of the allowed stored beam energy is then approximately 14\% with the most relaxed setting and approximately 450\% with the tightest setting. In conclusion, a significant improvement for the allowed stored beam energy can be achieved, if the TCLD8 can be set to sufficiently tight settings.


The simulations indicate that the potential for improvement with only one TCLD collimator per beam is rather limited. The maximum average losses in the cold LHC regions can be reduced by less than 50\% even if the smallest studied TCLD half gap of $10\,\sigma$ is applied. A significant improvement can be expected with the installation of two TCLD collimators. With the most relaxed collimator setting, the maximum average losses in the DS magnets are reduced by 80\%. With a setting of $12\,\sigma$, the losses in superconducting regions are reduced by almost two orders of magnitude. In this configuration, the risk of quench from collimation debris becomes irrelevant for operation and other constraints (e.g. collimator damage or the injector performance) limit the achievable beam intensity. Furthermore, the simulations show that the installation of the second TCLD provides a global loss protection. This means that possibly critical losses also in regions downstream of IR7 can be mitigated by the second TCLD collimator. This should also be considered in the discussion of the collimation system upgrades for HL-LHC.
\vspace{0.2cm}

To derive the allowed stored beam energy from these simulations, experimental input about the quench limit in operational conditions in required. This is discussed in detail in \chapref{chapter:quenchtest}. 


\newpage


\section{Operation with Ar and Xe Ions} \label{chap:xear}

The LHC injectors are capable of providing different heavy-ion species than \lead. Presently these lighter heavy ions are only used for fixed target experiments with beams extracted from the SPS. The NA61/Shine experiment has so far received \iso{40}{Ar}{18+} ions for fixed target experiments and \iso{129}{Xe}{54+} ions are scheduled for future operation \cite{IPAC16:TUPMR027,EDMS:1570447,CERN-2014-006}. 

At the time of writing, LHC operation with these ions is not envisaged, but some of the LHC experiments have shown interest in colliding other heavy-ion species than \lead. The reason for this interest is that, compared to \lead, bunches with higher intensities could be delivered to the LHC and hence the achievable luminosity could be higher~\cite{IPAC16:TUPMR027}. To study possible limitations from the LHC collimation system, an exploratory study of the cleaning performance with the ions \iso{40}{Ar}{18+} and \iso{129}{Xe}{54+} is conducted with the hiSixTrack-FLUKA coupling.

\subsection{Simulation Setup}

The cleaning simulations are carried out for B1H in the HL-LHC configuration without TCLDs at $7\,Z\,$TeV. The optics and collimator settings are identical to those used for the simulations with \lead beams and are listed in \tabref{tab:sets_hl}. In all simulations, a primary beam of $6.0 \times 10^6$ ions is assumed to impact the TCP at $b=1\,\mu$m.











% \begin{figure}[t]
%   \centering
%   \begin{tikzpicture}
%     \footnotesize
%     \node[anchor=south west,inner sep=0] (image) at (0,0) {\includegraphics[width=1.0\linewidth]{pictures/16072701.pdf}};
%     \node [fill=white,x={(image.south east)},y={(image.north west)}]       at (0.94,0.94)    {\lead};
%     \node [fill=white,x={(image.south east)},y={(image.north west)}]       at (0.94,0.755)    {\iso{129}{Xe}{54+}};
%     \node [fill=white,x={(image.south east)},y={(image.north west)}]       at (0.94,0.57)    { \iso{40}{Ar}{18+}};
%     \node [fill=white,x={(image.south east)},y={(image.north west)}]       at (0.94,0.39)    {\iso{4}{He}{2+}};
%     \node [fill=white,x={(image.south east)},y={(image.north west)}]       at (0.94,0.21)    {Protons};
%     %\node [draw,rotate=0 ,x={(image.south east)},y={(image.north west)}]                   at (0.22,0.96)    {text1};
%     %\node [draw,rotate=0 ,x={(image.south east)},y={(image.north west)},anchor=west]       at (0.22,0.80)    {text2};
%     %\draw[->,color=black,thick,x={(image.south east)},y={(image.north west)}]             (0.42,0.22) -- (0.37,0.23);
%   \end{tikzpicture}

%   %/media/phermes/local/hisix_results/HLLHC/Other_Ions/analysis/postprocessing/isotope_comparison_IR7.pdf
%   \end{figure}

\subsection{Simulation Result}

The loss maps simulated for the different heavy-ion species are compared in \figref{pic:16072601} (zoom to IR7) and \figref{pic:16072602} (full LHC ring). Peak, integrated and maximum average cleaning inefficiency in the DS clusters are summarized for all study cases in in \tabref{tab:hl_isotopes}. 


\begin{figure}[t]
  \centering
  \begin{tikzpicture}
    \footnotesize
    \node[anchor=south west,inner sep=0] (image) at (0,0) {\includegraphics[width=1.0\linewidth]{pictures/16090803.pdf}};


  % \node [draw,rotate=90,x={(image.south east)},y={(image.north west)}]                   at (0.50,0.50)    {text0};
  % \node [draw,rotate=0 ,x={(image.south east)},y={(image.north west)}]                   at (0.22,0.96)    {text1};
  % \node [draw,rotate=0 ,x={(image.south east)},y={(image.north west)},anchor=west]       at (0.22,0.80)    {text2};
  % \draw[->,color=black,thick,x={(image.south east)},y={(image.north west)}]             (0.42,0.22) -- (0.37,0.23);
   \node [fill=white,x={(image.south east)},y={(image.north west)}]        at (0.92,0.90)    {\lead};
    \node [fill=white,x={(image.south east)},y={(image.north west)}]       at (0.92,0.61)    {\iso{129}{Xe}{54+}};
    \node [fill=white,x={(image.south east)},y={(image.north west)}]       at (0.92,0.32)    { \iso{40}{Ar}{18+}};
    \node [fill=white,x={(image.south east)},y={(image.north west)}]       at (0.67,0.88)    {DS1};
    \node [fill=white,x={(image.south east)},y={(image.north west)}]       at (0.77,0.88)    {DS2};


%  \draw[help lines,step=.05,x={(image.south east)},y={(image.north west)}] (0,0) grid (1,1);
%  \draw[help lines,line width=.6pt,step=0.1,x={(image.south east)},y={(image.north west)}] (0,0) grid (1,1);
%  \foreach \x in {0,0.1,0.2,0.3,0.4,0.5,0.6,0.7,0.8,0.9,1.0}
%       \node[anchor=north,x={(image.south east)},y={(image.north west)}] at (\x,-0.01) {\x};
%  \foreach \y in {0.0,0.1,0.2,0.3,0.4,0.5,0.6,0.7,0.8,0.9,1.0}
%      \node[anchor=east,x={(image.south east)},y={(image.north west)}] at (-0.01,\y) {\y};

  \end{tikzpicture}
  \caption{Loss maps zoomed to IR7 simulated with the hiSixTrack-FLUKA coupling for the HL-LHC configuration at 7$\,Z\,$TeV with \lead ions, \iso{129}{Xe}{54+} ions and \iso{40}{Ar}{18+} ions.}
  \label{pic:16072601}

  %/media/phermes/local/hisix_results/HLLHC/Other_Ions/analysis/postprocessing/iso_comparison_IR7.pdf
  \end{figure}


\begin{table}[b]
\centering
\caption{Cleaning inefficiency for different isotopes in the HL-LHC configuration.}
\label{tab:hl_isotopes}
\begin{tabular}{cccccc}
\toprule
\begin{tabular}[c]{@{}c@{}}Isotope \\ \mbox{} \end{tabular} & 
\begin{tabular}[c]{@{}c@{}}$\eta^{max}_{cold}$\\ {[$10^{-2}$ m$^{-1}$]}\end{tabular} &  
\begin{tabular}[c]{@{}c@{}}$ \eta^{int}_{DS1}$\\ {[$10^{-2}$]}\end{tabular} &  
\begin{tabular}[c]{@{}c@{}}$ \eta^{int}_{DS2}$\\ {[$10^{-2}$]}\end{tabular} &  
\begin{tabular}[c]{@{}c@{}}$ \eta^{avg}_{DS1}$\\ {[$10^{-3}$ m$^{-1}$]}\end{tabular} &  
\begin{tabular}[c]{@{}c@{}}$ \eta^{avg}_{DS2}$\\ {[$10^{-3}$  m$^{-1}$]}\end{tabular} \\ \midrule
    \lead                   &      2.7   &  9.5   &  10.6    &  4.0   & 4.7  \\
    \iso{129}{Xe}{54+}      &      3.5   &  12.0  &  8.3     &  4.8   & 4.3  \\
    \iso{40}{Ar}{18+}       &      2.5   &  11.1  &  4.6     &  4.6   & 1.9  \\ \bottomrule
\end{tabular}
\end{table}


The global distribution of loss peaks is qualitatively similar for all study cases. Shape and amplitude of the losses in the DS clusters are slightly different. Among the simulated scenarios, the largest peak loss in the IR7 DS is simulated with \iso{129}{Xe}{54+} ions. However, the comparison of the integrated cleaning inefficiency shows that the sum of $\eta^{int}_{DS1}$ and $\eta^{int}_{DS2}$ is almost the same as for \lead ions. The amount of losses in the DS is hence similar, but the collimation debris is distributed differently. Also the maximum average cleaning inefficiency for \iso{129}{Xe}{54+} is slightly higher than for \lead, while for \iso{40}{Ar}{18+} it is slightly lower. 

A remarkable difference observed between the different simulation scenarios is the ratio of the losses in the DS1 and DS2. For the simulation with \lead ions, the peak, integrated and maximum average losses are similar in the DS1 and DS2, with slightly more losses in the DS2. 


With the lighter ions, the losses in the DS1 cluster become dominating. For \iso{129}{Xe}{54+}, the integrated losses in the DS1 are approximately higher by 44\% compared to the DS2. In the simulation with \iso{40}{Ar}{18+} ions, they are more than twice as high as in the DS2.




\begin{figure}[htbp]
  \centering
  \begin{tikzpicture}
    \footnotesize
    \node[anchor=south west,inner sep=0] (image) at (0,0) {\includegraphics[width=1.0\linewidth]{pictures/16090804.pdf}};

    \node[anchor=south west,inner sep=0,x={(image.south east)},y={(image.north west)}] (leg) at (0.35,0.24) {\includegraphics[width=0.15\linewidth]{pictures/legend.pdf}};



   \node [fill=white,x={(image.south east)},y={(image.north west)}]        at (0.93,0.96)    {\lead};
    \node [fill=white,x={(image.south east)},y={(image.north west)}]       at (0.93,0.65)    {\iso{129}{Xe}{54+}};
    \node [fill=white,x={(image.south east)},y={(image.north west)}]       at (0.93,0.33)    { \iso{40}{Ar}{18+}};
  % \node [draw,rotate=90,x={(image.south east)},y={(image.north west)}]                   at (0.50,0.50)    {text0};
  % \node [draw,rotate=0 ,x={(image.south east)},y={(image.north west)}]                   at (0.22,0.96)    {text1};
  % \node [draw,rotate=0 ,x={(image.south east)},y={(image.north west)},anchor=west]       at (0.22,0.80)    {text2};
  % \draw[->,color=black,thick,x={(image.south east)},y={(image.north west)}]             (0.42,0.22) -- (0.37,0.23);

%  \draw[help lines,step=.05,x={(image.south east)},y={(image.north west)}] (0,0) grid (1,1);
%  \draw[help lines,line width=.6pt,step=0.1,x={(image.south east)},y={(image.north west)}] (0,0) grid (1,1);
%  \foreach \x in {0,0.1,0.2,0.3,0.4,0.5,0.6,0.7,0.8,0.9,1.0}
%       \node[anchor=north,x={(image.south east)},y={(image.north west)}] at (\x,-0.01) {\x};
%  \foreach \y in {0.0,0.1,0.2,0.3,0.4,0.5,0.6,0.7,0.8,0.9,1.0}
%      \node[anchor=east,x={(image.south east)},y={(image.north west)}] at (-0.01,\y) {\y};

  \end{tikzpicture}
  \caption{Full ring loss maps simulated with the hiSixTrack-FLUKA coupling for the HL-LHC configuration at 7$\,Z\,$TeV with \lead ions, \iso{129}{Xe}{54+} ions and \iso{40}{Ar}{18+} ions.}  
  \label{pic:16072602}
  %/media/phermes/local/hisix_results/HLLHC/Other_Ions/analysis/postprocessing/iso_comparison_LHC.pdf
  \end{figure}


As discussed in \chapref{chap:STIER:full}, the probability that a residual heavy-ion fragment is lost at the DS1 instead of the DS2 location increases with the rigidity offset. As a consequence of the quantization of nucleons in the nuclei, the nuclear loss of the same amount of protons or neutrons results in a larger rigidity offset with respect to the main beam. In \tabref{tab:isohl:chi}, the $\chi$-values of particles generated from EMD1 and EMD2 are summarized for the different reference isotopes. The comparison shows that the rigidity offset associated to particles created in EMD processes is significantly increasing with decreasing mass of the main beam.  


% \begin{figure}[t]
%   \centering
%   \begin{tikzpicture}
%     \footnotesize
%     \node[anchor=south west,inner sep=0] (image) at (0,0) {\includegraphics[width=1.0\linewidth]{pictures/16072702.pdf}};
%     %\node [draw,rotate=90,x={(image.south east)},y={(image.north west)}]                   at (0.50,0.50)    {text0};
%     %\node [draw,rotate=0 ,x={(image.south east)},y={(image.north west)}]                   at (0.22,0.96)    {text1};
%     %\node [draw,rotate=0 ,x={(image.south east)},y={(image.north west)},anchor=west]       at (0.22,0.80)    {text2};
%     %\draw[->,color=black,thick,x={(image.south east)},y={(image.north west)}]             (0.42,0.22) -- (0.37,0.23);

%    \node [fill=white,x={(image.south east)},y={(image.north west)}]       at (0.94,0.97)    {\lead};
%     \node [fill=white,x={(image.south east)},y={(image.north west)}]       at (0.94,0.78)    {\iso{129}{Xe}{54+}};
%     \node [fill=white,x={(image.south east)},y={(image.north west)}]       at (0.94,0.59)    { \iso{40}{Ar}{18+}};
%     \node [fill=white,x={(image.south east)},y={(image.north west)}]       at (0.94,0.40)    {\iso{4}{He}{2+}};
%     \node [fill=white,x={(image.south east)},y={(image.north west)}]       at (0.94,0.21)    {Protons};
%   \end{tikzpicture}
%   \caption{Full ring loss maps simulated for the HL-LHC configuration at 7$\,Z\,$TeV with \lead ions, \iso{129}{Xe}{54+} ions, \iso{40}{Ar}{18+} ions, \iso{4}{He}{2+} ions and protons.}  
%   \label{pic:16072602}
%   %/media/phermes/local/hisix_results/HLLHC/Other_Ions/analysis/postprocessing/isotope_comparison_LHC.pdf
%   \end{figure}


%  \begin{table}[b]
% \centering
% \caption{Maximum stored beam energy $E^\text{max}_\text{tot}$ and the corresponding maximum number of ions stored in the machine, derived from the simulations at 7\,$Z$\,TeV. }
% \label{tab:intensity:limits}
% \begin{tabular}{cccccc}
%   \toprule
%    Quantity                  & Unit               & \lead            &     \iso{129}{Xe}{54+}        &       \iso{40}{Ar}{18+}      &      \iso{4}{He}{2+}                       \\ \midrule
% $E^{\text{max}}_\text{tot}$ & {[}MJ{]}             & $8.9 \pm 0.8$ & $6.9 \pm 0.7$ & $9.4 \pm 0.9$ & $314 \pm 29$                \\
% Max. number of ions &  {[}$10^{11}${]} & $1.0 \pm 0.1$ & $1.1 \pm 0.2$ & $4.7 \pm 0.5$ & $ (1.4 \pm 0.2) \times 10^3$ \\ \bottomrule
% \end{tabular}
% \end{table}


\begin{table}[t]
\centering
\caption{Rigidity offset (expressed in terms of $\chi$, assuming that $\delta=0$) of fragments generated by EMD1 and EMD2 for different reference isotopes.}
\label{tab:isohl:chi}
\begin{tabular}{ccc}
\toprule
    Ref. species            &   $\chi_\text{EMD1}$   &  $\chi_\text{EMD2}$     \\ \midrule
    \lead                   &   1.0048 &  1.0097 \\
    \iso{129}{Xe}{54+}      &   1.0078  & 1.0157 \\
    \iso{40}{Ar}{18+}       &   1.0256  & 1.0527 \\ \bottomrule
\end{tabular}
\end{table}



% \begin{figure}[htpb]
%   \centering
%   \begin{tikzpicture}
%     \small
%     \node[anchor=south west,inner sep=0] (image) at (0,0) {\includegraphics[width=0.6\linewidth]{pictures/16072603.pdf}};
%     \node [rotate=90,x={(image.south east)},y={(image.north west)}]                   at (-0.02,0.55)    {$\chi_{\text{EMD}}-1$};
%     %\node [draw,rotate=0 ,x={(image.south east)},y={(image.north west)}]                   at (0.22,0.96)    {text1};
%     %\node [draw,rotate=0 ,x={(image.south east)},y={(image.north west)},anchor=west]       at (0.22,0.80)    {text2};
%     %\draw[->,color=black,thick,x={(image.south east)},y={(image.north west)}]             (0.42,0.22) -- (0.37,0.23);
%   \end{tikzpicture}
%   \caption{$\chi_{\text{EMD}}-1$ as a function of $A$ for all isotopes lighter than \lead.}  
%   \label{pic:16072603}
%   %/home/phermes/Dropbox/codes/python/160726_chi_EMD/chiEMD.pdf
%   \end{figure}

\subsection{Summary}

In summary, the simulation with the hiSixTrack-FLUKA coupling shows that the loss pattern and the cleaning inefficiency for the studied heavy-ion species is comparable to that of \lead. Therefore, the same limitations in terms of stored beam energy apply similarly for the other ion species. On the other hand, especially for the scenario \iso{40}{Ar}{18+} beams, already today higher bunch intensities are available than those which may be reachable for \lead ions in \mbox{HL-LHC}~\cite{IPAC16:TUPMR027}. Compared to \lead beams, the higher bunch intensity could increase the achievable luminosity, even if the allowed total stored beam energy would be similar. 

Furthermore, the charge to mass ratio and therefore the relativistic $\gamma$ of \iso{40}{Ar}{18+} is higher than that for \lead at the same rigidity.  Assuming that the normalized emittance with Ar and Xe is the same as for \lead beams, this leads to a further increase of the luminosity.


To study potential limitations of the allowed stored beam energy with \lead, \iso{40}{Ar}{18+} and  \iso{129}{Xe}{54+} beams, it is required to have experimental input about the quench limit. This is further studied in \chapref{chap:perflimi}.



      
 

\newpage

\section{Summary and Conclusions}


The new heavy-ion collimation simulation tools STIER and the hiSixTrack-FLUKA coupling were applied to study the cleaning performance in different configurations of the LHC and HL-LHC. In the 2015 heavy-ion run, STIER was used to validate the collimation system and probe different strategies to reduce the IR7 DS losses. It was also used to derive mitigation strategies for losses at the tertiary collimator in IR2, which were successfully tested in the LHC. 

The hiSixTrack-FLUKA coupling was used for exploratory studies of the cleaning performance in HL-LHC. The simulations with TCLD collimators delivered new evidence which is going to be essential in the discussion of the required collimation system upgrades for HL-LHC. Furthermore, the cleaning inefficiency with other heavy-ion species than \lead was explored. 


These applications demonstrate the importance of heavy-ion collimation simulation tools. They are not only required to ensure safe and uninterrupted LHC operation, they also define the requirements for future upgrades which are often related to substantial budgetary decisions. 



In order to relate these findings to tangible intensity and hence luminosity limitations for the operation in the LHC, the quench limit of the IR7 DS magnets must be known. For this purpose, a dedicated heavy-ion collimation quench experiment was carried out at the end of the heavy-ion operation in 2015. This quench test and the resulting implications on the permitted stored beam energy in HL-LHC are presented in the next chapter. 


















\chapter{Heavy-Ion Collimation Quench Test in 2015}  \label{chapter:quenchtest}

To exploit the full potential of the LHC and achieve the highest possible (integrated) luminosity, the beam intensity has to be pushed to the maximum which is compatible with safe and uninterrupted operation. One of the limiting factors for the maximum stored beam energy is the possible quench of the superconducting DS magnets downstream of IR7 from collimation debris. As shown in \chapref{chap:quenchlim}, the quench limit of the LHC magnets is difficult to estimate and subject to large uncertainties. It also depends on the loss scenario, which reflects in the fact that the BLM thresholds are set with respect to the expected quench limit for UFO events~\cite{IPAC2011:TUPC137}. Unidentified Falling Objects (UFOs) are dust particles with size in the order of some \mum\, falling from the top of the beam pipe. The beam particles interact with the UFOs which leads to secondary showers that can possibly quench the superconducting magnets~\cite{IPAC2011:TUPC137}. 

The quench limit for other loss scenarios, such as the impact of collimation debris on the superconducting magnets can be experimentally accessed with beam in operational conditions. In such measurements, very high losses are produced at the primary collimator with the aim to quench the IR7 DS magnets with the collimation debris in a controlled manner. 

\begin{table}[b]
\centering
\caption{Key parameters of collimation quench tests carried out at the LHC. }
\label{tab:quenchtests}
\begin{tabular}{cccccccc} 
\toprule
\begin{tabular}[c]{@{}c@{}} Year \\ \mbox{} \end{tabular} & \begin{tabular}[c]{@{}c@{}}Energy\\ {[}$Z$ GeV{]}\end{tabular} & Particle & Method & \begin{tabular}[c]{@{}c@{}}$P_\text{max}$\\ {[}kW{]}\end{tabular} & \begin{tabular}[c]{@{}c@{}}$\tau_L$\\ {[}s{]}\end{tabular} & Quench & \begin{tabular}[c]{@{}c@{}}Ref.\\  \mbox{} \end{tabular} \\ \midrule
2011 & 3.5 & p & tune & 500 & 1 & No & \cite{CERN-ATS-NOTE-2011-042MD} \\
2013 & 4.0 & p & ADT & 1050 & 5-10 & No & \cite{CERN-ATS-NOTE-2014-0036} \\
2015 & 6.5 & p & ADT & 585 & 4 & No & \cite{CERN-ACC-NOTE-2016-0015}  \\
2011 & 3.5 & \lead & tune & 151 & 0.075 & No & \cite{CERN-ATS-NOTE-2012-081MD} \\
2015 & 6.37 & \lead & ADT & 15 & 14 & Yes & \cite{CERN-ACC-NOTE-2016-0015}  \\ \bottomrule
\end{tabular}
\end{table}


Several collimation quench tests have been carried out in the past to get experimental input for quench limit studies~\cite{CERN-ATS-NOTE-2011-042MD,CERN-ATS-NOTE-2012-081MD,HB2012:MOP245,CERN-ATS-NOTE-2014-0036,CERN-ACC-NOTE-2016-0015}. The most important key parameters of this and previous tests are summarized in \tabref{tab:quenchtests}. 

Quench test experiments are complemented by simulations of the experimental setup. Combined with simulations from the hiSixTrack-FLUKA coupling, the gathered information can be used to define upgrade requirements for the LHC collimation system. Detailed shower propagation and energy deposition studies with FLUKA allow for interrelating the measured BLM signal to the energy deposited in the magnet coils~\cite{IPAC15:TUPTY046}. 


While the quench tests in early operation used tune resonance crossing methods, which induce very fast losses at the primary collimator, the tests from 2013 and later could make use of the ADT. This led to a much better control of the losses and enabled a significant increase of the loss duration to probe the quench limit with steady state losses. 

In this chapter, the design, realization and analysis of the 2015 heavy-ion collimation quench test are presented. It was the first collimation quench test in which a magnet quench was achieved. The content of this chapter was partly published in \cite{ACC-NOTE-16-0031}.

\section{Preparation of the Experiment}

The preparation of the test includes to determine the beam parameters, target loss rates and machine modifications required for the successful realization of the experiment. The quench limit in the DS magnets downstream of IR7 is probed with collimation debris which is lost immediately after its generation in the primary collimators. Therefore, the quench limit in operational conditions can be tested with un-squeezed beams in the IRs (flat top at 6.37\,$Z$\,TeV). Accordingly the collimator settings correspond to the operational settings in the collision mode, except for the TCTs and physics debris collimators (see \tabref{tab:sets_qt}). 

\begin{table}[b]
\centering
\caption{Collimator settings in $\sigma$ applied in the 2015 heavy-ion collimation quench test.}
\label{tab:sets_qt}
\begin{tabular}{cccccc} 
\toprule
\multicolumn{2}{c}{IR7}    &          \multicolumn{2}{c}{IR3}          &    \multicolumn{2}{c}{IR1/2/5/8/6} \\ \midrule
TCP    &  5.5  &   TCP   &  15.0    &    TCT  &   37.0 \\ 
TCSG   &  8.0  &   TCSG  &  18.0    &    TCL  &   out  \\
TCLA   & 14.0  &   TCLA  &  20.0    &   TCDQ  &   9.1  \\ \bottomrule
\end{tabular}
\end{table}

The beam and plane to be used for the test can be freely chosen, whereas preference should be given to the horizontal plane in which the DS losses are typically higher. The decision was taken to use the horizontal plane of B2 (B2H) for the test to potentially benefit from synergies with the precedent proton quench test which was also carried out in this plane~\cite{CERN-ACC-NOTE-2016-0015}. 


\newpage
\subsection{Target Beam Loss Rate} \label{chap:targetlr}
 
The target beam loss rate at the primary collimator is an important measure that must be defined before the experiment is carried out. It determines the number of bunches that must be excited simultaneously and hence has direct implications on the filling scheme. 

As explained before, the quench limit is related to many uncertainties. The decision of the target loss rate was therefore based on the operational BLM thresholds. The latter are set to 1.5 times the assumed quench limit for UFO events, which is taken as a baseline. The qualification loss map for B2H measured at flat top in the 2015 heavy-ion run is shown in \figref{pic:16090710}. The highest BLM signal at the superconducting LHC magnets is measured at the BLMQI.09L7.B2I10\_MQ which monitors the losses at the quadrupole in cell 9 left of IP7. The relative measured BLM Signal with 1.3~s integration time (RS09) at this BLM is 
%
\begin{align}
\mathcal{B}_\text{MQ9L7}^\text{lm} = (1.6 \pm 0.4) \times 10^{-2} \, . 
\end{align}
%
The uncertainty is estimated by variations in the peak cleaning inefficiency between different loss map measurements. From the measured loss signal $\mathcal{B}_\text{MQ9L7}^\text{lm}$ and the operational BLM threshold for RS09, the expected BLM signal at quench $\mathcal{B}_\text{MQ9L7}^\text{q}$ can be extrapolated 
%
\begin{align}
    \frac{\mathcal{B}_\text{MQ9L7}^\text{q}}{\mathcal{B}_\text{MQ9L7}^\text{lm}} =   (1.13 \pm 0.29) \times 10^{2} \, . \label{qtscl}
\end{align}

\begin{figure}[b]
  \centering
  \begin{tikzpicture}
    \footnotesize
    \node[anchor=south west,inner sep=0] (image) at (0,0) {\includegraphics[width=1.0\linewidth]{pictures/16090711.pdf}};
  % \node [draw,rotate=90,x={(image.south east)},y={(image.north west)}]                   at (0.50,0.50)    {text0};
  % \node [draw,rotate=0 ,x={(image.south east)},y={(image.north west)}]                   at (0.22,0.96)    {text1};
  \node [fill=white ,x={(image.south east)},y={(image.north west)},anchor=west]       at (0.1,0.85)    {B2H qualification loss map 23.11.2015 23:32:07};
  \draw[->,color=black,thick,x={(image.south east)},y={(image.north west)}]             (0.2,0.70) -- (0.24,0.65);

  \node [fill=white ,x={(image.south east)},y={(image.north west)},anchor=west]       at (0.11,0.73)    {BLMQI.09L7.B2I10\_MQ};

 % \draw[help lines,step=.05,x={(image.south east)},y={(image.north west)}] (0,0) grid (1,1);
 % \draw[help lines,line width=.6pt,step=0.1,x={(image.south east)},y={(image.north west)}] (0,0) grid (1,1);
 % \foreach \x in {0,0.1,0.2,0.3,0.4,0.5,0.6,0.7,0.8,0.9,1.0}
 %      \node[anchor=north,x={(image.south east)},y={(image.north west)}] at (\x,-0.01) {\x};
 % \foreach \y in {0.0,0.1,0.2,0.3,0.4,0.5,0.6,0.7,0.8,0.9,1.0}
 %     \node[anchor=east,x={(image.south east)},y={(image.north west)}] at (-0.01,\y) {\y};

  \end{tikzpicture}
  \caption{B2H qualification loss map with \lead ions at $6.37\,Z\,$TeV used for the preparation of the heavy-ion collimation quench test.}  
  \label{pic:16090710}
  %/home/phermes/Dropbox/PhD/pictures/160714_quench_test/qlm.pdf
  \end{figure}


During the measurement of the qualification loss map, an intensity drop equivalent to a peak primary beam loss of $P_l=123\,$W was measured with the beam current transformers (BCT)~\cite{BIW10:TUPSM051}. The quantity $P_l$ can be scaled with the ratio given in \eqref{qtscl} to obtain a BLM signal equivalent to the assumed quench limit for UFO events yields
%
\begin{align}
  P_q^T = \frac{\mathcal{B}_\text{MQ9L7}^\text{q}}{\mathcal{B}_\text{MQ9L7}^\text{lm}} \, P_l = (13.9 \pm 3.6) \, \text{kW} \, . \label{eq:peakloss}
\end{align}
%
This number should not be considered to be the expected quench limit. Rather, it should be regarded in the context of the beam loss mechanism which is different from the UFO scenario to which the BLM threshold is adjusted. Even for the UFO event, the quench limit is uncertain. 

However, $P_q^T$ is an important measure and was the first step in the cascade of envisaged target loss rates, because it can give important information about the accuracy of the setting of operational BLM thresholds. In case a quench would not be achieved with a loss rate of $P_q^T$,  it was envisaged to increase the loss rate in steps to narrow down the real quench limit. 

The upper boundary for the achievable loss rate is given by the power load that the primary collimators can resist without being damaged. They are designed to withstand continuous proton losses of 487~kW~\cite{EPAC02:TUAGB01}. The analysis of the 2013 proton collimation quench test demonstrated that their physical integrity is not endangered by losses up to 1~MW~\cite{IPAC14:MOPRO043}. 

The dominating process of energy deposition in the collimator is different for protons and heavy ions. The power deposited by an impacting proton and \lead ion at $7\,Z\,$TeV along the primary collimator is shown in Fig.~21.7 in \citedr. The charge dependence of the Bethe-Bloch formula indicates that the energy deposited by the \lead ions is, in a narrow region around the track of the impacting ion, more driven by ionization losses than for protons. The peak energy deposited from \lead ions at this energy is approximately 57\,GeV/cm$^3$/charge. For protons it is approximately 6.5\,GeV/cm$^3$/charge, mainly due to hadronic showers produced during the passage through the collimator. To protect against surface damage from ionization losses, the peak loss rate at the primary collimator was conservatively limited to 100~kW, leaving enough margin to significantly outreach the power load derived in \eqref{eq:peakloss}.

From the envisaged peak power load, the number of bunches required for the experiment can be deduced. Conservatively assuming that one ion bunch carries $10^{10}$ charges with an energy of $6.37\,$TeV per charge, the energy per bunch yields $E_B=$10~kJ. The LHC beam position monitors (BPM)~\cite{PAC99:THAR3} require sufficiently populated bunches to accurately measure their position. Therefore, the LHC interlock system triggers a beam dump if the bunch intensity of a circulating bunch is below 30\% of its nominal value. With this constraint included, on can derive the required number of bunches $n_B^P$ to continuously induce a primary power load of 1\,kW over 10~s
\begin{align}
 n_B^P=1.4 \,\text{bunches/kW} \,.
\end{align}
From the qualification loss map and the associated loss rate, the expected BLM signals at a peak loss rate of 100~kW can be estimated for the full ring. This data is used to derive the increase of BLM thresholds required to permit the target loss rate during the experiment without triggering a protection beam dump. In the preparation phase of the experiment, the required modifications on the BLM thresholds (mainly the BLMs in IR7) were prepared and presented in \cite{BLM-ECR-0043}. 





\subsection{Experimental Schedule}


From the quantity $n_B^P$, the required number of bunches to achieve a given loss rate can be deduced. The chosen approach of a stepwise loss rate increase reflects in the filling schedule that was foreseen for the experiment, as shown in \tabref{tab:filling_scheme}. 

An initial fill at injection energy was foreseen to set up the \acrshort{ADT}. The control software of the latter is designed for the excitation of single bunches and had to be adjusted for an enlarged window covering the many bunches required to achieve the target loss rate. Also the time profile of the losses had to be modified to obtain continuous losses at a constant level over some seconds. More details on the ADT modifications are given in \cite{ACC-NOTE-16-0031}.  

In the second fill, four bunch trains of 24 bunches as well as 8 individual bunches were foreseen to be accelerated to top energy for a first quench attempt with a peak loss rate around 13.5~kW. This approximately corresponds to the loss rate required to reach the BLM signal equivalent to the assumed quench limit for UFO events. The individual bunches should allow for a first testing of the ADT excitation at top energy  with small intensities, without the risk of outreaching the envisaged loss rate. The four bunch trains carry enough particles to allow multiple attempts at a peak loss rate of approximately 13.5~kW. In the case a quench would not be achieved, two optional fills were foreseen to increase the loss rate to 50~kW and 100~kW respectively. 

	\begin{table}[b]
		\centering
		\caption{Fills foreseen for the heavy-ion collimation quench experiment.}
		\label{tab:filling_scheme}
		\begin{tabular}{cccc}
                  \toprule
		Fill & Bunches & E {[}$Z$ TeV{]} & $P_{max}${[}kW{]} \\ \midrule
		1    & 8     & 0.45          & $\approx 0.1$        \\
		2    & 8 + 4$\times$24  & 6.37          & 13.5       \\
		3    & 8$\times$24  & 6.37          & 50     \\
		4    & 8$\times$24  & 6.37          & 100        \\ \midrule
		\bottomrule
		\end{tabular}
	\end{table}



\section{Results}


\subsection{Analysis}

The heavy-ion collimation quench test was carried out the 13.12.2015 from 17:00h and ended with the quench at 22:08~h (LHC fill numbers 4722 \& 4723). 

The filling scheme and the envisaged peak loss rates were slightly modified during the experiment, due to unforeseen LHC downtime which significantly reduced the time available for the test. After the set up of the ADT in the first fill, the machine was filled with 12 bunch trains of 24 bunches and 2 single bunches which were accelerated to 6.37\,$Z$\,TeV. 

At the first excitation of 6 bunch trains, the MBB.9L7 quenched at loss rate of approximately 
%
\begin{align}
P_q \approx (15 \pm 1) \, \text{kW}.
\end{align} 
The beam was subsequently dumped by the quench protection system~\cite{EPAC08:WEPD010} (QPS). The intensity and energy evolution throughout the experiment is shown in \figref{pic:16071301}. Furthermore, the evolution of the beam intensity, the power loss derived from it and the BLM signal at the quench location is shown in \figref{pic:16071101}. 


\begin{figure}[htbp]
  \centering
  \begin{tikzpicture}
    \footnotesize
    \node[anchor=south west,inner sep=0] (image) at (0,0) {\includegraphics[width=1.0\linewidth]{pictures/16071302.pdf}};
    \node [x={(image.south east)},y={(image.north west)}]                   at (0.50,1.0)    {Heavy Ion Collimation Quench Test - 13.12.2015};
    \node [draw,fill=white,rotate=0 ,x={(image.south east)},y={(image.north west)},align=center]            at (0.22,0.35)    {ADT setup \\ at injection energy};
    \node [draw,fill=white,rotate=0 ,x={(image.south east)},y={(image.north west)},align=center]            at (0.40,0.6)    {Filling with 12x24 \\ bunch trains};

    \node [draw,fill=white,rotate=0 ,x={(image.south east)},y={(image.north west)},align=center]            at (0.64,0.6)    {Ramp};

    \node [draw,fill=white,rotate=0 ,x={(image.south east)},y={(image.north west)},align=center]            at (0.8,0.8)    {Quench};
    %\node [draw,rotate=0 ,x={(image.south east)},y={(image.north west)},anchor=west]       at (0.22,0.80)    {text2};
    \draw[-,color=black,x={(image.south east)},y={(image.north west)}]             (0.43,0.54) -- (0.5,0.5);

    \draw[-,color=black,x={(image.south east)},y={(image.north west)}]             (0.64,0.56) -- (0.68,0.52);

    \draw[-,color=black,x={(image.south east)},y={(image.north west)}]             (0.8,0.77) -- (0.835,0.735);

  \end{tikzpicture}
  \caption{Intensity and particle energy evolution during the quench test.}  
  \label{pic:16071301}
  %/afs/cern.ch/work/p/phermes/private/160112_ion_quenchtest/plots/plots/md_overview_thesis.pdf
  \end{figure}



	% \begin{table}[htbp]
	% 	\centering
	% 	\caption{Realized fills during the MD. }
	% 	\label{tab:filling_schemereal}
	% 	\begin{tabular}{cccc}
        %           \toprule
	% 	Fill & Bunches & E {[}$Z$ TeV{]} & $P_{max}${[}kW{]} \\ \midrule
	% 	1    & $3 \times 24$     & 0.45          & $0.6$        \\
	% 	2    & 2 + 12$\times$24  & 6.37          & 15.0       \\
	% 	\bottomrule
	% 	\end{tabular}
	% \end{table}




\begin{figure}[t]
  \centering
  \begin{tikzpicture}
    \node[anchor=south west,inner sep=0] (image) at (0,0) {\includegraphics[width=1.0\linewidth]{pictures/16071102.pdf}};
    %\node [draw,rotate=90,x={(image.south east)},y={(image.north west)}]                   at (0.50,0.50)    {text0};
    %\node [draw,rotate=0 ,x={(image.south east)},y={(image.north west)}]                   at (0.22,0.96)    {text1};
    %\node [draw,rotate=0 ,x={(image.south east)},y={(image.north west)},anchor=west]       at (0.22,0.80)    {text2};
    %\draw[->,color=black,thick,x={(image.south east)},y={(image.north west)}]             (0.42,0.22) -- (0.37,0.23);
  \end{tikzpicture}
  \caption{Power load on the TCP, intensity evolution and measured BLM signal (BLMEI.09L7.B2I30, RS09) during the final beam excitation in which the quench was achieved at the main dipole MBB.9L7.}  
  \label{pic:16071101}
  %/afs/cern.ch/work/p/phermes/private/160112_ion_quenchtest/plots/plots/power_load_thesis.pdf
  \end{figure}



\newpage
The uncertainty on the peak power loss is estimated based upon the fluctuations visible in the increasing power load evolution. The loss rate increased continuously over approximately 14~s when the peak power load was achieved and the quench occurred. In lack of better estimates, the measurement result can be used to scale the permitted stored beam energy in the LHC on the basis of the minimum beam lifetime. 

Considering the design value of $\tau_\text{DS}=12\,\text{min}$ for the minimum beam life time, and taking into account that the quench occurred with a loss rate of $P_q$ at the TCP, the maximum stored beam energy in the studied configuration yields
%
\begin{align}
  E_s^\text{max,QT} (6.37\,Z\,\text{TeV}) = P_q \, \tau = (10.8 \pm 0.8) \, \text{MJ} \, . \label{qt:es}
\end{align}  
%
This value is very close to the stored beam energy achieved in the 2015 heavy-ion run and therefore imposes a serious limitation for the achievable luminosity in future operation. Note that the derived intensity limit is only valid for 6.37$\,Z\,$TeV, because the quench limit for higher magnetic fields is expected to be lower. In \chapref{chap:perflimi}, intensity limitations for 7$\,Z\,$TeV are derived on the basis of the HL-LHC cleaning simulations presented in \chapref{chap:hllhc}. %The findings gathered in this quench experiment were also used to estimate potential performance limitations in the 2016 heavy-ion run~\cite{hermespbprep}.

\newpage

An important outcome of the experiment are the measured BLM signals at quench with respect to the operational BLM thresholds. The measured data is going to be used to adjust the BLM thresholds for future operation. The BLM signals at the BLMEI.09L7.B2I30 at the moment of the quench are shown, normalized to the present operational BLM thresholds, in \figref{pic:16071403}. The highest signal is measured for RS10 with 350\% of the applied threshold. For RS09 and RS11, the thresholds were exceeded by 60\% and 110\% respectively~\cite{ACC-NOTE-16-0031}. 


\begin{figure}[t]
  \centering
  \begin{tikzpicture}
    \footnotesize
    \node[anchor=south west,inner sep=0] (image) at (0,0) {\includegraphics[width=0.7\linewidth]{pictures/16071404.pdf}};
    \node [fill=white,x={(image.south east)},y={(image.north west)}]                   at (0.50,0.93)    {BLMEI.09L7.B2I30};
    %\node [draw,rotate=0 ,x={(image.south east)},y={(image.north west)}]                   at (0.22,0.96)    {text1};
    %\node [draw,rotate=0 ,x={(image.south east)},y={(image.north west)},anchor=west]       at (0.22,0.80)    {text2};
    %\draw[->,color=black,thick,x={(image.south east)},y={(image.north west)}]             (0.42,0.22) -- (0.37,0.23);
  \end{tikzpicture}
  \caption{BLM Signals at quench for different running sums. The maximum exceed is obtained for RS10 where the signal was measured to be at 350\% of the operational threshold.}  
  \label{pic:16071403}
  %/afs/cern.ch/work/p/phermes/private/160112_ion_quenchtest/plots/plots/blm_ratio.pdf
  \end{figure}



From the measured cleaning inefficiency at the MBB.9L7 and the known peak power loss at the TCP, a rough estimate of the peak power deposited in the latter can be extracted as:
%
\begin{align}
  P_\text{MBB} = \eta_Q \, P_q = (330 \pm 30)\,\text{W}. 
\end{align}
%
This value should be regarded with caution, knowing that the energy deposited is related to the BLM signals via a response function which is unknown. Better estimates for the power deposited can be obtained with a detailed shower deposition study carried out with FLUKA. The basis for this is a cleaning simulation with the hiSixTrack-FLUKA coupling in which the \texttt{toucMap} input is generated. This simulation is presented in the next section.


\newpage

\subsection{Simulations}
Further information from the experiment can be gathered if the cleaning performance in the quench test configuration is simulated with the hiSixTrack-FLUKA coupling. The simulated cleaning performance can be combined with the measured quench limit to extrapolate the maximum stored beam energy also for the LHC configurations studied in \chapref{chap:hllhc} and \chapref{chap:xear}. Furthermore, the simulation output can be used to perform a subsequent FLUKA shower propagation and energy deposition study to derive the power deposition in the superconducting magnet coils.

\subsubsection{Cleaning Simulation with the hiSixTrack-FLUKA Coupling}

\textit{Simulation Setup and Referene Loss Maps}
\vspace{0.2cm}

The cleaning simulation with the hiSixTrack-FLUKA coupling is carried out in the B2H plane with 6.0$\times 10^6$ initial \lead ions at an energy of 6.37$\,Z\,$TeV, starting from IP1. The simulated optics and collimator settings are identical to the settings applied in the measurement, listed in \chapref{tab:sets_qt}. The impact parameter at the TCP.C6R7.B2 was set to values between 0.5~$\mu$m and 2.0~$\mu$m on both collimator jaws. The simulated loss pattern is compared to the loss map measured during the quench test in \figref{pic:16071401} (full ring) and \figref{pic:16071402} (zoom to IR7). 
\vspace{0.2cm}

For comparison, also two measured B2H qualification loss maps are shown. One of them is the loss map measured at flat top (FT), already used for the quench test preparation in \chapref{chap:targetlr}. The second loss map was measured at the end of squeeze (EoS). It was chosen for the comparison, because the background signal is significantly lower. From IR7 to IR5, the optics is identical, so the measured loss maps should be comparable in between.

% hence both should be comparable to the quench test loss map. 
 \mbox{} \\
\textit{Global Loss Distribution}
\vspace{0.2cm}

The loss map simulated with the hiSixTrack-FLUKA coupling shows a good overall agreement with the measured loss maps. As explained in \chapref{firstICOS}, the simulated losses in the warm magnets of IR7 are not comparable to the measured BLM signals. The dominating losses in IR7, IR6 and IR3 are well modeled in the simulation and also the loss peak C5 (see \figref{pic:16071401}) in the cold region downstream of IR5 is accurately predicted. 
\vspace{0.2cm}

Remarkable differences are visible in the arcs between IR5 and IR6 (called C56) as well as between IR6 and IR7 (C67). The C56 loss peaks are only visible in the loss map measured during the quench test, while they are neither simulated nor measured in either of the qualification loss maps. In the EoS loss map, their amplitude should be above the noise level. 





\begin{figure}[htbp]
  \centering
  \begin{tikzpicture}
    \footnotesize
    \node[anchor=south west,inner sep=0] (image) at (0,0) {\includegraphics[width=1.0\linewidth]{pictures/16091103.pdf}};
    \node [fill=white,x={(image.south east)},y={(image.north west)},align=left]                   at (0.22,0.97)    {hiSixTrack-FLUKA coupling};
    \node [fill=white,x={(image.south east)},y={(image.north west)},align=left]                   at (0.19,0.735)    {Quench test loss map};
    \node [fill=white,x={(image.south east)},y={(image.north west)},align=left]                   at (0.26,0.50)    {Qualification loss map end of squeeze };
    \node [fill=white,x={(image.south east)},y={(image.north west)},align=left]                   at (0.18,0.465)    {24/11/2015 10:34:42};

    \node [fill=white,x={(image.south east)},y={(image.north west)},align=left]                   at (0.225,0.27)    {Qualification loss map flat top };
    \node [fill=white,x={(image.south east)},y={(image.north west)},align=left]                   at (0.18,0.235)    {23/11/2015 23:32:08};
    \footnotesize

    \node [fill=white,x={(image.south east)},y={(image.north west)}]                   at (0.525,0.72)    {C5};
    \node [fill=white,x={(image.south east)},y={(image.north west)}]                   at (0.58,0.67)    {C56};
    \node [fill=white,x={(image.south east)},y={(image.north west)}]                   at (0.70,0.67)    {C67};

    \node [fill=white,x={(image.south east)},y={(image.north west)},align=center]      at (0.9,0.72)    {Beam \\ direction};
    \draw[->,color=black,thick,x={(image.south east)},y={(image.north west)}]             (0.93,0.675) -- (0.87,0.675);
  \end{tikzpicture}
  \caption{Measured and simulated B2H loss maps for the B2H collimation quench test.}  
  \label{pic:16071401}
  %/home/phermes/Dropbox/PhD/pictures/160714_quench_test/comparison_qt.pdf
  \end{figure}



\begin{figure}[t]
  \centering
  \begin{tikzpicture}
    \footnotesize
    \node[anchor=south west,inner sep=0] (image) at (0,0) {\includegraphics[width=0.98\linewidth]{pictures/16091104.pdf}};
    \node [x={(image.south east)},y={(image.north west)}]                   at (0.29,0.685)    {DS1};
    \node [x={(image.south east)},y={(image.north west)}]                   at (0.17,0.685)    {DS2};
    \node [fill=white,x={(image.south east)},y={(image.north west)}]                   at (0.6,0.93)    {hiSixTrack-FLUKA coupling};
    \node [fill=white,x={(image.south east)},y={(image.north west)}]                   at (0.6,0.705)    {Quench test loss map};

    \node [fill=white,x={(image.south east)},y={(image.north west)},align=left]                   at (0.37,0.48)    {Qualification loss map end of squeeze - 24/11/2015 10:34:42};
%    \node [fill=white,x={(image.south east)},y={(image.north west)},align=left]                   at (0.18,0.465)    {24/11/2015 10:34:42};

    \node [fill=white,x={(image.south east)},y={(image.north west)},align=left]                   at (0.34,0.25)    {Qualification loss map flat top - 23/11/2015 23:32:08 };

    \node [fill=white,x={(image.south east)},y={(image.north west)},align=center]      at (0.35,0.92)    {Beam \\ direction};
    \draw[<-,color=black,thick,x={(image.south east)},y={(image.north west)}]             (0.32,0.88) -- (0.37,0.88);

%    \node [fill=white,x={(image.south east)},y={(image.north west)},align=left]                   at (0.18,0.235)    {23/11/2015 23:32:08};
%    \node [x={(image.south east)},y={(image.north west)}]                   at (0.6,0.315)    {Qualification loss map};
    %\node [draw,rotate=0 ,x={(image.south east)},y={(image.north west)}]                   at (0.22,0.96)    {text1};
    %\node [draw,rotate=0 ,x={(image.south east)},y={(image.north west)},anchor=west]       at (0.22,0.80)    {text2};
    %\draw[->,color=black,thick,x={(image.south east)},y={(image.north west)}]             (0.42,0.22) -- (0.37,0.23);
  \end{tikzpicture}
  \caption{Top panel: loss map in the 2015 heavy-ion quench test configuration simulated with the hiSixTrack-FLUKA coupling. Second panel: loss map measured at the moment of quench. Third panel: qualification loss map at the end of squeeze. Bottom panel: qualification loss map measured at flat top.}  
  \label{pic:16071402}
  %/home/phermes/Dropbox/PhD/pictures/160714_quench_test/comparison_qt_IR7.pdf
  \end{figure}


% \begin{figure}[h]
%   \centering
%   \begin{tikzpicture}
%     \node[anchor=south west,inner sep=0] (image) at (0,0) {\includegraphics[width=1.0\linewidth]{pictures/16071303.pdf}};
%     %\node [draw,rotate=90,x={(image.south east)},y={(image.north west)}]                   at (0.50,0.50)    {text0};
%     %\node [draw,rotate=0 ,x={(image.south east)},y={(image.north west)}]                   at (0.22,0.96)    {text1};
%     %\node [draw,rotate=0 ,x={(image.south east)},y={(image.north west)},anchor=west]       at (0.22,0.80)    {text2};
%     %\draw[->,color=black,thick,x={(image.south east)},y={(image.north west)}]             (0.42,0.22) -- (0.37,0.23);
%   \end{tikzpicture}
%   \caption{Figure caption}  
%   \label{pic:16071303}
%   %/home/phermes/Dropbox/Talks/160303_quenchtest_simulation/pictures/16030413.pdf
%   \end{figure}

\newpage
Some of the pronounced C67 loss peaks are simulated in the hiSixTrack-FLUKA coupling but not measured in the qualification loss maps. From the zoom to IR7 shown in \figref{pic:16071402}, it is visible that the measured losses in the TCLA collimators in IR7 are different between the EoS qualification loss map and the FT and quench test loss map. In the EoS loss map, the TCLA losses at some absorbers are lower by one order of magnitude. 

These discrepancies are not understood and subject to further investigation. One possible reason could be different reference orbits between the measurements. 

\mbox{} \\
\textit{Losses in the DS1 and DS2 cluster}
\vspace{0.2cm}

In the measured loss map at quench, the highest loss signal in the cold regions is occurs at the MBB.9L7 with a loss signal of $\mathcal{B}_\text{MBB9L7}^\text{QT} = (2.2 \pm 0.4) \times 10^{-2}$. The uncertainty is derived from the BLM signal variations observed during the excitation at which the quench occurred.
\vspace{0.2cm}

In the qualification loss map at flat top, the highest loss signal was measured at the MQY in the same cell, with a slightly lower loss signal of $\mathcal{B}_\text{MQ9L7}^\text{lm, FT} = (1.6 \pm 0.4) \times 10^{-2}$. For the EoS loss map, the measured signal yields $\mathcal{B}_\text{MQ9L7}^\text{lm, EoS} = (1.2 \pm 0.4) \times 10^{-2}$. This discrepancy is not considered important because the impact parameter and hence the fragment spectrum might be different in the measured loss maps. 
\vspace{0.2cm}

% \begin{figure}[h]
%   \centering
%   \begin{tikzpicture}
%     \node[anchor=south west,inner sep=0] (image) at (0,0) {\includegraphics[width=1.0\linewidth]{pictures/16071304.pdf}};
%     %\node [draw,rotate=90,x={(image.south east)},y={(image.north west)}]                   at (0.50,0.50)    {text0};
%     %\node [draw,rotate=0 ,x={(image.south east)},y={(image.north west)}]                   at (0.22,0.96)    {text1};
%     %\node [draw,rotate=0 ,x={(image.south east)},y={(image.north west)},anchor=west]       at (0.22,0.80)    {text2};
%     %\draw[->,color=black,thick,x={(image.south east)},y={(image.north west)}]             (0.42,0.22) -- (0.37,0.23);
%   \end{tikzpicture}
%   \caption{Figure caption}  
%   \label{pic:16071304}
%   %/home/phermes/Dropbox/Talks/160303_quenchtest_simulation/pictures/16030416.pdf
%   \end{figure}




In the simulated loss map, the highest loss peak in the cold regions is located at the MBB.9L7, in line with the measurement and the quench location. At this location, the cleaning inefficiency simulated with the hiSixTrack-FLUKA coupling is 
\begin{align}
\eta_\text{MBB9L7} = 2.4 \times 10^{-3}\, \text{m}^{-1} \, .
\end{align}

In order to better understand the energy deposited in the magnet coils, a detailed shower propagation study is carried out with FLUKA.

% The loss composition in the DS1 and DS2 is listed in \tabref{tab:quench:losscompo}. The losses in the DS1 are composed of many different isotopes. In the DS2, more than one third of the losses are induced by $^{205}$Pb$^{82+}$. In both clusters, the abundance of 


% \begin{table}[h]
%   \centering
%   \caption{Isotope composition (weighted with ion momentum) in the DS clusters in the quench test simulation.}
%   \label{tab:quench:losscompo}
%   \begin{tabular}{cccc}
% 		\toprule
%                 \multicolumn{2}{c}{DS1} & \multicolumn{2}{c}{DS2} \\ \midrule
% 		Isotope       & Fraction{[}\%{]} & Isotope  & Fraction{[}\%{]} \\ \midrule
% 		\iso{203}{Pb}{82+}       & 7.7              & \iso{205}{Pb}{82+}       & 36.8 \\
% 		\iso{202}{Pb}{82+}       & 6.9              & \iso{206}{Pb}{82+}       & 21.3 \\
%                 \iso{201}{Tl}{81+}       & 5.3              & \iso{203}{Tl}{81+}       & 17.3 \\
% 		\iso{200}{Tl}{81+}       & 4.5              & \iso{200}{Hg}{80+}       & 6.8  \\
% 		\iso{199}{Tl}{81+}       & 4.1              & \iso{202}{Tl}{81+}       & 6.6  \\ \bottomrule
%   \end{tabular}
% \end{table}






\subsubsection{Energy Deposition Simulation with FLUKA}

Based on the \texttt{toucMap} output from the hiSixTrack-FLUKA coupling, subsequent FLUKA simulations can be performed to study the energy deposited in the superconducting magnet coils and hence derive the quench limit in terms of the MQPD~\cite{IPAC15:TUPTY046}. 
\vspace{0.2cm}

A detailed model of the IR7 geometry is used to simulate the interaction of the particles with the TCP material and the subsequent propagation of residual fragments and electromagnetic and hadronic showers. The propagation of residual particles and their interaction with collimators, surrounding beam pipes and other machine hardware is included in the simulation. This allows to simulate the BLM signals at the collimators and in the DS. The most important outcome of the simulation is to estimate the energy deposited in the magnet coils at the time of quench. 
\vspace{0.2cm}

These simulations are carried out by the CERN FLUKA team and are very demanding in terms of computing time and post-processing of the data. At the time of writing, the energy deposition simulations by FLUKA are still ongoing.



\newpage

\section{Estimates on Performance Limitations} \label{chap:perflimi}

The experimental input gathered from the heavy-ion collimation quench test can be combined with the results of the cleaning simulations presented in \chapref{chap:hllhc}, to estimate the maximum allowed stored beam energy for the different configurations of HL-LHC. This analysis will determine whether the target luminosity is achievable and if the upgrade with the TCLD collimators is required for future heavy-ion operation. The extrapolation of the stored beam energy and luminosity is also presented for the prospective operation with \iso{129}{Xe}{54+} or \iso{40}{Ar}{18+}. 

\subsection{Strategy}

\subsubsection{Maximum Stored Beam Energy}
The baseline for the estimate of the achievable stored beam energy is the maximum stored beam energy at 6.37$\,Z\,$TeV derived from the heavy-ion collimation quench test. Following \eqref{qt:es}, the maximum stored beam energy at 6.37$\,Z\,$TeV is $(10.8 \pm 0.8)\,$MJ. The re-scaling to a beam energy of $7\,Z\,$TeV requires taking into account the reduced quench limit with the higher magnet current. As introduced in \chapref{chap:quenchlim}, the quench limit for steady state losses is quantified by the minimum quench power density (MQPD). The most recent estimates for the MQPD scaling from 6.5$\,Z\,$TeV to 7.0$\,Z\,$TeV are summarized in \tabref{tab:ql:est}. It is assumed that the MQPD is approximately 20\% lower at 6.5$\,Z\,$TeV than at $6.37\,Z\,$TeV~\cite{vervej:private}. The MQPD scaling factors are derived from simulations and should be considered tentative and uncertain~\cite{Verweij:col}. 
\vspace{0.2cm}

To estimate the energy deposited in the IR7 DS magnets in every simulated configuration, the maximum average cleaning inefficiency $\eta^{avg}_{max}$ is compared to $\eta^{avg,QT}_{max}$ from the cleaning simulation in the quench test configuration. The same impact parameter should be applied in all cases to make the simulated cleaning performances comparable. The simulations for HL-LHC were conducted with $b=1\,\mu$m, because the largest cleaning inefficiency in the DS2 was simulated with this impact parameter. The loss map simulated for the quench test scenario with an impact parameter of $b=1\,\mu$m is shown in \figref{pic:16090907}. 
\vspace{0.2cm}

The maximum allowed stored beam energy at 6.37$\,Z\,$TeV was derived from the quench test result using the design specification for the minimum beam lifetime, which is $\tau_\text{DR}=\,$12\,min. So far, the minimum beam lifetime $\tau$ in operation at top energy with heavy-ion beams dropped rarely to values close to 12\,min. Therefore, the extrapolation of the permitted stored beam energies for HL-LHC is also carried out for beam lifetimes up to $\tau=\,$60\,min. This requires re-scaling the maximum stored beam energy with the ratio $\tau/\tau_\text{DR}$. Such a comparison allows specifying a minimum beam lifetime required to reach the HL-LHC target.

\begin{table}[htbp]
\centering
\caption{Estimated quench limit scaling from 6.37$\,Z\,$TeV and 6.5$\,Z\,$TeV to 7.0$\,Z\,$TeV for the LHC main dipoles~\cite{vervej:private}.  Only the longest loss durations are shown, considering that collimation losses are assumed to be steady state losses. It should be kept in mind that these numbers have significant uncertainties.}
\label{tab:ql:est}

\begin{tabular}{lcc}
\toprule
Loss Duration [s]  &  $\frac{\text{MQPD}_{7.0}}{\text{MQPD}_{6.5}}$ & $\frac{\text{MQPD}_{7.0}}{\text{MQPD}_{6.37}}$    \\ \midrule
% 0.00001 & 1.46 \\
% 0.00004 & 1.46 \\
% 0.00008 & 1.46 \\
% 0.00016 & 1.46 \\
% 0.00032 & 1.48 \\
% 0.00064 & 1.48 \\
% 0.001   & 1.42 \\
% 0.00256 & 1.24 \\
% 0.005   & 1.26 \\
% 0.01024 & 1.28 \\
% 0.025   & 1.31 \\
% 0.08192 & 1.31 \\
0.25    & 0.74 & 0.59 \\
0.65536 & 0.71 & 0.57 \\
1.31072 & 0.68 & 0.54 \\
5.24288 & 0.60 & 0.48 \\
20.9715 & 0.55 & 0.44 \\
83.8861 & 0.56 & 0.45 \\ \bottomrule
\end{tabular}
\end{table}



\begin{figure}[htbp]
  \centering
  \begin{tikzpicture}
    \footnotesize
    \node[anchor=south west,inner sep=0] (image) at (0,0) {\includegraphics[width=1.0\linewidth]{pictures/16090907.pdf}};
    \node [fill=white,x={(image.south east)},y={(image.north west)},align=center]      at (0.3,0.80)    {Beam \\ direction};
    \draw[<-,color=black,thick,x={(image.south east)},y={(image.north west)}]             (0.265,0.72) -- (0.34,0.72);

  \end{tikzpicture}
  \caption{B2H cleaning inefficiency in IR7 simulated with the hiSixTrack-FLUKA coupling in the configuration of the heavy-ion collimation quench test at 6.37$\,Z\,$TeV with an impact parameter of $1\,\mu$m. }  
  \label{pic:16090907}
  %/home/phermes/Dropbox/PhD/pictures/160714_quench_test/refsimu.pdf
  \end{figure}


\begin{table}[htbp]
\centering
\caption{Maximum average cleaning inefficiencies in the IR7 DS clusters simulated with the hiSixTrack-FLUKA coupling for different LHC configurations. All simulations assume an impact parameter of 1\,$\mu$m at the TCP.}
\label{tab:etaavg2}
\begin{tabular}{lc cccc}
\toprule
Configuration& Ion &  TCLDs & $N_D$ {[}$\sigma${]} & $\eta^{avg}_{DS1}$ {[}m$^{-1}${]} &  $\eta^{avg}_{DS2}$ {[}m$^{-1}${]} \\ \midrule
2015 Quench Test  & \lead &  0  & -                      & 4.7 $\times$ 10$^{-3}$                       & 2.5 $\times$ $10^{-3}$                       \\ \midrule
HL-LHC   & \lead &  0  & -                               & 4.0 $\times$ 10$^{-3}$                       & 4.7 $\times$ $10^{-3}$                       \\
HL-LHC       & \lead &  1  & 15                          & 1.0 $\times$ 10$^{-3}$                       & 4.2 $\times$ 10$^{-3}$                       \\
HL-LHC       & \lead & 1  & 12                          & 3.6 $\times$ 10$^{-5}$                       & 2.7 $\times$ 10$^{-3}$                       \\
HL-LHC       & \lead & 1  & 10                          & 1.5 $\times$ 10$^{-5}$                       & 2.6 $\times$ 10$^{-3}$                       \\
HL-LHC      & \lead & 2  & 15                           & 1.0 $\times$ 10$^{-3}$                       & 1.4 $\times$ 10$^{-4}$                       \\
HL-LHC      & \lead & 2  & 12                           & 3.6 $\times$ 10$^{-5}$                       & 1.5 $\times$ 10$^{-6}$                       \\
HL-LHC      & \lead & 2  & 10                           & 1.5 $\times$ 10$^{-5}$                       & 2.5 $\times$ 10$^{-6}$                       \\ \midrule
HL-LHC           & \iso{129}{Xe}{54+}  & 0  & -                           & 4.8 $\times$ 10$^{-3}$                       & 4.3 $\times$ 10$^{-3}$                       \\
HL-LHC           & \iso{40}{Ar}{18+}  & 0  & -                           & 4.6 $\times$ 10$^{-3}$                       & 1.9 $\times$ 10$^{-3}$                       \\ \bottomrule
\end{tabular}
\end{table}


\newpage
The maximum permitted stored beam energy $E_s^\text{max,HL}$ in the studied scenario is then derived from the quench limit scaling, the ratio of maximum average cleaning inefficiency and the assumed beam lifetime by the following relation
%
%
\begin{align}
  E_s^\text{max,HL} = E_s^\text{max,QT} \times \frac{\tau}{\tau_\text{DR}} \times\frac{\text{MQPD}_{7.0}}{\text{MQPD}_{6.37}} \times \frac{\eta^{avg,QT}_{max}}{\eta^{avg}_{max}} \, . \label{eq:esmaxex}
\end{align}

The quantity $E_s^\text{max,QT}$ is the maximum stored beam energy at 6.37$\,Z\,$TeV derived from the heavy-ion collimation quench test (see \eqref{qt:es}). 
\vspace{0.2cm}

Given the rather large uncertainty of the quench limit estimate and the known limitations from the comparison of simulated cleaning inefficiencies, the estimates derived with this formula should be regarded as approximate. A better approach would be the simulation of energy deposition in the magnet, starting from the output generated with the hiSixTrack-FLUKA coupling. This type of simulation is envisaged for the future. 
\vspace{0.2cm}

For the MQPD re-scaling, the most conservative scaling factor of 0.44 is chosen (loss duration 20.9715\,s). As a comparison, the allowed stored beam energy is also estimated without accounting for the quench limit reduction. This value represents an upper limit. 

\newpage
Especially for the simulations with 2 TCLDs per beam, the extrapolation will deliver stored beam energies significantly above the HL-LHC target value of 24.1\,MJ for \lead beams. At the time of writing,  stored beam energies beyond the HL-LHC target are not reachable with the envisaged injector upgrades. The values for simulations with 2 TCLDs should rather be interpreted as a margin compared to the HL-LHC target, from perspective of the quench risk by collimation losses. But even if the injector performance should turn out to be better in the future, additional constraints arise from the collimation system, e.g. by the maximum power load the primary collimators can withstand without being damaged. 
%\vspace{0.1cm}

\subsubsection{Achievable Luminosity}

As shown in \eqref{eq:lumi}, the maximum stored beam energy cannot be directly translated into the luminosity. Assumptions have to be made about the bunch intensity and the number of bunches in the machine. The luminosity is proportional to the square of the bunch intensity, and linearly proportional to the number of bunches. The maximum luminosity can hence be reached if the stored beam energy is distributed over the fewest possible number of bunches. 
\vspace{0.1cm}

The baseline for the luminosity calculations presented below is the most recent estimate for the HL-LHC bunch intensity yielding $2.1\times 10^8$ ions/bunch for \lead ions~\cite{jowett:private}, \mbox{$6\times 10^8$ ions/bunch} for \iso{129}{Xe}{54+} ions and $5 \times 10^9$ for \iso{40}{Ar}{18+}  ions~\cite{IPAC16:TUPMR027}. Note that for Ar and Xe, the bunch intensity corresponds to the value that is achievable at the extraction of the SPS. The assumed number of bunches is derived from the maximum stored beam energy and the bunch intensity, although no filling scheme is yet existing for the different cases. 
\vspace{0.1cm}



All luminosities derived are calculated for IP2, assuming that all bunches are colliding. The crossing angle correction factor is not taken into account ($F=1$), so all luminosities should be regarded as upper boundaries. Furthermore, the assumed beam parameters are $\beta^* = 0.5\,$m, a normalized emittance of $\epsilon_N = 1.3\, \mu$m rad~\cite{jowett:private} and the nominal revolution frequency of 11.245\,kHz~\citedr.  The assumptions are made to provide the best comparability between the different study cases. 
%
\subsection{Results}
%
%
The maximum average cleaning inefficiencies in the two DS clusters derived from the different simulations are listed in \tabref{tab:etaavg2}. Based on these values, the maximum stored beam energy is derived with and without MQPD scaling by means of \eqref{eq:esmaxex}. The maximum stored beam energy is used to derive the achievable luminosity in IP2, as it was described above. The results of the study assuming a minimum beam lifetime of 12\,min are listed in \tabref{tab:esmax}. The results are also graphically represented for different lifetimes in \figref{pic:energy:nominal}, \figref{fig:escompa} and  \figref{pic:arxe:lumi}.

\newpage

\textit{Nominal HL-LHC without TCLD} \\ 
The allowed stored beam energy for the HL-LHC configuration without TCLDs and considering a minimum beam lifetime of 12\,min is extrapolated to be $(4.8\pm0.4)$\,MJ with MQPD scaling and $(10.8\pm0.8)\,$MJ without MQPD scaling. The nucleon-nucleon luminosity reachable with scaled MQPD yields \mbox{$(1.9 \pm 0.3)\times 10^{32}$\,cm$^{-2}$\,s$^{-1}$}. In the 2015 heavy-ion run, the peak nucleon-nucleon luminosity was approximately \mbox{$1.6 \times 10^{32}$\,cm$^{-2}$\,s$^{-1}$}. The HL-LHC target nucleon-nucleon luminosity derived with the same assumptions on emittance, $\beta^*$ and crossing angle is \mbox{$9.7 \times 10^{32}$\,cm$^{-2}$\,s$^{-1}$}. The illustration in \figref{pic:energy:nominal} shows also that the target value could not be reached, even if the minimum beam lifetime was 60\,min.




\begin{table}[t]
\centering
% \caption{Extrapolated maximum stored beam energy and achievable peak nucleon-nucleon luminosity in IP2 for different HL-LHC configurations. The MQPD scaling factor from 6.37$\,Z\,$TeV to 7.0$\,Z\,$TeV is assumed to be 0.44. The luminosity in IP2 is derived assuming a normalized emittance of $\epsilon_N=1.3\times 10^{-6}\,\mu$m rad, and a bunch intensity of $2.1\times 10^{8}$ ions/bunch for Pb, $5.0 \times 10^9$ ions/bunch for Ar and $6.0 \times 10^8$ ions/bunch for Xe. It is assumed that $\beta^*=0.5\,$m in ALICE and a crossing correction factor is not taken into account ($F=1$). All values presented consider a minimum beam lifetime of 12\,min. For operation after the LIU upgrade, a stored \lead beam energy of 18.0\,MJ and for HL-LHC a value of 24.1\,MJ are envisaged. Note that the stored beam energy and luminosity listed for 2 TCLDs at 12\,$\sigma$ and 15$\,\sigma$ are not reachable. }

\caption{Extrapolated maximum stored beam energy and achievable peak nucleon-nucleon luminosity in IP2 for different HL-LHC configurations with and without MQPD scaling. A minimum beam lifetime of 12\,min is assumed.% The MQPD scaling factor from 6.37$\,Z\,$TeV to 7.0$\,Z\,$TeV is assumed to be 0.44. The luminosity in IP2 is derived assuming a normalized emittance of $\epsilon_N=1.3\times 10^{-6}\,\mu$m rad, and a bunch intensity of $2.1\times 10^{8}$ ions/bunch for Pb, $5.0 \times 10^9$ ions/bunch for Ar and $6.0 \times 10^8$ ions/bunch for Xe. It is assumed that $\beta^*=0.5\,$m in ALICE and a crossing correction factor is not taken into account ($F=1$). All values presented consider a minimum beam lifetime of 12\,min. For operation after the LIU upgrade, a stored \lead beam energy of 18.0\,MJ and for HL-LHC a value of 24.1\,MJ are envisaged.
  The values derived for the case of 2 TCLDS with settings of 10$\,\sigma$ and 12$\,\sigma$, are not reachable in the LHC, because the required number of bunches can not be injected in the LHC. They are written in brackets and are only shown for comparison.}
\label{tab:esmax}
\footnotesize
\begin{tabular}{lc|ccc|ccc}
       \multicolumn{2}{c}{} & \multicolumn{3}{c}{Without MQPD scaling} & \multicolumn{3}{c}{With MQPD scaling} \\  \toprule

            & $N_D$                & $E_s^\text{max,HL}$       & $n_B$ & $\mathcal{L}_{\text{NN, IP2}}^\text{max}$                      & $E_s^\text{max,HL}$   & $n_B$        & $\mathcal{L}_{\text{NN, IP2}}^\text{max}$                           \\ 
            & {[}$\sigma${]}       & {[}MJ{]}                 &      &          [cm$^{-2}$ s$^{-1}$] &                     {[}MJ{]} &         &                               [cm$^{-2}$ s$^{-1}$]     \\ \midrule

\textit{HL Target}  & -                     & \textit{24.1}            & \textit{1248}   & \textit{9.7}$\mathit{\times 10^{32}}$  &  \textit{24.1}   & \textit{1248}   &  \textit{9.7}$\mathit{\times 10^{32}}$   \\ \midrule \midrule


No TCLD  & -                     & 10.8$\pm$0.8            & 560   & \err{4.3}{0.5}$\times 10^{32}$  &  4.8 $\pm$0.4   & 248   &  \err{1.9}{0.3}$\times 10^{32}$   \\ \midrule 
1 TCLD   & 15                    & 12.1$\pm$0.9            & 626   & \err{4.8}{0.9}$\times 10^{32}$  &  5.3$\pm$0.4    & 274   &  \err{2.2}{0.2}$\times 10^{32}$   \\ 
1 TCLD   & 12                    & 18.8$\pm$1.4            & 974   & \err{7.8}{0.9}$\times 10^{32}$  &  8.3$\pm$0.7    & 430   &  \err{3.3}{0.4}$\times 10^{32}$   \\ 
1 TCLD   & 10                    & 19.5$\pm$1.5            & 1010  & \err{7.8}{0.9}$\times 10^{32}$  &  8.7$\pm$0.7    & 450   &  \err{3.5}{0.4}$\times 10^{32}$   \\ \midrule
2 TCLD   & 15                    & 50.8$\pm$3.8            & 2630  & \err{2.0}{0.2}$\times 10^{33}$  &  22.4$\pm$1.7   & 1160  &  \err{9.1}{0.9}$\times 10^{32}$   \\ 
2 TCLD   & 12                    & [1410$\pm$105]            & n.r.  & [\err{5.7}{0.5}$\times 10^{34}$]  &  [623$\pm$47]     & n.r.  &  [\err{2.5}{0.2}$\times 10^{34}$]   \\ 
2 TCLD   & 10                    & [3384$\pm$251]            & n.r.  & [\err{1.4}{0.2}$\times 10^{35}$]  &  [1596$\pm$111]   & n.r.  &  [\err{6.4}{0.5}$\times 10^{34}$]   \\ \midrule
Xe       & -                     & 10.6$\pm$0.8            & 292   & \err{7.7}{0.7}$\times 10^{32}$  &  4.7$\pm$0.4    & 130   &  \err{3.3}{0.4}$\times 10^{32}$   \\ 
Ar       & -                     & 11.0$\pm$0.9            & 110   & \err{2.1}{0.2}$\times 10^{33}$  &  4.9$\pm$0.4    & 48    &  \err{9.1}{0.8}$\times 10^{32}$   \\ \bottomrule

\end{tabular} 
\end{table}






\begin{figure}[ht] 
\centering
\begin{minipage}[b]{.49\textwidth}
  \begin{tikzpicture}
    \footnotesize
    \node[anchor=south west,inner sep=0] (image) at (0,0) {\includegraphics[width=1.0\linewidth]{pictures/16091507.pdf}};
   \node [draw,fill=white,rotate=0 ,x={(image.south east)},y={(image.north west)}]         at (0.29,0.82)    {Nominal HL-LHC};
    \node [fill=white,rotate=0,x={(image.south east)},y={(image.north west)},align=center]         at (0.50,0.7)    {LIU \\ Baseline};
    \draw[-,color=black,thick,x={(image.south east)},y={(image.north west)}]       (0.57,0.70) -- (0.62,0.69);
    \draw[-,color=black,thick,x={(image.south east)},y={(image.north west)}]       (0.72,0.2) -- (0.78,0.2);
    \node [fill=white,rotate=0,x={(image.south east)},y={(image.north west)},align=center]         at (0.85,0.2)    {HL-LHC \\ Request};
  \end{tikzpicture}
\end{minipage}%
\caption{Maximum stored beam energy and achievable luminosity for HL-LHC with \lead ions without TCLD collimators.  The lower limits are derived from the rescaling including the reduced MQPD. The upper limit is only re-scaled with the cleaning inefficiency. The vertical black lines show the target values for operation after the LIU upgrade and the present optimistic baseline for HL-LHC.  The gray area shows stored beam energies which are not reachable with the presently envisaged injector upgrades.}
\label{pic:energy:nominal}
\end{figure}







\begin{figure}[htbp] 
\begin{minipage}[b]{.49\textwidth}
  \begin{tikzpicture}
    \footnotesize
    \node[anchor=south west,inner sep=0] (image) at (0,0) {\includegraphics[width=1.0\linewidth]{pictures/16091503.pdf}};
   \node [draw,fill=white,rotate=0 ,x={(image.south east)},y={(image.north west)}]         at (0.29,0.82)    {1 TCLD at $15\, \sigma$};
    \node [fill=white,rotate=0,x={(image.south east)},y={(image.north west)},align=center]         at (0.50,0.7)    {LIU \\ Baseline};
    \draw[-,color=black,thick,x={(image.south east)},y={(image.north west)}]       (0.57,0.70) -- (0.62,0.65);
    \draw[-,color=black,thick,x={(image.south east)},y={(image.north west)}]       (0.72,0.2) -- (0.78,0.2);
    \node [fill=white,rotate=0,x={(image.south east)},y={(image.north west)},align=center]         at (0.85,0.2)    {HL-LHC \\ Request};
  \end{tikzpicture}
\end{minipage}%
\hfill
\begin{minipage}[b]{.49\linewidth}
  \begin{tikzpicture}
    \footnotesize
    \node[anchor=south west,inner sep=0] (image) at (0,0) {\includegraphics[width=1.0\linewidth]{pictures/16091506.pdf}};
   \node [draw,fill=white,rotate=0 ,x={(image.south east)},y={(image.north west)}]         at (0.8,0.82)    {2 TCLD at $15\, \sigma$};
  \end{tikzpicture}
\end{minipage}
%
\\ \mbox{} \\ 
%
\begin{minipage}[b]{.49\textwidth}
  \begin{tikzpicture}
    \footnotesize
    \node[anchor=south west,inner sep=0] (image) at (0,0) {\includegraphics[width=1.0\linewidth]{pictures/16091502.pdf}};
   \node [draw,fill=white,rotate=0 ,x={(image.south east)},y={(image.north west)}]         at (0.29,0.82)    {1 TCLD at $12\, \sigma$};
  \end{tikzpicture}
\end{minipage}%
\hfill
\begin{minipage}[b]{.49\linewidth}
  \begin{tikzpicture}
    \footnotesize
    \node[anchor=south west,inner sep=0] (image) at (0,0) {\includegraphics[width=1.0\linewidth]{pictures/16091505.pdf}};
   \node [draw,fill=white,rotate=0 ,x={(image.south east)},y={(image.north west)}]         at (0.52,0.82)    {2 TCLD at $12\, \sigma$};
  \end{tikzpicture}
\end{minipage}
%
%\vspace{0.5cm}
\\ \mbox{} \\ 
%
\begin{minipage}[b]{.49\textwidth}
  \begin{tikzpicture}
    \footnotesize
    \node[anchor=south west,inner sep=0] (image) at (0,0) {\includegraphics[width=1.0\linewidth]{pictures/16091501.pdf}};
   \node [draw,fill=white,rotate=0 ,x={(image.south east)},y={(image.north west)}]         at (0.29,0.82)    {1 TCLD at $10\, \sigma$};
  \end{tikzpicture}
\end{minipage}%
\hfill
\begin{minipage}[b]{.49\linewidth}
  \begin{tikzpicture}
    \footnotesize
    \node[anchor=south west,inner sep=0] (image) at (0,0) {\includegraphics[width=1.0\linewidth]{pictures/16091504.pdf}};
   \node [draw,fill=white,rotate=0 ,x={(image.south east)},y={(image.north west)}]         at (0.52,0.82)    {2 TCLD at $10\, \sigma$};
  \end{tikzpicture}
\end{minipage}
\caption{Maximum stored beam energy as a function of the beam lifetime for different \mbox{HL-LHC} scenarios with \lead beams. Left column: simulations with one TCLD per beam. Right column: simulations with two TCLDs per beam. The gray areas show stored beam energies which are not reachable with the presently envisaged injector upgrades. The vertical black lines show the target values for LIU and HL-LHC. }
\label{fig:escompa}
\end{figure}

\vspace{0.2cm}

\textit{Nominal HL-LHC with 1 TCLD} \\ 
The improved cleaning performance with one TCLD collimator leads to an increase of the permitted stored beam energies. Including the MQPD scaling and assuming the design minimum beam lifetime of 12\,min, the maximum stored beam energy yields (5.3$\pm$0.4)\,MJ for a TCLD setting of 15\,$\sigma$ and (8.7$\pm$0.7)\,MJ for the tightest setting of 10\,$\sigma$. The achievable nucleon-nucleon luminosity in IP2 increases accordingly to \err{2.2}{0.2}$\times 10^{32}$\,cm$^{-2}$\,s$^{-1}$ for a TCLD half gap of $15\,\sigma$ and \err{3.5}{0.4}$\times 10^{32}$\,cm$^{-2}$\,s$^{-1}$ with a half gap of 10$\,\sigma$. 

\newpage
The HL-LHC target stored beam energy could possibly be reached, if the TCLD is set to half gaps smaller than 12$\,\sigma$ and the beam lifetime is larger than 38\,min.

% In both of these cases, the target value of  of 24.1\,MJ for the stored beam energy in HL-LHC can not be reached. This applies even for the values extracted without MQPD scaling, which yields (19.5$\pm 1.5$) with the tightest TCLD setting. The simulations hence indicate that the target beam intensity for the HL-LHC heavy-ion program can not be reached with the installation of only one TCLD collimator. 






 \mbox{} \\ 
\textit{Nominal HL-LHC with 2 TCLDs} \\ 
The drastic improvement of the cleaning performance with two TCLD collimators leads to significantly larger permitted stored beam energies. Already in the scenario with MQPD scaling, design minimum beam lifetime and the TCLDs set to the most relaxed setting of 15\,$\sigma$, the allowed stored beam energy is (22.4$\pm$1.7)\,MJ. This corresponds to an achievable luminosity of \err{9.1}{0.9}$\times 10^{32}$ cm$^{-2}$ s$^{-1}$, close to the design luminosity. All studied scenarios with tighter settings deliver stored beam energies beyond 600\,MJ, showing that the quench risk of collimation losses would not restrict the LHC performance any more. Evidently, such large beam intensities are not reachable and would be restricted by other limitations. The conclusion of this study is hence, that the HL-LHC goal is reachable when two TCLDs per beam are installed in IR7 and set to a small enough half gap between 12$\,\sigma$ and $15\,\sigma$.


\newpage
\textit{Argon and Xenon Ions} \\ 
The cleaning inefficiency  simulated for \iso{40}{Ar}{18+} and \iso{129}{Xe}{54+} are similar to the HL-LHC configuration with \lead beams, without TCLDs. Accordingly, the maximum stored beam energy is also similar for Ar and Xe. With MQPD scaling and a minimum beam lifetime of 12\,min, the allowed stored beam energy yields (4.9$\pm$0.4)\,MJ for Ar and $(4.7\pm 0.4)\,$MJ for Xe. 
%
%
\begin{figure}[t] 
\centering
\begin{minipage}[b]{.49\textwidth}
  \begin{tikzpicture}
    \footnotesize
    \node[anchor=south west,inner sep=0] (image) at (0,0) {\includegraphics[width=1.0\linewidth]{pictures/16091307.pdf}};
   \node [draw,fill=white,rotate=0 ,x={(image.south east)},y={(image.north west)}]         at (0.20,0.85)    {Argon};
  \end{tikzpicture}
\end{minipage}%
\hfill
\begin{minipage}[b]{.49\linewidth}
  \begin{tikzpicture}
    \footnotesize
    \node[anchor=south west,inner sep=0] (image) at (0,0) {\includegraphics[width=1.0\linewidth]{pictures/16091417.pdf}};
   \node [draw,fill=white,rotate=0 ,x={(image.south east)},y={(image.north west)}]         at (0.20,0.85)    {Argon};
  \end{tikzpicture}
\end{minipage}
\\ \mbox{} \\ 
\begin{minipage}[b]{.49\textwidth}
  \begin{tikzpicture}
    \footnotesize
    \node[anchor=south west,inner sep=0] (image) at (0,0) {\includegraphics[width=1.0\linewidth]{pictures/16091309.pdf}};
   \node [draw,fill=white,rotate=0 ,x={(image.south east)},y={(image.north west)}]         at (0.20,0.85)    {Xenon};
%
%
%
  \end{tikzpicture}
\end{minipage}%
\hfill
\begin{minipage}[b]{.49\linewidth}
  \begin{tikzpicture}
    \footnotesize
    \node[anchor=south west,inner sep=0] (image) at (0,0) {\includegraphics[width=1.0\linewidth]{pictures/16091419.pdf}};
   \node [draw,fill=white,rotate=0 ,x={(image.south east)},y={(image.north west)}]         at (0.20,0.85)    {Xenon};
    \draw[-,color=black,thick,x={(image.south east)},y={(image.north west)}]       (0.2,0.60) -- (0.34,0.65);
    \draw[-,color=black,thick,x={(image.south east)},y={(image.north west)}]       (0.55,0.25) -- (0.40,0.30);

    \node [fill=white,rotate=0,x={(image.south east)},y={(image.north west)},align=center]         at (0.20,0.6)    {LIU \\ Baseline};
    \node [fill=white,rotate=0,x={(image.south east)},y={(image.north west)},align=center]         at (0.55,0.25)    {HL-LHC \\ Request};
  \end{tikzpicture}
\end{minipage}
\caption{Left column: Maximum stored beam energy for Ar and Xe ions. Right column: achievable luminosity, assuming the same normalized emittance 1.3$\,\mu$m rad as for \lead beams and bunch intensities of $5 \times 10^9$ ions/bunch for Ar and $6\times 10^8$ ions/bunch for Xe. The vertical lines show the \lead design values after the LIU upgrade and for HL-LHC. }
\label{pic:arxe:lumi}
\end{figure}
%
%

However, for the case of Ar, the number of ions per bunch is higher by a factor of \mbox{almost 20}. Also the charge to mass ratio of Ar is larger than that of Pb, which increases the relativistic $\gamma$ factor, assuming the same emittance as for Pb beams. Therefore, the achievable nucleon-nucleon luminosity with Ar is significantly larger compared to Pb beams. Including MQPD scaling and assuming a minimum beam lifetime of 12\,min, the extrapolated nucleon-nucleon luminosity of Ar yields ($9.1\pm 0.8$)$\times 10^{32}$\,cm$^{-2}$\,s$^{-1}$. 


This is a significant improvement compared to ($1.9\pm 0.2$)$\times 10^{32}$\,cm$^{-2}$\,s$^{-1}$ for Pb beams, when MQPD scaling is considered and no TCLDs are installed. The comparison shows that with Ar ions, the envisaged HL-LHC nucleon-nucleon luminosity of $9.7 \times 10^{32}$\,cm$^{-2}$\,s$^{-1}$ can almost be reached. The maximum nucleon-nucleon luminosity for Xe ions, including MQPD scaling and assuming a beam lifetime of 12\,min, yields \mbox{$(3.3\pm 0.4)\times 10^{32}$}. This is only slightly larger than for Pb and significantly below the HL-LHC target. 

The comparison shows that higher luminosities are indeed possible when lighter ion species than \lead are collided, although the cleaning inefficiency is comparable. While for Xe, the potential for luminosity increase is rather limited, the estimated nucleon-nucleon luminosity for Ar is significantly larger than for Pb. It should be kept in mind, that these estimates depend crucially on the assumed bunch intensity. The latter could be lower in the LHC at top energy. This should be taken into account in an updated study, as soon as reliable data on the achievable Ar bunch intensity in the LHC is available. 


\newpage
\section{Summary and Conclusions}


The 2015 LHC heavy-ion collimation quench test was the first experiment of its kind in which a quench was achieved. The MBB.9L7 quenched from the collimation debris, when a power of $(15\pm 1)\,$kW was deposited at the horizontal TCP for B2. Extrapolating from the TCP power load and the design value of the minimum beam life time, the upper limit for the achievable stored beam energy at $6.37\,Z\,$TeV was derived to be (10.8$\pm$0.8)~MJ. 
\vspace{0.1cm}

The cleaning inefficiency of the LHC in the quench test configuration was simulated with the hiSixTrack-FLUKA coupling and the result is in good agreement with the measured loss pattern. The output produced is used for a detailed shower propagation and energy deposition simulation. This study is carried out by the CERN FLUKA team and is still ongoing. 
\vspace{0.1cm}

The experimentally gathered intensity limitation for 6.37$\,Z\,$TeV was used to extrapolate intensity limitations for HL-LHC with and without TCLD collimators. This was achieved by re-scaling the intensity limit found for 6.37$\,Z\,$TeV with a MQPD reduction factor and taking into account the maximum average cleaning inefficiency in the different scenarios. These studies show that the target intensity for HL-LHC can possibly not be reached if only one TCLD is installed per beam in IR7 and the minimum beam lifetime is 12$\,$min. This should be further investigated, possibly with more detailed shower propagation studies in FLUKA. With sufficiently tight TCLD settings and a lifetime above 38\,min, the target stored beam energy can possibly be reached.
\vspace{0.1cm}

With two TCLD collimators, the amount of collimation debris lost in the IR7 DS is reduced sufficiently to reach the HL-LHC goals if the TCLD half gap is set to sufficiently tight settings. With two TCLD collimators, the quench risk from collimation debris can potentially be eliminated as a limiting factor for the achievable stored heavy-ion beam energy. 
\vspace{0.2cm}

The studies with Ar and Xe ions have shown that higher nucleon-nucleon luminosities can be reached also without TCLDs, especially if Ar ions were to be used in the LHC. 
\vspace{0.2cm} 

Within the remaining inevitable uncertainties, the HL-LHC simulations with the hiSixTrack-FLUKA coupling have delivered important new findings. They are going to be taken as a reference for future discussions about the required collimation system upgrades for HL-LHC.


% ;All 
% ndings should be considered preliminary, since they depend crucially on di
% erent assumptions
% of which some are only vaguely known at the time of writing. 


% This is a very bad way to end ... try to be more positive about what you have achieved, here and in other places!! 

% Do not use the word "vague" anywhere!    Perhaps something like 

% Within the remaining inevitable uncertainties, we have therefore established  ... 


% All findings should be considered preliminary, since they depend crucially on different assumptions of which some are only vaguely known at the time of writing.



\chapter{Outlook} \label{chap:outlook}

This chapter discusses potential reasons for remaining discrepancies between measurements and simulations with STIER or the hiSixTrack-FLUKA coupling. The outcome can be used in future applications to improve the agreement with measurements or to study the significance of loss predictions with the new simulation tools.


\section{Possible Improvements for Future Simulations} \label{chap:impr}

The simulations with both STIER and the hiSixTrack-FLUKA coupling still show discrepancies to the measured loss maps. Some of the loss peaks which are measured are not simulated and vice versa. In this section, it is discussed, how these discrepancies could be understood and possible improvements for future upgrades of the simulation tools are outlined.

\subsection{Orbit Fluctuations}


\begin{figure}[p]
  \centering
  \begin{tikzpicture}
    \footnotesize
    \node[anchor=south west,inner sep=0] (image) at (0,0) {\includegraphics[width=1.0\linewidth]{pictures/16082902.pdf}};
  \node [fill=white,x={(image.south east)},y={(image.north west)}]                   at (0.87,0.92)    {2011 Measurement};
  \node [fill=white,x={(image.south east)},y={(image.north west)}]                   at (0.87,0.70)    {2011 Simulation};

  \node [fill=white,x={(image.south east)},y={(image.north west)}]                   at (0.87,0.47)    {2015 Measurement};
  \node [fill=white,x={(image.south east)},y={(image.north west)}]                   at (0.87,0.25)    {2015 Simulation};

  \node [fill=white,x={(image.south east)},y={(image.north west)}]                   at (0.43,0.657)    {A1};
  \node [fill=white,x={(image.south east)},y={(image.north west)}]                   at (0.56,0.657)    {A2};
  \node [fill=white,x={(image.south east)},y={(image.north west)}]                   at (0.74,0.657)    {A3};
  \node [fill=white,x={(image.south east)},y={(image.north west)}]                   at (0.92,0.657)    {A4};



  % \draw[help lines,step=.2] (0,0) grid (16.0,13.0);
  % \draw[help lines,line width=.6pt,step=1] (0,0) grid (16.0,9.0);
  % \foreach \x in {0,1,2,3,4,5,6,7,8,9,10,11,12,13,14,15,16}
  %      \node[anchor=north] at (\x,-0.1) {\x};
  % \foreach \y in {0,1,2,3,4,5,6,7,8,9,10,11,12,13}
  %     \node[anchor=east] at (-0.2,\y) {\y};

  \end{tikzpicture}
  \caption{Measured and simulated B1H loss maps of the 2011 heavy-ion run at 3.5$\,Z\,$TeV and the 2015 heavy-ion run at 6.37$\,Z\,$TeV, zoomed to IR7. The simulations are carried out with the hiSixTrack-FLUKA coupling.}  
  \label{pic:16082902}
  %/home/phermes/Dropbox/PhD/pictures/160617_hisix_STIER_2015/ionLM11_15.pdf
  \end{figure}


Field errors and misalignments of the LHC magnets can lead to variations of the closed orbit. As it was discussed in \chapref{chap:orbump}, the loss location can, depending on the projected penetration depth $d_p$ of the particle, be altered when the orbit does not coincide with the ideal orbit. For loss peaks which are simulated with either STIER or the hiSixTrack-FLUKA coupling, the simulation data can be consulted to find the projected penetration depth of the individual particles causing the losses and hence quantify the orbit offset required to alleviate these losses. 


\begin{figure}[b]
  \centering
  \begin{tikzpicture}
    \footnotesize
    \node[anchor=south west,inner sep=0] (image) at (0,0) {\includegraphics[width=1.0\linewidth]{pictures/16083002.pdf}};
  \node [draw,x={(image.south east)},y={(image.north west)},fill=white]               at (0.44,0.73)    {A1};

  \node [draw,x={(image.south east)},y={(image.north west)},fill=white]               at (0.935,0.73)    {A2};
  \node [draw,x={(image.south east)},y={(image.north west)},fill=white]               at (0.44,0.24)    {A3};
  \node [draw,x={(image.south east)},y={(image.north west)},fill=white]               at (0.935,0.24)    {A4};

  % \node [draw,rotate=0 ,x={(image.south east)},y={(image.north west)}]                   at (0.22,0.96)    {text1};
  \node [x={(image.south east)},y={(image.north west)},anchor=west]       at (0.11,0.84)    {$\Delta x_{50}^{2011}$};
  \draw[-,color=black,x={(image.south east)},y={(image.north west)}]             (0.176,0.785) -- (0.14,0.825);
  \draw[-,color=black,x={(image.south east)},y={(image.north west)}]             (0.31,0.785) -- (0.274,0.825);
  \node [x={(image.south east)},y={(image.north west)},anchor=west]       at (0.244,0.84)    {$\Delta x_{50}^{2015}$};


  % \draw[help lines,step=.05,x={(image.south east)},y={(image.north west)}] (0,0) grid (1,1);
  % \draw[help lines,line width=.6pt,step=0.1,x={(image.south east)},y={(image.north west)}] (0,0) grid (1,1);
  % \foreach \x in {0,0.1,0.2,0.3,0.4,0.5,0.6,0.7,0.8,0.9,1.0}
  %      \node[anchor=north,x={(image.south east)},y={(image.north west)}] at (\x,-0.01) {\x};
  % \foreach \y in {0.0,0.1,0.2,0.3,0.4,0.5,0.6,0.7,0.8,0.9,1.0}
  %     \node[anchor=east,x={(image.south east)},y={(image.north west)}] at (-0.01,\y) {\y};

  \end{tikzpicture}
  \caption{Loss reduction as a function of the orbit offset at the individual arc loss spikes A1 to A4 derived from the projected penetration depth. Data simulated with the hiSixTrack-FLUKA coupling for the 2011 and the 2015 heavy-ion run. }  
  \label{pic:16082901}
  %/media/phermes/local/hisix_results/2015/B1_squeezed/tools/postprocessing/orbitoffset.pdf
  \end{figure}

\vspace{0.2cm}

This principle can be illustrated at the example of the simulated and measured loss patterns for the 2011 and 2015 heavy-ion run. They are directly compared in \figref{pic:16082902}. The arc loss peaks A1 to A4 are both measured and simulated in the heavy-ion run of 2011. Also in the hiSixTrack-FLUKA coupling simulation for the 2015 heavy-ion run they are predicted. However, in the 2015 qualification loss maps only the A1 and A4 loss spikes are visible, while at the A2 and A3 locations loss peaks are not measured. These discrepancies could potentially arise from orbit offsets, which shall be studied in the following. 



\newpage
From the simulation data, the projected penetration depth for the particles impacting at the individual loss peaks can be derived. With this information, the expected loss reduction for a given orbit offset can be derived, assuming that the orbit offset shifts the beam away from the impacting side of the beam pipe. The so derived loss reduction as a function of the orbit offset $\Delta x$ is shown for the 2011 and the 2015 heavy-ion run in \figref{pic:16082901}. A key quantity in the following comparison is the orbit offset required to reduce the losses by 50\% (see \figref{pic:16082901}). This quantity is called $\Delta x_{50}$ in the following. The values for $\Delta x_{50}$ in the A1 to A4 loss peaks derived for the 2011 and 2015 heavy-ion run are listed in \tabref{tab:offsetdx}.



For the A1 loss peak, the required offset for 50\% loss reduction simulated for the 2011 configuration is approximately $\Delta x_{50}^{2011}=800$\mum. In the simulation for the 2015 heavy-ion run, this quantity yields $\Delta x_{50}^{2015}=1700$\mum. This means that a larger orbit offset is required to alleviate the same amount of the A1 losses in the 2015 configuration. For orbit offsets of more than $2500$\mum, the A1 loss peak is fully alleviated in both simulations. These values are rather large. 





\begin{table}[t]
\centering
\caption{Orbit offsets required to alleviate 50\% of the energy lost at the A1 to A4 loss peaks.}
\label{tab:offsetdx}
\begin{tabular}{lcccc}
\toprule
     & \multicolumn{4}{c}{$\Delta x_{50}$ {[}$\mu$m{]}} \\ \midrule
Year & A1          & A2         & A3        & A4        \\ \midrule
2011 & 800         & 650        & 190       & 550       \\
2015 & 1700        & 6          & 4         & 9         \\ \bottomrule
\end{tabular}
\end{table}

\newpage

For the A2 loss peak in the 2011 configuration which was both measured and simulated, $\Delta x_{50}^{2011}$ yields 650\mum. In the 2015 heavy-ion run, the measured qualification loss map does not show a loss peak at the A2 location. The analysis of the simulated loss pattern shows that $\Delta x_{50}^{2015}$ is only 6\mum\, for this simulation case. Hence, very small orbit variations can alleviate this loss peak which might be the reason for the discrepancy between simulation and measurement.

\vspace{0.2cm}

For the A3 loss peak, an orbit offset of $\approx 4$\mum\, is sufficient to alleviate half of the losses in the 2015 configuration. For the 2011 simulation, the required offset yields $\approx 190$\mum. Also for this loss spike, the discrepancy between measurement and simulation could be explained by small orbit fluctuations. 

\vspace{0.2cm}

For the A4 loss peak, the simulation in of the 2015 heavy-ion run predicts $\Delta x_{50}^{2015}=$9\mum\, and that for the 2011 heavy-ion run $\approx 550$\mum. In spite of the small penetration depth in the 2015 configuration, the A4 loss peak is measured in the qualification loss map. One possible reason could be that the particles which are not lost in the A2 and A3 arrive at the A4 loss peak with larger projected penetration depth and cause the losses measured at this location. 

\vspace{0.2cm}

In conclusion, the study demonstrates that the loss distribution of heavy-ion collimation losses can, depending on the machine configuration, be altered by small variations of the beam orbit.  A loss reduction of 50\% of the simulated loss amplitude at the locations A2, A3 and A4 can be expected by very small orbit offsets between 4\mum\, and 9\mum\, in the 2015 heavy-ion configuration. The real orbit during operation can drift by significantly larger amounts. Within one fill, the orbit can drift by 50\,$\mu$m~\cite{wenn:private}. The reference orbit that is commissioned at the beginning of a run can be even further away from the magnet centers. The simulation results indicate that this is possibly the reason why A2 and A3 were not measured during the 2015 heavy-ion run. At the location of the A4 loss peak, the orbit could be shifted towards the beam pipe or the particles which would otherwise have been lost at the A2 and A3 could be lost there. 

\vspace{0.2cm}

The studies show that it is important to take into account the projected penetration depth of simulated loss peaks. This can give valuable information to determine whether the simulated loss peak is probable to be measured or not. 

\newpage
Simulated loss spikes from particles impacting the aperture at a small projected penetration depth could be unobservable in the machine. As a counterexample, the impact parameter of the \iso{207}{Pb}{82+} fragments lost at the TCT2 in the study presented at \chapref{chap:ir2loss} yields approximately 2~mm. As it was shown before, these losses were well predicted in the STIER simulation.

\subsection{Aperture Misalignments}\label{aper:misalg}


\begin{figure}[b]
  \centering
  \begin{tikzpicture}
    \node[anchor=south west,inner sep=0] (image) at (0,0) {\includegraphics[width=0.7\linewidth]{pictures/16082903.pdf}};
  % \node [draw,rotate=90,x={(image.south east)},y={(image.north west)}]                   at (0.50,0.50)    {text0};
  % \node [draw,rotate=0 ,x={(image.south east)},y={(image.north west)}]                   at (0.22,0.96)    {text1};
  % \node [draw,rotate=0 ,x={(image.south east)},y={(image.north west)},anchor=west]       at (0.22,0.80)    {text2};
  % \draw[->,color=black,thick,x={(image.south east)},y={(image.north west)}]             (0.42,0.22) -- (0.37,0.23);


  % \draw[help lines,step=.2] (0,0) grid (16.0,13.0);
  % \draw[help lines,line width=.6pt,step=1] (0,0) grid (16.0,9.0);
  % \foreach \x in {0,1,2,3,4,5,6,7,8,9,10,11,12,13,14,15,16}
  %      \node[anchor=north] at (\x,-0.1) {\x};
  % \foreach \y in {0,1,2,3,4,5,6,7,8,9,10,11,12,13}
  %     \node[anchor=east] at (-0.2,\y) {\y};

  \end{tikzpicture}
  \caption{Measured horizontal aperture in the LHC magnets of IR7 and at the beginning of the arc between IR7 and IR8. }  
  \label{pic:16082903}
  %/media/phermes/local/hisix_results/2015/B1_squeezed/tools/postprocessing/measured_aperture.pdf
  \end{figure}


Besides orbit variations, also aperture misalignments of the LHC magnets can alter the loss position of secondary ion fragments. In the collimation simulations so far, only the ideal aperture is taken into account. Measurements of the aperture profile~\cite{EPAC08:TUPP091,lhcprojreport1056} in the LHC magnets show that in reality the aperture can be shifted significantly with respect to the reference. The measured aperture in IR7 is illustrated in \figref{pic:16082903}. Furthermore, the distribution of aperture misalignments with respect to the reference aperture in the arc magnets between IR7 and IR8 is illustrated in \figref{pic:16083001}.

\begin{figure}[t]
  \centering
  \begin{tikzpicture}
    \footnotesize
    \node[anchor=south west,inner sep=0] (image) at (0,0) {\includegraphics[width=0.6\linewidth]{pictures/16083001.pdf}};
  \node [x={(image.south east)},y={(image.north west)},align=center]   at (0.80,0.80)    {Standard\\ deviation: \\ 340\mum};
  % \node [draw,rotate=0 ,x={(image.south east)},y={(image.north west)}]                   at (0.22,0.96)    {text1};
  % \node [draw,rotate=0 ,x={(image.south east)},y={(image.north west)},anchor=west]       at (0.22,0.80)    {text2};
  % \draw[->,color=black,thick,x={(image.south east)},y={(image.north west)}]             (0.42,0.22) -- (0.37,0.23);


  % \draw[help lines,step=.2] (0,0) grid (16.0,13.0);
  % \draw[help lines,line width=.6pt,step=1] (0,0) grid (16.0,9.0);
  % \foreach \x in {0,1,2,3,4,5,6,7,8,9,10,11,12,13,14,15,16}
  %      \node[anchor=north] at (\x,-0.1) {\x};
  % \foreach \y in {0,1,2,3,4,5,6,7,8,9,10,11,12,13}
  %     \node[anchor=east] at (-0.2,\y) {\y};

  \end{tikzpicture}
  \caption{Distribution of aperture misalignments in the arc between IR7 and IR8.}  
  \label{pic:16083001}
  %/media/phermes/local/hisix_results/2015/B1_squeezed/tools/postprocessing/apertureoffset_abundance.pdf
  \end{figure}

The standard deviation of this distribution is approximately 340\mum. The study of the projected penetration depth discussed in the previous subsection shows that such aperture offsets are sufficient to alleviate losses which would occur with the nominal aperture. For future cleaning simulations with heavy-ion beams, a framework could be established which incorporates the information about the measured aperture. This would possibly allow for a better understanding of existing discrepancies between simulations and measurements.

\section{Summary and Conclusions}

In this chapter, it was shown that remaining discrepancies between measured and simulated heavy-ion loss maps can potentially be caused by orbit and aperture offsets. The projected penetration depth in the magnet aperture was used to study loss peaks in the arc downstream of IR7. Some of these loss peaks were simulated in the configuration of the 2015 heavy-ion run, but not measured during operation. The analysis unveiled that the projected penetration depth of the particles was smaller than typical orbit drifts or aperture misalignments, which might explain their absence in the measurement. 

Future simulations could take these effects into account and hence predict the cleaning inefficiency even better than the existing tools.



\chapter*{Summary} \label{chap:sum}
\addcontentsline{toc}{chapter}{Summary}
\chaptermark{Summary}


With the CERN Large Hadron Collider (LHC), particle beams of unprecedented momentum became available for fundamental research. Besides the proton program, the LHC is operated with \lead beams, making it the third heavy-ion collider ever operated worldwide. The unprecedented stored beam energies in the LHC are a potential danger, because small fractions of it are sufficient to provoke the superconducting magnets to lose their superconductivity (quench), or even damage the machine hardware. The LHC is therefore equipped with a multi-stage collimation system to protect the machine from undesired beam loss. However, particles with rigidity offsets can still scatter out of the collimators. They can subsequently be absorbed by the superconducting LHC magnets which are in danger to quench. The dominating contribution to the cleaning inefficiency of the collimation system, defined as the ratio of energy lost in the superconducting magnets to the particles impacting on the collimators, is given by particles with changed magnetic rigidity


\vspace{0.2cm}

In this thesis, a new software to simulate the collimation performance of the LHC with heavy-ion beams is presented. Compared to proton collimation, the efficiency of the LHC collimation system for \lead beams is worse by more than two orders of magnitude. The reason for this is the large amount of residual heavy-ion fragments with different mass to charge ratio that is generated when heavy ions interact with the collimators. When the fragments enter the superconducting LHC magnets, they can be deviated into the magnet aperture and deposit energy in the superconducting magnet coils. Sophisticated simulation tools are required to simulate the efficiency of the collimation system in order to ensure safe operation that is uninterrupted by quenches. Such simulations allow understanding potentially critical beam losses and can be used to develop loss mitigation strategies. Collimation simulation tools must accurately simulate the interaction of beam particles with the collimators and follow the trajectory of the particles scattered out of the collimation system (tracking) to determine their loss location. The result of such simulations is the predicted distribution of losses in the ring (loss map), which can be compared to the loss distribution measured in operation.
\vspace{0.2cm}

\newpage
Simulation software for heavy-ion collimation must hence take into account the nuclear and electromagnetic fragmentation inside the collimators and follow the particle trajectories of a wide range of different heavy ions inside the LHC magnets. The previous standard heavy-ion collimation simulation tool ICOSIM simulated the fragmentation in the LHC collimators in a simplified manner. In particular scattering and momentum changes from the fragmentation process are not taken into account in ICOSIM. Some important loss features that were measured in LHC operation could not be predicted with ICOSIM. 



\thispagestyle{plain}

To understand if the simplified fragmentation algorithm in ICOSIM is the origin of the observed discrepancies, a new simulation framework, referred to as SixTrack with Ion-Equivalent Rigidities (STIER), was established. STIER was developed in order to estimate the impact of the various approximations used in ICOSIM. In particular, it is distinguished by a more realistic fragmentation model during the first passage through a collimator. In the STIER approach, the fragmentation at the most important primary collimator is simulated with the simulation software for particle-matter interaction FLUKA. The resulting distribution of heavy-ion fragments (including the correct angular coordinates and the particle energy) is then tracked as protons with ion-equivalent rigidities in the proton tracking software SixTrack. The particles start from the primary collimator. All other collimators are treated as perfect absorbers.
\vspace{0.2cm}

The STIER simulations have demonstrated that the accuracy of the cleaning simulations can be significantly improved if changes in transverse angle and in energy from the fragmentation process are included. In spite of not treating fragmentation in subsequent collimators, the STIER simulations are in good agreement with the measured data. It was therefore used to validate the collimation system in the 2015 LHC heavy-ion run at 6.37$\,Z\,$TeV. In this operational period, high losses at the tertiary collimator close to the \mbox{ALICE} experiment occurred. STIER was employed to study the origin of these losses, showing that they are dominated by the isotope \iso{207}{Pb}{82+} scattered out of the left jaw of the primary collimator. Based on this outcome, strategies for loss mitigation were worked out. They were later-on experimentally tested in the LHC. In these experiments, the LHC behaved as predicted by STIER and the loss reduction anticipated by the simulation could be quantitatively confirmed.
\vspace{0.2cm}

The experience obtained with STIER was used to implement heavy-ion SixTrack (hiSixTrack), a generic tracking tool supporting arbitrary ion types. The software is based on the standard version of SixTrack, but is extended to store information about the species of the tracked particle. In order to accurately simulate the magnetic bending for particles with different masses and charges, symplectic tracking maps were derived from a new accelerator Hamiltonian that is adapted for the treatment of isotopes different from the reference species. The new multi-isotopic tracking maps are implemented in hiSixTrack and have been \mbox{studied to verify their symplecticity}. 
\vspace{0.2cm}

The particle tracks simulated with hiSixTrack have been benchmarked against STIER. With hiSixTrack, a new tracking tool became available that allows the symplectic tracking of arbitrary heavy-ion species with their physical momentum, mass and charge. 
\vspace{0.2cm}

To allow simulating the fragmentation at the LHC collimators, hiSixTrack was coupled with FLUKA. The basis for this development was the SixTrack-FLUKA coupling, which was already implemented to actively exchange protons between both codes. The framework was adapted to exchange heavy ions between hiSixTrack and FLUKA. Today, the hiSixTrack-FLUKA coupling is the standard simulation software for heavy-ion collimation used at CERN. The cleaning simulations with the hiSixTrack-FLUKA coupling show an even better agreement with the measured data than STIER, mainly due to the inclusion of fragmentation at all collimators. 
\vspace{0.2cm} 

The hiSixTrack-FLUKA coupling was applied to study the cleaning performance in the future High Luminosity LHC (HL-LHC). Keeping the cleaning inefficiency at a low and tractable value is particularly important for this scenario, because the stored beam energies are going to be significantly increased. The simulations were also carried out with additional collimators (TCLD) in the dispersion suppressor of the betatron collimation region IR7, which are envisaged to be installed for HL-LHC.  The analysis shows that the cleaning performance can be improved if the new TCLD collimators were installed in the LHC. With one TCLD collimator per beam, the cleaning inefficiency can be reduced by approximately 45\%, if the collimator is set to a sufficiently small half gap. With two TCLD collimators per beam, the quench risk from particles scattered out of the collimators can be reduced by more than two orders of magnitude. 
\vspace{0.2cm} 

The cleaning performance was also simulated with other heavy-ion species than \lead. The study showed that the cleaning efficiency for \iso{129}{Xe}{54+}\, and \iso{40}{Ar}{18+}, which are presently the most probable candidates for operation, is similar to what can be expected with \lead.
\vspace{0.2cm} 

\thispagestyle{plain}


As a part of this thesis, a dedicated heavy-ion collimation quench test was designed and conducted in the LHC, to quantify the quench risk from collimation losses. In this experiment, carried out with \lead beams at 6.37$\,Z\,$TeV in late 2015, high steady state particle losses were induced at the primary collimator. The aim of the experiment was to quantify at which power load on the collimator the generated heavy-ion fragments would induce a quench of the superconducting LHC magnets. A magnet quench was achieved with a beam loss rate of $(15 \pm 1)$~kW. Assuming a minimum beam lifetime of 12\,min (which is the design specification), an approximate estimate for the maximum stored beam energy at 6.37$\,Z\,$TeV can be derived, which yields $(10.8 \pm 0.8)$~MJ. 
\vspace{0.2cm} 

The experimentally accessed quench limit was combined with the hiSixTrack-FLUKA coupling simulations to derive limitations in terms of stored beam energy also for future operation. Taking into account that the quench limit at $7.0\,Z\,$TeV is lower than at $6.37\,Z\,$TeV, the analysis showed that the target stored beam energy, and hence luminosity for HL-LHC cannot be reached without TCLD collimators, if a minimum beam lifetime of 12\,min is assumed. With one TCLD collimator per beam, the HL-LHC target can be only reached if the minimum beam lifetime was larger than 38$\,$min. With two TCLD collimators per beam, the cleaning performance is improved significantly, such that the stored beam energy is not limited by the quench risk from collimation losses any more. The analysis further showed that the target luminosity can almost be reached also without TCLD collimators if the LHC were to be operated with  \iso{40}{Ar}{18+} ions. The main reason for this improvement is the significantly larger bunch intensity compared to what is achievable with \lead beams.
\vspace{0.2cm} 

In a brief outlook, remaining discrepancies between simulated and measured loss maps were studied, which can be used for future improvements of the simulation tools. 



\thispagestyle{plain}
\vspace{0.2cm} 

In summary, this thesis has introduced STIER and the hiSixTrack-FLUKA coupling as new simulation tools for heavy-ion collimation. They have been benchmarked against measured data, were used to make accurate predictions about the machine behavior,  delivered input for the analysis of experimental data from the LHC and were used to define the required collimation system upgrades for HL-LHC. 
\vspace{0.2cm} 

At the time of writing, the hiSixTrack-FLUKA coupling is the most sophisticated tool to simulate heavy-ion collimation. It is going to be used for future heavy-ion collimation simulations and might be indispensable to understand and avoid potential limitations from beam losses, which need to be overcome to exploit the full potential of the LHC in heavy-ion operation. 




\chapter*{Zusammenfassung} \label{chap:sum}
\addcontentsline{toc}{chapter}{Zusammenfassung}
\chaptermark{Zusammenfassung}

\thispagestyle{empty}

\begin{otherlanguage}{german}

Mit dem CERN Large Hadron Collider (LHC) wurden Teilchenstrahlen von vorher \mbox{unerreichter} \mbox{Energie} f\"{u}r die Grundlagenforschung zug\"{a}nglich. Neben dem Forschungsprogramm mit Protonenstrahlen wird der LHC mit Schwerionenstrahlen betrieben, womit er der dritte Speicherring weltweit ist, der Colliding-Beam-Experimente mit Schwerionen erm\"{o}glicht. Die gro\ss e Strahlenergie, die im LHC gespeichert wird stellt ein Risiko f\"{u}r den Beschleuniger dar, da selbst kleinste Mengen dieser Energie ausreichen k\"{o}nnen, um einen \"{U}bergang der supraleitenden Magnete in den normalleitenden Zustand zu induzieren (Quench), oder sogar den Beschleuniger zu besch\"{a}digen. Aus diesem Grund ist der LHC mit einem mehrstufigen \mbox{Kollimationssystem} \mbox{ausgestattet}, das den Beschleuniger vor  unerw\"{u}nschten Strahlverlusten sch\"{u}tzen soll. Trotzdem k\"{o}nnen Teilchen mit abweichenden magnetischen Rigidt\"{a}ten aus den Kollimatoren gestreut werden und anschlie\ss end in den supraleitenden Magneten verloren gehen. Die Ineffizienz des Kollimationssystems wird definiert als die in supraleitenden Magneten verlorene Strahlenergie im Verh\"{a}ltnis zur Strahlenergie, die in den Kollimatoren verloren geht. Der gr\"{o}\ss te Teil der Ineffizienz des Kollimationssystems stammt von Teilchen mit abweichender Rigidit\"{a}t. 

%Der gr\"{o}\ss te Teil der Ineffizienz des Kollimationssystems, die definiert ist als das Verh\"{a}ltnis aus der verlorenen Strahlenergie in den supraleitenden Magneten zu jener im Kollimationssysem, 



%With the CERN Large Hadron Collider (LHC), particle beams of unprecedented momentum became available for fundamental research. Besides the proton program, the LHC is operated with \lead beams, making it the third heavy-ion collider ever operated worldwide. The unprecedented stored beam energies in the LHC are a potential danger, because small fractions of it are sufficient to provoke the superconducting magnets to lose their superconductivity (quench), or even damage the machine hardware. The LHC is therefore equipped with a multi-stage collimation system to protect the machine from undesired beam loss. However, particles with different rigidities can still scatter out of the collimators and can subsequently be absorbed by the superconducting LHC magnets which are in danger to quench. The dominating contribution to the cleaning inefficiency of the collimation system, defined as the ratio of energy lost in the superconducting magnets to the particles impacting on the collimators, is given by particles with changed magnetic rigidity

In dieser Dissertation wird eine Software zur Simulation der Kollimationseffizienz mit \mbox{Schwerionenstrahlen} vorgestellt. Im Vergleich zur Kollimation von Protonen ist die Effizienz des Kollimationssystems mit Schwerionenstrahlen um zwei Gr\"{o}\ss enordnungen reduziert. Die Ursache daf\"{u}r ist die gro\ss e Menge an Fragmenten mit anderen Massen und Ladungen, die entsteht wenn Schwerionen mit dem Kollimatormaterial wechselwirken.  Wenn die Fragmente die supraleitenden LHC Magnete erreichen, werden sie in die Magnetapertur gelenkt und gehen dort verloren. Um einen sicheren und ununterbrochenen Betrieb des Beschleunigers zu gew\"{a}hrleisten werden anspruchsvolle Simulationen ben\"{o}tigt, die die Effizienz des Kollimationssystems simulieren k\"{o}nnen. Diese Simulationen erlauben es  Strahlverluste, die m\"{o}glicherweise gef\"{a}hrlich sein k\"{o}nnten, zu verstehen und Strategien zu entwickeln um sie zu reduzieren. Simulationen der Effizienz des Kollimationssystems simulieren die Wechselwirkung der Strahlteilchen mit den Kollimatoren und berechnen die Trajektorie der aus den Kollimatoren gestreuten Teilchen durch die Magnete des Beschleunigers. 




\vspace{0.2cm}
%
%In this thesis, a new software to simulate the collimation performance of the LHC with heavy-ion beams is presented. Compared to proton collimation, the efficiency of the LHC collimation system for \lead beams is worse by more than two orders of magnitude. The reason for this is the large amount of residual heavy-ion fragments with different mass to charge ratios that is generated when heavy-ions interact with the collimators. When the fragments enter the superconducting LHC magnets, they can be deviated into the magnet aperture and hence be lost. Sophisticated simulation tools are required to simulate the efficiency of the collimation system in order to ensure safe and uninterrupted operation. They allow understanding potentially critical beam losses and can be used to develop loss mitigation strategies. Such tools simulate the interaction of beam particles with the collimators and follow the trajectory of the particles scattered out of the collimation system (tracking). 
\vspace{0.2cm}

Eine Software f\"{u}r die Simulation von Schwerionenkollimation muss deshalb Wechselwirkungen \"{u}ber die elektromagnetische und die starke Kraft mit dem Kollimatormaterial simulieren und die Trajektorie einer Vielzahl verschiedener Fragmente in den Magneten des LHC berechnen. ICOSIM, die bisherige Referenz f\"{u}r diese Art von Simulationen, berechnet die Fragmentierung in den Kollimationen in einem vereinfachten Modell. Insbesondere Streuung und der Transfer von Energie auf den erzeugten Atomkern wurden nicht ber\"{u}cksichtigt. Wichtige Strukturen in der gemessenen Verteilung der Fragmente im LHC wurden von ICOSIM nicht vorausgesagt.
\vspace{0.2cm}


%Simulation software for heavy-ion collimation must take into account the nuclear and electromagnetic fragmentation inside the collimators and follow the particle trajectories of a wide range of different heavy ions inside the LHC magnets. The previous standard heavy-ion collimation simulation tool ICOSIM simulated the fragmentation in the LHC collimators in a simplified manner. In particular scattering and momentum changes from the fragmentation process are not taken into account in ICOSIM. Some important loss features that were measured in LHC operation could not be predicted with ICOSIM. 


Um zu verstehen ob die Ursache f\"{u}r die beobachteten Diskrepanzen in dem vereinfachten \mbox{Modell} zur Simulation der Fragmentierung zu suchen ist, wurde eine neue Software entwickelt, die die Bezeichnung SixTrack with Ion-Equivalent Rigidities (STIER) tr\"{a}gt. STIER zeichnet sich insbesondere durch ein realistischeres Modell zur Simulation der Fragmentierung aus, das bei der ersten Wechselwirkung mit einem Kollimator genutzt wird. STIER basiert auf der Software \mbox{FLUKA} zur Simulation der Wechselwirkung von Teilchen mit Materie, sowie auf SixTrack, einer Software f\"{u}r Teilchentransport in den Magneten eines Beschleunigers. In STIER wird die Wechselwirkung der Teilchenstrahlen mit dem prim\"{a}ren Kollimator (der der gr\"{o}\ss ten Menge einfallender Teilchen ausgesetzt ist) mit \mbox{FLUKA} simuliert. Anschlie\ss end wird die Trajektorie der aus dem Kollimator gestreuten Teilchen mit SixTrack berechnet, wobei die Schwerionentrajektorien als Protonentrajektorien berechnen werden und den Protonen eine Rigidit\"{a}t zugewiesen wird, die jener des Schwerions entspricht. Im Gegensatz zu ICOSIM werden die Streuwinkel und die korrekten Teilchenenergien ber\"{u}cksichtigt. Alle anderen Kollimatoren werden in STIER als perfekte Absorber betrachtet.
\vspace{0.2cm}

%To understand if the simplified fragmentation algorithm in ICOSIM is the origin of the observed discrepancies, a new simulation framework, referred to as SixTrack with Ion-Equivalent Rigidities (STIER), was established. STIER was developed in order to estimate the impact of the various approximations used in ICOSIM, and in particular use a more realistic fragmentation model during the first passage through a colliamtor. It is based on the proton tracking tool SixTrack and the Monte-Carlo event generator FLUKA. In the STIER approach, the fragmentation at the most important primary collimator is simulated in FLUKA. The resulting distribution of heavy-ion fragments (including the correct angular coordinates and the particle energy) is then tracked in SixTrack as protons of the same rigidity as the ion fragments, starting from the primary collimator. Other collimators are treated as perfect absorbers.

Die Simulationen mit STIER haben aufgezeigt, dass die \"{U}bereinstimmung mit der gemessenen Verteilung von Strahlverlusten verbessert wird, wenn \"{A}nderungen des transversalen Winkels und der Energie ber\"{u}cksichtigt werden. Obwohl die Interaktionen mit allen weiteren Kollimatoren nicht simuliert werden, zeigen die Simulationen mit  STIER eine gute \"{U}bereinstimmung mit den gemessenen Verteilungen von Kollimationsverlusten. Aufgrund dessen wurde STIER im Jahre 2015 genutzt um das \mbox{Kollimationssystem} f\"{u}r den Betrieb mit \lead Schwerionen bei $6.37\,Z\,$TeV zu validieren. W\"{a}hrend des Betriebes wurden gro\ss e Strahlverluste am terti\"{a}ren Kollimator nahe des ALICE Experimentes gemessen. Mittels STIER wurde aufgezeigt, dass der dominierende Anteil an diesen Verlusten durch \iso{207}{Pb}{82+} verursacht wird, das auf der linken Seite des prim\"{a}ren Kollimators erzeugt wird. Diese Erkenntnisse wurden genutzt um Strategien zur Verringerung dieser Verluste zu entwickeln, die sp\"{a}ter im LHC experimentell getestet wurden. Die Experimente best\"{a}tigen die von STIER getroffenen Voraussagen.
\vspace{0.2cm}

%Bei der experimentellen \"{U}berpr\"{u}fung best\"{a}tigten die von STIER prophezeite Reduzierung der Verluste.

%The STIER simulations have demonstrated that the accuracy of the cleaning simulations can be significantly improved if changes in transverse angle and in energy from the fragmentation process are included. In spite of not treating fragmentation in subsequent collimators, the STIER simulations are in good agreement with the measured data. It was therefore used to validate the collimation system in the 2015 LHC heavy-ion run at 6.37$\,Z\,$TeV. In this operational period, high losses at the tertiary collimator close to the \mbox{ALICE} experiment occurred. STIER was employed to study the origin of these losses, showing that they are dominated by the isotope \iso{207}{Pb}{82+} scattered out of the left jaw of the primary collimator. Based on this outcome, strategies for loss mitigation were worked out. They were later-on experimentally tested in the LHC. In these experiments, the LHC behaved as predicted by STIER and the loss reduction anticipated by the simulation could be quantitatively confirmed.

\thispagestyle{plain}

\newpage

Um die \"{U}bereinstimmung mit den gemessenen Daten weiter zu verbessern, wurde als Nachfolger von STIER die neue Software heavy-ion SixTrack (hiSixTrack) entwickelt. Diese erm\"{o}glicht die Berechnung der Trajektorie von beliebigen Schwerionen. Die Software hiSixTrack ist eine Erweiterung von SixTrack, die zus\"{a}tzlich Informationen \"{u}ber Masse und Ladung der Teilchen speichert. Um die magnetische Ablenkung von Teilchen mit unterschiedlichen Massen und Ladungen akkurat zu berechnen wurden symplektische Transformationsregeln von einer \mbox{verallgemeinerten} Hamilton-Funktion abgeleitet. Diese ist geeignet die Dynamik von Teilchen mit Masse und Ladung zu berechnen, die nicht notwendigerweise mit jenen des \mbox{Referenzteilchens} \"{u}bereinstimmen m\"{u}ssen. Die neuen generalisierten Transformationsregeln wurden in hiSixTrack implementiert, nachdem ihre Symplektizit\"{a}t sichergestellt wurde. Die mit hiSixTrack berechneten Trajektorien wurden mit von STIER simulierten Trajektorien verglichen um die Fehlerfreiheit der Implementierung zu gew\"{a}hrleisten. Mit hiSixTrack steht nun eine neue Software zur Verf\"{u}gung, die zur symplektischen Berechnung von Trajektorien beliebiger Teilchenarten verwendet werden kann. 
\vspace{0.2cm}
%The experience obtained with STIER was used to implement heavy-ion SixTrack (hiSixTrack), a generic tracking tool supporting arbitrary ion types. The software is based on the standard version of SixTrack, but is extended to store information about the species of the tracked particle. In order to accurately simulate the magnetic bending for particles with different masses and charges, symplectic tracking maps were derived from a new accelerator Hamiltonian that is adapted for the treatment of isotopes different from the reference species. The new multi-isotopic tracking maps have been studied to verify their symplecticity and implemented in hiSixTrack. The particle tracks simulated with hiSixTrack have been benchmarked against STIER. With hiSixTrack, a new tracking tool became available that allows the symplectic tracking of arbitrary heavy-ion species with their physical momentum, mass and charge. 


Um die Wechswelwirkung der Schwerionen mit den Kollimatoren zu simulieren wurde \mbox{hiSixTrack} mit FLUKA gekoppelt. Als Grundlage f\"{u}r diese Entwicklung diente das SixTrack-FLUKA coupling, das bereits implementiert war um Protonen zwischen SixTrack und FLUKA auszutauschen. Die zugrundeliegende Software wurde angepasst um den Austausch von Schwerionen zwischen hiSixTrack und FLUKA zu erm\"{o}glichen. Heute ist das hiSixTrack-FLUKA coupling die Standardsoftware zur Simulation von Schwerionenkollimation am CERN. Dank der Ber\"{u}cksichtigung von Fragmentierungsprozessen an allen Kollimatoren zeigt das hiSixTrack-FLUKA coupling eine noch bessere \"{U}bereinstimmung mit den gemessenen Daten als STIER. 
\vspace{0.2cm}
%To allow simulating the fragmentation at the LHC collimators, hiSixTrack was coupled with FLUKA. The basis for this development was the SixTrack-FLUKA coupling, which was already implemented to actively exchange protons between both codes. The framework was adapted to exchange heavy-ions between hiSixTrack and FLUKA. Today, the hiSixTrack-FLUKA coupling is the standard simulation software for heavy-ion collimation used at CERN. The cleaning simulations with the hiSixTrack-FLUKA coupling show an even better agreement with the measured data than STIER, mainly due to the inclusion of fragmentation at all collimators. 
 
Das hiSixTrack-FLUKA coupling wurde genutzt um die Effizienz des Kollimationssystems im High Luminosity LHC (HL-LHC) zu simulieren. Aufgrund der, im Vergleich zum LHC, stark vergr\"{o}\ss erten gespeicherten Strahlenergie in HL-LHC ist es f\"{u}r letzteren besonders wichtig die Strahlverluste zu minimieren und kontrollieren. Die Effizienz des Kollimationssystems wurde au\ss erdem  mit zus\"{a}tzlichen Kollimatoren des Typs TCLD simuliert, deren Installation f\"{u}r \mbox{HL-LHC} vorgesehen ist. Die Simulationen zeigen, dass mit der Installation eines TCLD Kollimators pro Strahl das Risiko eines strahlinduzierten Quenches um etwa 45\% reduziert werden kann. Mit zwei TCLD Kollimatoren pro Strahl kann das Risiko eines Quenches sogar um zwei Gr\"{o}\ss enordnungen reduziert werden. 
\vspace{0.2cm}
%The hiSixTrack-FLUKA coupling was applied to study the cleaning performance in the future High Luminosity LHC (HL-LHC). Keeping the cleaning inefficiency at a low and tractable value is particularly important for this scenario, because the stored beam energies are going to be significantly increased. The simulations are therefore also carried out with additional collimators (TCLD) in the dispersion suppressor of the betatron collimation region IR7, which are envisaged to be installed for HL-LHC.  The analysis shows that the cleaning performance can be improved if the new TCLD collimators were installed in the LHC. With one TCLD collimator per beam, the cleaning inefficiency can be reduced by approximately 45\%, if the collimator is set to a sufficiently small half gap. With two TCLD collimators per beam, the quench risk from particles scattered out of the collimators can be reduced by more than two orders of magnitude. 

%The cleaning performance was also simulated with other heavy-ion species than \lead. The study showed that the cleaning efficiency for \iso{129}{Xe}{54+}\, and \iso{40}{Ar}{18+}, which are presently the most probable candidates for operation, is similar to what can be expected with \lead.

Die Kollimationseffizienz wurde auch mit anderen Nukliden als \lead untersucht. Die Simulationen wurden mit \iso{129}{Xe}{54+}\, und \iso{40}{Ar}{18+} durchgef\"{u}hrt und zeigen, dass die erwarteten Strahlverluste  zu jenen bei Betrieb mit \lead vergleichbar sind. 
\vspace{0.2cm}

%\newpage
\thispagestyle{plain}

Im Rahmen dieser Dissertation wurde weiterhin ein Experiment zur Quantifizierung des Risikos eines strahlinduzierten Quenches durch Kollimationsverluste konzipiert und durchgef\"{u}hrt. In diesem Experiment wurden hohe Strahlverluste eines \lead Strahls bei einer Energie von 6.37$\,Z\,$TeV am prim\"{a}ren Kollimator erzeugt. So konnte eine Verlustrate am prim\"{a}ren Kollimator identifiziert werden, bei der die Schwerionenfragmente einen Quench der supraleitenden Magnete induzieren. Dies geschah bei einer Verlustrate von $(15 \pm 1)$~kW. Auf dieser Grundlage kann die maximale gespeicherte Strahlenergie bei 6.37$\,Z\,$TeV abgesch\"{a}tzt werden. Unter Ber\"{u}cksichtigung der minimalen Strahllebenszeit von 12\,min (f\"{u}r die das Kollimationssystem konzipiert ist) ergibt sich f\"{u}r \lead Strahlen mit einer Teilchenenergie von 6.37$\,Z\,$TeV eine Limitierung der gespeicherten Strahlenergie auf $(10.8 \pm 0.8)$~MJ.
\vspace{0.2cm}

%As a part of this thesis, a dedicated heavy-ion collimation quench test was designed and conducted in the LHC, to quantify the quench risk from collimation losses. In this experiment, carried out with \lead beams, high steady state heavy-ion losses at 6.37$\,Z\,$TeV were induced at the primary collimator with the aim of quantifying at which power load on the collimator the superconducting magnets quench, due to the generated heavy-ion fragments. A magnet quench was achieved with a beam loss rate of $(15 \pm 1)$~kW. From this value, an approximate estimate for the maximum stored beam energy at 6.37$\,Z\,$TeV can be derived, which yields $(10.8 \pm 0.8)$~MJ. 

Die gewonnenen Daten wurden anschlie\ss end mit Simulationsergebnissen des hiSixTrack-FLUKA coupling kombiniert um die maximale Strahlenergie f\"{u}r den zuk\"{u}nftigen Betrieb des LHC abzusch\"{a}tzen. Unter Ber\"{u}cksichtigung der Reduzierung der n\"{o}tigen Verlustrate zur Induzierung eines Quenches von 6.37$\,Z\,$TeV zu $7.0\,Z\,$TeV wurde gezeigt, dass die vorgesehene gespeicherte Strahlenergie und damit Luminosit\"{a}t ohne TCLD Kollimatoren nicht erreicht werden kann. Dies gilt, wenn eine \mbox{minimale} Strahllebensdauer von 12\,min angenommen wird. Mit einem TCLD Kollimator pro Strahl k\"{o}nnen die Zielparameter f\"{u}r HL-LHC nur erreicht werden, wenn die \mbox{minimale} Strahllebenszeit gr\"{o}\ss er als 38$\,$min ist. Mit zwei TCLD Kollimatoren kann die \mbox{Verlustrate} so stark reduziert werden, dass die maximale gepeicherte Strahlenergie nicht mehr durch das Risiko eines strahlinduzierten Quenches limitiert wird. Weiterhin wurde festgestellt, dass die beabsichtigte Luminosit\"{a}t fast auch ohne TCLD Kollimatoren erreicht werden kann, wenn der LHC mit \iso{40}{Ar}{18+} betrieben w\"{u}rde. Der Hauptgrund daf\"{u}r ist die im Vergleich zu \lead gr\"{o}\ss ere Bunchintensit\"{a}t, die mit diesen Schwerionen erreicht werden kann.
\vspace{0.2cm}

Abschlie\ss end wurden die verbleibenden Diskrepanzen zwischen Simulation und Messung diskutiert. Die gewonne Erkenntnisse k\"{o}nnen m\"{o}glicherweise in Zukunft genutzt werden um die \"{U}bereinstimmung mit den gemessenen Strahlverlusten weiter zu verbessern. %f\"{u}r zuk\"{u}nftige Verbessungen der Simulationen genutzt werden.
\vspace{0.2cm}
%The experimentally accessed quench limit was combined with the hiSixTrack-FLUKA coupling simulations to derive limitations in terms of stored beam energy for future operation. Taking into account that the quench limit at $7.0\,Z\,$TeV is lower than at $6.37\,Z\,$TeV, the analysis showed that the target stored beam energy and hence luminosity for HL-LHC cannot be reached without TCLD collimators, if a minimum beam lifetime of 12\,min is assumed. With one TCLD collimator per beam, the HL-LHC target can be only reached if the minimum beam lifetime was larger than 38$\,$min, while the design specification is 12\,min. With two TCLD collimators, the cleaning performance is improved significantly, such that the stored beam energy is not limited by the quench risk from collimation losses any more. The analysis further showed that the target luminosity can potentially be reached also without TCLD collimators if the LHC were to be operated with  \iso{40}{Ar}{18+} ions. The main reason for this improvement is the significantly larger bunch intensity compared to \lead. 

%In a brief outlook, remaining discrepancies between simulated and measured loss maps are studied, which can be used for future improvements of the simulation tools. 



\thispagestyle{plain}

Zusammenfassend wurden in dieser Arbeit STIER und das hiSixTrack-FLUKA coupling als neue Software zur Simulation der Effizienz von Schwerionenkollimation vorgestellt. STIER und das hiSixTrack-FLUKA coupling wurden mit gemessenen Daten verglichen, wurden genutzt um physikalische Prozesse im LHC vorherzusagen und lieferten substanzielle Erkenntnisse \"{u}ber die ben\"{o}tigten Erweiterungen des Kollimationssystems f\"{u}r HL-LHC.  
\vspace{0.2cm}
%In summary, this thesis has introduced STIER and the hiSixTrack-FLUKA coupling as new simulation tools for heavy-ion collimation. They have been benchmarked against measured data, were used to make accurate predictions about the machine behaviour,  delivered input for the analysis of experimental data from the LHC and were used to define the required collimation system upgrades for HL-LHC. 

Heute ist das hiSixTrack-FLUKA coupling die Software zur Simulation von Schwerionenkollimation mit der besten \"{U}bereinstimmung zum Experiment. Zuk\"{u}nftige Simulationen mit dem hiSixTrack-FLUKA coupling k\"{o}nnten unverzichtbar sein um potentielle Limitationen durch Strahlverluste zu \"{u}berwinden. Damit k\"{o}nnte das hiSixTrack-FLUKA coupling dazu beitragen das volle Potential des LHC im Schwerionenbetrieb auszusch\"{o}pfen. 

%At the time of writing, the hiSixTrack-FLUKA coupling is the most sophisticated tool to simulate heavy-ion collimation. It is going to be used for future heavy-ion collimation simulations and might be indespensable to understand and avoid potential limitations from beam losses, which need to be overcome to exploit the full potential of the LHC in heavy-ion operation. 
 

\end{otherlanguage}