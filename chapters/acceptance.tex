\chapter{Detector Acceptance}\label{chap:6}
%
\section*{Introduction}
Based upon the presented optical configurations, the detector acceptance for very forward particles can be calculated. This gives information about the measurability of scattered particles as a function of $|t|$ and $\xi$. The results of these calculations depend on numerous assumptions which are indicated for each acceptance distribution. Different parameter sets for the acceptance calculations are discussed and compared.  

%pictures of detectors with particles.
% for simplicity and for efficiancy, detectors could be  upgraded to with second horizontal
  
\section{Approach}\label{chap:acceptance_simulations}
As part of this thesis, two tracking algorithms for the acceptance simulation have been implemented independently, in collaboration with R. Schicker. The algorithms are based on the two different platforms Wolfram Mathematica~\cite{trott2004mathematica} and C~\cite{ritchie1975c}, which gives the possibility to cross-check the simulation algorithms for their correctness. Using the same accelerator settings as input, both algorithms reproduce the same detector acceptances as a function of $|t|$ and $\xi$. The general structure of one simulation at a given $|t|$ and $\xi$ is:
%
\begin{enumerate}
\item Definition of the diffractive parameters $|t|$ and $\xi$ and of the number of particles $n_{\text{IP}}$. 
%
\item Generation of the particle position in a Gaussian distribution as a function of the RMS beam size at the IP. Creation of a random azimuthal angle, and calculation of the resulting $\theta_x^*$ and $\theta_y^*$ for the given value $|t|$ and $\xi$.
%
\item Transport of the particle from the IP to the two RP detector positions by using the known matrix elements of the respective optical configuration.
%
\item Increase a counter if the particle is within the acceptance of both detectors.
%
\item Repeat the points 2. to 4. for a number of $n_{\text{IP}}$ particles. The number of detected particles at fixed $|t|,\xi$ is defined as $n_d(|t|,\xi)$.
%
\item Calculate the acceptance for the given $|t|,\xi$ by 
%
\begin{align}
\mathcal{A}(|t|,\xi) = \frac{n_d(|t|,\xi)}{n_{\text{IP}}} \, .
\end{align}
%
%\item Repeat this for different $|t|,\xi$ values in a chosen range.
\end{enumerate}
%
%The algorithms generate an array of data which can be read out to be visualized by means of external software.
%
% The corresponding properties were assumed, based on empiric information or guesses. A brief overview of the relevant parameters with the assumed properties is given in the following.

The properties of several parameters which are of importance for the acceptance simulations are only vaguely known. Therefore, the corresponding properties are assumed, based on empiric information or guesses. The following passages give a brief overview over the relevant parameters with the assumed properties.


\paragraph{Detector Geometry}
The geometry of the Roman Pot detectors plays a significant role for the measurement. As shown in \figref{rpdesign}, the detectors used for TOTEM and ALFA host two vertical and one horizontal edgeless silicon sub-detector. The detector in the horizontal plane is installed on the outer part of the two beam pipes (see \figref{rp_layout_2}). 

%One reason for this positioning is, the availability of space (contrary to the region between the two pipes of \mbox{Beam 1} and \mbox{Beam 2}). 
\mbox{In the high $\beta^*$-configurations} for TOTEM and ALFA, the phase advances are matched to \mbox{$\Delta \psi_y=0.25$} in the vertical and $\Delta \psi_x=0.5$ in the horizontal plane, so the optics are parallel-to-point focusing in the vertical plane and point-to-point focusing in the horizontal plane~\cite{Hinterberger2008}. The latter configuration is independent of the scattering angle and only depending on the initial position. In vertical direction, the scattering angle is projected into an amplitude $y$, which is scaled by the optical length $L_y$. For a given $|t|$-value, the particles in vertical direction are equally distributed and have the largest amplitude if $\theta_x^* = 0$. The horizontal distribution is a Gaussian, scaled by the factor $v_x$. For a given value of $\xi$, this distribution is horizontally shifted, depending on the local dispersion function $D^{\text{loc}}_x$. Since the local horizontal dispersion is always of the same sign, the installation of a second horizontal detector is unnecessary. The particle tracks on the detector with such a configuration is shown on the left panel of \figref{fig:particles_detector}. 

The ALICE high $\beta^*$-optics, however, were not matched to these phase advances. The particles with given $|t|$-value hit the detector in an elliptic distribution, as shown in the right panel of \figref{fig:particles_detector}. Therefore it would be advantageous to have horizontal measurements on both beam sides. For the acceptance simulations, a Roman Pot with two detectors in the horizontal plane is considered (see \figref{fig:particles_detector}). However, it is not yet known if such a device is producable.
%
\paragraph{Detector Distance from Beam Centre}
The minimum distance of the detector to the beam centre has a large impact on the acceptance. As a general guideline for high-intensity pp-operation, the Roman Pots must be retracted by $3\,\sigma$ with respect to the TCTs~\cite{RSchickerPrivate}. The TOTEM detectors are allowed to be moved to a distance of 14$\,\sigma$ to the beam centre in high-intensity operation~\cite{Deile_Private}. This distance is considered for the ALICE acceptance calculations as well. The impact of different detector distances is discussed.
%
\paragraph{Emittance}
The edgeless silocon detectors are assumed to be at a distance of 14$\,\sigma$ from the beam centre. This is a relative distance which depends on the beam emittance, as shown in \eqref{eq:beam_size_no_dispersion}. Thus, for the diffractive measurements, the smallest possible emittances are desirable.  
%
\minipageswidthtb{IR2Tracking-Comparison05025}{IR2Tracking-Comparison}{
Particle hits in the edgeless silicon detectors for three sets, A, B, and C of $|t|$,$\xi$-values, with $|t|_\text{A} < |t|_\text{B} < |t|_\text{C}$ and $\xi_\text{A} < \xi_\text{B} < \xi_\text{C}$. Left: Optics with $\Delta \psi_x=0.5$ and $\Delta \psi_y = 0.25$. Right: Optics with $0.25<\Delta \psi_{x,y}<0.5$. The 10$\sigma$ beam-profile is shown at the centre.}{fig:particles_detector}{0.46}{t}    
\vspace{-0.5cm}
\paragraph{Outer Boundaries and Beam Pipe Losses}
%
Particles with very large $|t|$-values have large amplitudes in $x$ and $y$ when they arrive at the RP station. The outer limit for the detectability of the particles is assumed to be at a distance of 2.9$\,$cm from the beam centre. Furthermore, losses can occur in the beam pipe before the detector stations. These losses become significant for very large $|t|$ and $\xi$-values. In the presented simulations, those losses are not considered.
\paragraph{}
The \textit{standard conditions} for the acceptance simulations are: 
\begin{itemize}
\item The Design Report emittance at LHC design energy, $\epsilon_N = 3.75 \,\mu$m$\,$rad, is considered.
\item The RPs are located at distances of $\Delta s=-180\,$m and $\Delta s = -220\,$m from the IP for \mbox{Beam 2} and at distances of $\Delta s = +150\,$m and $\Delta s = +220\,$m from the IP for \mbox{Beam 1}.
\item The edgeless silicon detectors in a Roman Pot are installed on both sides, in the horizontal and vertical plane. They are operated at a distance of 14$\,\sigma$ from the beam centre.
\item The outer limit for the detection is a distance of 2.9$\,$cm from the beam centre. The beam pipe losses between IP and RP are not taken into account. 
\end{itemize}

\section{Acceptance Simulations}
The acceptances of the different optical configurations are shown as a function of $|t|$ and $\xi$. The considered range for the fractional momentum loss is \mbox{$0\leq \xi \leq 0.10$}. \mbox{As shown} in \figref{fig:18m_LxLy_rhs}, the local dispersion of the $\beta^*=18\,$m optics takes values up to \mbox{$D^{\text{loc}}=0.12\,$m}. With a momentum loss of $\xi=0.10$, this corresponds to an additional deviation from the ideal trajectory of 1.2$\,$cm. The LHC beam screen provides a horizontal aperture of approximately 47$\,$mm in total~\cite{anashin2004vacuum}. Therefore, particles with small $|t|$-values and a fractional momentum loss of $\xi < 0.1$ will most probably not be lost before the RP. Detailed studies on the particle losses between IP and detector could be done using the tracking module PTC~\cite{schmidt2005mad}. 
\subsection{Nominal Collision Optics}
\minipageswidthtb{Acceptance/acc_a_10_standard_all}{Acceptance/acc_c_10_standard_all2}{Proton acceptance for the nominal $\beta^*=10\,$m optics as a function of $|t|$ and $\xi=\Delta p/p$. \mbox{Left}: Acceptance for the ALICE A-side (Beam 2), right: Acceptance for the C-side (Beam 1). Note the different scale of the $z$-axis with respect to the following acceptance distributions. The contour lines correspond to acceptance steps of 0.05.}{fig:10m_c_p_acc}{0.5}{b}
%
The nominal ALICE optics are not optimized for very forward measurements. The acceptance rises very slowly with increasing $|t|$ (see \figref{fig:10m_c_p_acc}). In order to be able to compare the acceptance of different optics to each other, the $|t|$-value with $\mathcal{A}=50\%$, for $\xi=0$, denoted by $|t|_{50\%}$, \mbox{is considered as a reference}. For the measurement on the ALICE A-side, this threshold is reached at $|t|_{50\%} \approx 18\,$GeV$^2$, for the measurement on the ALICE C-side at $|t|_{50\%} \approx 13 \,$GeV$^2$. The acceptances of the two sides are different, which is can be explained by the different detector positions left and right and the rather non-symmetric optics at the standard configuration.

\subsection{The $\beta^*=18\,$m Optics}

\begin{figure}[b]
\begin{minipage}[t]{.5\textwidth}
%\centering
\includegraphics[width=\textwidth]{pictures/Acceptance/18m_a_acc_standard_all}
\includegraphics[width=\textwidth]{pictures/Acceptance/18m_a_acc_standard_t1_xi010}
\end{minipage}
\hfill
\begin{minipage}[t]{0.5\textwidth}
\centering
\includegraphics[width=\textwidth]{pictures/Acceptance/18m_c_acc_standard_all2}
\includegraphics[width=\textwidth]{pictures/Acceptance/acc18_c_t_0_1}
\end{minipage}
\caption{Acceptances of the $\beta^*=18\,$m optics for different $|t|$- and $\xi=\Delta p/p$ ranges in standard conditions for the ALICE A-side (left) and the ALICE C-side (right). The contour lines correspond to acceptance steps of 0.05.}
\label{fig:18m_acc_ac}
\end{figure}
\figref{fig:18m_acc_ac} shows the acceptance of the $\beta^*=18\,$m optics at the two sides of IP2. The reference of $\mathcal{A}=50\%$ for elastic events is reached at $|t|_{50\%}=0.44\,$GeV$^2$ on both, ALICE A-side and C-side. The acceptances of the two sides do not differ significantly. A region with an acceptance of $\mathcal{A}=100\%$ is in a range of $0.7\,$GeV$^2 \leq |t| \leq 4.1\,$GeV$^2$ for elastic events. This region reaches over a broader range in $\xi$, for the measurements on the ALICE A-side. Due to the elliptic shapes of the particle distributions with given $(|t|,\xi)$, an acceptance of $\mathcal{A}=100\%$ with small \mbox{$\xi$-values} can only be reached if the beam is instrumented on both sides in the horizontal direction. \mbox{The acceptances with only one horizontal detector} are discussed in \chapref{chap:acceptance_with_one_horizontal_detector}.


\figref{fig:10_18_comparison} shows a comparison of the acceptance evolution for the nominal ALICE optics and the $\beta^*=18\,$m optics in the range  $0\,$GeV$^2 \leq |t| \leq 16\,$GeV$^2$ with $\xi=0$. It is clearly visible that the matched optics improve the measurability at very low $|t|$-values in comparison to the the nominal optics. The drop of the acceptance for $|t|>4\,$GeV$^2$ is due to the outer limit of the detectability. The acceptance evolution for $|t|>$4$\,$GeV$^2$ shows a step at $|t| \approx 8\,$GeV$^2$, which can be traced back to the condition that the particle should be detectable in both RP stations.

With the given boundary conditions (standard conditions), an acceptance of $\mathcal{A}=100\%$ can not be reached with the $\beta^*=10\,$m optics.

For the ALICE A-side, the corresponding acceptance evolution is not shown. However, as shown in \figref{fig:18m_acc_ac}, the acceptance evolution for the measurement on the ALICE A-side with the $\beta^*=18\,$m optics is similar to the measurement on the ALICE C-side. 
 
%
\begin{figure}[t]
  \centering
  \includegraphics[width=0.7\textwidth]{pictures/Acceptance/10m_18m}
  \caption{Comparison of the acceptances for elastic events ($\xi=0$) with the nominal $\beta^*=10\,$m optics and with the $\beta^*=18\,$m optics for the measurement on the ALICE C-side.}
  \label{fig:10_18_comparison} 
\end{figure}
%
\newpage
\subsection{The $\beta^*=30\,$m Optics}
\vspace{-0.7cm}
\begin{figure}[h]
\begin{minipage}[t]{.5\textwidth}
%\centering
  \includegraphics[width=\textwidth]{pictures/Acceptance/30m_c_acc_standard_all}
\end{minipage}
\hfill
\begin{minipage}[t]{0.5\textwidth}
\centering
\includegraphics[width=\textwidth]{pictures/Acceptance/30m_c_acc_standard_t1_xi10}
\end{minipage}
\caption{Acceptance of the $\beta^*=$30$\,$m optics as a function of $|t|$ and $\xi=\Delta p/p$ in different $|t|$-ranges for the ALICE C-side. The contour lines correspond to acceptance steps of 0.05.}
\label{fig:30m_acc_ac}
\end{figure}
The detector acceptance as a function of $|t|$ and $\xi$ for the $\beta^*=$30$\,$m optics is shown for the ALICE C-side in \figref{fig:30m_acc_ac}. The acceptance distribution is comparable to the calculations for the $\beta^*=18\,$m optics, but shifted to smaller $|t|$-values. The direct comparison between the two optics is shown for elastic events in \figref{fig:18_30_comparison}. With this optical configuration, the $\mathcal{A}=50\%$ threshold is already reached for $|t|_{50\%}=0.24\,$GeV$^2$.
\vspace{-0.3cm}
\begin{figure}[h]
  \centering
  \includegraphics[width=0.7\textwidth]{pictures/Acceptance/18m_30m_bw}
  \caption{Comparison of the acceptances for elastic events ($\xi=0$) on the ALICE C-side with the $\beta^*=18\,$m optics and the $\beta^*=30\,$m optics.}
  \label{fig:18_30_comparison} 
\end{figure}

\section{Special Conditions}\label{chap:non_standard_conditions}
The shown acceptances were calculated with the standard \mbox{conditions defined in \chapref{chap:acceptance_simulations}}. \mbox{In the following chapter, the individual parameters} are varied separately in reasonable limits, and the impact on the acceptance is discussed. This gives an impression on the possible uncertainties on the acceptance by the choices which were made. With the exception of the detector geometry, \mbox{all assumptions in the standard} conditions are conservative choices which lead to the worst possible acceptance for a given optical configuration.
\subsection{Beam Emittance}

The considered standard emittance was the Design Report value of \mbox{$\epsilon_N = 3.75\,\mu$m~rad}. As it will be discussed in \chapref{chap:different_emittance_scenarios}, lower emittances than this are reachable in the LHC. 
\mbox{\figref{fig:acc_func_emittance} shows the acceptance} of the $\beta^*=$18$\,$m optics for \mbox{Beam 1} with different emittances while all other parameters remain at standard conditions. As expected, the $|t|_{50\%}$-value decreases with decreasing emittance. With the smallest considered emittance of $\epsilon_N=2.25\,\mu$m$\,$rad, the $\mathcal{A}=$50$\,$\% threshold is reached for \mbox{$|t|_{50\%}=0.26\,$GeV$^2$}.
\begin{figure}[h]
\centering
\includegraphics[width=0.7\textwidth]{pictures/Acceptance/18m_emittance_comparison_bw}
    \caption{Acceptance of the $\beta^*=18\,$m optics on the ALICE C-side for elastic events with different emittances. The black solid line corresponds to the standard conditions.}
    \label{fig:acc_func_emittance}
\end{figure} 
\newpage
\subsection{Detector Distance}
In the standard conditions, a distance of $D_{\text{db}}=14 \,\sigma$ between the beam centre and the detector edge is considered. This was outlined to be a conservative value~\cite{RSchickerPrivate}%r.schicker, private discussion march 2013
. The acceptance at the ALICE C-side for elastic events with three different detector distances \mbox{($D_{\text{db}}=$10$\,\sigma$, $12\,\sigma$, $14\,\sigma$)} is shown in \figref{fig:acc_func_sigma} while using standard conditions for the remaining parameters. 

The $|t|_{50\%}$-value decreases with decreasing detector distance. With the smallest considered detector distance of $D_{\text{db}}=10\,\sigma$, the $\mathcal{A}=50\%$ threshold is reached for $|t|_{50\%}\approx 0.24\,$GeV$^2$.   
\begin{figure}[h]
\centering
\includegraphics[width=0.7\textwidth]{pictures/Acceptance/18m_detector}
    \caption{Acceptances of the $\beta^*=18\,$m optics for elastic events on the ALICE C-side for different detector distances $D_{\text{db}}$ of the edgeless silicon detectors from the beam centre. \mbox{The black, solid line represents} the acceptance in standard conditions.}
    \label{fig:acc_func_sigma}
\end{figure}

\subsection{Detector Geometry}\label{chap:acceptance_with_one_horizontal_detector}
\begin{figure}[h]
\begin{minipage}[t]{.5\textwidth}
%\centering
  \includegraphics[width=\textwidth]{pictures/Acceptance/18m_c_acc_3_detectors_0_12}
\end{minipage}
\hfill
\begin{minipage}[t]{0.5\textwidth}
\centering
\includegraphics[width=\textwidth]{pictures/Acceptance/18m_c_acc_3_detectors_0_1}
\end{minipage}
\caption{Acceptance of the $\beta^*=18\,$m optics on the ALICE C-side with one instead of two horizontal detectors. Note the different $z$-axis ranges between left and right. The contour lines correspond to acceptance steps of 0.05.}
\label{fig:non_standard_detector}
\end{figure}
The detector layout which is considered in the standard conditions presumes one horizontal detector at either of the two sides of the beam (for $x<0$ and $x>0$).  However, this is a \mbox{hypothetical}, ideal detector layout, which is not applied in the Roman Pot detectors used by TOTEM and ALFA (\figref{fig:particles_detector} shows the detector layout in the standard conditions and \figref{rpdesign} shows the detector layout of the RPs which are installed in IR1 and IR5). It is not yet known whether the installation of a second set of horizontal sub-detectors in the Roman Pot stations is feasible or not. The detector layout in the standard conditions was an optimistic assumption.

\newpage

The result of the acceptance calculations, using the detector design of the existing RPs with  one set of edegeless silicon detectors in horizontal direction~\cite{LHCAcceleratorExperiments}, is shown in \figref{fig:non_standard_detector}. The horizontal set of the edgeless silicon sub-detectors is installed at the outer side of the respective beam pipe\footnote{In this sense, the outer side  means the horizontal side of the beam pipe which is not between the two pipes, in which \mbox{Beam 1} and \mbox{Beam 2} are moving, respectively.}. For the measurements with Beam 1 \mbox{(ALICE C-side)}, the horizontal detector would be placed at the side with $x>0$, and vice versa for the measurement with Beam 2 \mbox{(ALICE A-side)}.

The sign of the horizontal vertical dispersion function at the RP location is such, that particles with momentum loss $\xi>0$ are deviated towards the remaining horizontal edgeless silicon detector. Furthermore, the vertical detectors have an overlap with the horizontal detectors, as shown in \figref{rpdesign}. Thus, the corrected acceptance $\mathcal{A}_c$ is related to the standard acceptance $\mathcal{A}$ by
\begin{align}
\mathcal{A}_c > 0.75\,\mathcal{A} \, .
\end{align}
This scaling is clearly visible when comparing the acceptance in standard conditions in \figref{fig:30m_acc_ac} with the corrected acceptances shown in \figref{fig:non_standard_detector}.  \mbox{The $\mathcal{A}=$50$\,$\% threshold for} elastic events is reached at $|t|_{50\%}=0.49\,$GeV$^2$ with standard conditions being applied for the remaining parameters.

\newpage
\section{Conclusion}
The acceptance studies show that the detector acceptance of the RPs is significantly improved with the \mbox{$\beta^*=18\,$m} optics, compared to the nominal $\beta^*=10\,$m configuration in ALICE. \mbox{However}, a clear statement on the reachable $|t|$-acceptance can not be given, because the relevant parameters are not fully known yet. Based on different assumptions on the beam parameters, the detector layout, and the possible detector settings, a possible $|t|_{50\%}$-range of
\begin{align}
0.24 \, \text{GeV}^2 \leq |t|_{\text{50}\%} \leq 0.49 \, \text{GeV}^2 \, ,
\end{align}
was deduced for the measurements on the ALICE C-side with the $\beta^*=18\,$m optics. The lower limit is rather optimistic, even if combinations of optimistic parameters (e.g. reduced emittance and smaller detector distance in terms of $\sigma$ at the same time) are not covered. Especially due to fact that the optics are intended to be implemented during high-intensity operation, the detector-distance from the beam centre should be evaluated very conservatively.  

The same considerations for the $\beta^*=30\,$m optics lead to an acceptance range
%
\begin{align}
0.13 \, \text{GeV}^2 \leq |t|_{\text{50}\%} \leq 0.28 \, \text{GeV}^2 \, .
\end{align}
%
Since no concrete strategy for the required phase advance compensation has been worked out yet, the acceptance of rematched optics with $\beta^*=18\,$m, which require lower phase advance compensations, is discussed in \chapref{chap:considerations_on_ir2_phase_advance}.

