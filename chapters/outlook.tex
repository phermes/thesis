\chapter{Future LHC Configurations}






\section{Heavy-Ion Operation in 2016}


% put cleaning inefficiency in context with the quench limit and stored beam energy
% show beamlinegraphics with and without tclds
% compare settings used in STIER with hiSixTrack [12sigma]





\section{HL-LHC Upgrade with TCLD Collimators} \label{chap:hllhc}
%


\subsection{LIU and High Luminosity LHC}

For the future of the LHC, a drastic increase of the luminosity is foreseen to increase the potential for discoveries. The current baseline foresees the commissioning of the High Luminosity LHC (HL-LHC) adapted for a luminosity five times beyond the design value as of 2025, after a period of dedicated hardware upgrades~\cite{hllhc:asdhep}.
%
	\begin{table}[b]
		\centering
		\caption{Comparison of heavy-ion beam parameters: design parameters~\cite{CERN-2004-003-V1}, achieved in the 2015 heavy-ion run, envisaged parameters for LIU~\cite{IPAC11:WEPS017,IPAC2015:THPF093} and HL-LHC~\cite{}}
		\label{tab:LIU_HL_Parameters}
		
		\begin{tabular}{cccccccc}
			\toprule
			                        & $E$             & $n_B$   & $\epsilon_N$  & $I_B$             & $E^\text{tot}$     \\ %\midrule
			                        & {[}$Z$ TeV{]}	  &         &  $\mu$m rad   &  [$10^7$ ions]     & [MJ]             \\ \midrule
			Design                  & 7.0             & 592     & 3.75 & $7$               & 3.81             \\
			2015                    & 6.37            & 518     & 1.5$\pm 0.15$ & 22$\pm3$              & 9.54$\pm1.30$             \\
			LIU baseline            & 7.0             & 1152    & 1.2  & $17$              & 18.0             \\
			HL-LHC request          & 7.0             & 1248    & 1.3 & $21 $             & 24.1             \\ \bottomrule
		\end{tabular}
	\end{table}
%
An essential keystone for this aim is the increase of the bunch intensity and the possible number of bunches that can be stored in the machine. This is made possible by the LHC injectors upgrade (LIU)~\cite{IPAC11:WEPS017,IPAC2015:THPF093}, a modification of the injector chain with hardware upgrades and even the replacement of LINAC2 by a new H$^-$ accelerator LINAC4~\cite{IPAC2015:THPF093}. The schedule to reach the HL-LHC parameters foresees a stepwise luminosity increase in which double of the design luminosity shall be reached after the LIU in long shutdown 2~\cite{hllhc:asdhep}.

The luminosity increase is envisaged for both, proton and heavy-ion beams. The relevant parameters are compared for the design values, the values achieved in 2015 and present optimistic estimates for the LIU and HL-LHC upgrade in \tabref{tab:LIU_HL_Parameters}. 

This comparison makes apparent that the stored beam energies are envisaged to increase drastically to values more than six times the deisgn values. Note that the target bunch intensity for LIU has already been exceeded in the 2015 heavy-ion run. Inevitably, the increased stored beam energy will increase the collimation losses, especially in the IR7 DS regions. The collimation quench test unveiled that with the present collimation system the largest acceptable stored beam energy is very close to the values yet achieved in the 2015 heavy-ion run. 


\begin{figure}[t]  
    \centering
    \includegraphics[width=1.0\textwidth]{pictures/16042602.pdf}
    \caption{LHC beam line sequence in IR7. Top: Nominal sequence. Middle: sequence with the main dipoles in cell 8 and cell 10 replaced by two 11~T dipoles and the TCLD collimators in between. Bottom: horizontal dispersion function.}  
    \label{pic:16042602}
    %/home/phermes/Dropbox/codes/madx/160426_blg_hllhc/tcld.pdf
\end{figure}

The safe and uninterrupted operation with HL-LHC intensities shall be ensured by means of additional collimators in the DS region downstream of IR7. The required space would be made available by replacing a main dipole with a maximum magnetic field of 8.3~T dipole by two stronger dipoles (with a maximum field strength of 11~T). In the center of the two dipoles a new collimator, denominated TCLD, could be installed (see \figref{pic:16042602}). To provide a large cross section for absorption, the present layout foresees tungsten as the active material of the TCLD collimators. In view of the two most critical loss locations in the cells 9 and 11 downstream of IR7, the proposal foresees the new collimators to be installed immediately upstream of them in the cells 8 and 10~\cite{hb08:wgd08,Bruce2014}.

As presented in \cite{Bruce2014}, the TCLD collimators have been demonstrated to significantly improve the cleaning inefficiency for proton beams. In this chapter the cleaning inefficiency in the HL-LHC configuration is studied with STIER and hiSixTrack for heavy-ion beams with and without the TCLD collimators. Given that especially the construction of the the stronger dipole magnets requires significant financial resources, the present estimate for the integrated cost of the installation of one TCLD is presently approximately 7 mio CHF. Therefore, a solution in which only one TCLD is required (installed in cell 8), is desirable. The presented  simulations are therefore carried out for three cases to compare the limiting cleaning inefficiency and evaluate the risk of quenches for the three scenarios:
%
\begin{itemize}
  \item 0 TCLDs: reference simulation for the present layout without additional collimators,
  \item 1 TCLD: layout with an additional TCLD collimator in cell 8, 
  \item 2 TCLDs: two TCLD collimators are installed cell 8 and cell 10. 
\end{itemize}
%




\subsection{Consistency Check with STIER}


\begin{figure}[t]
  \centering
  \begin{tikzpicture}
    \node[anchor=south west,inner sep=0] (image) at (0,0) {\includegraphics[width=0.99\linewidth]{pictures/16072301.pdf}};
    %\node [draw,rotate=90,x={(image.south east)},y={(image.north west)}]                   at (0.50,0.50)    {text0};
    %\node [draw,rotate=0 ,x={(image.south east)},y={(image.north west)}]                   at (0.22,0.96)    {text1};
    %\node [draw,rotate=0 ,x={(image.south east)},y={(image.north west)},anchor=west]       at (0.22,0.80)    {text2};
    %\draw[->,color=black,thick,x={(image.south east)},y={(image.north west)}]             (0.42,0.22) -- (0.37,0.23);
  \end{tikzpicture}
  \caption{Cleaning inefficiency of HL-LHC at 7$\,Z\,$TeV simulated with STIER (top) and hiSixTrack(bottom).}  
  \label{pic:16072301}
  %/afs/cern.ch/work/p/phermes/private/STIER_results/analysis/TCLD_comparison_IR7_HS_ST.pdf
  \end{figure}

The first simulation of the cleaning inefficiency with heavy-ions in the HL-LHC configuration was conducted with the STIER simulation tool . The optics used for the simulation is identical to the settings used in the 2011 heavy-ion run, due to a lack of concrete optics proposals at the time the simulation was conducted. Hitherto a change of the optics in IR7 is not envisaged. Therefore, the cleaning inefficiency simulated in the IR7 DS should not be affected by eventual optics changes coming up in the experimental IRs. The collimator half gaps applied in the simulation are the design settings summarized in \tabref{tab:14070901} and the TCLD collimators set to a half gap of $10\,\sigma$. The presented simulations are carried out for B1H with $10^6$ initial \lead ions at an energy of 7$\,Z\,$TeV, impacting the TCP at an impact parameter of 10$\,\mu$m. The STIER simulation with these parameters predicts a significant improvement of the cleaning performance with the TCLD collimators, as presented in detail in~\cite{phermes_ipac2015_1}. In this section, only the simulation result without DS collimators is presented to validate the hiSixTrack simulations. In the next sections the cleaning performance with TCLD collimators shall be presented in a more thorough analysis with hiSixTrack, based on more recent estimates for the collimator settings, optics and including also secondary particles generated in the TCLDs. 


The cleaning inefficiency in the HL-LHC configuration without TCLD collimators as simulated with STIER and hiSixTrack is shown in \figref{pic:16072301}. Both simulations take into account identical settings of the collimators with the TCP at 6.0$\,\sigma$ and the same impact parameter $b=10\,\mu$m at the TCP jaw. The simulated losses at the TCP and the TCSG collimators downstream is different between the two simulations, with the reasons discussed in \chapref{lm:benchmark}. The same reason applies for the apparent discrepancy of the loss rate in the IR7 DS clusters, which are simulated to be smaller in hiSixTrack than in STIER. The more accurate accounting  of the TCP losses leads to a higher loss rate at the latter, leading to an apparent reduction of the losses at the remaining locations. 

The shape of the DS loss clusters is, however, similar in both simulations confirming the accuracy of hiSixTrack. Additional losses are simulated with the latter slightly upstream of the DS1 cluster, which could be traced back to secondary fragments generated in the TCSG collimator. The same applies for the apparent broadening of the A1 loss peak in hiSixTrack. In conclusion, the STIER result can be reproduced with hiSixTrack with additional features expected from the changed simulation apporach.  




% \begin{figure}[t]
%   \centering
%   \begin{tikzpicture}
%     \node[anchor=south west,inner sep=0] (image) at (0,0) {\includegraphics[width=1.0\linewidth]{pictures/16061303.pdf}};
%     %
%     %  \node [draw,rotate=90,x={(image.south east)},y={(image.north west)}]  at (0.50,0.50)  {text0};
%     %  pure text 
%     %  \node [draw,rotate=0 ,x={(image.south east)},y={(image.north west)}]       at (0.22,0.965)  {text1};
%     %  \node [draw,rotate=0 ,x={(image.south east)},y={(image.north west)},anchor=west]       at (0.22,0.8)    {text2};
%     %
%   \end{tikzpicture}
%   \caption{Zoom to IR7 of the cleaning inefficiency simulated with STIER in B1H for HL-LHC without, with one and with two TCLD collimators.}  
%   \label{pic:16061303}
%   %/afs/cern.ch/work/p/phermes/private/STIER_results/analysis/TCLD_comparison_IR7.pdf
%   \end{figure}


% \begin{figure}[t]
%   \centering
%   \begin{tikzpicture}
%     \footnotesize
%     \node[anchor=south west,inner sep=0] (image) at (0,0) {\includegraphics[width=1.0\linewidth]{pictures/16061304.pdf}};
%     %
% %    \draw[x={(image.south east)},y={(image.north west)},step=0.1] (0.0,0.0) grid (1.0,1.0);
%     %  pure text 
%     \node [x={(image.south east)},y={(image.north west)}]       at (0.90,0.96)  {0 TCLDs};
%     \node [x={(image.south east)},y={(image.north west)}]       at (0.90,0.34)  {2 TCLDs};
%    \node [x={(image.south east)},y={(image.north west)}]       at (0.90,0.65)  {1 TCLD};
%     %  \node [draw,rotate=0 ,x={(image.south east)},y={(image.north west)},anchor=west]       at (0.22,0.8)    {text2};
%     %
%   \end{tikzpicture}
%   \caption{Cleaning inefficiency in the full LHC ring simulated with STIER in B1H for HL-LHC without, with one and with two TCLD collimators.}  
%   \label{pic:16061304}
%   %/afs/cern.ch/work/p/phermes/private/STIER_results/analysis/TCLD_comparison_LHC.pdf
%   \end{figure}







\subsection{Simulation with hiSixTrack}

\begin{table}[htbp]
\centering
\caption{Collimator settings in $\sigma$ used for the HL-LHC simulations in hiSixTrack.}
\label{tab:sets_hl}
\begin{tabular}{cccccc} 
\toprule
IR7    &       &   IR3   &          &    \multicolumn{2}{c}{IR1/2/5/8/6} \\ \midrule
TCP    &  5.7  &   TCP   &  15.0    &    TCT  &   10.9 \\ 
TCSG   &  7.7  &   TCSG  &  18.0    &    TCL  &   12.0 \\
TCLA   & 10.0  &   TCLA  &  20.0    &   TCDQ  &   9.0  \\ 
       &       &         &          &   TCSP  &   8.5  \\ \bottomrule
\end{tabular}
\end{table}

Compared to the STIER simulation that was carried out for one particular plane, with a defined impact parameter and fixed TCLD settings~\cite{phermes_ipac2015_1}, the hiSixTrack simulation is carried out for a variety of different settings to define the worst case scenario and thoroughly study the cleaning performance for this case. First, the cleaning performance without TCLD collimators is evaluated for a defined impact parameter for the both planes and beams. The most critical plane is chosen and the cleaning inefficiency is systematically evaluated for different impact parameters. Again the most critical impact parameter is selected and the cleaning inefficiency is compared with and without one or two TCLD collimators. The study is conducted for different possible half gaps of the TCLD collimators. At the time of study, the decision of installing one TCLD collimator was already taken, based on the requirements for proton beams. 

The main aim of this study is thus to clarify whether the heavy-ion losses can be reduced to a satisfactory low extent with only one TCLD, such that the target beam intensity can be achieved. All simulations are carried out with the HL-LHC optics V.1.2 and the collimator settings summarized in \tabref{tab:sets_hl}.

\subsubsection{Selection of Beam and Plane}


\subsubsection{Impact Parameters}


Based on the results of the previous simulations, the simulation is carried out for B1H with different impact parameters. The IR7 loss maps simulated for $b=0.1\,\mu$m, $1.0\,\mu$m, $10.0\,\mu$m and $100\,\mu$m are shown in \figref{pic:16062201}. 


\begin{figure}[t]
  \centering
  \begin{tikzpicture}
    \node[anchor=south west,inner sep=0] (image) at (0,0) {\includegraphics[width=1.0\linewidth]{pictures/16072501.pdf}};
    %\node [draw,rotate=90,x={(image.south east)},y={(image.north west)}]                   at (0.50,0.50)    {text0};
    %\node [draw,rotate=0 ,x={(image.south east)},y={(image.north west)}]                   at (0.22,0.96)    {text1};
    %\node [draw,rotate=0 ,x={(image.south east)},y={(image.north west)},anchor=west]       at (0.22,0.80)    {text2};
    %\draw[->,color=black,thick,x={(image.south east)},y={(image.north west)}]             (0.42,0.22) -- (0.37,0.23);
  \end{tikzpicture}
  \caption{Comparison of the cleaning inefficiency simulated with the hiSixTrack-FLUKA coupling for four different impact parameters \mbox{$b=0.1\mu$m, $1.0\,\mu$m, 10.0$\,\mu$m, 100.0$\,\mu$m}. }  
  \label{pic:16062201}
  %/media/phermes/local/hisix_results/HLLHC/B1H/analysis/postprocessing/output/comparison_impact_IR7.pdf
  \end{figure}


\begin{figure}[htbp]
  \centering
  \begin{tikzpicture}
    \footnotesize
    \node[anchor=south west,inner sep=0] (image) at (0,0) {\includegraphics[width=1.0\linewidth]{pictures/16062701.pdf}};
    \node [x={(image.south east)},y={(image.north west)},anchor=west]       at (0.25,0.92)    {$^{3}$H$^{1+}$};
    \draw[->,color=black,thick,x={(image.south east)},y={(image.north west)}]             (0.28,0.91) -- (0.315,0.90);
    \node [fill=white,x={(image.south east)},y={(image.north west)},anchor=west]       at (0.37,0.93)    {$b=0.1\,\mu$m};
    \node [fill=white,x={(image.south east)},y={(image.north west)},anchor=west]       at (0.37,0.71)    {$b=1.0\,\mu$m};
    \node [fill=white,x={(image.south east)},y={(image.north west)},anchor=west]       at (0.37,0.47)    {$b=10\,\mu$m};
    \node [fill=white,x={(image.south east)},y={(image.north west)},anchor=west]       at (0.37,0.25)    {$b=100\,\mu$m};
    %\draw[->,color=black,thick,x={(image.south east)},y={(image.north west)}]             (0.42,0.22) -- (0.37,0.23);
  \end{tikzpicture}
  \caption{Distribution of $\chi$ and $A$ in the DS1 (left) and DS2 (right) clusters with different impact parameters from $b=0.1\,\mu$m to $100\,\mu$m.}  
  \label{pic:16062204}
  %/media/phermes/local/hisix_results/HLLHC/B1H/analysis/postprocessing/chi_A_DS1_DS2.pdf
  \end{figure}

The comparison shows, as for the STIER study case at 3.5\,$Z\,$TeV qualitatively similar loss patterns, which are quantitatively different. The worst cleaning inefficiency with the highest losses at the IR7 DS magnets is obtained for an impact parameter of $b=1.0\,\mu$m. In \figref{pic:16062204}, the distribution of $\chi$ and $A$ in both loss clusters DS1 and DS2 is shown for the four impact parameters. Both distributions are normalized to yield one if the values of all data points are integrated. In all studied cases, the distribution of $\chi$ in the DS1 cluster is broader than in the DS2 cluster, in line with the observation made in the STIER simulation in the 2011 configuration. Pronounced peaks are visible in all simulations from the isotope \iso{3}{H}{1+} with an amplitude increasing from $b=1\,\mu$m to $b=100$\mum, but larger for 0.1\mum than for 1.0\mum. This observation is, like the better cleaning inefficiency for 0.1\mum than with 1.0\mum unexpected, because the fragmentation rate should be lower for smaller impact parameters. This behaviour is studied in detail later-on in this section. 



\begin{figure}[t]
  \centering
  \begin{tikzpicture}
    \node[anchor=south west,inner sep=0] (image) at (0,0) {\includegraphics[width=0.6\linewidth]{pictures/16070806.pdf}};
    %\node [draw,rotate=90,x={(image.south east)},y={(image.north west)}]                   at (0.50,0.50)    {text0};
    %\node [draw,rotate=0 ,x={(image.south east)},y={(image.north west)}]                   at (0.22,0.96)    {text1};
    %\node [draw,rotate=0 ,x={(image.south east)},y={(image.north west)},anchor=west]       at (0.22,0.80)    {text2};
    %\draw[->,color=black,thick,x={(image.south east)},y={(image.north west)}]             (0.42,0.22) -- (0.37,0.23);
  \end{tikzpicture}
  \caption{Cleaning inefficiency for B1H for different impact parameters.}  
  \label{pic:16062203}
  %/media/phermes/local/hisix_results/HLLHC/B1H/analysis/postprocessing/output/eta_vs_b.pdf
  \end{figure}


A further observation is the diminished density in the $\chi$ histogram with increasing impact parameter. The losses for the larger impact parameters are caused by less different isotopes which is also visible in the distribution of nuclear mass numbers.  For the largest impact parameter, the largest fradction of the losses arises from tritium ions with the spectrum of isotopes lost in the DS ending at $A=85$. Along with this observation comes the larger amount of energy lost in the TCP, leading to a reduced cleaning inefficiency in the DS clusters. The cleaning inefficiency does not change significantly from $b=10$\mum to 100\mum, because at 10\mum the primary heavy-ions traverse almost the full length of the collimator (for the presented setup they do so at $b=13$\mum). 

Additional cleaning simulations show a complete picture of the cleaning evolution with the impact parameter. The result is shown in \figref{pic:16062203} and confirms the expectation of better cleaning performance at larger impact parameters. However, the better cleaning performance at $b=0.1$\mum than at 1.0\mum is not understood. 

In the assumption that the underlying physical process is related the secondary impact of out-scattered \lead ions at the TCP in subsequent turns, a dedicated simulation was conducted. To avoide eventual effect of the phase space shift with the associated increase of the impact parameter (see \chapref{chap:pha_shift}), the simulations are repeated for 0.1\mum and 1.0\mum with the sextupoles switched off. In this scenario, the cleaning inefficiency for $b=1.0$\mum is still worse than for 0.1\mum. The distribution of impact parameters at the TCP for impacts at subsequent turns is shown in \figref{fig:b_subs}. 













% \begin{figure}[htbp]
%   \centering
%   \begin{tikzpicture}
%     \node[anchor=south west,inner sep=0] (image) at (0,0) {\includegraphics[width=0.9\linewidth]{pictures/16070805.pdf}};
%     %\node [draw,rotate=90,x={(image.south east)},y={(image.north west)}]                   at (0.50,0.50)    {text0};
%     %\node [draw,rotate=0 ,x={(image.south east)},y={(image.north west)}]                   at (0.22,0.96)    {text1};
%     %\node [draw,rotate=0 ,x={(image.south east)},y={(image.north west)},anchor=west]       at (0.22,0.80)    {text2};
%     %\draw[->,color=black,thick,x={(image.south east)},y={(image.north west)}]             (0.42,0.22) -- (0.37,0.23);
%   \end{tikzpicture}
%   \caption{Figure caption}  
%   \label{pic:16070804}
%   %/media/phermes/local/hisix_results/HLLHC/B1H/analysis/postprocessing/impactpar.pdf
%   \end{figure}






% \subsubsection{Cleaning with 1 TCLD Collimator at different settings}



% \begin{figure}[htbp]
%   \centering
%   \begin{tikzpicture}
%     \node[anchor=south west,inner sep=0] (image) at (0,0) {\includegraphics[width=1.0\linewidth]{pictures/16072201.pdf}};
%     %\node [draw,rotate=90,x={(image.south east)},y={(image.north west)}]                   at (0.50,0.50)    {text0};
%     %\node [draw,rotate=0 ,x={(image.south east)},y={(image.north west)}]                   at (0.22,0.96)    {text1};
%     %\node [draw,rotate=0 ,x={(image.south east)},y={(image.north west)},anchor=west]       at (0.22,0.80)    {text2};
%     %\draw[->,color=black,thick,x={(image.south east)},y={(image.north west)}]             (0.42,0.22) -- (0.37,0.23);
%   \end{tikzpicture}
%   \caption{Cleaning inefficiency with one TCLD collimator set to half gaps of $10\,\sigma$ (top), $12\,\sigma$ (middle) and $15\,\sigma$ (bottom).}  
%   \label{pic:16072201}
%   %/media/phermes/local/hisix_results/HLLHC/B1H/analysis/postprocessing/comparison_1TCLD_IR7.pdf
%   \end{figure}









% \subsubsection{hiSixTrack Simulation}

% With $\beta^*=0.48\,$m in IP1, and IP5, $\beta^*=0.50\,$m in IP2 and $\beta^*=3.00\,$m in IP8.





% \begin{figure}[t]  
%     \centering
%     \includegraphics[width=1.0\textwidth]{pictures/16053001.pdf}
% %    \includegraphics[width=0.9\textwidth]{pictures/16053001.pdf}
%     \caption{Figure caption}  
%     \label{pic:16053001}
%     %/media/phermes/ph3tboffice/ph1tbwd/hisix_results/HLLHC/B1H/analysis/postprocessing/examples/TCLD_comparison_LHC.pdf
% \end{figure}



% \begin{figure}[t]  
%     \centering
%     \includegraphics[width=1\textwidth]{pictures/16053003.pdf}
%     \caption{Figure caption}  
%     \label{pic:16053002}
%     %/media/phermes/ph3tboffice/ph1tbwd/hisix_results/HLLHC/B1H/analysis/postprocessing/examples/TCLD_comparison_IR7.pdf
% \end{figure}

\section{Operation with other Heavy-Ion Species}

The LHC injectors are capable of providing heavy-ios of species different from \lead to the LHC. The NA61/Shine experiment has so far received \iso{40}{Ar}{18+} ions for fixed target experiments and \iso{129}{Xe}{54+} ions are scheduled for future operation \cite{EDMS:1570447,CERN-2014-006}. 

The cleaning performance with these ion species are carried out in the HL-LHC configuration with the optical and collimator settings identical to those used in \chapref{chap:hllhc} with the primary beam impacting with $b=1\,\mu$m. The loss maps simulated for the different heavy-ion species and for protons are compared in \figref{pic:16072601} (zoom to IR7) and \figref{pic:16072602} (full LHC ring). 

The proton simulation result is obtained with the standard proton SixTrack using BeamLossPattern aperture check and carried out in a HL-LHC simulation campaign by D. Mirarchi, presented in~\cite{IPAC16:WEPMW030}. 

\begin{figure}[t]
  \centering
  \begin{tikzpicture}
    \node[anchor=south west,inner sep=0] (image) at (0,0) {\includegraphics[width=1.0\linewidth]{pictures/16072701.pdf}};
    %\node [draw,rotate=90,x={(image.south east)},y={(image.north west)}]                   at (0.50,0.50)    {text0};
    %\node [draw,rotate=0 ,x={(image.south east)},y={(image.north west)}]                   at (0.22,0.96)    {text1};
    %\node [draw,rotate=0 ,x={(image.south east)},y={(image.north west)},anchor=west]       at (0.22,0.80)    {text2};
    %\draw[->,color=black,thick,x={(image.south east)},y={(image.north west)}]             (0.42,0.22) -- (0.37,0.23);
  \end{tikzpicture}
  \caption{Loss maps zoomed to IR7 simulated for the HL-LHC configuration with \lead ions, \iso{129}{Xe}{54+} ions, \iso{40}{Ar}{18+} ions, \iso{4}{He}{2+} ions and protons.   }
  \label{pic:16072601}
  %/media/phermes/local/hisix_results/HLLHC/Other_Ions/analysis/postprocessing/isotope_comparison_IR7.pdf
  \end{figure}

The losses in the IR7 region are very similar for all simulations with heavy-ions. Slight differences in terms of shape of the loss patterns in the DS clusters are visible between the different heavy-ion species but the cleaning inefficiency for the different species are in the same order of magnitude, as summarized in \tabref{tab:hl_isotopes}. Unsurprisingly, the loss peaks in the remaining ring are at the same locations where the dispersion function takes local maxima. They are, however, at different amplitudes due to differences in the fragmentation spectra for the different isotopes. 



\begin{figure}[t]
  \centering
  \begin{tikzpicture}
    \node[anchor=south west,inner sep=0] (image) at (0,0) {\includegraphics[width=1.0\linewidth]{pictures/16072702.pdf}};
    %\node [draw,rotate=90,x={(image.south east)},y={(image.north west)}]                   at (0.50,0.50)    {text0};
    %\node [draw,rotate=0 ,x={(image.south east)},y={(image.north west)}]                   at (0.22,0.96)    {text1};
    %\node [draw,rotate=0 ,x={(image.south east)},y={(image.north west)},anchor=west]       at (0.22,0.80)    {text2};
    %\draw[->,color=black,thick,x={(image.south east)},y={(image.north west)}]             (0.42,0.22) -- (0.37,0.23);
  \end{tikzpicture}
  \caption{Full ring loss maps simulated for the HL-LHC configuration with \lead ions, \iso{129}{Xe}{54+} ions, \iso{40}{Ar}{18+} ions, \iso{4}{He}{2+} ions and protons.}  
  \label{pic:16072602}
  %/media/phermes/local/hisix_results/HLLHC/Other_Ions/analysis/postprocessing/isotope_comparison_LHC.pdf
  \end{figure}


\begin{table}[t]
\centering
\caption{Cleaning inefficiency for different isotopes in the HL-LHC configuration.}
\label{tab:hl_isotopes}
\begin{tabular}{ccccc}
\toprule
\begin{tabular}[c]{@{}c@{}}Isotope \\ \mbox{} \end{tabular} & 
\begin{tabular}[c]{@{}c@{}}$\eta^{max}_{cold}$\\ {[$10^{-2}/$m]}\end{tabular} &  
\begin{tabular}[c]{@{}c@{}}$ \eta^{int}_{DS1}$\\ {[$10^{-2}$]}\end{tabular} &  
\begin{tabular}[c]{@{}c@{}}$ \eta^{int}_{DS2}$\\ {[$10^{-2}$]}\end{tabular} &  
\begin{tabular}[c]{@{}c@{}}$ \eta^{int}_{cold}$\\ {[$10^{-2}$]}\end{tabular}     \\ \midrule
    \lead                   &      2.7   &  9.5   &  10.6    &  22     \\
    \iso{129}{Xe}{54+}      &      3.5   &  12.0  &  8.3     &  28     \\
    \iso{40}{Ar}{18+}       &      2.5   &  11.1  &  4.6     &  17     \\
    \iso{4}{He}{2+}         &      0.07  &  0.6   &  0.2     &  0.9     \\
    p                       &  6.5$\cdot10^{-3}$  &  $3.9\cdot10^{-2}$   &  $3.5\cdot10^{-2}$ &  $9 \cdot 10^{-2}$   \\ \bottomrule
\end{tabular}
\end{table}
 

The worst cleaning inefficiency in terms of peak loss and integrated loss in the DS1 is simulated for \iso{129}{Xe}{54+}, but the difference to the cleaning inefficiency for \lead is small. A remarkable difference between \lead and the lighter heavy-ion species is the ratio of the integrated losses in the DS1 and DS2. While for \lead the quantities $\eta^{int}_{DS1}$ and $\eta^{int}_{DS2}$ take similar values with a slight overweight towards DS2, the lighter ion fragments have a clear tendency to have higher integrated losses in the first DS cluster. For \iso{40}{Ar}{18+}, the integrated losses in the DS1 are more than twice as high as in the DS2. For \iso{4}{He}{2+} the ratio is approximately three. This observation is an inevitable consequence of the quantization of nucleons in the nuclei, which implies a larger rigidity offset for the same amount of neutron loss for lighter isotopes. 

\begin{table}[b]
\centering
\caption{Cleaning inefficiency for different isotopes in the HL-LHC configuration.}
\label{tab:isohl:chi}
\begin{tabular}{ccc}
\toprule
    Ref. species            &   $\chi_\text{EMD1}-1$ [$10^{-2}$]   &  $\chi_\text{EMD2}-1$ [$10^{-2}$]     \\ \midrule
    \lead                   &   0.48  &  0.97 \\
    \iso{129}{Xe}{54+}      &   0.78  &  1.57 \\
    \iso{40}{Ar}{18+}       &   2.56  &  5.27 \\ 
    \iso{4}{He}{2+}         &   33.3  &  -    \\ \bottomrule
\end{tabular}
\end{table}

In \tabref{tab:isohl:chi}, the $\chi-1$-values of particles generated from EMD1 and EMD2 are summarized as a measure of the rigidity offset for the different reference species. Furthermore, $\chi-1$ for the EMD1 and EMD2 isotopes is given as a function of $A$ for different isotopes in \figref{pic:16072603}. The $\chi-1$ value for the EMD isotopes increases continuously with decreasing mass number of the reference isotope. With increasing rigidity offset, the probability that the residual heavy-ion fragment is lost closesly to the TCP increases. 



Therefore, the loss location of these isotopes is closer to the TCP, the lighter the ion species of the main beam. 

Thus, the observed asymmetry between the integrated signals in the DS cluster is related 



\begin{figure}[b]
  \centering
  \begin{tikzpicture}
    \small
    \node[anchor=south west,inner sep=0] (image) at (0,0) {\includegraphics[width=0.6\linewidth]{pictures/16072603.pdf}};
    \node [rotate=90,x={(image.south east)},y={(image.north west)}]                   at (-0.02,0.55)    {$\chi_{\text{EMD}}-1$};
    %\node [draw,rotate=0 ,x={(image.south east)},y={(image.north west)}]                   at (0.22,0.96)    {text1};
    %\node [draw,rotate=0 ,x={(image.south east)},y={(image.north west)},anchor=west]       at (0.22,0.80)    {text2};
    %\draw[->,color=black,thick,x={(image.south east)},y={(image.north west)}]             (0.42,0.22) -- (0.37,0.23);
  \end{tikzpicture}
  \caption{$\chi_{\text{EMD}}-1$ as a function of $A$ for all isotopes lighter than \lead.}  
  \label{pic:16072603}
  %/home/phermes/Dropbox/codes/python/160726_chi_EMD/chiEMD.pdf
  \end{figure}


 

In conclusion, the simulation with hiSixTrack shows that the cleaning inefficiency for other heavy-ion species that could be possibly injected into the LHC is comparable to that of \lead. Therefore, the same limitations in terms of stored beam energy apply similarly for the other ion species. 


\chapter{Outlook}



\subsection{}



\section{Heavy-Ion Loss Maps with Aperture and Orbit Uncertainties}



\chapter{Summary and Conclusions}