\chapter{The Large Hadron Collider}\label{thelhc}
%
\section*{Introduction}
%
\begin{figure}[b]
  \centering
  \includegraphics[width=0.6\textwidth]{pictures/14050501.png}
  \caption{LHC Beam Momentum vs. Stored Energy in comparison to other Particle Accelerators~\cite{Christiane:1260465}. In both means, the LHC is far beyond previous accelerators. The maximum stored energy in the LHC will be 362\,MJ. } \label{pic:14050501} 
\end{figure}
%
The Large Hadron Collider (LHC) is the world's largest and most particle accelerator, designed to store and accelerate proton and \lead beams at unprecedented energies of 7$\,Z\,$TeV. The LHC is a synchrotron of 26.7\,km length, installed in the underground tunnel of the former LEP\footnote{Large Electron Positron collider} accelerator at the CERN\footnote{Centre Europ\'{e}en pour la Recherche Nucl\'{e}aire} research center in proximity to Geneva, Switzerland. Besides the Relativistic Heavy-Ion Collider (RHIC) at the Brookhaven National Laboratory in Long Island (USA), it is one of the two heavy-ion colliders operating worldwide and the third heavy-ion collider ever built~\cite{Fischer2014}. In the first operational period (LHC run 1), the LHC reached energies up to 4$\,Z\,$TeV and collected an integrated luminosity of 29.2\,fb$^{-1}$~\cite{lamont_moyab101} with proton beams and xx.x\,fb$^{-1}$ with \lead beams. With the collected data, the discovery of the long sought Higgs Boson could be announced in July 2012~\cite{}. After a phase of machine and detector upgrades from 2013 to 2015, the LHC re-started and accelerated proton beams to the unprecedented energy of $6.5\,$TeV.
%
%
%
%Breaking many barriers in terms of beam intensity and energy, the topic of collimation was never as important as for the LHC.  
%
%
%\newpage

\section{The LHC Accelerator Complex}

  \begin{figure}[t]
  \centering
  \includegraphics[width=1.0\textwidth]{pictures/14052201.png}
  \caption{ The CERN Accelerator Complex }  
  \label{pic:14052201}
  \end{figure}

High energy accelerators like the LHC can not be operated independently, but are installed at the end of a complex chain of injectors which pre-accelerate and shape the beam for the requirements of the end using machine. The CERN accelerator complex is schematially illustrated in Fig.~\ref{pic:14052201}. The generation of proton beam starts at an ion source 

% PH: 140516
Heavy-ion beams originate from the ion source which is connected upstream of the linear accelerator LINAC3. The ions are generated from a block of isotopically pure $Pb^{208}$ by means of microwave heating~\cite{}. The source delivers ions at a momentum\footnote{For clarity, in this chapter the momementum is given in natural units. All given momenta shall correspond to the correct unit of eV$/c$. Furthermore, the energies for the non fully stripped ions are given in terms of momentum per nucleon, while for the fully stripped ions, the general convention of using the momentum per charge is followed.} of 2.5\,keV/$u$, which are sent to a spectrometer in order to extract the desired Pb$^{+27}$ charge state. After the filtering, a multi-stage RF system accelerates the selected ion species to a momentum of $4.2\,$MeV/$u$. The following stripper foil removes more electrons, such that an ion beam of Pb$^{+53}$ is extracted from LINAC3 and transferred to the circular accelerator LEIR (Low Energy Ion Ring). In the latter, the ion beams are cooled, e.g. the transverse emittance (see \chapref{}) is reduced by an adiabatic process using electron scattering. In parallel, the beam is accelerated to a momentum of 72$\,$keV$/u$ at which it is extracted and transferred into the Proton Synchrotron (PS). At this machine, the ions are accelerated to a momentum of 5.9$\,$GeV$/u$ and sent to the Super Proton Synchroton (SPS). 

A final stripper foil installed at the PS-SPS transfer line removes the remaining electrons from the ions, such that a beam of \lead is injected into the SPS. The SPS provides the acceleration to the energy of $450\,Z\,$GeV at which the beams are injected into the LHC.


\section{LHC Layout}
\subsection{Global Layout}

\figref{pic:15032201} shows the LHC layout with its eight straigt insertion regions (IR), four of which host the main experiments (IR1, IR2, IR5 and IR8) and four providing operational functionalities. The straight sections are seperated by eight arc regions, in which the particle beams are transported from IR to IR by means of a total of 1232 superconducting dipole magnets and 392 superconducting quadrupole magnets in a so-called FODO lattice. 



  \begin{figure}[t]
  \centering
  \includegraphics[width=0.7\textwidth]{pictures/15092509.pdf}
  \caption{The layout of the LHC. Based upon \cite{Bruning2012705,CERN-2004-003-V1}. }  
  \label{pic:15032201}
  %/home/phermes/Dropbox/codes/latex/150305_eps2pgf/test.pdf
  \end{figure}


\subsection{Insertion Region Layout}




\section{Beam Parameters}

%%%%%\multicolumn{1}{l|}{$^{208}_{82}$Pb}

The LHC Beam Parameters for protons and lead ions are summarized in \tabref{tab:14052101}. In heavy ion operation, the design luminosity is significantly smaller than for proton operation, which can be traced back to the lower number of bunches and the smaller number of bunches per beam. 

%\begin{table}[htb

% \begin{table}[htbp]
% \caption{Comparison of LHC beam parameters in proton and lead mode~\citedr. }
% \begin{center}
% \begin{tabular}{lccc}
% \toprule
% \midrule
%  & & Proton & $^{208}$Pb$^{+82}$ \\ \midrule
% Total energy &[GeV]\phantom{abc} & 7000 & 574000 \\ \midrule
% Energy per nucleon &[GeV]\phantom{abc} & 7000 & 2759 \\ \midrule
% Lorentz factor $\gamma$ &\phantom{abc} & 7461 & 2963.5 \\ \midrule
% Normalized tr. emittance  &[$\mu$m rad]\phantom{abc} & 7461 & 2963.5 \\ \midrule
% Stored Energy per beam  &[MJ]\phantom{abc} & 362 & 3.81 \\ \midrule
% Particles per bunch  &\phantom{abc} & $1.15 \cdot 10^{11}$ & $7 \cdot 10^{7}$  \\ \midrule
% Number of bunches  &\phantom{abc} & 2808 & 592  \\ \midrule
% Design luminosity  &[cm$^{-2}\,$s$^{-1}$]\phantom{abc} & $1\cdot 10^{34}$ & $5.4\cdot 10^{25}$  \\ \midrule
% \bottomrule
% \end{tabular}
% \end{center}
% \label{tab:14052101}
% \end{table}

\section{The LHC as a Nucleus-Nucleus Collider}

The LHC is one of the two operating heavy-ion colliders worldwide. Other than the LHC only the Relativistic Heavy-Ion Collider (RHIC) at the Brookhaven National Laboratory in Long Island is presently in operation, but provides energies far below the LHC design collision energy. The beam parameters for the LHC in heavy-ion operation are summarized in \tabref{tab:lhc_parameters}. Especially in terms intensity, emittance and luminosity, the LHC has outreached the design values. The number of 

\begin{tiny}
\begin{table*}[t]
\centering
\caption{Comparison of the LHC design beam parameters for heavy-ion beams and proton beams in comparison to the parameters typically achieved in the LHC heavy-ion runs.~\cite{CERN-2004-003-V1,pPbref01,jowett-RLIUP13,PbPbref01,Jowett:1492972}. The parameters given for p-Pb operation refer to the \lead beam.}
\label{tab:lhc_parameters}
\begin{tabular}{cc|cc|cccc}

\multicolumn{2}{c|}{} &  \multicolumn{2}{c|}{Nominal} & \multicolumn{4}{c}{Achieved in the LHC} \\ \toprule

\multicolumn{2}{c|}{Year}     &  &  & 2010 & 2011 & 2013 & 2015 \\% \midrule
\multicolumn{2}{c|}{Species}  & p-p & Pb-Pb & Pb-Pb & \multicolumn{1}{c}{Pb-Pb} & p-Pb & \multicolumn{1}{c}{Pb-Pb} \\ \midrule


$E$ & {[}TeV{]} & 7 & 7$\,Z$ & 3.5$\,Z$ & 3.5$\, Z$ & 4.0$\, Z$ & 6.37$\,Z$\\

$\gamma$ & & 7460.5 & 2963.5 & 1481.8 & 1481.8 & 1693.4 & 2696.8\\

$n_b$ & & 2808 & 592 & 137 & 358 & 338 & 111\\

$n_p$ &{[}$10^8${]} & 1.15$\times 10^{3}$ & 0.7 & 1.12 & $1.20 \pm 0.25$ & $1.40\pm0.27$ & $1.11$\\

$\epsilon_N$ & {[}$\mu$m$\,$rad{]} & 3.75 & 1.5 & 2.0 & $1.7\pm0.2$ & - & 1.11 \\

$E_s$ &{[}MJ{]} & 362 & 3.81 & 0.71 & 1.98 & 2.18 & 1.11\\

$\mathcal{L}_\text{peak}$ &[$10^{27}\,$cm$^{-2}\,$s$^{-1}$] & $1.0\times 10^7$ & \begin{tabular}[c]{@{}l@{}}1 (Pb-Pb)\\ 115(p-Pb)\end{tabular} & 0.03 & 0.5 & 110 & 111\\ 

\bottomrule
\end{tabular}
\end{table*}
\end{tiny}