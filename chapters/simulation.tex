\chapter{Simulation Tools}\label{chap:simulation_tools}
\section*{Introduction}
Very important contributions to the excellent performance of the LHC came from theoretical simulations. Many tools have been developed and used to predict important physics aspects in the machine. For collimation, particularly software for particle tracking and simulations of particle-matter interaction are important. The prior requires also a detailed model for optics computation, as the configuration of the magnetic lattice is crucial for the particle motion in the machine. In order to have a complete picture of the collimation efficiency, information must be exchanged between the tracking tool and the particle-matter interaction, which can be realized in different manners. In this chapter, software tools important for the simulation of heavy-ion collimation are presented with their different advantages and limitations. 






\section{MAD-X}
MAD-X (Methodical Accelerator Design)~\cite{MADXref01} is the standard tool to simulate beam dynamics and compute beam optics in particle accelerators. The software is a complete migration of MAD-8 (written in FORTRAN77) to C++ and was introduced in 2002 for the design and simulation of the LHC optics~\cite{MADXref02}.

The software computes the evolution of the betatronic functions by means of transfer matrices for the individual lattice elements. The structure of the machine and the strengths of the magnets are given by the user by means of dedicated input files. A matching function provides the functionality to adjust specific variables such that previously defined constraints are fulfilled. An aperture model of the machine can be processed and compared with the beam position and dimensions to evaluate the available normalized aperture throughout the machine. A dedicated function produces the required optics input for SixTrack (see next Chapter). 



% ----------------------------------- PARTICLE MATTER INTERACTION ----------------------------------

\section{FLUKA}


% \begin{figure}[t]
% \centering
% \resizebox{1.0\textwidth}{!}{\input{pictures/15010501.pgf}}
% \caption{}
% \label{pic:15010501}
% %/home/phermes/Dropbox/codes/python/140105_plot_deltakin/my-awesome-plot.pgf
% \end{figure}


\section{ICOSIM}
ICOSIM (Ion Collimation Simulation) was the first tool developed for the simulation of heavy-ion beam collimation in the LHC. Before the first LHC heavy-ion run took place in 2010  




\section{SixTrack}
SixTrack~\cite{SixTrackref01,SixTrackref02,SixTrackref03,SixTrackref04} is the present standard tool for the simulation of proton collimation. It is designed to provide symplectic six-dimensional tracking of relativistic proton beams in high energy synchrotrons over many turns. Designed and maintained at CERN, the software is subject to regular updates providing new features for dedicated functions or improved physics models. 

The tracking algorithm is based on symplectic transfer maps which are computed for all lattice elements in thin lens approximation. Chromatic effects are simulated up to 20th order, making SixTrack an excellent tool to provide accurate tracking for off-momentum beam particles. 


  \begin{figure}[t]
  \centering
  \includegraphics[width=0.8\textwidth]{pictures/15070701.png}
  \caption{}  
  \label{pic:15070701}
  %/home/phermes/Desktop/tra.png
  \end{figure}

The collimation subroutine provides an integrated environment for 6D tracking together with a Monte-Carlo Module to simulate the interaction of the protons with the material of the collimation devices. The individual particle tracks are compared to a detailed model of the LHC aperture. If a beam particle is identified to intercept the aperture of the beam pipe, the loss location is saved. By default, this is done on a post-processing level, but a recently developed integrated on-line aperture check~\cite{} provides the same functionality without the time- and space-consuming write-out of the individual particle tracks. The aperture check is first carried out at dedicated markers. If the aperture is intercepted at a marker, the particle track between this and the previous marker are interpolated as a straight line and the aperture check is repeated on this line in steps of $10\,$cm. The latter is therefore the precision at which aperture losses are simulated. 

Particles interacting with the collimators are considered to be lost if they undergo nuclear inelastic scattering. 

\begin{itemize}
\item good agreement for protons

\end{itemize}





\subsection{Input}

The input for SixTrack is essentially based on two files: \textbf{fort.2} and \textbf{fort.3}.



\section{SixTrack-FLUKA Coupling}
