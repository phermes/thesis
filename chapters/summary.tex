\chapter*{Summary}\label{chap:8}
\markboth{SUMMARY}{SUMMARY}
\addcontentsline{toc}{chapter}{Summary}

In this thesis, the development and evaluation of an optical configuration with large $\beta^*$-value in the LHC region around the ALICE detector was described. This configuration should allow the measurement of very forward particles in parallel to the nominal proton-proton physics program in ALICE and the remaining LHC experiments. Very forward particles are emitted from the Interaction Point, due to diffractive or elastic scattering events, probably related to the exchange of one or multiple Pomerons. The detection of these particles can be provided by means of Roman Pot detectors, installed in the very forward region.

The optics design must optimize the measurability of the very forward particles, but respect the many constraints which are imposed by the present configuration of the LHC. For the optimization of the measurement, the $\beta^*$-value should be maximized, and the betatron phase advance between the Interaction Point and the prospective Roman Pot detector positions must be matched to a value as closely as possible to $\Delta \psi =0.25$.

Based on conservative assumptions on the reachable emittance, the reachable crossing angle, and the required beam-beam separation, a value of $\beta^*=18\,$m is considered, which is compatible with the LHC design filling scheme with 2808 bunches \mbox{(corresponding to a bunch spacing of 25$\,$ns)}. The beam line around the ALICE detector is analyzed for suitable positions for the Roman Pot detector placement. The proposed locations are at longitudinal distances of -180$\,$m/-220$\,$m from the Interaction Point for the measurement with Beam 2, and at $+150\,$m/$+220\,$m for the measurement with Beam 1. The installation of two detector stations is necessary for the reconstruction, because the phase advance between Interaction Point and Roman Pot detector can not be matched to the desired value.
Optics with $\beta^*=18\,$m were matched and optimized for best performance in terms of measurability of very low four-momentum transfers $|t|$. A phase advance of $\Delta \psi = 0.34$ to the second detector station can be reached for both beams in both transverse directions. The optics are compatible with the \mbox{magnet} limitations, aperture constraints and separation requirements. \mbox{In order to provide} a sufficiently large beam-beam separation at the parasitic bunch encounters, a crossing scheme, providing a full crossing angle of $\theta_C = 300\,\mu$rad, was matched. 

\newpage

Very large phase advance compensations must be provided by other parts of the machine. Due to the required applicability in high-intensity operation, this phase advance compensation must be feasible between IP1 and IP5. So far, no concrete strategy for this compensation was worked out. With the proposed configuration, the luminosity in ALICE is reduced, but is still too large to match the detector requirements.


The $|t|$- and $\xi$-acceptance of the matched optics with the proposed detector locations is studied. The results of these studies are strongly depending on the assumptions which are made for the beam properties, the detector layout and detector distance from the beam centre. In the rather conservative standard conditions, the $\beta^*=18\,$m optics reach an acceptance of $\mathcal{A}=$50$\%$ for elastic events ($\xi=0$) at a value of $|t|_{50\%}=0.44\,$GeV$^2$, instead of $|t|_{50\%}\approx 13\,$GeV$^2$ with the nominal optical configuration in Insertion Region 2. Taking into account either more optimistic or more pessimistic measurement scenarios, an interval with upper and lower limits of the reachable acceptance with $\beta^*=18\,$m can be given
\begin{align}
0.24\,\text{GeV}^2 \leq |t|_{50\%} \leq 0.78\,\text{GeV}^2.
\end{align} 


Besides the $\beta^*=18\,$m optics, a configuration with $\beta^*=30\,$m is briefly discussed, which is compatible with a bunch spacing of 50$\,$ns. The phase advance from Interaction Point to Roman Pot detectors of this configuration is matched to $\Delta \psi = 1/3$. The acceptances of this configuration are better than for the $\beta^*=18\,$m case. However, this configuration is most likely not implementable during high-intensity operation of the LHC, with the largest number of bunches. 


Additional studies which have to be carried out before real implementation of the optics are outlined. A continuous un-squeeze for the transition from injection optics to the $\beta^*=18\,$m optics must be developed. The optics are compatible with the prospective new ALICE beam pipe dimensions. An algorithm for the reconstruction of the $|t|$- and $\xi$-values is discussed. Lower emittances would enhance the potential of high $\beta^*$-optics and give the possibility to apply larger $\beta^*$-values. Significant reductions of the required external phase advance compensation can be reached at the expense of an increased acceptance threshold.


In conclusion, an optical configuration with increased $\beta^*$-value and matched phase advance, compatible to long time running conditions, was found to be feasible for ALICE. The concrete details of such a configuration still depend on subsequent studies on phase advance compensation and on the reachable emittance in high-intensity operation. 
Several essential steps for the implementation still have to be taken, which will be carried out if ALICE decides to implement a high $\beta^*$-configuration.
For the presented $\beta^*=18\,$m optics, an interval for the reachable acceptances as a function of $|t|$ can be given. Ongoing studies~\cite{schicker_unpublished} indicate that acceptances in this range are eligible for the study of additional physics topics at the LHC. 
 