\chapter{SixTrack with Ion-Equivalent Rigidities }\label{chap:stier}
%
\section*{Introduction}
The design of the LHC collimation system forsees the primary collimator in the betatron collimation region IR7 as the first collimator in the collimation hierarchy. It should be the only collimator exposed to particles of the main beam and is therefore in general the device with the highest amount of particle losses in the whole LHC ring. The loss distribution measured during qualification loss maps is therefore dominated by secondary ion fragments originating from an interaction of the main beam with the TCP. Tertiary particles generated from scattering and fragmentation in collimators downstream of the TCP should only contribute marginally to the final loss map. 

Based on these assumptions, an intermediate approach for the simulation of heavy-ion collimation was established. Instead of using the simplified fragmentation model at all collimators in ICOSIM, this approach relies on a detailed fragmentation simulation at the TCP using a Monte-Carlo event generator, such as \textsc{FLUKA}, followed by a tracking of the heavy-ion fragments by means of SixTrack. Given that SixTrack is not designed for the tracking of ions different from protons, the mass to charge ratios of the many different isotopes are taken into account by using an effective momentum offset. In this case, the ions are tracked as protons with ion equivalent rigidities. The approach is referred to as SixTrack with Ion-Equivalent Rigidities (STIER). 

In this chapter, the STIER approach is described in detail and the obtained loss maps are benchmarked with qualification loss maps measured in the 2011 heavy-ion run. The analysis result is also compared to loss maps simulated by means of ICOSIM. Based on the results, the requirements for a new heavy-ion collimation simulation software are outlined. Parts of the contents in this chapter have been published in ~\cite{hermes_nim}.

\newpage
\section{Simulation Approach}
%
  \begin{figure}[h]
  \centering
  \includegraphics[width=0.5\textwidth]{pictures/15063001.pdf}
  \caption{Schematics of a STIER simulation: The phase space parameters of the particles impacting on the collimator jaw of the primary collimator are computed using MAD-X. With the obtained angle of incidence, the fragmentation of the heavy-ion beam at the collimator material is simulated using the Monte-Carlo event generator FLUKA. The outcoming ion fragments are tracked as protons with ion-equivalent rigidities by means of SixTrack.}  
  \label{pic:15062601}
  %/home/phermes/Dropbox/PhD/pictures/STIER-schematics/thesis/simulation-overview4.pdf
  \end{figure}
%
STIER relies upon three consecutive simualtions as shown in \figref{pic:15062601}. First, an initial optics computation is carried out using MAD-X to determine the phase space parameters and thus the angle of incidence of the primary beam halo at the TCP. With this information, the fragmentation of the main beam at the primary collimator is simulated by means of FLUKA. The third and last step includes the tracking of the resulting ion fragments using protons in SixTrack starting at the TCP with ion-equivalent rigidities. 

In the following sub-sections, the three STIER stages are described in detail and important simulation results for every stage are summarized. Some of the results are important also for the development of the advanced heavy-ion collimation tool that is described in the next chapter.
%
%
\subsection{Optics Calculation}
%
\begin{figure}[htpb]
  \centering
   \def\svgwidth{0.5\linewidth}
   \input{pictures/hybrid_pictures/14070304.pdf_tex}
  \caption{Phase space diagram for at the TCP locations. The particles hitting the collimator edge (indicated by the red dots) have a certain angle $x'= \alpha \, \sqrt{ \frac{\epsilon}{\beta_x} }  $. }
  \label{pic:14070304}
\end{figure}
%
The tool MAD-X is used to obtain the optical functions at the location of the primary collimator. Besides, the \texttt{fort.2} input for the SixTrack simulation is generated by means of MAD-X. Particles which hit the primary collimator have specific properties in phase space, as illustrated in \figref{pic:14070304}. Of particular importance for the sub-sequent fragmentation simulation is the angle of incidence at the collimator which is defined by the normalized half gap $N_P$, the geometric emittance $\epsilon_x$ and the Twiss parameters $\beta_x,\alpha_x$ at the location of the TCP:
\begin{align}
x' = \pm N_P \, \alpha_x \, \sqrt{\frac{\epsilon_x}{\beta_x}} \, . \label{eq:angle_of_incidence}
\end{align}
%
In the 2011 heavy-ion run the collimator settings were based upon a normalized emittance of \mbox{$\epsilon_N = 1.4\,\mu$m$\,$rad} corresponding approximately to the geometric emittance as for proton beams at $3.5\,$TeV.
%
The resulting angles of incidence for 3.5$\,Z\,$TeV and the hypothetical case of $7\,Z\,$TeV are shown in \tabref{tab:optical_functions}.
%
%
\begin{table}[t]
\centering
\caption{Summary of the parameters used to calculate the angle of incidence at the primary collimator. The presented emittance $\epsilon_x=\frac{\epsilon_N}{\beta \gamma}$ is the geometric emittance of the heavy-ion beam. The Twiss parameters $\beta_x$ and $\alpha_x$ are computed using MAD-X. The angle $x'$ is calculated by means of \eqref{eq:angle_of_incidence}.}
\label{tab:optical_functions}
\begin{tabular}{ccccccc}
\toprule
\begin{tabular}[c]{@{}c@{}}E\\ {[}$Z$ GeV{]}\end{tabular} & \begin{tabular}[c]{@{}c@{}}$\epsilon_x$\\ {[}$\mu$m rad {]}\end{tabular}& \begin{tabular}[c]{@{}c@{}}$\beta_x$\\ {[}m{]}\end{tabular} & \begin{tabular}[c]{@{}c@{}}$\alpha_x$\\ {[} {]}\end{tabular} & \begin{tabular}[c]{@{}c@{}}$\gamma$\\ {[} {]}\end{tabular} & \begin{tabular}[c]{@{}c@{}}$N_p$\\ {[}$\sigma${]}\end{tabular} & \begin{tabular}[c]{@{}c@{}}$x'$\\ {[}rad{]}\end{tabular} \\ \midrule
3500 & $9.38 \times 10^{-10}$  & 148.46 & 2.04 & 1482.78 & 5.70 & 2.912$\times 10^{-5}$ \\
7000 & $x \times 10^{-10}$  & 148.46 & 2.04 & 2964.50 & 5.70 & $\times 10^{-5}$ \\ \bottomrule
\end{tabular}
\end{table}
 
\subsection{Fragmentation Simulation}
%
\begin{figure}[htb]
  \centering
   \def\svgwidth{0.65\linewidth}
   \input{pictures/hybrid_pictures/14112701.pdf_tex}
  \caption{Schematic illustration of the set-up for the fragmentation simulation.}
  \label{pic:14112701}
\end{figure}
%
The fragmentation simulation in STIER is carried out by means of a Monte-Carlo event generator such as Geant4~\cite{Agostinelli2003250,1610988} or \textsc{FLUKA}. The latter is used for the study case presented below. A primary \lead beam of $10^7$ particles at an energy of $3.5\,Z\,$TeV is simulated to hit a carbon target at an angle $x'$ and an impact parameter $b$. Based on the results of the prior optics computation the angle of incidence is set to be $x'=2.912\times 10^{-5}$. Assuming that the diffusion processes in heavy-ion and proton beams are similar, prior SixTrack simulations with protons deliver values for the impact parameter in the range $1\,\mu \text{m} \leq b \leq 10\,\mu$m. 

The collimator is modelled as a 60$\,$cm long block of carbon as shown in \figref{pic:14112701}. At the end of the collimator jaw, the ions are selected and their energies, angles and species are written to an output file. 

\begin{figure*}[t]
\begin{minipage}[t]{0.495\textwidth}
\includegraphics[width=\textwidth]{pictures/15092402.pdf}
\end{minipage}
\begin{minipage}[t]{0.495\textwidth}
\includegraphics[width=\textwidth]{pictures/15092404.pdf}
\end{minipage}
\caption{Spectrum of ions obtained in the fragmentation simulation of \lead~ions at $E=3.5\,Z\,$TeV impacting the material of the primary collimator with an impact parameter $b=3\,\mu$m. Left: Spectrum of $\chi_N$ as defined in \eqref{eq:normalized_chi}. The black line represents the case $\chi=1$. Right: Energy fraction carried by the individual isotopes, normalized with the total ion energy which leaks out of the collimator, as calculated by means of \eqref{eq:energyfraction}. The most important ions are those with $A,Z$ close to the main beam and very light ion fragments, such as protons and $\alpha$ particles. A detailed overview of the energy fraction carried by the most important isotopes is given in \tabref{tab:importance}.}
\label{fig:stier_fragmentation}
\end{figure*}

\figref{pic:15091701} shows the spectrum of isotopes generated in the fragmentation simulation for STIER with \mbox{$b=3\,\mu$m}. The left side shows the normalized $\chi$ defined as
\begin{align} 
\chi_N=\begin{cases} \chi & \text{if } \chi<1 \\ \chi^{-1} & \text{if } \chi>1 \end{cases} \, . \label{eq:normalized_chi}
\end{align}
The right side shows the energy fraction $e_f$ carried by the individual isotopes, normalized by the total outleaking ion energy:
\begin{align}
e_f(A,Z) = \frac{n(A,Z) \, A }{\sum_{A,Z} n(A,Z) \, A} \,, \label{eq:energyfraction}
\end{align}
where $n(A,Z)$ is the number of ions of the species $A,Z$ obtained from the fragmentation simulation. The individual ion energy is scaled with $A$ which roughly takes into account the energy weight, assuming that all ions have approximately the same energy per nucleon. The isotopes carrying the largest amount of energy amongst all out-coming ions are summarized for the three cases, $b=1\,\mu$m, $b=3\,\mu$m and $b=10\,\mu$m in \tabref{tab:importance}.

As expected, the number of very small ion fragments increases with increasing $b$, since distance an ion travels through the collimator material scales linearly with $b$, such that an ion can be fragmented several times throughout the passage. For the case of $b=1\,\mu$m, the amount of energy carried by protons and \lead ions is in the same order of magnitude, with 4.7\% of the energy carried by protons and 3.6\% carried by \lead. At $b=10\,\mu$m, approximately 40\% of the outcoming beam energy is carried by protons and 16\% by $^4$He$^{2+}$ ions, while the contribution of particles of the main beam has diminished by three orders of magnitude. In conclusion, the impact parameter plays an important role for the isotopic spectrum generated in the process of heavy-ion fragmentation. The impact on the resulting loss distribution is studied in the next chapter.
%
%
\begin{table}[b]
\centering
\caption{Isotopes with the largest energetic fractions leaking out the collimator material from the initial fragmentation simulation. }
\small
\setlength\tabcolsep{2.5pt}
\label{tab:importance}
\begin{tabular}{ccccccc}
\toprule
          \multicolumn{1}{c}{}     &     &  \multicolumn{5}{c}{Energetic Fraction $e_f$ with}                    \\

\multicolumn{1}{c}{Isotope}       &   & \multicolumn{1}{c}{$b=1\,\mu$m} &            & \multicolumn{1}{c}{$b=3\,\mu$m}      &       & \multicolumn{1}{c}{$b=10\,\mu$m}            \\ \midrule
$^1$H$^{1+}$      & \phantom{a} & $4.7 \times 10^{-2}$ & \phantom{a} & $6.3 \times 10^{-2}$ & \phantom{a} & $4.0 \times 10^{-1}$  \\
$^2$H$^{1+}$      &  & $2.1 \times 10^{-2}$ & & $2.5 \times 10^{-2}$ & & $1.2 \times 10^{-1}$  \\
$^3$H$^{1+}$      &  & $1.5 \times 10^{-2}$ & & $1.7 \times 10^{-2}$ & & $7.4 \times 10^{-3}$  \\
$^3$He$^{2+}$     &  & $5.8 \times 10^{-3}$ & & $8.1 \times 10^{-3}$ & & $4.9 \times 10^{-3}$ \\
$^4$He$^{2+}$     &  & $3.6 \times 10^{-2}$ & & $4.2 \times 10^{-2}$ & & $1.6 \times 10^{-1}$  \\
$^{205}$Pb$^{82+}$&  & $7.1 \times 10^{-3}$ & & $2.3 \times 10^{-3}$ & & $1.1 \times 10^{-5}$  \\
$^{206}$Pb$^{82+}$&  & $1.7 \times 10^{-2}$ & & $5.0 \times 10^{-3}$ & & $ 1.4 \times 10^{-5}$ \\
$^{207}$Pb$^{82+}$&  & $3.3 \times 10^{-2}$ & & $ 8.4\times 10^{-3}$ & & $1.0 \times 10^{-5}$  \\
$^{208}$Pb$^{82+}$&  & $3.6 \times 10^{-2}$ & & $3.6\times 10^{-2}$  & & $1.4 \times 10^{-5}$  \\

\bottomrule
\end{tabular}
\end{table}
%
%
%

A very important quantity for the setup of the new heavy-ion collimation software is the momentum per nucleon of the ion fragments (see \chapref{chap:hisix_coupling}). The distribution momentum per nucleon $p_A$ of the outcoming ions is shown in \figref{pic:15091701}. Most of the ions have $p_A$ close to the momentum per nucleon of the incoming \lead beam, which is at 1379.8$\,$GeV/$A$. Parts of the fermi-motion in the nucleus can be transferred to individual ion fragments, such that the ion can have significantly larger momentum per nucleon. The highest $p_A$ that was obtained from the simulation is 3200$\,$GeV$/A$.

  \begin{figure}[t]
  \centering
  \includegraphics[width=0.5\textwidth]{pictures/15091701.pdf}
  \caption{Momentum per nucleon of the ion fragments in the fragmentation simulation. }  
  \label{pic:15091701}
  %/media/phermes/ph3tboffice/ph1tbwd/FLUKA_results/150903_Ions_Perpendicular_lx/mpnuc.pdf
  \end{figure}


\subsection{Tracking in SixTrack}

SixTrack is designed for proton tracking, so the tracking does not include effects of isotopic dispersion. Based on \eqref{eq:d_effective}, the effective momentum offset $\delta_\text{eff}$ of an ion of the reference species to have the same magnetic bending behaviour as an arbitrary ion with $\delta,\chi$, is given by
%
\begin{align}
\delta_\text{eff} = \frac{(1+\delta)}{\chi} -1 \,. 
\end{align}
%
If the reference species is a proton of momentum $P_0$, the applicable effective proton momentum to simulate the tracking behaviour of an ion with mass and charge $m$ and $Z$ is given by (see \eqref{eq:brho_brho0})
\begin{align}
  P_E = P_0 \, (1+\delta_\text{eff}) = \frac{P}{Z} \, .
\end{align}

\begin{itemize}
\item Collimators set to perfect absorbers
\item Track all isotopes from the fragmentation simulation.
\end{itemize}


\section{Simulation Results}
\subsection{Full Heavy-Ion Loss Map Simulations}

%\subsection{Loss Map simulations} \label{sec:lm_simulations}
\begin{figure*}[htbp]
\begin{minipage}[t]{0.495\textwidth}
\includegraphics[width=\textwidth]{pictures/15092301.pdf}
\end{minipage}
\begin{minipage}[t]{0.495\textwidth}
\includegraphics[width=\textwidth]{pictures/15092302.pdf}
\end{minipage}
\caption{Comparison of loss map simulations using ICOSIM (top row), a simplified STIER approach (second row), a full STIER simulation (third row) and the measured loss maps during the 2011 LHC heavy-ion run at 3.5\,$Z\,$TeV. The right graph in the bottom row also shows the locally generated dispersion function $D_x$ starting at $D_x=0$ at the TCP. The left column shows the loss map over the full LHC ring, while the right column shows the same loss map zoomed into the betatron collimation region IR7. The STIER simulations are carried out assuming an impact parameter of $b=3\,\mu$m.}
\label{fig:comparison_lossmaps}
\end{figure*}


\textit{The result of the ICOSIM simulation is shown in the first row Fig.~\ref{fig:comparison_lossmaps} for the full LHC ring (left) and zoomed into IR7 and the following arcs (right). Fragments scattered out of the collimators are in this model absorbed closely to the collimators where the fragmentation occurred and do not continue moving inside the machine for long distances. Losses in warm regions are visible only in IR3 but not in IR7. The mass number of the heaviest created and tracked fragment is $A_\text{max} = 90$ which can be traced back to the simplified fragmentation algorithm. The high measured loss peaks (see bottom row in Fig.~\ref{fig:comparison_lossmaps}) in the arcs between IR7 and IR8 are, with one exception, not reproduced by ICOSIM. The most critical losses in the dispersion suppressor located in the cells 8 and 9 of IR7 are visible in the simulated loss map. The local cleaning inefficiency in this region peaks at approximately $\eta=10^{-2}$, which is comparable to the measured loss distribution. However, it shall again be emphasized that the BLM signals with limited coverage can not be compared quantitatively to the simulated loss maps. The longitudinal extensions of the loss clusters in the DS do not fully match with the ones of the measured losses. 
%
%
The loss maps from the simplified STIER simulation are shown in the second row of Fig.~\ref{fig:comparison_lossmaps}. The losses are more broadly distributed over the LHC ring when the full spectrum of fragments is included. This indicates that light isotopes are by no means only lost locally in the secondary collimators or in the warm regions surrounding them. Parts of the losses in the warm region of IR7 are reproduced in this simulation approach, originating mainly from very light isotopes at $|\delta_\text{eff}|\gg0$. The two clusters of losses in the IR7 DS are increased in their intensity and longitudinal extension, since the isotopes which are neglected in ICOSIM are now included and are lost in these regions. The loss peak with the largest amplitude is still in the IR7 DS but is increased to \mbox{$\eta_\text{max} = 10^{-1}$}. One of the loss peaks in the arcs is, as in ICOSIM, reproduced in this simulation. Also here, the simulated peak intensity is increased with respect to ICOSIM.}


%\subsection{Isotope Contributions to Loss Peaks} \label{chap:isocontributions}


\begin{table*}[htbp]
\caption{STIER simulated contributions on the total deposited energy at the warm magnets in IR7 (W1), the two loss clusters in the IR7 DS (C1 and C2) and in the arcs downstream of IR7 (A1,A2,A3,A4) as shown in Fig.~\ref{fig:comparison_lossmaps} }
\small
\setlength\tabcolsep{2.5pt}
\centering
\label{tab:contrib}
\begin{tabular}{cc|cc|cc|cc|cc|cc|cc}
\toprule
\multicolumn{2}{c}{W1}                 & \multicolumn{2}{c}{C1}                 & \multicolumn{2}{c}{C2}         & \multicolumn{2}{c}{A1}                 & \multicolumn{2}{c}{A2}                 & \multicolumn{2}{c}{A3}                 & \multicolumn{2}{c}{A4}                    \\ \midrule
Ion            &  (\%) & Ion            &  (\%) & Ion  & (\%)   &   Ion            &  (\%) & Ion            &  (\%) & Ion            &  (\%) & Ion            &  (\%) \\ \midrule
$^{1}$H$^{1+}$ & 57.0              & $^{3}$H$^{1+}$ & 8.6              & $^{206}$Pb$^{82+}$ & 34.0            & $^{204}$Tl$^{81+}$ & 61.0              & $^{204}$Tl$^{81+}$ & 74.6              & $^{204}$Tl$^{81+}$ & 86.6              & $^{204}$Tl$^{81+}$ & 86.7              \\

$^{3}$H$^{1+}$ & 38.0              & $^{4}$He$^{2+}$ & 4.5              & $^{205}$Pb$^{82+}$ & 16.2          & $^{206}$Pb$^{82+}$ & 18.7              & $^{206}$Pb$^{82+}$ & 10.3              & $^{199}$Au$^{79+}$ & 6.7               & $^{199}$Au$^{79+}$ & 7.2               \\

$^{2}$H$^{1+}$ & 2.6               & $^{2}$H$^{1+}$ & 3.2               & $^{204}$Pb$^{82+}$ & 11.6            & $^{199}$Au$^{79+}$ & 7.4               & $^{199}$Au$^{79+}$ & 5.7               & $^{206}$Pb$^{82+}$ & 2.2               & $^{206}$Pb$^{82+}$ & 1.7               \\

$^{3}$He$^{2+}$     & 1.4               & $^{203}$Pb$^{82+}$ & 3.2               & $^{203}$Tl$^{81+}$ & 8.7         & $^{1}$H$^{3+}$     & 3.5               & $^{201}$Hg$^{80+}$ & 2.3               & $^{194}$Ir$^{77+}$ & 1.2               & $^{202}$Hg$^{80+}$ & 1.6              
\\ \bottomrule             
\end{tabular}
\end{table*}





\textit{The loss maps generated with the full STIER simulation are shown in comparison to the measurement and the ICOSIM loss map in the third row of Fig.~\ref{fig:comparison_lossmaps}. This approach shows the best agreement with the measured loss distribution. The four highest loss peaks in the arcs become visible when the angles and energies of the fragments are included. 
%
A larger amount of the losses in the warm IR7 magnets (W1) is visible, coming mainly from the very light fragments scattered out of the collimator (see Tab.~\ref{tab:contrib} for the mainly contributing isotopes). The two loss clusters (C1 and C2) in the IR7 DS are modeled with a correct longitudinal extension and order of magnitude (the mainly contributing isotopes to C1 and C2 are also listed in Tab.~\ref{tab:contrib}). It is remarkable that the second loss cluster C2 is dominated by heavy fragments of Pb, created by electromagnetic dissociation (the three isotopes contribute to 61.8\,\% of the total deposited energy), while the highest fraction of C1 is due to very light isotopes (H and He ions) where the four most important isotopes only compose only 19.5\,\% of the energy deposited in C1. With 1227 different isotope species lost in C1, the deposited energy is shared between a much larger number of isotopes than at C2, where only 334 different isotopes are absorbed. 
%
The distributions of the quantity $\chi$ for the isotopes lost in the regions C1,C2 are graphically represented in Fig.~\ref{fig:15032102}. As expected, the aperture in the cold region C1 captures a very broad range of isotopes. In the cold region C2, mostly isotopes with mass to charge ratios close to the reference ion species contribute to the total deposited energy. 
%
The graph at the right bottom of Fig.~\ref{fig:comparison_lossmaps} shows also the locally generated dispersion function $D_x$ starting at the primary collimator. The dispersion increases from $D_x\approx0\,$m in the warm IR7 magnets to $D_x \approx 1\,$m at the end of the C1. At the second loss cluster C2, the dispersion increases to even $D_x=2.4\,$m. We conclude that the isotope distribution shown in Fig.~\ref{fig:15032102} can be explained by the fact that the isotopes with large momentum offsets are removed from the beam already in the C1, while the isotopes with rigidities closer to the main beam are not sufficiently off-momentum to be intercepted by the aperture at this location. When the dispersion further increases at C2, also the isotopes with smaller $|\delta_\text{eff}|$ are lost in the aperture, while the isotopes with $A<190$ have already been removed from the beam in C1. Here it shall be pointed out that the fraction of light isotopes depends strongly on the chosen impact parameter for the fragmentation simulation.}



\begin{figure}[b]
\centering
 \includegraphics[width=0.6\textwidth]{pictures/15092501.pdf}
% \includegraphics[width=0.5\textwidth]{pictures/15062202.pdf}
 % \includegraphics[width=0.5\textwidth]{pictures/15032103.pdf}

\caption{Fractions of the total deposited energies in the regions C1 and C2 as a function of the relative mass to charge ratio $\chi$. The data is extracted from the full STIER simulation with $b=3\,\mu$m.}
\label{fig:15032102}
\end{figure}

\textit{Contrary to this, the isotope composition at the arc loss peaks (A1,A2,A3,A4) is very homogeneous (see Tab.~\ref{tab:contrib}). At the passage through the arcs, the locally generated dispersion function is beating with amplitudes between $0.8\,\text{m} < D_x < 2.5\,\text{m}$. The loss peaks A1-A4 are located at local maxima of the function $D_x$. The similarity of their isotopic composition can be explained by the fact that, especially for isotopes with rigidities closely to the main beam, the individual starting conditions at the TCP (starting angle and collimator jaw) can partly compensate or enhance dispersive effects, thus some ions of the same species can travel for longer distances than others. It shall be emphasized that except $^1$H$^{3+}$, all isotopes listed for A1-A4 in Tab.~\ref{tab:contrib} are also included in the ICOSIM simulation. We conclude that the decisive properties determining at which of the four locations a particle is absorbed are the angle and momentum at which the particle is emitted at the TCP, which are not included in ICOSIM. Hence, an accurate simulation of heavy ion loss patterns requires inevitably the incorporation of angular and energetic shifts by fragmentation at the collimator, which should be considered for future simulations.
%
The fraction of nucleons which is absorbed in the aperture was calculated to be $f_\text{glob}=0.167$ almost two orders of magnitude larger than for comparable simulations with proton beams where this quantity takes typically values in the order of $f_\text{glob}=0.002$~\cite{Bruce2014a}.
%
Parts of the losses far downstream of the TCP are longitudinally shifted with respect to the measured loss peaks in the same region of the LHC. These shifts could come from small aperture displacements in the real machine, or offsets of the closed orbit by the same amount. An analysis of the particle trajectories at the corresponding loss locations shows that displacements as small as $\Delta a =300\,\mu$m are sufficient to shift the loss location of the impacting ions, while at the loss peaks A1-A4 displacements of $\Delta a > 600\,\mu$m are required for a significant reduction of the loss peak amplitude. Another possible explanation is that a non-equal amount of secondary ion fragments is produced at the two collimator jaws, which is studied in the next sub-chapter.}

\subsection{Dependence on the starting Collimator Jaw}

\begin{figure}[b]
  \begin{center}
\includegraphics[width=0.6\textwidth]{pictures/15092502.pdf}
\caption{STIER simulations starting at the left and right collimator jaw, shown in comparison to the measured loss map. The simulations are carried out considering an impact parameter of $3\,\mu$m. }
\label{fig:15032201.pdf}
  \end{center}
\end{figure}

\textit{An intrinsic advantage of the STIER setup is the possibility to study the loss behaviour of isotopes starting at the individual collimator jaws. So far, all STIER results assume the same amount of ions impacting on the two TCP jaws. However, studies with proton beams have shown that the beam halo in the real machine can impact the two collimator jaws asymmetrically, as discussed in \cite{Bruce2014a} and seen from SixTrack simulations. This process is very hard to model accurately in simulations, as it depends on the interplay of a large number of machine imperfections that are not well known. In Fig.~\ref{fig:15032201.pdf}, the loss map is shown, as simulated with STIER with the same settings as above for particles starting at the left and right collimator jaw separately. A different behaviour of the losses can be expected because the betatronic motion and the dispersion can amplify or compensate each other, depending on the starting conditions of the ion.
\\
From the obtained loss maps it can be seen that the simulation result for the particles starting at the right collimator jaw is in much better agreement with the measured data than for particles starting at the left jaw. The largest fraction of the loss peaks between IR8 and IR1 as well as between IR1 and IR2 which are unobserved in the measurement but visible in STIER come from particles starting at the left jaw. However, still one intense peak, visible in the simulation for the right jaw, is unobserved in the measurement, which might come from aperture or orbit displacements as discussed above. In conclusion, the discrepancies of STIER in the regions far downstream of the TCP might come from both, asymmetric losses at the two collimator jaws and small beam displacements relative to the aperture. }


\subsection{Dependence on the Impact Parameter}

\textit{In the real machine, the impact parameter $b$ of the ions hitting the collimators may vary. For the STIER simulations presented so far, an impact parameter of 3$\,\mu$m was assumed, based on previous proton studies. As shown in Tab.~\ref{tab:particle_matter_interaction}, the interaction lengths for fragmentation processes by EMD and NF are smaller than the length of the collimator jaw of the primary collimator (60$\,$cm). With increasing impact parameter, the traversed distance of the heavy-ion beam inside the material increases, leading to a drastic change of the fragmentation rates into the different isotopes. In particular the production of very light ion fragments, such as protons, $\alpha$-particles and neutrons increases, while the rate of surviving ions of the main beam drops significantly with increasing $b$. An overview of the energetic fraction carried by different out-coming isotopes after the ion-collimator interaction is given in Tab.~\ref{tab:isotope_abundances}. Note that the values are scaled with the ion mass, so the ion abundance is reduced by a factor of $A$ with respect to the given values. With an impact parameter of $b=3\,\mu$m, the two most important isotopes are protons and $\alpha$ particles followed by \lead ions of the initial species. 
\\
The loss maps as simulated with STIER for the three different impact parameters are compared in Fig.~\ref{fig:15062502.pdf}. The loss patterns are qualitatively similar but the loss peak intensities differ quantitatively. For the smallest impact parameter, the highest losses occur at the primary collimator in IR7. The main contribution of these losses comes from \lead ions that were not fragmented but scattered at small angles in the TCP. They move inside the machine for one or multiple turns until they are intercepted by the global bottleneck, which is the primary collimator. In reality, however, these ions are again subject to fragmentation and scattering inside of the TCP, which is not considered in the presented simulation. The losses at the DS region peaks at $\eta=10^{-3}$, which is smaller than in the other simulations, due to the large amount ions of the main beam surviving the initial passage through the TCP. The remaining losses in the aperture and the other collimators are located at elements which are also subject to losses for the other impact parameters, but the loss amplitudes are smaller. For the cases of $b=3\,\mu$m and $b=10\,\mu$m the highest losses are visible at the secondary collimator, which is consistent with the loss pattern measured during operation. Both simulated loss patterns are dominated by ion fragments instead of ions of the main beam. The losses in the DS clusters peak at $\eta=10^{-2}$ for $b=3\,\mu$m and are larger by a factor of 3 for the case of $b=10\,\mu$m. In the latter case, the production yield of effectively off-momentum isotopes is highest among the studied scenarios, which becomes apparent by the comparatively small amount of fragments captured by the collimation system.
\\
We conclude that the impact parameter does seem to influence the cleaning inefficiency but not on the longitudinal distribution of the losses. }



\begin{figure}[t]
  \begin{center}
\includegraphics[width=0.6\textwidth]{pictures/15092504.pdf}
\caption{Ion loss maps as simulated with the full STIER approach for three different impact parameters \mbox{$b=1\,\mu$m}, \mbox{$b=3\,\mu$m}, \mbox{$b=10\,\mu$m}.}
\label{fig:15062502.pdf}
  \end{center}
\end{figure}

\section{Chromatic Tracking in ICOSIM and SixTrack}
\begin{figure*}[t]
\begin{minipage}[t]{0.49\textwidth}
\includegraphics[width=\textwidth]{pictures/15092506.pdf}
\end{minipage}
\begin{minipage}[t]{0.49\textwidth}
\includegraphics[width=\textwidth]{pictures/15092507.pdf}
\end{minipage}
\caption{Comparison of the tracking behavior of ICOSIM and SixTrack for the two isotopes $^{8}$Li$^{3+}$ (left) and $^{207}$Pb$^{82+}$ (right) with identical starting conditions at the right jaw of the IR7 horizontal TCP. Note the different scales for the dispersion function and the computed horizontal position. The elements of the LHC beam line are shown on top of the graphs. }
\label{fig:15032202.pdf}
\end{figure*}



\textit{The chromatic modeling in ICOSIM is done in linear approximation, while SixTrack provides tracking up to order of 20. The expected deviations of the computed particle trajectories are small for particles with $|\delta_\text{eff}| \approx 0$, but increase for particles with large effective momentum offsets. To study this effect, we perform two tracking simulations with identical starting conditions at the TCP using SixTrack and ICOSIM. The tracks are computed for the two different isotopes, $^{8}$Li$^{3+}$ and $^{207}$Pb$^{82+}$, assuming that both start at the TCP with the same angle (the angle of incidence used for the fragmentation simulation) with the momentum per nucleon of the initial \lead beam. The difference in the bending behavior results from the different mass to charge ratios, quantified as $\chi_\text{Li}=1.054$ and $\chi_\text{Pb}=0.995$. The tracks are illustrated in Fig.~\ref{fig:15032202.pdf}. 
\\
For the $^{207}$Pb$^{82+}$ ion, with small $\delta_\text{eff}$, the simulated tracks are different by 200$\,\mu$m after a distance of 3$\,$km. With the light $^{8}$Li$^{3+}$ fragment, significant deviations between the simulated tracks occur after only a few magnets. The particle tracks differ by approximately 3$\,$mm after only 450$\,$m. 
\\
Such deviations can significantly change the simulated loss maps. Given that there are numerous ions with even larger effective momentum offsets (e.g. for protons $\chi_p = 0.39$) the tracking of such fragments must be carried out with a more elaborated chromatic tracking algorithm than used in ICOSIM. The light fragments can, as discussed in Sec.~\ref{sec:lm_simulations} contribute significantly to the losses not only locally at the collimators. Based on this result, SixTrack appears to be the better candidate to provide heavy-ion tracking, especially if the light fragments should be included in the simulation. }

