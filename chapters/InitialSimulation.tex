\chapter{SixTrack with Ion-Equivalent Rigidities }\label{chap:stier}
%
%
%\section*{Challenges in Heavy-Ion Collimation}
%
% The design of the LHC collimation system forsees the primary collimator in the betatron cleaning insertion IR7 as the first collimator in the cleaning hierarchy. It should be the only collimator exposed to particles of the main beam and is under normal circumstances the location with the highest amount of lost particles in the LHC ring. Besides the losses in the primary collimator, the measured qualification loss maps is expected to be dominated by secondary ion fragments from the interaction of the main beam with the material of the primary collimator. Tertiary particles generated from scattering and fragmentation in collimators downstream of the TCP should only contribute marginally to the final loss map. 

% Based on these assumptions, an intermediate simulation tool for heavy-ion cleaning was established. Instead of using a simplified fragmentation model for all collimators, this approach relies upon a detailed fragmentation simulation at the TCP, while the particle-matter interaction with other collimators is neglected. The resulting heavy-ion distribution starting at the TCP is tracked as protons with equivalent momenta in SixTrack, to take into account for the rigidities of the different heavy ions. The approach is referred to as SixTrack with Ion-Equivalent Rigidities (STIER). Initially developped to study the effect of the simplifications in ICOSIM and to determine the requirements for an improved simulation tool, STIER simulations proved to be in good agreement with the measured losses. It was used in the 2015 heavy-ion run to validate the collimator settings and to develop loss mitigation strategies, which were successfully tested in operation (see \chapref{}).

The benchmarking of ICOSIM simulations against measured LHC beam loss patterns unveiled that the simulation result shows discrepancies with respect to the measurement. Possible reasons might be the simplified fragmentation algorithm in ICOSIM, which does not take into account the transverse momentum transfer and changes of the kinetic energy from the fragmentation process. Also, the simplified tracking algorithm and contributions from light ions to the measured loss patterns could be reasons for the observed discrepancies.  


As a part of this thesis, the simulation tool SixTrack with Ion-Equivalent Rigidities (STIER) was developed, in order to verify or falsify these hypotheses. The aforementioned physics aspects can be individually probed with STIER. Comparisons to measurements and the ICOSIM simulation result allow for conclusions on their relevance for accurate predictions of heavy-ion loss patterns. Based on the results, the requirements for a further improved collimation simulation software are outlined. The content of this chapter has been published in part in~\cite{phermes_hb2014,NIM:819}.



\section{Efficiency of Staged Collimation for Heavy Ions} \label{colleff:ions}

Before the new simulation tool for heavy-ion fragmentation and tracking is presented, the origin for the worse cleaning performance with heavy-ions compared to protons shall be studied. 

The loss location of out-scattered particles that inevitably leave the primary collimator in IR7 depends on the type of interaction the particles have undergone. By virtue of its design, the collimation system is most efficient if the particles have been subject to small changes in rigidity, but received transverse momentum transfers large enough to scatter them into the secondary collimators. This is described by the relation defined in \eqref{dx:secon}. 

Among all superconducting LHC magnets, those in the IR7 DS are exposed to highest amount of collimation debris. It is therefore the region with the highest risk of beam-induced quenches.

Particles lost in the IR7 DS have been insufficiently scattered in the primary collimator, but have acquired rigidity offsets $1+\delta_\text{eff} = (1+\delta)/\chi$ that are outside of the rigidity acceptance of the dispersion suppressor magnets in IR7. A rough estimate for the latter is given by 
%
\begin{align}
  \delta_\text{eff}^\text{max}  = \pm A_g \, \tilde{D}_x \, . \label{rigacc}
\end{align}
%
where $A_g$ is the horizontal aperture in the magnet (approximately 22~mm in the IR7 DS) and $\tilde{D}_x$ is the horizontal dispersion generated locally between the TCP and the DS magnet considered. This relation is only valid in linear approximation and for particles without betatron offset. For real particles with betatron offsets, the acceptance may be reduced or enhanced, such that the expression in \eqref{rigacc} is only approximate and the real value for $\delta_\text{eff}^\text{max}$ becomes a distribution, rather than a constant.



\begin{figure}[htbp]
  \centering
  \begin{tikzpicture}
    \small
    \node[anchor=south west,inner sep=0] (image) at (0,0) {\includegraphics[width=1.0\linewidth]{pictures/16080301.pdf}};
    \node [x={(image.south east)},y={(image.north west)}]                   at (0.85,0.65)    {Protons};
    \node [x={(image.south east)},y={(image.north west)}]                   at (0.85,0.976)    {\lead};
    %\node [draw,rotate=0 ,x={(image.south east)},y={(image.north west)}]                   at (0.22,0.96)    {text1};
    \node [rotate=90 ,x={(image.south east)},y={(image.north west)},anchor=west]       at (1.02,0.45)    {Abundance [a.u.]};
    \draw[<-,color=black,x={(image.south east)},y={(image.north west)}]       (0.80,0.52) -- (0.84,0.47);
    \draw[<-,color=black,x={(image.south east)},y={(image.north west)}]       (0.87,0.53) -- (0.84,0.47);
    \node [x={(image.south east)},y={(image.north west)},anchor=west,align=center]       at (0.78,0.44)    {Angular \\ acceptance};
    \node [fill=white,x={(image.south east)},y={(image.north west)},anchor=west,align=center]       at (0.58,0.41)    {Rigidity \\ acceptance};
    \draw[->,color=black,x={(image.south east)},y={(image.north west)}]       (0.69,0.42) -- (0.725,0.44);
    \draw[->,color=black,x={(image.south east)},y={(image.north west)}]       (0.69,0.42) -- (0.755,0.39);

    \node [x={(image.south east)},y={(image.north west)}]                   at (0.5,0.10)    {Protons};
    \node [x={(image.south east)},y={(image.north west)}, align=center]                   at (0.40,0.25)    {\lead \\ fragments};

        \filldraw[fill=white,draw=white,x={(image.south east)},y={(image.north west)}] (0,0.1) rectangle (0.049,0.3);

    \node [rotate=90,x={(image.south east)},y={(image.north west)}]                   at (0.025,0.20)    {Relative Abundance};

    \node [draw,x={(image.south east)},y={(image.north west)}]                   at (0.3,0.33)    {Particles inside the angular acceptance};


   % \draw[help lines,step=.05,x={(image.south east)},y={(image.north west)}] (0,0) grid (1,1);
   %  \draw[help lines,line width=.6pt,step=0.1,x={(image.south east)},y={(image.north west)}] (0,0) grid (1,1);
   %  \foreach \x in {0,0.1,0.2,0.3,0.4,0.5,0.6,0.7,0.8,0.9,1.0}
   %     \node[anchor=north,x={(image.south east)},y={(image.north west)}] at (\x,-0.01) {\x};
   %  \foreach \y in {0.0,0.1,0.2,0.3,0.4,0.5,0.6,0.7,0.8,0.9,1.0}
   %     \node[anchor=east,x={(image.south east)},y={(image.north west)}] at (-0.01,\y) {\y};
  \end{tikzpicture}
  \caption{Top and middle plot: FLUKA simulated map of transverse angular kick received at the passage through a 10.3~cm thick carbon target vs particle momentum per nucleon for an initial beam of \lead ions (top) and protons (middle). The horizontal lines represent the TCSG acceptance and the vertical lines the rigidity acceptance of the MQ.11R7.B1. All data points are weighted with the total energy per bin. The bottom plot shows a projected histogram of all particles inside the TCSG acceptance as a function of the momentum per nucleon. Both histograms are weighted with the particle energy and are normalized to cover a surface of 1. }  
  \label{pic:16072101} %/media/phermes/ph3tboffice/ph1tbwd/FLUKA_results/150506_HeavyIon_3500GeV_perpendicular_3um/scattering_protons_ions.pdf
  \end{figure}




% Therefore, the effectiveness of the LHC multi-stage collimation system with low residual collimation losses in the IR7 DS region depends is best if the particles leaving the primary collimator 


% different cleaning situations for protons:
%    1. proton lost in primary collimator
%    2. elastic proton scattering in primary collimator to an angle x' large enough such that the particle is
%       intercepted by the secondary collimator
%    3. elastic proton scattered in primary collimator (small losses from ionization and showering) but 
%       uncaptured of the secondary collimator, particle is captured at one of the next passages
%    4. inelastic scattering (single diffractive events), particles lose significant amounts of energy and
%       are lost in the DS region downstream of the collimator 


In the following, the distribution in angle and energy of residual particles created by \lead ions and protons impacting the TCP is studied by FLUKA simulations. In both cases, a particle beam with an energy of $3.5\,Z\,$TeV is simulated to perpendicularly hit a carbon target of 10.3$\,$cm thickness. This is comparable to the distance particles traverse in the material of the primary collimator with impact parameters of $3\,\mu$m at an angle of 29.1$\,\mu$rad.

The resulting distribution in terms of the momentum per nucleon and the scattering angle $\Delta x'$ of all  particles scattered out of the collimator material is shown for both simulations in the top and middle plots of Fig.~\ref{pic:16072101}. The horizontal lines show the minimum angular kick $\Delta x'$ required such that a particle intercepts the secondary collimator. The vertical lines show the rigidity acceptance of the MQ.11R7.B1 around the nominal beam energy of $3.5\,Z\,$TeV. Assuming an aperture of $A_g = \pm 22~\,$mm and a local dispersion function of $\tilde{D}_x = 2.4\,$m, the rigidity acceptance yields approximately \mbox{$\delta_\text{eff}^\text{max}=9 \times 10^{-3}$}. Particles not lost in this magnet are lost at locations downstream of the DS clusters. Note also that the dispersion function in this magnet is larger than in the DS1 cluster upstream, so the rigidity acceptance of the DS1 is larger.

The comparison demonstrates that the number of particles outside the rigidity acceptance of the DS magnets but inside the angular acceptance of the TCSG is significantly larger for the heavy-ion fragments than for out-scattered protons. The bottom plot of \figref{pic:16072101} shows the projected number of nucleons inside the angular acceptance of the TCSG collimators. For rigidities beyond $\pm \delta_\text{eff}^\text{max}$, the fraction of heavy ions (black line) is larger by up to three orders of magnitude compared to the proton distribution. 

The integral of the distribution outside the rigidity acceptance in the bottom plot of \figref{pic:16072101} yields approximately $5.1 \times 10^{-3}$ for protons and $8.6 \times 10^{-1}$ for heavy ions. This difference is the origin of the larger cleaning inefficiency with \lead ions, which is two orders of magnitude above the cleaning inefficiency for proton beams. Given the drastic impact on the cleaning inefficiency, the effect of fragmentation and the motion of the ion fragments in the LHC must be accurately modeled in a simulation tool for heavy-ion collimation.







%\newpage
\section{The STIER Simulation Tool}

The design of the LHC collimation system foresees the TCPs in IR7 as the first collimators in the cleaning hierarchy. For every plane (H,V,S), the respective TCP should be the only collimator exposed to particles of the main beam. It is under normal circumstances the location with the highest amount of lost particles in the LHC. The pattern of collimation losses should hence be dominated by residual fragments generated from the interaction of the main beam particles with the TCP. It should hence be possible to accurately describe the loss distribution by a simulation model which takes into account only the fragmentation at the primary collimator.

Based on these assumptions, a new simulation tool for heavy-ion cleaning was established as an alternative to ICOSIM. Instead of using a simplified fragmentation model for all collimators, the approach relies upon a detailed fragmentation simulation only at the TCP. This includes taking into account all residual fragments and kicks in angle and energy from the fragmentation process. Other collimators are treated as perfect absorbers. 

The resulting distribution of residual heavy-ion fragments is tracked as protons with ion-equivalent rigidities in SixTrack, starting from the TCP. The approach is referred to as SixTrack with Ion-Equivalent Rigidities (STIER). Initially developed to study the effect of the simplifications used in ICOSIM, and to determine the requirements for an improved simulation tool, STIER simulations proved to be in good agreement with the measured loss patterns. It was used in the 2015 heavy-ion run to validate the collimator settings and to develop loss mitigation strategies, which were successfully tested in operation (see \chapref{chap:sim_meas}).

%
  \begin{figure}[b]
  \centering
  \includegraphics[width=0.5\textwidth]{pictures/15063001.pdf}
  \caption{Three stages of the STIER simulation setup.}  
  \label{pic:15062601}
  %/home/phermes/Dropbox/PhD/pictures/STIER-schematics/thesis/simulation-overview4.pdf
  \end{figure}
%

STIER relies upon three consecutive simulation steps shown in \figref{pic:15062601}. In the first step, the phase space properties of the particles impacting the collimator jaws of the TCP are determined by means of MAD-X. The angle of incidence is then used as an input for the following simulation step in which the interaction of the primary heavy-ion beam with the material of the primary collimator is simulated with FLUKA. The information about the out-scattered ion fragments is then converted to input for SixTrack where the ions are tracked as protons with momenta that match the rigidities of the individual ion fragments.

In the following sub-sections, the three STIER stages are described in detail and important simulation results for every stage are summarized. Some of the results are also important for the development of a further advanced heavy-ion collimation tool described in \chapref{chap:hisix}.
%
%
\subsection{Optics Calculation}
%
\begin{figure}[htpb]
  \centering
   \def\svgwidth{0.6\linewidth}
   \input{pictures/hybrid_pictures/16021601.pdf_tex}
  \caption{Phase space diagram of particles at a maximum normalized betatron amplitude $N_P$. Particles hitting the left and right jaw of the TCP have specific coordinates in phase space.}
  \label{pic:14070304}
\end{figure}
%
As a first step of the STIER simulation approach, the optical functions are computed with MAD-X. Particles impacting the primary collimator have specific properties in phase space, as illustrated in \figref{pic:14070304}. The collimator jaw is assumed to be in parallel to the nominal closed orbit (i.e. it is not aligned to the divergence of the beam envelope). Furthermore, it is assumed that the diffusion is slow enough for particles to hit the TCP close to their maximum spacial amplitude. Then one can conclude from \figref{pic:14070304} that the particles impacting the collimator do so with a non-zero angle of incidence, if $\alpha \neq 0$. This is illustrated in \figref{pic:14112701}. 

The angle of incidence $x_{l/r}'$ at the left ($x>0$ for B1 and vice versa for B2) and right collimator jaw is defined by the normalized half gap $N_P$, the geometric emittance $\epsilon_x$ and the Twiss parameters $\beta_x,\alpha_x$ at the location of the TCP as follows~\cite{wiedemann1999particle}
\begin{align}
x_{l/r}' = \mp N_P \, \alpha_x \, \sqrt{\frac{\epsilon_x}{\beta_x}} \, . \label{eq:angle_of_incidence}
\end{align}
%
For LHC emittances and collimator settings, the distance the primary beam particles traverse in the collimator material scales, in good approximation, linearly with the impact parameters. They were found in SixTrack simulations for protons to be between 1\mum\, and 10\mum\,\cite{Bruce2014a}.  Assuming that the diffusion for heavy-ion beams is similar, and taking into account the angle of incidence for $3.5\,Z\,$TeV presented in \tabref{tab:optical_functions}, the traversed distances lie in the range from $\approx 3\,$cm to $30\,$cm. This is in the same order of magnitude as the mean free path for nuclear interactions (see \tabref{tab:physics_ions_matter}). The fragment composition of the out-scattered heavy ions hence depends strongly on the angle of incidence and on the impact parameter.


% The geometrical collimator settings are 
% While the settings for proton assume a normalized emittance of $\epsilon_x^{p}=3.5\,\mu$m, the design normalized emittance of the heavy-ion beams is $\epsilon_x^{\text{Pb}}=1.5\,\mu$m. 

% For the past heavy-ion runs in 2011, 2013 and 2015, the geometrical collimator gaps of the previous proton runs were kept for heavy-ion operation. 

% than for protons mainly due to the electron cooling in LEIR. 

% Given the different masses, the geometrical emittances for heavy-ion beams and 


% The equivalent emittance of heavy-ion beams is given in \tabref{tab:optical_functions}.
%
A summary of the Twiss parameters at the TCP, the collimator opening and the resulting angles of incidence for different scenarios are shown in \tabref{tab:optical_functions}. Note that the optics in IR7 remains unchanged during the LHC cycle and for the different configurations. 

%
%
\begin{table}[b]
\centering
\caption{Summary of the parameters used to calculate the angle of incidence at the primary collimator for different LHC configurations. The Twiss parameters $\beta_x$ and $\alpha_x$ are computed using MAD-X. The angle $x'$ is calculated by means of \eqref{eq:angle_of_incidence}.}
\label{tab:optical_functions}
\begin{tabular}{cccccccc}
\toprule
Year & \begin{tabular}[c]{@{}c@{}}$E$\\ {[}$Z$ GeV{]}\end{tabular} & \begin{tabular}[c]{@{}c@{}}$\beta_x$\\ {[}m{]}\end{tabular} & \begin{tabular}[c]{@{}c@{}}$\alpha_x$\\ {[} {]}\end{tabular} & %
\begin{tabular}[c]{@{}c@{}}$\epsilon_N$\\ {[}$\mu$m rad{]}\end{tabular} & %
\begin{tabular}[c]{@{}c@{}}$\gamma$\\ {[} {]}\end{tabular} & \begin{tabular}[c]{@{}c@{}}$N_p$\\ {[}$\sigma${]}\end{tabular} & \begin{tabular}[c]{@{}c@{}}$x'$\\ {[}rad{]}\end{tabular} \\ \midrule
2010/2011 & 3500 &  148.46 & 2.04 & 1.50 & 1482.8 & 5.7 & $-2.9\times 10^{-5}$ \\
2015 & 6370 &  148.46 & 2.04 & 1.41 & 2696.8 & 5.5 & $-2.1\times 10^{-5}$ \\
Design & 7000 &  148.46 & 2.04 & 1.50 & 2964.5 & 5.7 & $-2.1\times 10^{-5}$ \\ \bottomrule
\end{tabular}
\end{table}
%
 %%%%%%%%%%%%%%%%%%%%%%%%%%%%%%%%%%%%%%%%%%%%%%%%%%%%%%%%%%%%%%%%%%%%%%%%%%%%%%%%%%%%%%%%%%%%%%%%%%%%%%%%%%%%%%%%%%%%%%%%%%%%%%%%%%%%%%%%%%%%%%%%
 %
 %  FRAGMEnTATION SIMULATION
 %  
 %
 %
 %%%%%%%%%%%%%%%%%%%%%%%%%%%%%%%%%%%%%%%%%%%%%%%%%%%%%%%%%%%%%%%%%%%%%%%%%%%%%%%%%%%%%%%%%%%%%%%%%%%%%%%%%%%%%%%%%%%%%%%%%%%%%%%%%%%%%%%%%%%%%%%%
%
\subsection{Fragmentation Simulation} \label{chap:STIERfrag}
%
\subsubsection{Simulation Setup}
%
\begin{figure}[t]
  \centering
  \def\svgwidth{0.65\linewidth}
  \input{pictures/hybrid_pictures/14112701.pdf_tex}
  \caption{
    Geometry used for the FLUKA simulation of the fragmentation
    at the TCP.%
  }
  \label{pic:14112701}
\end{figure}
%
The fragmentation at the TCP is simulated with FLUKA. The primary collimator is modeled as a simple rectangular carbon cuboid of 60~cm length (see \figref{pic:14112701}). Alternatively, the more accurately modeled collimator geometry of the FEDB can be used. Comparisons between fragmentation simulations using the two geometries have shown no significant discrepancy in the resulting spectrum of heavy-ion fragments. The density of the carbon composite is set to 1.61~g/cm$^3$ to account for the CFC material used for the TCPs. 

Species, energy and transverse momentum of the heavy ions arriving at the end of the collimator jaw are saved to an output file. Other particles than protons or heavy ions are ignored. They are most probably lost in the warm aperture immediately downstream of the TCP. While this approximation is valid for simulations of the collimation performance, detailed shower simulations of energy deposition and radiation dose in the collimation region IR7 take them into account~\cite{NucDataSheet:120}.  

The FLUKA simulation input file is adjusted to take into account for electromagnetic dissociation, nuclear fragmentation using the DPMJET-III model~\cite{MC2000:DPMJETIII} and nuclear evaporation. 

In the presented simulations of the 2011 cleaning performance, the primary beam is simulated as $10^7$ particles of \lead at an energy of $3.5\,Z\,$TeV, impacting the TCP at an angle of incidence of $x'_{r,l}= \pm 2.91\times 10^{-5}\,$rad. They have been carried out for three different impact parameters: $b=1$\mum, $b = 3$\mum\, and $b=10$\mum. The beam is simulated to impact with fixed $x,x'$ (in contrast to a distribution in $x,x'$) and no amplitude in $y,y'$. This setup is called pencil beam.

\subsubsection{Fragment Spectrum}


\begin{figure}[htbp]
  \centering
  \begin{tikzpicture}
    \node[anchor=south west,inner sep=0] (image) at (0.1,0) {\includegraphics[width=0.75\linewidth]{pictures/16081208.pdf}};
    %\node [draw,rotate=90,x={(image.south east)},y={(image.north west)}]                   at (0.50,0.50)    {text0};
    %\node [draw,rotate=0 ,x={(image.south east)},y={(image.north west)}]                   at (0.22,0.96)    {text1};
    %\node [draw,rotate=0 ,x={(image.south east)},y={(image.north west)},anchor=west]       at (0.22,0.80)    {text2};
    %\draw[->,color=black,thick,x={(image.south east)},y={(image.north west)}]             (0.42,0.22) -- (0.37,0.23);
  \end{tikzpicture}
  \caption{Energy fraction carried by the individual isotopes in the fragmentation of $10^7$ \lead ions impacting the TCP at an impact parameter of $b=3$\mum. The data is normalized to the total energy of all ions scattered out of the  collimator.}  
  \label{fig:stier_fragmentation}
  %/media/phermes/ph3tboffice/ph1tbwd/FLUKA_results/150127_HeavyIon_3500GeV_5e6particles_b03um/output/isotopeMap_thesis.pdf
  \end{figure}





% \begin{figure}[htbp]
%   \centering
%   \begin{tikzpicture}
%     \node[anchor=south west,inner sep=0] (image) at (0,0) {\includegraphics[width=0.7\linewidth]{pictures/15092404.pdf}};
%     \node [fill=white,rotate=90,x={(image.south east)},y={(image.north west)}]                   at (0.975,0.58)    {Energy fraction};
%     %\node [draw,rotate=0 ,x={(image.south east)},y={(image.north west)}]                   at (0.22,0.96)    {text1};
%     %\node [draw,rotate=0 ,x={(image.south east)},y={(image.north west)},anchor=west]       at (0.22,0.80)    {text2};
%     %\draw[->,color=black,thick,x={(image.south east)},y={(image.north west)}]             (0.42,0.22) -- (0.37,0.23);
%   \end{tikzpicture}
%   \caption{Energy fraction carried by the individual isotopes, normalized with the total ion energy which leaks out of the collimator.}  
%   \label{fig:stier_fragmentation}
%   %/media/phermes/ph3tboffice/ph1tbwd/FLUKA_results/141013_HeavyIon_3500GeV_1e6particles_b1um/A_distribution.pdf
%   \end{figure}



\begin{figure}[htbp]
  \centering
  \begin{tikzpicture}
    \node[anchor=south west,inner sep=0] (image) at (0,0) {\includegraphics[width=0.7\linewidth]{pictures/16080303.pdf}};
    %\node [draw,rotate=90,x={(image.south east)},y={(image.north west)}]                   at (0.50,0.50)    {text0};
    %\node [draw,rotate=0 ,x={(image.south east)},y={(image.north west)}]                   at (0.22,0.96)    {text1};
    %\node [draw,rotate=0 ,x={(image.south east)},y={(image.north west)},anchor=west]       at (0.22,0.80)    {text2};
    %\draw[->,color=black,thick,x={(image.south east)},y={(image.north west)}]             (0.42,0.22) -- (0.37,0.23);
  \end{tikzpicture}
  \caption{Mass numbers of ion fragments simulated by FLUKA simulated for \lead ions impacting the TCP at three different impact parameters $b=1$\mum, $b=3$\mum\, and $b=10$\mum. The individual contribution of each isotope is weighted with the momentum. The distributions are normalized to the total energy carried by all ions.}  
  \label{pic:16080302}
  %/media/phermes/ph3tboffice/ph1tbwd/FLUKA_results/141013_HeavyIon_3500GeV_1e6particles_b1um/A_distribution.pdf
  \end{figure}


From the FLUKA simulations for the three impact parameters $1$\mum, $3$\mum\, and $10$\mum, the isotope spectra  can be derived for the different simulation cases. In \figref{fig:stier_fragmentation}, the fraction of energy carried by the different isotopes is shown for the simulation with $b=$$3$\mum. The fragment spectrum covers almost the full range of nuclei lighter than \lead, with over 3000 different fragments created. 

In \figref{pic:16080302}, the distribution of $A$, weighted with the particle momentum, is shown for the simulations with the three impact parameters. With an impact parameter of $b=1$\mum, the distribution is clearly dominated by heavy fragments, with masses closely to the main beam. For $b=3$\mum, the production yield of lighter ion fragments increases. For \mbox{$b=10$\mum}\, the spectrum is dominated by very light fragments such as helium nuclei, tritium, deuterium and protons. In all simulations, the main amount of energy is carried by either very light ion fragments or isotopes with $A$ and $Z$ close to the main beam. 
\newpage
A detailed overview of the energy fraction carried by the most important isotopes is given for the different impact parameters in \tabref{tab:importance}.

%
%
\begin{table}[b]
\centering
\caption{Isotopes with the largest energetic fractions scattered out of the collimator material from the initial fragmentation simulation. }
\small
\setlength\tabcolsep{2.5pt}
\label{tab:importance}
\begin{tabular}{ccccccc}
\toprule
          \multicolumn{1}{c}{}     &     &  \multicolumn{5}{c}{Energetic Fraction}                    \\

\multicolumn{1}{c}{Isotope}       &   & \multicolumn{1}{c}{$b=1\,\mu$m} &            & \multicolumn{1}{c}{$b=3\,\mu$m}      &       & \multicolumn{1}{c}{$b=10\,\mu$m}            \\ \midrule
$^1$H$^{1+}$      & \phantom{a} & $4.7 \times 10^{-2}$ & \phantom{a} & $6.3 \times 10^{-2}$ & \phantom{a} & $4.0 \times 10^{-1}$  \\
$^2$H$^{1+}$      &  & $2.1 \times 10^{-2}$ & & $2.5 \times 10^{-2}$ & & $1.2 \times 10^{-1}$  \\
$^3$H$^{1+}$      &  & $1.5 \times 10^{-2}$ & & $1.7 \times 10^{-2}$ & & $7.4 \times 10^{-3}$  \\
$^3$He$^{2+}$     &  & $5.8 \times 10^{-3}$ & & $8.1 \times 10^{-3}$ & & $4.9 \times 10^{-3}$ \\
$^4$He$^{2+}$     &  & $3.6 \times 10^{-2}$ & & $4.2 \times 10^{-2}$ & & $1.6 \times 10^{-1}$  \\
$^{205}$Pb$^{82+}$&  & $7.1 \times 10^{-3}$ & & $2.3 \times 10^{-3}$ & & $1.1 \times 10^{-5}$  \\
$^{206}$Pb$^{82+}$&  & $1.7 \times 10^{-2}$ & & $5.0 \times 10^{-3}$ & & $ 1.4 \times 10^{-5}$ \\
$^{207}$Pb$^{82+}$&  & $3.3 \times 10^{-2}$ & & $ 8.4\times 10^{-3}$ & & $1.0 \times 10^{-5}$  \\
$^{208}$Pb$^{82+}$&  & $3.6 \times 10^{-1}$ & & $3.6\times 10^{-2}$  & & $1.4 \times 10^{-5}$  \\

\bottomrule
\end{tabular}
\end{table}
%
%
%



The reason for the increasing number of light ion fragments with larger impact parameters is the linear relation between the impact parameter and the distance the particles travel through the material. The probability of fragmentation increases exponentially with the traversed distance through the collimator. As shown in \tabref{tab:importance}, for an impact parameter of $b=1\,\mu$m, the amount of energy carried by protons is approximately one order of magnitude below the energy carried by out-scattered \lead ions. At $b=10\,\mu$m, approximately 40\% of the out-scattered beam energy is carried by protons and 16\% by $^4$He$^{2+}$ ions, while the contribution of particles of the main beam is lower by four orders of magnitude. This demonstrates that the impact parameter is important for the isotope spectrum generated at the passage through the collimator. The impact on the resulting loss distribution is studied in the following chapters.


\begin{figure}[htbp]
  \centering
  \begin{tikzpicture}
    \footnotesize
    \node[anchor=south west,inner sep=0] (image) at (0,0) {\includegraphics[width=0.6\linewidth]{pictures/16090202.pdf}};
    \node [fill=white,rotate=90 ,x={(image.south east)},y={(image.north west)},align=center]                   at (0.49,0.81)    {$P_0/A_0 =$\\$1379.8\,$GeV/$c$};
  \end{tikzpicture}
  \caption{Momentum per nucleon of ion fragments with $A>5$ (blue) and $A<5$ (green) generated in the interaction of $10^6$ ions of \lead with the TCP material for an impact parameter of $3\,$\mum. Both distributions are normalized such that the sum of all data points yields one. The vertical line shows the momentum per nucleon of the reference particle.}  
  \label{pic:15091701}
  %/media/phermes/1E10-926A/160902_PA/PAAbundance.pdf
  \end{figure}


In ICOSIM it is assumed that the momentum per nucleon $P/A$ of the fragments generated in NF and EMD processes is similar to the momentum per nucleon of the main beam ($\delta \approx 0$). The FLUKA simulation allows to analyze the spectrum of $P/A$ for the out-scattered ions. This spectrum is shown for ion fragments with $A<5$ and $A>5$ in \figref{pic:15091701}. The comparison shows that the heavy fragments have indeed mostly $P/A$ values close to that of the main beam ($1379.8\,$GeV$/c$). Light fragments, however, have a broad distribution in terms of mass per nucleon and some fragments have significantly larger $P/A$ than the main beam. Parts of the Fermi-motion in the nucleus can be transferred to individual ion fragments, which can lead to a significant gain in momentum for the light fragments. The largest $P/A$ obtained from the FLUKA simulation is approximately 1950$\,$GeV$/c, A$. This broad spectrum of momenta indicates that an improved heavy-ion collimation tool should include the change of kinetic energy from the fragmentation process, especially if light fragments shall be included in the simulation.


% \begin{figure}[htbp]  
%     \centering
%     \includegraphics[width=0.6\textwidth]{pictures/16021602.pdf}
%     \caption{Transverse angles of the two ion species $^{207}$Pb$^{82+}$ and $^{4}$He$^{2+}$.}  
%     \label{pic:16021601}
%     %/media/phermes/ph3tboffice/ph1tbwd/FLUKA_results/150506_HeavyIon_3500GeV_perpendicular_3um/transverse_angle.pdf
% \end{figure}









\subsection{Heavy-Ion Tracking in STIER}

SixTrack is designed for the tracking of protons, so the tracking algorithm does not include effects of dispersion from a different mass to charge ratio. Following \eqref{eq:d_effective}, the rigidity offset $1+\delta_\text{eff}$ of an ion fragment with momentum per mass offset $\delta$ and mass to charge offset $\chi$, can be described as
%
\begin{align}
1+\delta_\text{eff} = \frac{(1+\delta)}{\chi}  \,. 
\end{align}
%
If the reference particle is a proton, the applicable proton momentum to obtain the same rigidity as the heavy ion to be tracked yields:
\begin{align}
  P_E = P_0 \, (1+\delta_\text{eff}) = \frac{P}{Z} \, .
\end{align}
%
The transverse motion of a heavy ion not matched to the magnetic lattice can therefore be accurately simulated by assigning the ion-equivalent momentum $P_E$ to a proton. From the FLUKA simulation output, the momentum and charge of every ion fragment can be extracted and converted into proton momenta to track the heavy-ion distribution in STIER. Furthermore, the transverse positions and angles are taken into account for the initial conditions. 

\newpage
Losses in the aperture of the LHC magnets are identified with \texttt{BeamLossPattern}, which is also used for proton studies with SixTrack. Momenta of ions generated from the fragmentation of \lead span over two orders of magnitude. Therefore, particle losses are weighted with the total energy of the impacting nuclei, as described in \eqref{eq:eta:ions}.

The Monte-Carlo routine implemented in SixTrack to simulate the proton interaction with the collimators must be avoided because it is not adapted to heavy ions. In STIER, the collimators are therefore set to perfect absorbers. This approach will lead to an overestimation of the collimator losses which should be considered in the analysis of the simulated loss patterns. 

Accelerating RF cavities are switched off, because the acceleration depends on the particle charge, which is not incorporated in the framework. However, it is expected that the number of turns that a heavy-ion fragment can perform in the machine without being lost is very small compared to the synchrotron period of $\approx 500$ turns.



\section{Simulation Results} \label{chap:STIERresults}
\subsection{Full Heavy-Ion Loss Map Simulations} \label{chap:STIER:full}

% \subsection{Loss Map simulations} \label{sec:lm_simulations}
% \begin{figure*}[htbp]
% \begin{minipage}[t]{0.495\textwidth}
% \includegraphics[width=\textwidth]{pictures/15102001.pdf}
% \end{minipage}
% \begin{minipage}[t]{0.495\textwidth}
% %\includegraphics[width=\textwidth]{pictures/15102002.pdf}

%  \begin{tikzpicture}
%     \tiny
%     \node[anchor=south west,inner sep=0] (image) at (0,0) {\includegraphics[width=1\linewidth]{pictures/15102002.pdf}};
%     \node [fill=white,x={(image.south east)},y={(image.north west)}]                   at (0.30,0.48)    {DS1};
%     \node [x={(image.south east)},y={(image.north west)}]                   at (0.35,0.48)    {DS2};
%     %\node [draw,rotate=0 ,x={(image.south east)},y={(image.north west)},anchor=west]       at (0.22,0.80)    {text2};
%   \end{tikzpicture}

% \end{minipage}
% \caption{Comparison of the B1H loss map simulations using ICOSIM (top row), the simplified STIER approach (second row), a full STIER simulation (third row) and the measured B1H qualification loss maps during the 2011 LHC heavy-ion run at 3.5\,$Z\,$TeV. The right graph in the bottom row also shows the locally generated dispersion function $D_x$ starting at $D_x=0$ at the TCP. The left column shows the loss map over the full LHC ring, while the right column shows the same loss map zoomed into the betatron collimation region IR7. The STIER simulations are carried out assuming an impact parameter of $b=3\,\mu$m. The losses in the DS are denominated DS1 and DS2, those in the arc A1 to A4.}
% \label{fig:comparison_lossmaps}
% \end{figure*}



\begin{figure}[htbp]
  \centering
  \begin{tikzpicture}
    \footnotesize
    \node[anchor=south west,inner sep=0] (image) at (0,0) {\includegraphics[width=1.0\linewidth]{pictures/16090102.pdf}};
  \node [fill=white,x={(image.south east)},y={(image.north west)}]   at (0.60,0.930)    {ICOSIM};
  \node [fill=white,x={(image.south east)},y={(image.north west)}]   at (0.60,0.700)    {Simplified STIER};
  \node [fill=white,x={(image.south east)},y={(image.north west)}]   at (0.60,0.475)    {Full STIER};
  \node [fill=white,x={(image.south east)},y={(image.north west)}]   at (0.60,0.255)    {Measurement};

  \node [x={(image.south east)},y={(image.north west)}]   at (0.31,0.435)    {DS1};
  \node [x={(image.south east)},y={(image.north west)}]   at (0.35,0.435)    {DS2};
  \node [x={(image.south east)},y={(image.north west)}]   at (0.4,0.435)    {A1};
  \node [x={(image.south east)},y={(image.north west)}]   at (0.525,0.435)    {A2};  
  \node [x={(image.south east)},y={(image.north west)}]   at (0.69,0.435)    {A3};
  \node [x={(image.south east)},y={(image.north west)}]   at (0.86,0.435)    {A4};
  \node [fill=white,x={(image.south east)},y={(image.north west)}]   at (0.125,0.475)    {W1};

  \node [fill=white,x={(image.south east)},y={(image.north west)}]   at (0.78,0.245)    {$\tilde{D}_x$};

  % \node [draw,rotate=0 ,x={(image.south east)},y={(image.north west)}]                   at (0.22,0.96)    {text1};
  % \node [draw,rotate=0 ,x={(image.south east)},y={(image.north west)},anchor=west]       at (0.22,0.80)    {text2};
  % \draw[->,color=black,thick,x={(image.south east)},y={(image.north west)}]             (0.42,0.22) -- (0.37,0.23);

 % \draw[help lines,step=.05,x={(image.south east)},y={(image.north west)}] (0,0) grid (1,1);
 % \draw[help lines,line width=.6pt,step=0.1,x={(image.south east)},y={(image.north west)}] (0,0) grid (1,1);
 % \foreach \x in {0,0.1,0.2,0.3,0.4,0.5,0.6,0.7,0.8,0.9,1.0}
 %      \node[anchor=north,x={(image.south east)},y={(image.north west)}] at (\x,-0.01) {\x};
 % \foreach \y in {0.0,0.1,0.2,0.3,0.4,0.5,0.6,0.7,0.8,0.9,1.0}
 %     \node[anchor=east,x={(image.south east)},y={(image.north west)}] at (-0.01,\y) {\y};

  \end{tikzpicture}
  \caption{Comparison of the B1H loss map simulations using ICOSIM (top row), the simplified STIER approach (second row), a full STIER simulation (third row) and the measured B1H qualification loss maps (bottom) for the 2011 LHC heavy-ion run at 3.5\,$Z\,$TeV. The bottom row also shows the locally generated dispersion function $\tilde{D}_x$ from the TCP. The STIER simulations are carried out assuming an impact parameter of $b=3\,\mu$m. }  
  \label{fig:comparison_lossmapsIR7}
  %/media/phermes/local/141003_compareICOSIMSixTrackMeasurement/IR7comparisonThesis.pdf
  \end{figure}




\begin{figure}[htbp]
  \centering
  \begin{tikzpicture}
    \footnotesize
    \node[anchor=south west,inner sep=0] (image) at (0,0) {\includegraphics[width=1.0\linewidth]{pictures/16090103.pdf}};
  % \node [draw,rotate=90,x={(image.south east)},y={(image.north west)}]                   at (0.50,0.50)    {text0};
  % \node [draw,rotate=0 ,x={(image.south east)},y={(image.north west)}]                   at (0.22,0.96)    {text1};
  % \node [draw,rotate=0 ,x={(image.south east)},y={(image.north west)},anchor=west]       at (0.22,0.80)    {text2};
  \draw[->,color=black,x={(image.south east)},y={(image.north west)}]             (0.45,0.215) -- (0.52,0.215);
  \node [x={(image.south east)},y={(image.north west)}]   at (0.48,0.225)    {Beam};

 % \draw[help lines,step=.05,x={(image.south east)},y={(image.north west)}] (0,0) grid (1,1);
 % \draw[help lines,line width=.6pt,step=0.1,x={(image.south east)},y={(image.north west)}] (0,0) grid (1,1);
 % \foreach \x in {0,0.1,0.2,0.3,0.4,0.5,0.6,0.7,0.8,0.9,1.0}
 %      \node[anchor=north,x={(image.south east)},y={(image.north west)}] at (\x,-0.01) {\x};
 % \foreach \y in {0.0,0.1,0.2,0.3,0.4,0.5,0.6,0.7,0.8,0.9,1.0}
 %     \node[anchor=east,x={(image.south east)},y={(image.north west)}] at (-0.01,\y) {\y};

  \node [x={(image.south east)},y={(image.north west)}]   at (0.20,0.961)    {ICOSIM};
  \node [x={(image.south east)},y={(image.north west)}]   at (0.20,0.725)    {Simplified STIER};
  \node [x={(image.south east)},y={(image.north west)}]   at (0.20,0.490)    {Full STIER};
  \node [x={(image.south east)},y={(image.north west)}]   at (0.20,0.255)    {Measurement};

  \node [x={(image.south east)},y={(image.north west)}]   at (0.10,0.215)    {A7};
  \node [x={(image.south east)},y={(image.north west)}]   at (0.21,0.215)    {A8};
  \node [x={(image.south east)},y={(image.north west)}]   at (0.245,0.215)    {A9};

  \node [x={(image.south east)},y={(image.north west)}]   at (0.90,0.215)    {A5/A6};

  \node [x={(image.south east)},y={(image.north west)}]   at (0.965,0.42)    {S78};
  \node [x={(image.south east)},y={(image.north west)}]   at (0.14,0.42)    {S12};


  \end{tikzpicture}
  \caption{Comparison of the B1H loss map simulations using ICOSIM (top row), the simplified STIER approach (second row), a full STIER simulation (third row) and the measured B1H qualification loss maps during the 2011 LHC heavy-ion run at 3.5\,$Z\,$TeV. The STIER simulations are carried out assuming an impact parameter of $b=3\,\mu$m. }  
  \label{fig:comparison_lossmapsLHC}
  %/media/phermes/local/141003_compareICOSIMSixTrackMeasurement/IR7LHCcomparisonThesis.pdf
  \end{figure}






In this section, the simulated B1H loss maps in the 2011 configuration at $3.5\,Z\,$TeV (see \tabref{tab:betastar} and \tabref{tab:14070901}) from ICOSIM and STIER are compared to the measured B1H qualification loss map. The STIER simulation is carried out with an impact parameter of $3$\mum. 

To compare with ICOSIM, an additional STIER simulation with simplified out-scattering from the TCP is carried out. Angular and energetic kicks from the fragmentation simulation are not taken into account. The approach is referred to as \textit{simplified STIER}. ICOSIM and the simplified STIER approach are not fully comparable because of the different aperture checks, different tracking routines and the losses from subsequent fragmentation processes included in ICOSIM but not in STIER. Furthermore, simplified STIER includes light ion fragments. Their contribution to the loss pattern is studied separately later-on. 

The result of the ICOSIM simulation presented before in \chapref{firstICOS} is shown in the first row of Fig.~\ref{fig:comparison_lossmapsIR7} (zoomed to IR7) and \figref{fig:comparison_lossmapsLHC} (full LHC ring). The measured losses are shown as a comparison to all simulations in the bottom row of these figures.

\subsubsection{Simplified STIER}

The loss map simulated with the simplified STIER approach is shown in the second row of \figref{fig:comparison_lossmapsIR7} and \figref{fig:comparison_lossmapsLHC}. The comparison unveils that some additional loss features are visible compared to ICOSIM. This applies, for example, to the S12 losses in the arc between \mbox{IR1 and IR2}. 

Also the A6 loss peak downstream of IR8, which is also measured with the BLMs, is visible in the simplified STIER approach, but not in ICOSIM. Further analysis shows that mainly \iso{206}{Pb}{82+} ions are lost in the A6 peak in the simplified STIER simulation. Possibly, this discrepancy can be explained by the different tracking algorithms applied in the two simulation approaches. The error from the simplified tracking in ICOSIM is further studied in \chapref{chap:crotr}.

Compared to ICOSIM, the two loss clusters in the IR7 DS (called DS1 and DS2) simulated with the simplified STIER approach are at larger amplitudes. The measured A2 loss peak in the arc downstream of IR7 is simulated in ICOSIM and the simplified STIER approach. Neither of the loss peaks A1, A3 and A4 is simulated in ICOSIM or the simplified STIER simulation. From this finding one can conclude, that STIER can indeed be used to produce loss maps comparable to ICOSIM. Both approaches show discrepancies to the measurement. Apparently the sole inclusion of light ion fragments in the simulation framework does not improve the agreement. With the full STIER simulation, it can now be studied if the agreement improves if the angles and energies of the ion fragments are taken into account.

%In conclusion, the sole inclusion of light fragments changes the composition of losses in the IR7 DS, but cannot explain the discrepancies of ICOSIM with respect to the loss clusters A1 to A4. 



\subsubsection{Full STIER}

The loss maps simulated with the full STIER approach are shown in the third row of Fig.~\ref{fig:comparison_lossmapsIR7} and \figref{fig:comparison_lossmapsLHC}. Compared to ICOSIM and the simplified STIER approach, it shows a significantly better agreement to the measured data. It shall again be emphasized that the simulated loss patterns should not be compared quantitatively to the measurements. The characteristic loss locations are discussed and analyzed below.

\mbox{} \\ 
\textit{Warm Region W1}
\\ 
The losses in the warm region W1 are more broadly distributed than in the simplified STIER approach. The isotopes lost in the W1 region are listed, sorted by their energetic contribution to the loss cluster, in Tab.~\ref{tab:contrib}. The analysis shows that the simulated losses are mainly composed of protons and other light ions. 

It shall again be emphasized that losses in warm regions simulated with STIER are not comparable to measured loss patterns in these regions. In reality, the BLM signals are caused by hadronic and electromagnetic showers starting from the collimators, which are not simulated in STIER. To define the risk of beam induced quenches, these losses are irrelevant, because all magnets in this region are normal conducting. If these losses should be understood by simulations, detailed shower propagation simulations can be carried out with FLUKA. 

\begin{table*}[t]
\caption{STIER simulated contributions on the total deposited energy at the warm magnets in IR7 (W1), the two loss clusters in the IR7 DS (DS1 and DS2) and in the arcs downstream of IR7 (A1, A2, A3, A4) as shown in Fig.~\ref{fig:comparison_lossmapsIR7} }
\small
\setlength\tabcolsep{2.5pt}
\centering
\label{tab:contrib}
\begin{tabular}{cc|cc|cc|cc|cc|cc|cc}
\toprule
\multicolumn{2}{c}{W1}                 & \multicolumn{2}{c}{DS1}                 & \multicolumn{2}{c}{DS2}         & \multicolumn{2}{c}{A1}                 & \multicolumn{2}{c}{A2}                 & \multicolumn{2}{c}{A3}                 & \multicolumn{2}{c}{A4}                    \\ \midrule
Ion            &  (\%) & Ion            &  (\%) & Ion  & (\%)   &   Ion            &  (\%) & Ion            &  (\%) & Ion            &  (\%) & Ion            &  (\%) \\ \midrule
$^{1}$H$^{1+}$ & 57.0              & $^{3}$H$^{1+}$ & 8.6              & $^{206}$Pb$^{82+}$ & 34.0            & $^{204}$Tl$^{81+}$ & 61.0              & $^{204}$Tl$^{81+}$ & 74.6              & $^{204}$Tl$^{81+}$ & 86.6              & $^{204}$Tl$^{81+}$ & 86.7              \\

$^{3}$H$^{1+}$ & 38.0              & $^{4}$He$^{2+}$ & 4.5              & $^{205}$Pb$^{82+}$ & 16.2          & $^{206}$Pb$^{82+}$ & 18.7              & $^{206}$Pb$^{82+}$ & 10.3              & $^{199}$Au$^{79+}$ & 6.7               & $^{199}$Au$^{79+}$ & 7.2               \\

$^{2}$H$^{1+}$ & 2.6               & $^{2}$H$^{1+}$ & 3.2               & $^{204}$Pb$^{82+}$ & 11.6            & $^{199}$Au$^{79+}$ & 7.4               & $^{199}$Au$^{79+}$ & 5.7               & $^{206}$Pb$^{82+}$ & 2.2               & $^{206}$Pb$^{82+}$ & 1.7               \\

$^{3}$He$^{2+}$     & 1.4               & $^{203}$Pb$^{82+}$ & 3.2               & $^{203}$Tl$^{81+}$ & 8.7         & $^{1}$H$^{3+}$     & 3.5               & $^{201}$Hg$^{80+}$ & 2.3               & $^{194}$Ir$^{77+}$ & 1.2               & $^{202}$Hg$^{80+}$ & 1.6              
\\ \bottomrule             
\end{tabular}
\end{table*}

\newpage
%\mbox{} \\ 
\textit{Dispersion Suppressor DS1 and DS2}

\vspace{0.2cm}

The two loss clusters DS1 and DS2 in the IR7 DS are accurately modeled in STIER. The isotopes with the highest energy contribution to the losses in the DS1 and DS2 are listed in Tab.~\ref{tab:contrib}. The highest fraction of the DS1 losses is caused by very light isotopes (H and He ions). The four most important isotopes compose only $\approx$19.5\,\% of the energy deposited in the DS1. The DS2 loss cluster is dominated by three different isotopes of Pb which contribute to 61.8\,\% of the total energy lost in this cluster.  With 1227 different isotope species lost in the DS1, the deposited energy is shared between a much larger number of isotopes than at the DS2, where only 334 different isotopes are absorbed. 

\vspace{0.2cm}
The distributions of the quantity $\chi$ for the isotopes lost in the regions DS1 and DS2 are graphically represented in Fig.~\ref{fig:15032102}. As expected, the aperture in the cold region DS1 captures a very broad range of mass to charge ratios. The heavy ions lost in the DS2 have mostly mass to charge ratios closer to the reference heavy-ion species. 

To understand this difference, the locally generated dispersion function from the TCP to the DS1 and DS2 is shown in the bottom plot of Fig.~\ref{fig:comparison_lossmapsIR7}. The dispersion increases from $\tilde{D}_x\approx0\,$m in the warm IR7 magnets to $\tilde{D}_x \approx 1\,$m at the end of the DS1 and reaches $\tilde{D}_x=2.4\,$m a the end of the DS2. Particles with large rigidity offsets are outside of the rigidity acceptance of the DS1 magnets and hence lost. For particles with rigidities closer to the main beam, the dispersion in the DS1 is not large enough to guide them into the magnet aperture. They continue to move further downstream and can be lost in the DS2, where the local dispersion function is higher and hence the rigidity acceptance is smaller. 




% \begin{figure}[t]
% \centering
%  \includegraphics[width=0.6\textwidth]{pictures/15092501.pdf}
% % \includegraphics[width=0.5\textwidth]{pictures/15062202.pdf}
%  % \includegraphics[width=0.5\textwidth]{pictures/15032103.pdf}


% \end{figure}




\begin{figure}[t]
  \centering
  \begin{tikzpicture}
    \footnotesize
    \node[anchor=south west,inner sep=0] (image) at (0,0) {\includegraphics[width=0.6\linewidth]{pictures/15092501.pdf}};
    \node [fill=white,x={(image.south east)},y={(image.north west)}]                   at (0.90,0.92)    {DS1};
    \node [fill=white,x={(image.south east)},y={(image.north west)}]                   at (0.90,0.85)    {DS2};
  \end{tikzpicture}
\caption{Momentum weighted abundance of $\chi$ for fragments lost in the DS1 and DS2 clusters. The abundances are normalized by the total momentum of all particles lost in the DS1 or DS2. The data is extracted from the full STIER simulation with $b=3\,\mu$m.}
\label{fig:15032102}
  \end{figure}

\newpage

\textit{Arc Losses A1 to A4}
\\
The comparison between the simplified and the full STIER simulation shows that the three measured loss spikes A1, A3 and A4 become visible when the angles and energies of the fragments are included. The isotope composition at these locations is very similar (see Tab.~\ref{tab:contrib}). This can be explained by the fact that the local dispersion function at all of these loss locations is similar. The individual starting conditions at the TCP (starting angle and collimator jaw) can partly add or subtract to dispersive effects. Therefore, the loss location of particles of the same species is different, if the starting angle and energy is taken into account.

This statement is supported by the fact that, except $^1$H$^{3+}$, all isotopes lost in A1 to A4 are also included in the ICOSIM simulation but the loss spikes are not simulated.  The study underlines that it is crucial to incorporate angular and energetic shifts by fragmentation, to accurately simulate heavy-ion beam loss patterns. This should be considered for future simulations.

\mbox{} \\ 
\textit{Losses Downstream of A4}
\\
Further downstream, the A5/A6 loss peaks between IR8 and IR7 and the A7 peak between IR1 and IR2 are both simulated and measured. At some locations, STIER predicts losses which are not measured. This applies to the loss peaks S78 in the arc between IR7 and IR8 and S12 in the arc between IR1 and IR2. On the contrary, the A8/A9 loss peaks between IR2 and IR3  are measured, but not simulated by STIER. Instead, two distinct loss peaks at other locations between IR2 and IR3 are simulated. The losses at the IR8 TCT are overestimated in the simulation, while STIER does not predict the losses measured at the IR1 TCT.

These remaining discrepancies could arise from small aperture displacements or orbit fluctuations in the real machine. An analysis of the particle trajectories at the corresponding loss locations shows that aperture or orbit displacements as small as $\approx 300\,\mu$m are sufficient to shift the loss location of the impacting ions. 

As it is discussed in more detail in \chapref{aper:misalg}, the measured magnet aperture varies by some hundred \mum. Also, the closed orbit which is commissioned at the beginning of a LHC run can have offsets in this order of magnitude~\cite{Redaelli:private}. In addition, orbit drifts in the order of 50\mum\, can be observed during one fill or between different fills~\cite{wenn:private} within one run. The combination of these effects can possibly result in the observed discrepancies between the STIER simulation and the measurement. 

Another reason for losses which are measured but not simulated could be fragments which are generated in secondary collimator interactions after the passage of the TCP. This can be studied with a simulation tool that includes subsequent fragmentation. Such a tool is presented in \chapref{chap:hisix}.


\subsection{Dependence on the starting Collimator Jaw}

\begin{figure}[t]
  \begin{center}
\includegraphics[width=0.7\textwidth]{pictures/15092502.pdf}
\caption{STIER simulations in the 2011 configuration with \lead beams at 3.5\,$Z$\,TeV starting at the left and right collimator jaw, shown in comparison to the measured loss map. The simulations are carried out considering an impact parameter of $3\,\mu$m. }
\label{fig:15032201.pdf}
  \end{center}
\end{figure}

STIER can be used to study the loss behavior of isotopes starting at the individual collimator jaws. So far, all STIER results assume the same amount of ions impacting on the two TCP jaws. In reality the beam halo can impact the two collimator jaws asymmetrically, because the collimator jaw alignment and the reference orbit can only be accurate to some \mum. Furthermore, non-linear magnetic fields can tilt or shift the phase space ellipse (see \chapref{chap:pha_shift}).

In Fig.~\ref{fig:15032201.pdf}, the loss map is shown, as simulated with STIER with the same settings as above, for particles starting at the left and right collimator jaw separately. A different behavior of the losses can be expected because the betatron motion and the dispersive offset can add or subtract, depending on the starting conditions of the ion.
\\
From the obtained loss maps it can be seen that the simulation result for the particles starting at the right collimator jaw is in much better agreement with the measured data than the simulation of particles starting at the left jaw. The largest fraction of the loss peaks between IR8 and IR1 as well as between IR1 and IR2, which are not observed in the measurement but visible in STIER, is composed of particles starting at the left jaw. However, still one intense peak between IR2 and IR3 is visible in the simulation for the right jaw, which is not observed in the measurement. 

In conclusion, the loss patters simulated for particles starting from the individual collimator jaws are different. The separate study of the loss distribution for fragments starting from the individual TCP jaws shall turn out to be very useful to mitigate losses in the 2015 heavy-ion run, as described in \chapref{chap:ir2loss}.


\subsection{Simulations with different Impact Parameters} \label{subsec:impactparam}

\begin{figure}[b]
  \centering
  \begin{tikzpicture}
    \footnotesize
    \node[anchor=south west,inner sep=0] (image) at (0,0) {\includegraphics[width=1.0\linewidth]{pictures/16090107.pdf}};
  \node [x={(image.south east)},y={(image.north west)}]                   at (0.15,0.950)    {$b=1\mu$m};
  \node [x={(image.south east)},y={(image.north west)}]                   at (0.15,0.638)    {$b=3\mu$m};
  \node [x={(image.south east)},y={(image.north west)}]                   at (0.15,0.330)    {$b=10\mu$m};
  % \node [draw,rotate=0 ,x={(image.south east)},y={(image.north west)}]                   at (0.22,0.96)    {text1};
  % \node [draw,rotate=0 ,x={(image.south east)},y={(image.north west)},anchor=west]       at (0.22,0.80)    {text2};
  % \draw[->,color=black,thick,x={(image.south east)},y={(image.north west)}]             (0.42,0.22) -- (0.37,0.23);

%  \draw[help lines,step=.05,x={(image.south east)},y={(image.north west)}] (0,0) grid (1,1);
%  \draw[help lines,line width=.6pt,step=0.1,x={(image.south east)},y={(image.north west)}] (0,0) grid (1,1);
%  \foreach \x in {0,0.1,0.2,0.3,0.4,0.5,0.6,0.7,0.8,0.9,1.0}
%       \node[anchor=north,x={(image.south east)},y={(image.north west)}] at (\x,-0.01) {\x};
%  \foreach \y in {0.0,0.1,0.2,0.3,0.4,0.5,0.6,0.7,0.8,0.9,1.0}
%      \node[anchor=east,x={(image.south east)},y={(image.north west)}] at (-0.01,\y) {\y};

  \end{tikzpicture}
  \caption{B1H loss maps for the 2011 configuration with \lead beams at 3.5\,$Z$\,TeV simulated with the full STIER approach for the impact parameters \mbox{$b=1\,\mu$m}, \mbox{$b=3\,\mu$m}, \mbox{$b=10\,\mu$m}.}  
  \label{fig:15062502.pdf}
  %/media/phermes/local/141003_compareICOSIMSixTrackMeasurement/LHCcomparisonThesis_b.pdf
  \end{figure}



\begin{figure}[htbp]
  \centering
  \begin{tikzpicture}
    \footnotesize
    \node[anchor=south west,inner sep=0] (image) at (0,0) {\includegraphics[width=1.0\linewidth]{pictures/16090108.pdf}};
  \node [x={(image.south east)},y={(image.north west)}]                   at (0.75,0.890)    {$b=1\mu$m};
  \node [x={(image.south east)},y={(image.north west)}]                   at (0.75,0.60)    {$b=3\mu$m};
  \node [x={(image.south east)},y={(image.north west)}]                   at (0.75,0.310)    {$b=10\mu$m};
  % \node [draw,rotate=90,x={(image.south east)},y={(image.north west)}]                   at (0.50,0.50)    {text0};
  % \node [draw,rotate=0 ,x={(image.south east)},y={(image.north west)}]                   at (0.22,0.96)    {text1};
  % \node [draw,rotate=0 ,x={(image.south east)},y={(image.north west)},anchor=west]       at (0.22,0.80)    {text2};
  % \draw[->,color=black,thick,x={(image.south east)},y={(image.north west)}]             (0.42,0.22) -- (0.37,0.23);

%  \draw[help lines,step=.05,x={(image.south east)},y={(image.north west)}] (0,0) grid (1,1);
%  \draw[help lines,line width=.6pt,step=0.1,x={(image.south east)},y={(image.north west)}] (0,0) grid (1,1);
%  \foreach \x in {0,0.1,0.2,0.3,0.4,0.5,0.6,0.7,0.8,0.9,1.0}
%       \node[anchor=north,x={(image.south east)},y={(image.north west)}] at (\x,-0.01) {\x};
%  \foreach \y in {0.0,0.1,0.2,0.3,0.4,0.5,0.6,0.7,0.8,0.9,1.0}
%      \node[anchor=east,x={(image.south east)},y={(image.north west)}] at (-0.01,\y) {\y};

  \end{tikzpicture}
  \caption{B1H loss maps zoomed to IR7 for the 2011 configuration with \lead beams at 3.5\,$Z$\,TeV simulated with the full STIER approach for the impact parameters \mbox{$b=1\,\mu$m}, \mbox{$b=3\,\mu$m}, \mbox{$b=10\,\mu$m}.}  
  \label{pic:16090105}
  %/media/phermes/local/141003_compareICOSIMSixTrackMeasurement/IR7comparisonThesis_b.pdf
  \end{figure}




% \begin{figure}[t]
%   \begin{center}
% \includegraphics[width=0.6\textwidth]{pictures/15092504.pdf}
% \caption{Heavy-ion loss maps for the 2011 configuration with \lead beams at 3.5\,$Z$\,TeV as simulated with the full STIER approach for three different impact parameters \mbox{$b=1\,\mu$m}, \mbox{$b=3\,\mu$m}, \mbox{$b=10\,\mu$m}.}
% \label{fig:15062502.pdf}
%   \end{center}
% \end{figure}

In the real machine, the impact parameter $b$ of the ions hitting the collimators may vary. For the STIER simulations presented so far, an impact parameter of 3$\,\mu$m was assumed, based on previous proton studies.
As shown in \chapref{chap:STIERfrag}, the production yield for very light fragments increases with increasing impact parameter. The difference in the fragment spectrum will lead to a change in the simulated loss maps, which is studied in this section.

\newpage
The loss maps simulated with the full STIER approach for the three different impact parameters $1\,\mu$m, $3\,\mu$m and $10\,\mu$m are compared in Fig.~\ref{fig:15062502.pdf} and \figref{pic:16090105}. The loss patterns are qualitatively similar but the loss peak amplitudes differ quantitatively. 

For the smallest studied impact parameter of 1$\,\mu$m, the highest losses occur at the primary collimator in IR7. The main contribution of these losses arises from \lead ions that were not fragmented but scattered at small angles in the TCP. They circulate inside the machine for one or multiple turns until they are intercepted again by the primary collimator. In reality, these ions are again subject to fragmentation and scattering inside of the TCP, which is not considered in STIER. In the latter, they are assumed to be absorbed in the TCP.

The losses at the DS region peak at $\eta=10^{-3}$, which is smaller than in the other simulations, due to the large amount of \lead ions surviving the initial passage through the TCP. The remaining losses in the aperture and the other collimators are located at elements which are also subject to losses for the other impact parameters, but the loss amplitudes are smaller. 

For the study cases with $b=3\,\mu$m and $b=10\,\mu$m, the highest losses are visible at a secondary collimator. The losses in the IR7 DS clusters peak at $\eta=10^{-2}$ for $b=3\,\mu$m. For $b=10\,\mu$m the peak loss in the cold LHC regions is simulated in the DS1, while the peak loss in the DS2 is smaller by two orders of magnitude. This indicates that the residual ion fragments generated with $b=10$\mum\, have larger rigidity offsets and are mostly lost closer to the TCP than in the other simulated scenarios. The A1 to A4 loss spikes are significantly lower for the simulation with $b=10$\mum\, than with $b=3$\mum. As it was discussed in the previous section, these losses are mainly composed of heavy nuclei with masses close to \lead. These are significantly less abundant when the larger impact parameter is applied.

In conclusion, the choice of impact parameter is very important to accurately simulate the cleaning inefficiency of heavy-ion beams. For the smallest impact parameter, the losses at the TCP and hence the cleaning performance are overestimated, because ion fragmentation at subsequent turns is not included. This should be better modeled with an improved tool taking into account multiple fragmentation, presented in \chapref{chap:hisix}.






% \subsection{Loss Classification of Residual Heavy-Ion Fragments}

% \begin{itemize}
% %
%    \item \textbf{Elastic scattering}: The particle rigidity is very close to the rigidity of the main beam. In the case of heavy ions, the particle species has not changed. The particle is scattered to larger transverse betatron amplitudes. If the transverse angular kick is sufficient, the particle is intercepted by the secondary collimators. Otherwise, they are intercepted by the TCP or a TCSG on a subsequent turn.
% %
%    \item \textbf{Inelastic interactions with small rigidity change}: Particles leaving the TCP with small momentum offsets and scattered at small angles continue moving through the magnetic lattice. Many of them are intercepted by the TCP in the momentum collimation region IR3, but - depending on their initial conditions - they can also be lost at other locations, such as TCT collimators. Typical rigidity offsets for such particles are in the order of $10^{-4}$. An important example for such particles in \lead operation are $^{207}$Pb$^{82+}$ ions, which can continue to move in the machine for several km. 
% %
%    \item \textbf{Inelastic interactions with large rigidity change}: Residual particles produced in nuclear interactions (single diffractive dissociation for protons and nuclear fragmentation or multiple EMD for heavy ions) can have rigidity offsets in the order of $10^{-3}$ or above. Depending on their initial conditions they may be beyond the momentum acceptance of the DS region in IR7 and therefore lost in these magnets. The losses from these particles compose the characteristic DS clusters in this region. The transverse momentum can partly compensate for the dispersive effect, such that the loss location for particles of the same rigidity may alter. In ICOSIM the latter effect was considered negligible, but in the studies presented in the following chapters it is demonstrated to have a considerable influence on the real loss distribution.
% %
%    \item \textbf{Inelastic interactions with extreme rigidity change}:
% Heavy-ion fragments with mass to charge ratios very different from the main beam (significantly above $10^{-1}$) are subject to large dispersive offsets already after the passage of the normal conducting magnets in the straight section of IR7. Typically these ions are very light (protons, deuterium, tritium, $^{4}$He$^{2+}$,...) and have large transverse momenta compared to the main beam. In consequence, these particles are lost in IR7, partly in the secondary collimators, partly in the aperture of the normal conducting elements and the surrounding beam pipes. 
% %
% \end{itemize}


\subsection{Loss Distribution of individual Isotopes}



\begin{figure}[htbp]
  \centering
  \begin{tikzpicture}
    \footnotesize
    \node[anchor=south west,inner sep=0] (image) at (0,0) {\includegraphics[width=1.0\linewidth]{pictures/16090501.pdf}};

  % \draw[->,color=black,thick,x={(image.south east)},y={(image.north west)}]             (0.42,0.22) -- (0.37,0.23);
    \filldraw[fill=white,draw=white,x={(image.south east)},y={(image.north west)}] (0,0.4) rectangle (0.04,0.6);
    \node [rotate=90 ,x={(image.south east)},y={(image.north west)},anchor=west]       at (0.02,0.475)    {Abundance};
    \node [rotate=90 ,x={(image.south east)},y={(image.north west)},anchor=west]       at (0.02,0.075)    {Abundance};
    \node [rotate=90 ,x={(image.south east)},y={(image.north west)},anchor=west]       at (0.02,0.275)    {Abundance};
    \node [rotate=90 ,x={(image.south east)},y={(image.north west)},anchor=west]       at (0.02,0.675)    {Abundance};
    \node [rotate=90 ,x={(image.south east)},y={(image.north west)},anchor=west]       at (0.02,0.85)    {Abundance};

   % \node [rotate=90 ,x={(image.south east)},y={(image.north west)},anchor=west]       at (0.02,0.475)    {Abundance};


    % \draw[help lines,step=.05,x={(image.south east)},y={(image.north west)}] (0,0) grid (1,1);
    % \draw[help lines,line width=.6pt,step=0.1,x={(image.south east)},y={(image.north west)}] (0,0) grid (1,1);
    % \foreach \x in {0,0.1,0.2,0.3,0.4,0.5,0.6,0.7,0.8,0.9,1.0}
    %    \node[anchor=north,x={(image.south east)},y={(image.north west)}] at (\x,-0.01) {\x};
    % \foreach \y in {0.0,0.1,0.2,0.3,0.4,0.5,0.6,0.7,0.8,0.9,1.0}
    %    \node[anchor=east,x={(image.south east)},y={(image.north west)}] at (-0.01,\y) {\y};

  \end{tikzpicture}
  \caption{STIER simulated loss locations of the isotopes $^{207}$Pb$^{82+}$, $^{204}$Tl$^{81+}$, $^{206}$Pb$^{82+}$, $^{4}$He$^{2+}$, $^{1}$H$^{1+}$ (from top to bottom) in the 2011 heavy-ion configuration at 3.5\,$Z$\,TeV. The vertical axis describes the number of lost particles normalized by the total number of particles of this species.}  
  \label{pic:16021501}
   %/home/phermes/Dropbox/PhD/pictures/160215_STIER_isotopes/isotopes.pdf 
  \end{figure}


% \begin{figure}[htbp]
%   \centering
%   \begin{tikzpicture}
%     \node[anchor=south west,inner sep=0] (image) at (0,0) {\includegraphics[width=1.0\linewidth]{pictures/16090501.pdf}};
%   % \node [draw,rotate=90,x={(image.south east)},y={(image.north west)}]                   at (0.50,0.50)    {text0};
%   % \node [draw,rotate=0 ,x={(image.south east)},y={(image.north west)}]                   at (0.22,0.96)    {text1};
%   % \node [draw,rotate=0 ,x={(image.south east)},y={(image.north west)},anchor=west]       at (0.22,0.80)    {text2};
%   % \draw[->,color=black,thick,x={(image.south east)},y={(image.north west)}]             (0.42,0.22) -- (0.37,0.23);

% %  \draw[help lines,step=.05,x={(image.south east)},y={(image.north west)}] (0,0) grid (1,1);
% %  \draw[help lines,line width=.6pt,step=0.1,x={(image.south east)},y={(image.north west)}] (0,0) grid (1,1);
% %  \foreach \x in {0,0.1,0.2,0.3,0.4,0.5,0.6,0.7,0.8,0.9,1.0}
% %       \node[anchor=north,x={(image.south east)},y={(image.north west)}] at (\x,-0.01) {\x};
% %  \foreach \y in {0.0,0.1,0.2,0.3,0.4,0.5,0.6,0.7,0.8,0.9,1.0}
% %      \node[anchor=east,x={(image.south east)},y={(image.north west)}] at (-0.01,\y) {\y};

%   \end{tikzpicture}
%   \caption{Figure caption}  
%   \label{pic:16090501}
%   %/home/phermes/Dropbox/PhD/pictures/160215_STIER_isotopes/isotopes.pdf
%   \end{figure}






STIER allows studying of the distribution of losses for the individual isotopes. With the previous studies it became apparent that the measured loss pattern may be affected by light ion fragments not only in the warm regions just downstream of the collimators, but also in the superconducting magnets. In \figref{pic:16021501}, the loss maps for five different isotopes are compared. Three heavy nuclei ($^{207}$Pb$^{82+}$, \iso{204}{Tl}{81+} and \iso{206}{Pb}{82+}) are studied, because they are relevant for important loss spikes simulated. In addition, two very light fragments (protons and \iso{4}{He}{2+}) are simulated to study their loss distribution. The loss peaks are normalized by the total number of ions of the respective species lost in the ring. 

The top plot shows the loss pattern for the isotope $^{207}$Pb$^{82+}$ with $\chi = 1.0048$. This isotope is generated via EMD with a relatively high cross section and has a small offset in rigidity compared to the main beam. It can hence travel in the machine over long distances. As expected, the losses are globally distributed, showing that the isotope is within the momentum acceptance of the arcs. The losses are mainly localized at collimators. Some losses are visible at the secondary collimators in IR7. The dominating part is simulated to be lost at the TCP in IR3, which is set to a momentum cut of $2\times 10^{-3}$. Furthermore, losses are simulated at the TCTs in IR2 and IR8. It shall be shown later that for a different particle momentum and different optics, a significantly larger fraction of this isotope is lost at the TCT in IR2 (see \chapref{chap:ir2loss}). 

The isotope \iso{204}{Tl}{81+} with $\chi = 1.0072$ is the dominant species lost at the A1 loss peak. The global distribution for this isotope also shows a defined peak at the horizontal TCT in IR8 and significant losses in superconducting magnets from IR7 to IR2. 

The fragments of the isotope \iso{206}{Pb}{82+} with $\chi=1.0097$ are absorbed in the LHC aperture before they can reach the TCP in IR3. This isotope is generated via second order EMD or two first order EMD processes and dominates the losses in the DS2. Most of these ion fragments are lost in the latter, followed by the IR7 TCSGs and the arc region between IR7 and IR8.  

In the domain of very large rigidity offsets, the losses are mostly localized in IR7 as shown on the example of \iso{4}{He}{2+} and \iso{1}{H}{1+}. Both of them are produced with high abundances when \lead interacts with the primary collimator. All protons are lost in the IR7 collimation system, so they are not contributing to the losses in cold regions. Small fractions of the isotope \iso{4}{He}{2+} are also lost in the IR7 DS. 





\section{Chromatic Tracking in ICOSIM and SixTrack} \label{chap:crotr}

\begin{figure*}[b]
\begin{minipage}[t]{0.49\textwidth}
\includegraphics[width=\textwidth]{pictures/15092506.pdf}
\end{minipage}
\begin{minipage}[t]{0.49\textwidth}
\includegraphics[width=\textwidth]{pictures/15092507.pdf}
\end{minipage}
\caption{Comparison of the simulated tracks with ICOSIM and STIER for the two isotopes $^{8}$Li$^{3+}$ (left) and $^{207}$Pb$^{82+}$ (right) with identical starting conditions at the right jaw of the IR7 horizontal TCP. Note the different scales applied in the two plots.}
\label{fig:15032202.pdf}
\end{figure*}

ICOSIM tracks off-rigidity heavy ions by means of matrix multiplication with chromatic effects modeled in linear order. This approximation is valid for small rigidity offsets and becomes less accurate when  $\delta_\text{eff}$ increases. SixTrack uses tracking maps derived from the accelerator Hamiltonian and does not truncate the contributions of dispersion (see \chapref{chap:hisix}).

To study the accuracy and importance of the chromatic modeling, tracking simulations of off-rigid isotopes are carried out with identical optics and initial conditions in STIER and ICOSIM. The isotopes \iso{8}{Li}{3+} (with $\delta_\text{eff} = 0.054$) and \iso{207}{Pb}{82+} (with $\delta_\text{eff} = -0.005$) were selected to study the tracking accuracy with different rigidity offsets. Assuming that the chromatic modeling in STIER is more accurate to that of ICOSIM, the comparison allows to quantify the tracking error implied by the linear truncation of the equation of motion in the ICOSIM implementation. 

Both tracking simulations start at the edge of the right jaw of the horizontal TCP in IR7 with a starting angle identical to the angle of incidence at the collimator derived from the phase space parameters. The simulated tracks are compared in \figref{fig:15032202.pdf}. For the isotope \iso{207}{Pb}{82+}, the tracks simulated with STIER and ICOSIM are similar. At a distance of 2~km from the TCP, the horizontal positions simulated with the two codes differ by 200~$\mu$m. 

For the isotope \iso{8}{Li}{3+}, the horizontal position differs by 3~mm after a longitudinal distance of 450~m. This demonstrates the drastic impact of the linear approximation on the simulated track for particles with large rigidity offsets. In spite of the fact that these ions are lost closely to the TCP, the error on the simulated track may cause a large shift in their simulated loss position. 

As demonstrated previously, particles with large rigidity offsets are created in the collimators. Therefore, a heavy-ion collimation simulation tool should rather use a tracking algorithm that is similar to SixTrack.




\section{Summary and Conclusions}


The STIER simulation tool was developed to assess the importance of the different approximations used in ICOSIM. The latter is implemented with a simplified fragmentation simulation where changes in angle and energy from the fragmentation process are not included. It was studied whether the accuracy can be increased by a more detailed fragmentation simulation only at the TCP. STIER cleaning simulations are  based on combining FLUKA for an accurate fragmentation simulation and SixTrack where the heavy-ion fragments are tracked as protons with ion-equivalent rigidities. 

The application of STIER to the 2011 heavy-ion run demonstrated that the tool can reproduce the measured loss patterns with an unprecedented accuracy. The studies have demonstrated that it is crucial to include the angle and energy of the particles that are scattered out of the TCP into the tracking simulation. 

Given the good agreement with the measured data, it appears to be a valid approximation of simulating the fragmentation only in the TCP. Therefore, shortly after its introduction in 2014, STIER became the standard simulation tool for LHC heavy-ion collimation simulations at the LHC. It was used directly in machine studies during the 2015 heavy-ion operation at 6.37$\,Z\,$TeV (see \chapref{chap:sim_meas}). 

Nevertheless, the comparison of the loss maps with different impact parameters shows that the cleaning simulations can be further improved if also collimator interactions after the first TCP passage are included in the simulation setup. This functionality is provided by the successor of STIER, which is presented in the next chapter.