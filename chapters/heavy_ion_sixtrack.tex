\chapter{Heavy-Ion SixTrack} \label{chap:hisix}

\section*{Introduction}
The previous study with the STIER simulation tool shows that SixTrack can serve the purpose of accurately tracking of multi-isotopic heavy-ion beams. However, in the STIER approach the different heavy-ions are tracked as protons. In order to keep track of the particle species, extensive pre- and post-processing is required. 

Furthermore, the active coupling with a software for particle-matter interaction such as FLUKA requires to track the physical heavy-ions in SixTrack to allow for the exchange of information on the particle species. In this chapter, the implementation of the new heavy-ion SixTrack is described, including the derivation of symplectic tracking maps for multi-isotopic particle beams, the modification of the SixTrack tracking routine to account for the isotopic dispersion and the linking to the SixTrack-FLUKA active coupling to allow for fragmentation simulations at all collimators. The tracking and the fragmentation routine are individually benchmarked against STIER and the main version of FLUKA.

\section{Requirements and Implementation Strategy}

The following list summarizes the implementation tasks to be carried out in order to make heavy-ion SixTrack (hiSixTrack) operationable for heavy-ion collimation simulations.
%
\begin{itemize}
        \item hiSixTrack must provide additional arrays storing information about the particle mass $m$, charge $Z$, and nuclear mass number $A$. In the present implementation it is assumed that the particle charge corresponds to the nuclear charge number (e.g. nuclei from which all electrons are removed). For the heavy-ion beams circulating in the LHC this is a valid assumption. If required, an adequate extension to define the particle charge differently than the nuclear charge number is fairly easy to implement. 
	\item The reference species must be defined in a designated input option, preferably given in the fort.3 file. Furthermore, the particle species of the initial bunch to be tracked must be read from an initial distribution file.
	\item The tracking maps must be modified to take into account the change of magnetic rigidity for different ions while maintaining their symplecticity. Instead of using effective proton momenta as it was done for STIER, the tracking maps should be implemented such that the momentum of the tracked ion conforms to the physical momentum. 
	\item The on-line coupling between SixTrack and FLUKA must be adapted to allow for the exchange of ions different than protons. The FLUKA input must be changed to take into account fragmentation processes by EMD, NF, and nulcear evaporation. 
	\item The dump of particle losses must be changed to include information of the ions lost at the collimators and at the aperture. This will allow for studies of loss locations of individual isotopes and analysis of the isotopic composition of losses. 

\end{itemize}

Even though collimation simulations are typically carried out for some hundrets of turns, other simulations of the particle behaviour in a high energy ring can require $10^6$ turns or even more. The demand on hiSixTrack is thus above the typical requirements on a collimation simulation software for which it would suffice to have enough numeric precision for a tracking over 500 turns and without the need to include RF cavities. The software is therefore going enable the possiblity for many different studies with heavy-ion beams.

\section{Magnetic Tracking of Multiple Isotopes}
%
The tracking maps for multi-isotopic particle beams can be derived from an appropriate Hamiltonian which incorporates information about the particle mass and charge with respect to the reference species. In this section, a consistent mathematical framework for the derivation of trackiing maps is introduced based on a  generalized Hamiltonian applicable for particles different from the reference species. The derivation of this Hamiltonian follows the same approach as for the mono-isotopic Hamiltonian, under consideration of the mathematical description of the magnetic rigidities treated in \chapref{}. Once derived, the multi-isotopic Hamiltonian is applied to vector potentials specific to the LHC beam line elements, to derive the corresponding tracking maps for multi-isotopic particle beams.
%
\subsection{Hamiltonian Formalism of Particle Motion} \label{chap:hamiltonian}


A particle moving in the accelerator lattice has three degrees of freedom $i=1,2,3$. The dynamical behaviour is thus described by a set of six coordinates, three of which being the generalized position coordinates $\vect{q} = \{q_i\} = (x,y,z)$. In the studied scenario, the latter are identical to the coordinates of the reference frame previously defined in Chap.~\ref{chap:refframe}. The canoncial conjugate is the generalized momentum $\vect{p} = (p_x,p_y,p_z)$. For convenience, the generalized velocities are defined as $\dot{\vect{q}} = (\dot{x},\dot{y},\dot{z})$, where $\dot{q_i} = \frac{d q_i}{dt}$ and the time $t$ is the independent variable.

The temporal evolution of the particle coordinates during their motion in electromagnetic fields obeys Hamilton's equation of motion, so it can be described by the Hamiltonian formalism. In the Hamiltonian formulation of mechanics the motion of a particle with $N$ degrees of freedom is described by the time evolution of a set of the $2\,N$ variables:
\begin{align}
\mathbf{x} = (q_1,p_1,q_2,p_2,...,p_N,q_N)^T \, . \label{eq:phasevector}
\end{align}
The longitudinal coordinate $s(t)$ is monotonically and smoothly raising in time, so that Hamilton's equations can also be expressed using $s$ as the independent parameter~\cite{feynmanlectures,rees_symplecticity}:
%
\begin{align}
\frac{\mathrm{d} q_k}{\mathrm{d}s} = \frac{\partial H}{\partial p_k} \quad \quad \quad \frac{\mathrm{d} p_k}{\mathrm{d}s} = -\frac{\partial H}{\partial q_k} \quad \quad \quad k=1,2,...,N\, , \label{eq:hamiltons}
\end{align}
%
where $q_k,p_k$ is the set of canonic conjugate variables corresponding to the degree of freedom $k$ and $H=H(p_k,q_k,s)$ is the Hamiltonian with the independent variable $s$. Using the vector notation introduced in \eqref{eq:phasevector}, Hamilton's equations can be expressed in a simple manner:
%
\begin{align}
\frac{\mathrm{d} \mathbf{x}}{\mathrm{d}s} = \mathbf{S} \, \frac{\partial H}{\partial \mathbf{x}} \, , \quad \quad \text{with} \quad \quad \left( \frac{\partial H}{\partial \mathbf{x}} \right)_i = \frac{\partial H}{\partial x_i} \, . \label{eq:symplecticform}
\end{align}
%
where $\mathbf{S}$ is a rearranging matrix, the so-called \emph{symplectic matrix}:
\begin{align}
\mathbf{S}
=
\begin{pmatrix}
\mathbf{s} & \mathbf{0}  & \cdots  & \mathbf{0} \\ 
\mathbf{0} & \mathbf{s} &  & \mathbf{0} \\ 
\vdots &  & \ddots  & \vdots \\ 
\mathbf{0} & \hdots & \mathbf{0} & \mathbf{s}
\end{pmatrix} \, ,
 \quad \quad \text{with} \quad \quad \mathbf{s} = 
 \begin{pmatrix}
0 & 1\\ 
-1 &  0
\end{pmatrix} \, \quad \text{and}
 \quad \quad \mathbf{0} = 
 \begin{pmatrix}
0 &  0\\ 
0 &  0
\end{pmatrix} \, .
\end{align}
The particular shape of this matrix is determined by the specific ordering used for $\mathbf{x}$ and the representation of Hamilton's equations in \eqref{eq:symplecticform} ar referred to as their \textit{symplectic form}. Frequently, the set of canonical variables is subject to transformations $\mathcal{T}$:
\begin{align}
\mathcal{T}:  \quad \mathbf{x} = (q_1,p_1,q_2,p_2,...,p_N,q_N)^T \quad \rightarrow \quad \mathbf{X} = (Q_1,P_1,Q_2,P_2,...,P_N,Q_N)^T \, .
\end{align}



\input{pictures/16070314.tex}
  % #PHTHESIS FILE ORIGIN
  %/home/phermes/Dropbox/PhD/pictures/16060301_symplecticity/puretikz/drawing.tex
% \begin{figure}[t]  
%     \centering
%     \includegraphics[width=0.7\textwidth]{pictures/16060301.pdf}
%     \caption{Illustration of the transformation from $(x,x')$ to $(X,X')$. Symplecticity implies that the phase space volume $a$ in the original system equals the volume $A$ in the transformed system.}  
%     \label{pic:16060301}
%     %/home/phermes/Dropbox/PhD/pictures/16060301_symplecticity/annotated/drawing-1_annotated.pdf
% \end{figure}


The transformation is called canonical or symplectic if the new set of variables $\mathbf{X}$ is also obeing Hamiltonian's equations with respect to a new Hamiltonian $\mathcal{K}(Q_k,P_k,s)$:
\begin{align}
\frac{\mathrm{d} Q_k}{\mathrm{d}s} = \frac{\partial \mathcal{K}}{\partial P_k} \quad \quad \quad \frac{\mathrm{d} P_k}{\mathrm{d}s} = -\frac{\partial \mathcal{K}}{\partial Q_k} \quad \quad \quad k=1,2,...,N\, ,
\end{align}
This new Hamiltonian can be derived through a generating function as described in detail in ~\cite{}. 

The Jacobian matrix $\mathcal{J}$ of the transformation $\mathcal{T}$ is defined by~\cite{CERN-SL-95-12} 
\begin{align}
\mathcal{J}_{ij} = \left( \frac{\partial \mathbf{X}}{\partial \mathbf{x}} \right)_{i,j} = \frac{\partial X_i}{\partial x_j} \, , \quad \quad \quad i,j=1,2,...N \, .
\end{align}
One can show, that a given transformation is symplectic (or canonicial) if the Jacobian matrix obeys the \emph{symplectic condition}~\cite{CERN-SL-95-12}:
\begin{align}
\mathcal{J}^T \, \mathbf{S} \, \mathcal{J} =  \mathbf{S} \, .
\end{align} 
Thus, the symplectic condition provides a tool for direct testing of the canonicality of a transformation or mapping. Sympecticity corresponds to a conservation of phase space volume throughout the transformation of a particle bunch~\cite{wolski2014beam}, as illustrated in \figref{fig:sympl} and in line with Liouville's theorem. 


\subsection{The Accelerator Hamiltonian for Multi-Isotopic Ion Beams}
\subsubsection{Prologue}

Literature~\cite{DESY-85-084,DESY-87-036,CERN-SL-95-12,DESY-95-189,wolski2014beam} has so far discussed the accelerator Hamiltonian and the resulting equations of motion in the mono-isotopic scenario, which is a valid approximation for pure proton, electron or positron beams, as well as for heavy-ion beams in which only a small fraction of ions different from the reference species are present. In heavy-ion collimation studies with a large fraction of energy carried by isotopes of a particle species unmatched to the magnetic lattice, a more generic approach should be chosen.

The derivation of the generalized multi-isotopic accelerator Hamiltonian follows the same approach as the mono-isotopic derivation presented in \cite{ProceedingsCAS1995}. If not indicated differently, the basic definitions used below are taken from this reference. Fundamental differences are introduced with the normalization of the dynamic variables and the re-definition of $\delta$ to be in line with \eqref{eq:15010701}.

In the following derivation, the particle mass, charge and nuclear mass number of the tracked particle are given as $m,q=Z\,e, A$, while the equivalent quantities for the reference particle are given as $m_0,q_0,A_0$. Eventual ambiguities between the particle charge $q$ and canonical \mbox{coordinate $\vect{q}$} shall be ruled out by using the vector notation for the latter, or well $q_i$ when referring to one particular coordinate.

\subsubsection{Derivation of the Multi-Isotopic Hamiltonian}

The generic Hamiltonian $H$ is given by~\cite{berz:beam_physics}:
%
\begin{align}
  H(\vect{p},\vect{q},t) = p_i \dot{q}_i - \mathcal{L}(\vect{q},\dot{\vect{q}},t) \, , \label{eq:hamilton01}
\end{align}
%
where $\mathcal{L}$ is the Lagrangian of the particle. For an arbitrary particle of the species $^AX^{Z+}$ moving in an electromagnetic field defined by the magnetic vector potential $\vect{A}$ and the electric (scalar) potential $\phi$, the Lagrangian is given by
%
\begin{align}
  \mathcal{L}(\vect{q},\dot{\vect{q}},t) = - \frac{m c^2}{\gamma} - q \, \phi + q \, \dot{\vect{q}} \vect{A} \, . \label{eq:lagrangian01}
\end{align}
%
The canonical momentum is then defined by Hamilton's variational principle:
%
\begin{align}
  p_i = \pd{\mathcal{L}}{\dot{q_k}} = m \dot{q}_i \gamma + q  A_i \, .
\end{align}
%
Merging the \eqsref{eq:hamilton01}{eq:lagrangian01} yields for the Hamiltonian:
%
\begin{align}
  H(\vect{p},\vect{q},t) = \sqrt{(\vect{p}- q \vect{A})^2 + m^2c^4} + q \, \phi \, . \label{eq:h02}
\end{align}
%
The Hamiltonian represents the total energy of the particle. It is advantageous to transform the independent variables from $t$ to $s(t)$, which is valid because $s(t)$ is continuously increasing \mbox{with $t$}. The new dynamic variables can be obtained by comparing the action functional $S$ before and after the transformation. Following the Euler-Lagrange equations, the temporal evolution of the canonical coordinates is such that the action functional is minimized. The action functional is given by the expression:
%
\begin{align}
  S= \int_{t_0}^{t_1} \mathcal{L}(\vect{q},\dot{\vect{q}}, t)\,  \mathrm{d}t \, .
\end{align}
%
With the relation defined in \eqref{eq:hamilton01}, the action functional yields:
%
\begin{align}
  S = \int_{t_0}^{t_1} (p_x \dot{x} + p_y \dot{y} + p_z \dot{z} - H) \, \mathrm{d} t \, ,
\end{align}
for the set of canonical coordinates 
%
\begin{align}
  (x,p_x), (y,p_y), (z,p_z) \, .
\end{align}
%
%
After the transformation of the independent variable $t \rightarrow s$, the action functional is given by
%
\begin{align}
  S = \int_{s_0}^{s_1} (p_x x' + p_y y' - H t' + p_z) \, \mathrm{d} s \, ,
\end{align}
where $q_i' = \dd{q_i}{s}$. The direct comparison of the original and the transformed action functional shows that the new set of canonical coordinates is given by
%
\begin{align}
  (x,p_x), (y,p_y), (-t,H) \, ,
\end{align}
%
with respect to the transformed Hamiltonian $\tilde{H}$:
%
\begin{align}
  \tilde{H} = -p_z \, .
\end{align}
%
Elementary transformations of \eqref{eq:h02} and taking into account that $H$ represents the full ion energy $E$, the new Hamiltonian yields:
%
\begin{align}
  \tilde{H} = -p_z = - \sqrt{ \frac{(E-q\phi)^2}{c^2} - m^2c^2 - (p_x - q A_x)^2 - (p_y -q  A_y)^2} - q  A_z \, .
\end{align}
%
The Hamiltonian should be expandable to allow for the analytical treatment of complex vector potentials. This requires the dynamic variables in the square root to be small, which can be achieved with the following set of transformations:
%
\begin{alignat}{4}
p_i &\rightarrow \tilde{p}_i = \frac{p_i}{P_0} \, \frac{m_0}{m} \quad \quad &\tilde{H} &\rightarrow \bar{H} = \frac{\tilde{H}}{P_0}  \frac{m_0}{m} \, , \notag \\
q A_i &\rightarrow \chi a_i = \chi \frac{q_0 A_i}{P_0}  \quad \quad &E &\rightarrow \tilde{E} = \frac{E}{P_0} \, \frac{m_0}{m} \, , \label{eq:normalization_H}
\end{alignat}
%
Note that this is a different definition than usually given in literature~\cite{ProceedingsCAS1995}, and takes into account that the momentum of the tracked ion may be significantly different from the reference momentum. The chosen definition with the ratio of mass with respect to the mass of the reference particle takes into account for these difference and delivers small quantities even for a large spread of masses.

Expressed in terms of the new coordinates, and assuming that a gauge can be found such that $\phi=0$, the transformed Hamiltonian is given by
%
\begin{align}
\bar{H} = - \sqrt{ \frac{m_0^2}{m^2} \, \left( \frac{E^2 - m^2 c^4}{P_0^2c^2} \right)   - (\tilde{p}_x - \chi a_x)^2 - (\tilde{p}_y- \chi a_y)^2   } - \chi a_z \, .
\end{align}
% 
Using \eqref{eq:p_over_p0} and the relativistic energy-momentum relation the latter can be simplified to
\begin{align}
\bar{H} = - \sqrt{(1+\delta)^2  - (\tilde{p}_x - \chi a_x)^2 - (\tilde{p}_y-\chi a_y)^2 } - \chi a_z \, .
\end{align}
%
The longitudinal motion can be described in a more convenient manner by means of another transformation to new canonical variables:
%
\begin{align}
(x,\tilde{p}_x), (y,\tilde{p}_y), (-t,\tilde{E}) \rightarrow (X,P_x), (Y,P_y), (\sigma,p_\sigma) \, .
\end{align}
%
The transformation can be provided using the generating function of the second order
\begin{align}
F_2 = x P_x + y P_y + (s-\beta_0 ct) \, \left( p_\sigma + \frac{E_0}{\beta_0 P_0 c} \right) \, ,
\end{align}
%
from which the transformed variables ${Q_i,P_i}$ and the new Hamiltonian $K$ follow from the following relations~\cite{ProceedingsCAS1995}:
%
% [CHECKED]
%
\begin{align}
\tilde{p}_i = \PD{F_2}{q_i} \quad \quad \quad Q_i = \PD{F_2}{P_i} \quad \quad \quad K = \bar{H} + \PD{F_2}{z} = \bar{H}+p_\sigma \, .
\end{align}
%
The transformed coordinates are then related to the initial coordinates as
%
\begin{alignat}{5}
X  &= x  \, ,          \quad \quad  &Y   &&= y  \, ,           \quad \quad &\sigma   &&= \phantom{m} s - \beta_0 ct \, ,  \\ \label{eq:sigmadefinition}
P_x&= \tilde{p}_x  \, , \quad \quad  &P_y &&= \tilde{p}_y \, ,  \quad \quad &p_\sigma &&= \frac{\frac{m_0}{m} \, E - E_0}{\beta_0 P_0 c} \, .
\end{alignat}
%
with the new Hamiltonian 
%
\begin{align}
K = p_\sigma - \sqrt{(1+\delta)^2 - (P_x - \chi a_x)^2 - (P_y-\chi a_y)^2} - \chi a_z \, .
\end{align}
%
The new longitudinal coordinate $\sigma$ with the canonical conjugate $p_\sigma$ describes the difference in arrival time with respect to the reference particle. 
%
After a last transformation for convenience: $P_i \rightarrow p_i$, $K \rightarrow H$, the final generic accelerator Hamiltonian is written as
\begin{align}
H = p_\sigma - \sqrt{(1+\delta)^2 - (p_x - \chi a_x)^2 - (p_y-\chi a_y)^2} - \chi a_z \, .
\end{align}


%
In the general case of a curved coordinate system, the Hamiltonian changes to~\cite{Fjellstrom:1642385}
%
\begin{align}
H = p_\sigma - (1+h_x(s)\,x) \, \left(  \sqrt{ (1+\delta)^2  - (p_x - \chi a_x(s))^2 - (p_y-\chi a_y(s))^2} + \chi a_s(s)  \right) \, , \label{eq:rawHamiltonian}
\end{align}
%
where $h_x(s) = \frac{1}{\rho(s)}$ is the radius of curvature of the particle trajectory. The longitudinal magnetic vector potential with respect to $s$ is defined by the following relation:
%
\begin{align}
p_s = \frac{m_0 \, \gamma \, \dot{s}}{P_0} \, (1+h_xx)^2 + q (1+h_xx) \, \chi \, a_s \, .
\end{align}
%
For the case of a straight coordinate system with $h_x=0$, the quantity $p_s$ is identical to $p_z$. In the mono-isotopic limit of $m \rightarrow m_0$ and $q \rightarrow q_0$, all derived equations converge into the standard expressions derived from the mono-isotopic Hamiltonian~\cite{CERN-SL-95-12}.

\subsubsection{Expansion}


The Hamiltonian presented in ~\eqref{eq:rawHamiltonian} is exact, thus without the usage of approximations. Depending on the complexity of the electromagnetic field of the beam-line element and the corresponding boundary conditions it can be useful to expand the square root in $\frac{(p_x-\chi a_x)^2 + (p_y - \chi a_y)^2}{(1+\delta)^2}$. By virtue of the normalization applied in \eqref{eq:normalization_H}, this is a small quantity, such that the second order Taylor expansion delivers a good approximation of the physical problem:
\begin{align}
H \approx p_\sigma - (1+h_x(s)x) \left[ (1+\delta) \left( 1 - \frac{1}{2} \frac{(p_x - \chi a_x(s))^2 + (p_y - \chi a_y(s))^2 }{(1+\delta)^2} \right) + \chi a_s(s) \right] \, . \label{eq:expanded_hamiltonian}
\end{align}  
The accuracy of tracking maps derived from the expanded Hamiltonian is studied for the example of the drift space in \cite{Fjellstrom:1642385}.


\section{Tracking Maps for Beam-Line Elements}\label{chap:trackingmaps}

Based on the Hamiltonian for multi-isotopic particle beams, in this chapter the tracking maps for the individual beam line elements are derived. The technical implemenetation of the individual maps into hiSixTrack is presented in \chapref{chap:implement}. Their symplecticity is demonstrated by means of the Jacobian matrix in \chapref{chap:sympl}.


\subsection{Drift Space}
% A drift space is a field-free region, in which the particles move on a straight line. If a particle enters a drift space region of length $L$ with the initial coordinates $x_i,x_i'$, the final angle and position $x_f,x_f'$ are given by the trivial relation
% \begin{align}
% \begin{pmatrix} x_f \\ x_f' \end{pmatrix} = \begin{pmatrix} 1 & L \\ 0 & 1 \end{pmatrix} \begin{pmatrix} x_i \\ x_i' \end{pmatrix} \, .
% \end{align}
% The full set of symplectic tracking maps is obtained by means of the Hamiltonian. 


\subsubsection{Exact Hamiltonian}
A drift space is defined by the absence of electromagnetic fields, thus the vector potential is zero in all directions. With regard to \eqref{eq:rawHamiltonian}, the Hamiltonian is then given by
\begin{align}
H = p_\sigma - \sqrt{(1+\delta)^2 - p_x^2 -p_y^2}  = p_\sigma - p_z\, . \label{eq:full_H_drift}
\end{align}
The equations of motion derived from this Hamiltonian are
\begin{alignat}{4}
x' &= \TD{x}{z} = \PD{H}{p_x} = \frac{p_x}{ \sqrt{(1+\delta)^2 - p_x^2 -p_y^2} } = \frac{p_x}{p_z} \quad \quad \quad \quad &p_x' &&= -\PD{H}{x} = 0 \, , \\
y' &= \TD{y}{z} = \PD{H}{p_y} = \frac{p_y}{\sqrt{(1+\delta)^2 - p_x^2 -p_y^2}} = \frac{p_x}{p_z} \quad \quad &p_y' &&= -\PD{H}{y} = 0 \, , \\
\sigma' &=  \TD{\sigma}{z} = \PD{H}{p_\sigma} = \left( 1 - \frac{\beta_0}{\beta_z}  \right)     &p_\sigma' &&= -\PD{H}{\sigma} = 0 \, , 
\end{alignat}
where $\beta_z$ is defined as 
\begin{align}
\beta_z = \beta \, \frac{p_z}{1+\delta}\,.
\end{align}
%
Starting from the equations of motion, the transformation of the initial set of coordinates $(\mathbf{q}^I,\mathbf{p}^I)$ at the beginning of the drift space is related to their final coordinates $(\mathbf{q}^F,\mathbf{p}^F)$ by the following set of equations, referred to as the transfer map:
%
\begin{alignat}{4}
x^F & = x^I + x^{I}  L \quad \quad \quad \quad \quad \quad &p_x^F &= p_x^I \, , \\
y^F & = y^I + y^{I} L \quad \quad &p_y^F &= p_y^I \, , \\
\sigma^F & = \sigma^I + \left(1 - \frac{\beta_0}{\beta_z}\right) L \quad \quad &p_\sigma^F &= p_\sigma^I \, ,
\end{alignat}
%
where, $L$ denotes the length of the drift space.
\subsubsection{Expanded Hamiltonian}
Combining \eqref{eq:expanded_hamiltonian} and \eqref{eq:full_H_drift} yields for the expanded Hamiltonian
\begin{align}
H \approx p_\sigma - \delta + \frac{1}{2} \, \frac{p_x^2+p_y^2}{(1+\delta)} \, . \label{eq:exp_drift}
\end{align}
Hamilton's equations of motion are 
\begin{alignat}{4}
x' &= \TD{x}{z} = \PD{H}{p_x} = \frac{p_x}{ (1+\delta) } &p_x' &&= -\PD{H}{x} = 0 \, , \\
y' &= \TD{y}{z} = \PD{H}{p_y} = \frac{p_y}{ (1+\delta) } \quad \quad &p_y' &&= -\PD{H}{y} = 0 \, , \\
\sigma' &=  \TD{\sigma}{z} = \PD{H}{p_\sigma} =  1 - \frac{\beta_0}{\beta} \left( 1 + \frac{1}{2} \, \frac{p_x^2 + p_y^2}{(1+\delta)^2}  \right)   \quad \quad \quad  &p_\sigma' &&= -\PD{H}{\sigma} = 0 \, , 
\end{alignat}
Note the different definition of $x'$ with respect to the exact Hamiltonian. The resulting tracking map for the drift space from the expanded Hamiltonian yields:
%
\begin{alignat}{4}
x^F &= x^I +   \frac{p_x^I}{1+\delta} \, L  &p_x^F &&= p_x^I \, , \\
y^F &= y^I +   \frac{p_y^I}{1+\delta} \, L  &p_y^F &&= p_y^I \, , \\
\sigma^F &=  \sigma^I - L \, \frac{\beta_0}{\beta} \left( 1 + \frac{1}{2} \, \frac{(p_x^I)^2 + (p_y^I)^2}{(1+\delta)^2}  \right)   \quad \quad \quad  &p_\sigma^F &&=  p_\sigma^I \, .
\end{alignat}
%
The comparison of $x'$ for the expanded and the exact drift space unveils that they agree in the limit of small $p_x$ and $p_y$. A detailed analysis is presented in \cite{Fjellstrom:1642385}. Since 2013, the exact Hamiltonian is implementated in SixTrack.

The modification of tracking maps for the drift space is not necessary for hiSixTrack, because the key quantity $\beta_0/\beta$ is defined in SixTrack as follows:
\begin{align}
\frac{\beta_0}{\beta} = \frac{E}{p \, c} \, \frac{p_0 \, c}{E} \, ,
\end{align}
which is also applicable for multi-isotopic heavy-ion beams.

\subsection{Dipole}
For simplicity, parts of the following derivations are only considered for a horizontal bending, but they are also valid for vertical bendings by permuting $x$ and $y$. The uniform magnetic field in a horizontal bending dipole can be described by the vector potential~\cite{wolski2014beam}
\begin{align}
A_x = 0 \, , \quad \quad \quad A_y =0 \, , \quad \quad \quad A_s = -B_y x \, \left( 1- \frac{h_x x}{2 (1+h_x x)} \right)\, .
\end{align}
Ideally, the vertical magnetic field $B_y$ is matched to the reference momentum and charge such that the bending radius of the reference particle yields $\rho_0=h_x^{-1}$. In reality, the magnet strength may differ from the reference, such that the ideal particle is bent with the radius $\rho^i=k_0^{-1}$ and 
\begin{align}
B_y = \frac{P_0 k_0}{q_0} \,  .
\end{align}
The resulting exact Hamiltonian is then given by
\begin{align}
H = p_\sigma - (1+h_x x)\, p_z + \chi \, k_0 \left( x + \frac{h_x x^2}{2} \right) \, ,
\end{align}
%
with
%
\begin{align}
  p_z = \sqrt{(1+\delta)^2 - (p_x - \chi a_x)^2 -(p_y - \chi a_y)^2} \, .
\end{align}
%
Omitting non-linear and constant terms delivers for the expanded Hamiltonian 
\begin{align}
H \approx p_\sigma - \delta - (h_x x) (1+\delta) + \frac{1}{2} \frac{p_x^2 + p_y^2}{(1+\delta)} + \chi \, k_0 \, \left(x + \frac{h_x x^2}{2}\right) \, . \label{eq:exp_dipole}
\end{align}


\subsubsection{Thick Dipole}
%
With the expanded Hamiltonian and considering that $\delta$ is a function of $p_\sigma$ with the derivative $\frac{\mathrm{d} \delta}{\mathrm{d} p_\sigma} = \frac{\beta_0}{\beta}$, the equations of motion become
%
% CHECKED FOR ACCURACY 28.06.16 
% MATHEMATICA NOTEBOOK: DIPOLE_new
%
\begin{alignat}{4}
  x' &=  \PD{H}{p_x} = \frac{p_x}{1+\delta} \quad \quad \quad \quad &p_x' &= -\PD{H}{x} = h_x \, (1+\delta) - \chi \, k_0 \, (1+h_x x)  \, , \label{eq:dipoleequationofmotion} \\ 
  y' &= \PD{H}{p_y} = \frac{p_y}{1+\delta} \quad \quad &p_y' &= -\PD{H}{y} = 0 ,  \\
  \sigma' &=  \PD{H}{p_\sigma} =  1 - \frac{\beta_0}{\beta} \left( 1+h_x x + \frac{1}{2} \frac{p_x^2 + p_y^2}{(1+\delta)^2} \right) \quad \quad  &p_\sigma' &= -\PD{H}{\sigma} = 0 \, . 
\end{alignat}
%
In the vertical direction, the dipole acts like a drift space in the vertical direction where no bending force is present. 

Starting from \eqref{eq:dipoleequationofmotion}, the horizontal motion can be described by the differential equation
\begin{align}
x''(s) + \frac{\chi \, h_x \, k_0}{(1+\delta)} \, x = \frac{h_x \, \delta}{(1+\delta)} + \frac{h_x - \chi \, k_0}{(1+\delta)} \, . \label{eq:diffeqdipole}
\end{align}
%
The homogenious part of the equation describes an oscillation with frequency $\omega_x=\sqrt{\frac{\chi \, h_x \, k_0}{1+\delta}}$. Note that the inhomogenious part of the differential equation \eqref{eq:diffeqdipole} represents the dispersion in the magnet. Compared to the corresponding mono-isotopic equation presented in \cite{DESY-95-189}, an additional term proportional to $(\chi-1)$ appears, which takes account for the isotopic dispersion. For particles of the reference species this term vanishes. 
%

The following quantities are defined for convenience:
\begin{alignat}{4}
S_x &= \sin \omega_x L \quad \quad \quad &C_x &=  \cos \omega_x L \, \\ \omega_x^2 &= \frac{\chi \, h_x \, k_0}{1+\delta}   &\Omega_x &= \frac{1+\delta}{k_0 \,\chi} - \frac{1}{h_x} \, .
\end{alignat}
%
The transfer map is then given by 
\begin{align}
% 
% x is correct 28.06.16
% SYMPLECTICITY FOR 4D CASE CHECKED
%
x^F &= x^I \, C_x + p_x^I \, \omega_x^{-1} \frac{S_x}{(1+\delta)} + \Omega_x \, \left(1 - C_x \right) \, , \label{eq:solution_thick_dipole}\\
%
%
p_x^F &= -x^I \, \omega_x \, (1+\delta) \, S_x + p_x^I \, C_x + (1+\delta) \, \Omega_x \, \omega_x \, S_x \, , \\
%
%p_x &\rightarrow - x \, \omega_x \, \tilde{S}_x + p_x \, C_x + \omega_x \,   \Omega_x \, \tilde{S}_x \, , \\ 
y^F &=  y^I + (y')^I \, L , \\
p_y^F &= p_y^I \, . \label{eq:tckdp4} \\ 
%
\sigma^F &= \sigma^I + L \, \left[ 1 - \frac{\beta_0}{\beta} - \frac{\beta_0}{\beta} \, \frac{1}{2} \, \left(\frac{p^I_y}{1+\delta} \right)^2 \right]  - \\ 
 & \phantom{ \sigma^I +aaa} \frac{\beta_0}{\beta} \, \left[ \frac{S_x}{\omega_x} \, \left(x^I - \Omega_x \right) + \frac{p_x^I \, (1-C_x)}{(1+\delta) \, \omega_x^2} + L \, \Omega_x  \right] - \\
 & \phantom{ \sigma^I + aaa} \frac{1}{8} \, \frac{\beta_0}{\beta} \, \frac{1}{(1+\delta)^2} \, \Bigg[ - 2 \, p_x^I \, (1+\delta) \, (x^I - \Omega_x) + \\  
 & \phantom{ \sigma^I + aaaaaaaaaaaaaaaaaa} 2 \, L \, \left( \left(p_x^I\right)^2 +(1+\delta)^2\,\omega_x^2\,\left(x^I-\Omega_x\right)^2 \right) +  \\ 
 & \phantom{ \sigma^I + aaaaaaaaaaaaaaaaaa} 2 \, p_x^I \, (1+\delta) \, \left(x^I-\Omega_x\right)\, \cos \left(2 \, \omega_x \, L \right) + \notag \\
 & \phantom{ \sigma^I + aaaaaaaaaaaaaaaaaa} \frac{\sin \left(2 \, \omega_x \, L \right)}{\omega_x} \, \left( \left(p_x^I\right)^2 - (1+\delta)^2 \, \omega^2 \, \left(x^I - \Omega_x\right) \right) \Bigg]  \\
%
% \sigma &\rightarrow \sigma + L\left(1 - \frac{\beta_0}{\beta}\right) \\
%   & \qquad\, -\frac{\beta_0}{\beta} \Bigg[ \frac{h_x S_x}{\sqrt{G_x}} \cdot x + \frac{1-C_x}{h_x} \cdot p_x
%   + \frac{h_y S_y}{\sqrt{G_y}} \cdot y + \frac{1-C_y}{h_y} \cdot p_y
%   + \delta \left(2L - \frac{S_x}{\sqrt{G_x}} - \frac{S_y}{\sqrt{G_y}} \right) \Bigg] \\
%   & \qquad\, - \frac{1}{4}\frac{\beta_0}{\beta} \Bigg[ G_x \left(L-\frac{C_xS_x}{\sqrt{G_x}} \right)
%   \left(x - \frac{\delta}{h_x}\right)^2
%   + \left(L+\frac{C_xS_x}{\sqrt{G_x}} \right) \frac{p_x^2}{(1+\delta)^2}
%   -\left(x-\frac{\delta}{h_x}\right) \frac{2S_x^2}{1+\delta} \cdot p_x \\
%   & \qquad\, + G_y \left(L-\frac{C_yS_y}{\sqrt{G_y}} \right)
%   \left(y - \frac{\delta}{h_y}\right)^2 + \left(L+\frac{C_yS_y}{\sqrt{G_y}}\right) 
%   \frac{p_y^2}{(1+\delta)^2}
%   -\left(y-\frac{\delta}{h_y}\right)\frac{2S_y^2}{1+\delta} \cdot p_y \Bigg] \\
%
p_\sigma^F & = p_\sigma^I \, .
\end{align} 


\subsubsection{Thin Dipole}
Thick lens tracking is very demanding in terms of time and computing power. Also, the exact motion of the particle inside the magnet is often not required and the global tracking through a large accelerator like the LHC can be well approximated by thin lenses. The tracking routine used in SixTrack for collimation studies is based on thin lens tracking, so the derivation of thin lens tracking maps is of particular interest. 

The Hamiltonian in \eqref{eq:exp_dipole} can be decomposed into the expanded Hamiltonion of a drift $H_D$, defined in \eqref{eq:exp_drift}, and the contribution from electromagnetic fields $H_L$ as follows:
%
\begin{align}
  H = H_D - h_x \, x \, (1+\delta) + \chi \, k_0 \, \left(x + \frac{h_x \, x^2}{2} \right) = H_D + H_L \, .
\end{align}
%
In the thin lens approximation, this Hamiltonian is changed to~\cite{DESY-95-189}:
%
\begin{align}
 H = H_D + \bar{\delta}(s-s_0) \, L \, H_L \, , \label{eq:thin_H}
\end{align}
%
where $\bar{\delta}(s-s_0)$ is the $\delta$ distribution which is non-zero only at the center $s_0$ of the magnet~\cite{dirac:1958}. 
%
Starting from this Hamiltonian, the equations of motion are given by:
\begin{align}
  x'    &= \frac{p_x}{(1+\delta)} \, , \label{eq:motion_thin_dipole}\\
  p_x'  &= L \, \bar{\delta}(s-s_0) \, \left[ h_x (1+\delta) - \chi \, k_0 \, (1+h_x \, x)  \right] \, ,\\
  y'    &= \frac{p_y}{(1+\delta)} \, ,\\
  p_y'  &= 0 \, , \\
  \sigma' &= 1 - \frac{\beta_0}{\beta} - \frac{\beta_0}{\beta} \, \left[ \frac{1}{2} \left(x'^2 + y'^2\right) \right] - \frac{\beta_0}{\beta} \, h_x \, x \, (1+\delta) \, \bar{\delta}(s-s_0) \, L \, , \\
  p_\sigma ' &= 0 \, .
\end{align}
%
The solution of the differential equations in the thin lens approximation are obtained by integrating from $s-\epsilon$ to $s+\epsilon$ in the limit of $\epsilon \rightarrow 0$. The tracking map for $x$ with the equation of motion given in \eqref{eq:motion_thin_dipole} is obtained as follows:
%
\begin{align}
  x^F - x^I = \lim_{\epsilon \rightarrow 0} \left[ \int_{s_0 - \epsilon}^{s_0+\epsilon} \frac{p_x}{(1+\delta)} \, \mathrm{d}s \right] = 0 \, .
\end{align}
%
Applying the same approach for the remaining quantities, the transformation rules for the dipole in thin lens approximation are given by:
\begin{alignat}{4}
x^F &= x^I \, , \label{eq:thindip01} \\ 
p_x^F &= p_x^I + L \left[ h_x \, (1+\delta) - k_0 \, \chi \, (1 + h_x \, x^I)  \right]\, \label{eq:thindipolekick} ,\\ 
y^F &= y^I \, , \\
p_y^F &= p_y^I\, ,\\ 
\sigma^F &= \sigma^I - \frac{\beta_0}{\beta} \, h_x \, x^I \, L \, , \\
p_\sigma^F & = p_\sigma^I \, \label{eq:thindip02} .
\end{alignat}
%
The transverse kick depends on the initial horizontal offset $x^I$, which is known as weak focusing~\cite{wolski2014beam}. Two particles with the same set of $\chi$ and $\delta$ are bent differently in the same magnet when they have two different initial offsets $x_1,x_2$, as illustrated in \figref{pic:15092201}. Weak focusing refers to the effect that the trajectories of two particles starting at different positions are focussed to a defined focal point. 

%
The effect of dispersion is taken into account by the dependence on $\delta$ and $\chi$. The chromatic dispersion, which is a pure function of the particle velocity, and the isotopic dispersion which only depends on the mass to charge ratio are independent effects which can compensate or enhance each other. In the mono-isotopic case $\chi \rightarrow 1$, and with perfectly matched magnetic field $k_0=h_x$, the \eqref{eq:thindipolekick} yields the well known shape (see \cite{CERN-SL-95-12})
\begin{align}
p_x^F = p_x^I + \delta h_x L - h_x^2 \, L \, x^I \, .
\end{align} 


\input{pictures/16070702.tex}
  % #PHTHESIS FILE ORIGIN
  %/home/phermes/Dropbox/PhD/pictures/150922_weak_focusing/tikz/drawing.tex




  % \begin{figure}[t]
  % \centering
  % \includegraphics[width=0.4\textwidth]{pictures/15092201.pdf}
  % \caption{Bending behaviour in a magnetic dipole field for two particles starting with different initial conditions.}  
  % \label{pic:15092201}
  % %/home/phermes/Dropbox/PhD/pictures/150922_weak_focusing/drawing-compiled.pdf
  % \end{figure}

\subsection{Thin Transverse Kicker Magnet}

  \begin{figure}[b]
  \centering
  \includegraphics[width=0.7\textwidth]{pictures/15112403.pdf}
  \caption{Schematics of a transverse kicker magnet. Kicker magnets are dipole magnets with unbent reference trajectory, thus $h_x=0$. }  
  \label{pic:15112403}
  %/home/phermes/Dropbox/PhD/pictures/151125_kickermagnet/tikz_drawing-crop.pdf
  \end{figure}


Transverse kicker magnets are used in accelerators as the LHC to control the beam orbit. Technically they are identical to bending magnets, with the exception that $h_x=0$, so the reference trajectory in these magnets is not bent. The Hamiltonian for a transverse kicker magnet in thin lens approximation is given by:
%
\begin{align}
  H=  H_D + \chi \, k_0 \, L \, \bar{\delta}(s-s_0) \, .
\end{align}
%
The resulting equations of motion lead to the following transport map:
\begin{alignat}{4}
p_x^F & = p_x^I - k_0 \, \chi \, L\, \label{eq:thindipolekick} ,\\ 
p_y^F & = p_y^I\, ,\\ 
p_\sigma^F & = p_\sigma^I \, .
\end{alignat}




\subsection{Quadrupole}
%
The vector potential of a quadrupole magnet is given by~\cite{CERN-SL-95-12}:
\begin{align}
A_x = 0 \quad \quad \quad A_y = 0 \quad \quad \quad A_s = - \frac{1}{2} \, g \, (y^2 -x^2) \, .
\end{align}
%
Expressed in the normalized coordinates, the longitudinal vector potential becomes
%
\begin{align}
a_s = - \frac{1}{2} \frac{q_0}{P_0} g  \, (y^2 -x^2)  = - \frac{1}{2} \, k \,  (y^2 -x^2) \, .
\end{align}
The quantity $k=\frac{q_0}{P_0} g$ is the normalized quadrupole gradient which has the unit $[k] = \text{m}^{-2}$. The optics of a machine in a certain configuration is defined by a full set of $k_i$ with $i= 1,...,N_q$, where $N_q$ is the number of quadrupoles in the machine. Thanks to the definition of the normalized quadrupole strength, the machine optics can be described by identical values valid for different energies, even if in reality the magnet currents are ramped with increasing beam energy. 

The reference trajectory passes through the center of the quadrupole where no magnetic field is present and is hence straight with $h_x=0$. The exact Hamiltonian of a quadrupole yields 
\begin{align}
H = p_\sigma - \sqrt{(1+\delta)^2 - p_x^2 -p_y^2} + \frac{1}{2} \, k \, \chi\, (x^2 -y^2) \, .
\end{align}

\subsubsection{Thick quadrupole}
%
For the solution of the equations of motion, the following quantities are defined:
\begin{align}
K = k \, \chi \, ,\quad \quad \quad \quad  \omega^2 = |K| \, .
\end{align}
%
The expanded Hamiltonian for the quadrupole is then given by
\begin{align}
H = p_\sigma + \frac{1}{2} \frac{p_x^2+p_y^2}{(1+\delta)} + \frac{1}{2} \, K \, (x^2 -y^2) -\delta \, . \label{eq:quad_exp_H}
\end{align}
Hamilton's equations deliver the following equations of motion
%
\begin{alignat}{4}
x'      &= \PD{H}{p_x} = \frac{p_x}{(1+\delta)} \quad \quad \quad &p_x' &= - \PD{H}{x} = - K x \, ,  \\
y'      &= \PD{H}{p_y} = \frac{p_y}{(1+\delta)} \quad \quad \quad &p_y' &= - \PD{H}{y} = \phantom{-} K y \, , \\
\sigma' &= \PD{H}{p_\sigma} = 1- \frac{\beta_0}{\beta} \, \left[ 1 + \frac{1}{2} \frac{p_x^2 + p_y^2}{(1+\delta)^2}  \right] \quad \quad \quad &p_\sigma' &= - \PD{H}{\sigma} =  \phantom{-}  0 \, .
\end{alignat}
Using these relations, the transverse motion can be described by two differential equations of the same type
\begin{align}
x'' + \frac{K}{1+\delta} x &= 0 \, , \label{eq:quadeq1} \\
y'' - \frac{K}{1+\delta} y &= 0 \, .
\end{align}
%
This is the well-known Hill equation. The transverse transport map is the general solution of the two differential equations 
\begin{alignat}{8}
x^F &= C_x x^I &&+ S_x \frac{p_x^I}{(1+\delta)} \quad \quad \quad \quad &p_x^F &= C_x p_x^I &&-  S_x \, \omega^2 x^I \, (1+\delta) \, , \\ 
y^F &= C_y y^I &&+ S_y \frac{p_y^I}{(1+\delta)} \, &p_y^F &= C_y p_y^I &&+  S_y \, \omega^2 y^I  \, (1+\delta) \, ,  %\\
% \sigma & \rightarrow \sigma && &p_\sigma &\rightarrow \sigma + &&\left( 1 - \frac{\beta_0}{\beta} \right) L  \, & \\
%  & && & & &&-\frac{\omega^2}{4} \frac{\beta_0}{\beta} \left( [S_x C_x -L] x^2 - [S_y C_y -L] y^2 \right) \notag \\
%   & && & & &&-\frac{\omega^2}{2} \frac{\beta_0}{\beta} \left( -S_x^2 \frac{x p_x}{1+\delta} +S_y^2 \frac{y p_y}{1+\delta} \right) \notag \\
%  & && & & &&-\frac{1}{4} \frac{\beta_0}{\beta} \left( [L+S_xC_x] \frac{p_x^2}{(1+\delta)^2} + [L+S_yC_y] \frac{p_y^2}{(1+\delta)^2} \right) . \notag
\end{alignat}
The quantities $C_u$ and $S_u$ are defined as follows:
\begin{alignat}{4}
C_x = &\begin{cases}  \cos \left( \omega L \right)  & \text{if} \quad  K>0 \\ 
\cosh \left( \omega L \right)  & \text{if} \quad  K<0 \end{cases} \quad \quad \quad&S_x = &\begin{cases}  \omega^{-1}  \sin \left( \omega L \right)  & \text{if} \quad  K>0 \\ \omega^{-1}\sinh \left( \omega L \right)  & \text{if} \quad  K<0 \end{cases} \label{eq:quad_solution1}
\, , \\ 
C_y = &\begin{cases}  \cosh \left( \omega L \right)  & \text{if} \quad  K>0 \\ 
\cos \left( \omega L \right)  & \text{if} \quad  K<0 \end{cases} \quad \quad \quad&S_y = &\begin{cases}  \omega^{-1}  \sinh \left( \omega L \right)  & \text{if} \quad  K>0 \\ \omega^{-1}\sin \left( \omega L \right)   & \text{if} \quad  K<0 \end{cases} \label{eq:quad_solution2}
\, , 
\end{alignat}




\subsubsection{Thin Lens Approximation}

Following the Eqs.~(\ref{eq:thin_H}) and (\ref{eq:quad_exp_H}), the following Hamiltonian can be derived for the quadrupole in thin lens approximation:
%
\begin{align}
  H = H_D + \frac{1}{2} \, \bar{\delta}(s-s_0) \, L \, K \, (x^2-y^2) \, .
\end{align}
The corresponding transfer map becomes:
\begin{alignat}{8}
x^F &= x^I \, ,  \quad \quad \quad \quad &p_x^F &=  p_x^I &&-  K L \, x^I \, , \\ 
y^F &= y^I \, ,  \quad \quad \quad \quad &p_y^F &=  p_y^I &&+  K L \, y^I \, , \\
\sigma^F &= \sigma^I \, ,  \quad \quad \quad \quad &p_\sigma^F &=  p_\sigma^I \, &&  &.
\end{alignat}
This transfer map corresponds to a focusing lens in horizontal and a defocusing lens in vertical direction. Compared to the transverse kick $\Delta p_{x} = - k \, L\, x^I$ of the reference isotope, the kick for arbitrary ions scales linearly with $\chi$:
\begin{align}
\Delta p_{x} = - \chi \, k \, L \, x^I \, .
\end{align}
The focal length for the different isotopes varies with $\chi$, in line with the expectation.

\subsection{Thin Multipole}
Higher order magnetic fields than the presented dipole and quadrupole fields are described in a more generic way:
%
\begin{align}
  B_y + i B_x = \sum_{n=1}^{\infty} (b_n + i a_n) \, \left( \frac{x+iy}{\rho_0} \right)^{n-1} \, . \label{eq:multiB}
\end{align}
%
In this context, $n$ the multipole order, $b_n,a_n$ are the multipole coefficients which describe the field orientation for the contribution of each multipole order~\cite{wolski:lectures} and the quantity $\rho_0$ is a reference radius. The well-known dipole and quadrupole fields described in the previous chapters correspond to the multipole orders $n_D=1$ and $n_Q=2$. In a perfect multipole of the order $n$, all multipole coefficients $m\neq 0$ yield zero.

The magnetic field described in \eqref{eq:multiB} corresponds to the following vector potential:
%
\begin{align}
A_x = 0 \, , \quad \quad \quad A_y = 0 \, , \quad \quad \quad A_z = - \text{Re} \sum_{n=1}^{\infty} (b_n + i a_n) \frac{(x+i y)^n}{n \, \rho_0^{n-1}} \, .
\end{align} 
%
Inserting this vector potential the Hamiltonian in thin lens approximation yields
%
\begin{align}
  H = H_D - \frac{q_0}{P_0} \, \chi \, L \, \bar{\delta}(s-s_0) \, \text{Re} \, \left[ \sum_{n=1}^\infty (b_n + i \, a_n) \, \frac{(x+i \, y)^n}{n \, \rho_0^{n-1}}  \right] 
\end{align}
%
The resulting tracking map for the thin multipole is:
%
\begin{alignat}{8}
x^F & =  x^I \, ,  \quad \quad \quad \quad &p_x^F &=   p_x^I &&-  \chi \, L \, \text{Re} \left[ \sum_{n=1}^\infty (k_n+ i \, \hat{k}_n) \, (x+i\,y)^{n-1}    \right] \\ 
y^F & =  y^I \, ,  \quad \quad \quad \quad &p_y^F &=  p_y^I && -  \chi \, L \, \text{Re} \left[ \sum_{n=1}^\infty \, i (k_n+ i \, \hat{k}_n) \, (x+i\,y)^{n-1}    \right] \\ 
\sigma^F & =  \sigma^I \, ,  \quad \quad \quad \quad &p_\sigma^F &\rightarrow  p_\sigma^I \,, &&  &
\end{alignat}
%
where $k_n$ and $\hat{k}_n$ are defined as:
%
\begin{align}
  k_n = \frac{q_0}{P_0} \, \frac{a_n}{\rho_0^{n-1}} \quad \text{and} \quad   \hat{k}_n = \frac{q_0}{P_0} \, \frac{b_n}{\rho_0^{n-1}} \, .
\end{align}

\subsection{Accelerating RF Cavity}
  % \begin{figure}[t]
  % \centering
  % \includegraphics[width=0.8\textwidth]{pictures/15081101.jpg}
  % \caption{Principle of the beam acceleration by means of RF cavities. The structure generates a longitudinal electric field. The wavelength of this }  
  % \label{pic:15081101}
  % %/home/phermes/Documents/1435503941_77cff2f9fcf76f625a60aa3135ed2c87.jpg
  % \end{figure}


The energy gain $\Delta E$ a particle receives at the position $s$ inside an accelerating cavity operated at frequency $f$ and wave number $k = \frac{\omega}{c} = 2 \pi f$ can be \mbox{approximated by}:
%
\begin{align}
  \Delta E = q  V(s) \, \sin \left( k \, \frac{\sigma}{\beta_0} + \phi \right) \, .
\end{align}
%
Where $V(s)$ is the longitudinal electric field acting on the particle at the position $s$. The dependence of $V$ on $s$ complicates the solution of the equations of motion and is therefore not explicitly considered in the magnetic vector potential. Rather, the electric field is averaged over the length of the cavity which is summarized in the mean electric field $U$~\cite{wolski2014beam}:
%
\begin{align}
  U = \frac{1}{L} \, \int_{-L/2}^{L/2} V(s) \, \mathrm{d} s \, .
\end{align}
%
This approximation leads to the following vector potential for a cavity~\cite{CERN-SL-95-12}:
%
\begin{align}
A_x = A_y =0 \, \quad \quad A_s =  \frac{U}{\omega} \cos \left(k \frac{\sigma}{\beta_0} + \phi  \right) \, .
\end{align}
%
The resulting expanded Hamiltonian for a thin cavity is then given by:
%
\begin{align}
  H = H_D + \chi \, \frac{q_0}{P_0} \, \frac{U}{\omega} \, \cos \left(k \frac{\sigma}{\beta_0} + \phi  \right) \, L \, \bar{\delta}(s-s_0) \, .
\end{align}

%
Using the substitution $\tilde{U} = \frac{q_0}{P_0} U$, the transfer map can be deduced as:
%
\begin{alignat}{8}
x^F & =  x^I \, ,  \quad \quad \quad \quad &p_x^F &=   p_x^I && \\ 
y^F & =  y^I \, ,  \quad \quad \quad \quad &p_y^F &=  p_y^I &&  \\ 
\sigma^F & =  \sigma^I \, ,  \quad \quad \quad \quad &p_\sigma^F &\rightarrow  p_\sigma^I +\chi \, U_n \, \frac{1}{\beta_0}\, \sin \left( k \frac{\sigma^I}{\beta_0} + \phi \right) \, .   &&  &
\end{alignat}
%
The change in $p_\sigma$ is, as expected, proportional to $q\, \frac{m_0}{m}$.




\section{Multi-Isotope Tracking in hiSixTrack}

Based on the presented tracking maps derived from the generic Hamiltonian, the proton tracking software SixTrack is made compatible with multi-isotope tracking. Contrary to the STIER approach, hiSixTrack is developed with the aim to track the heavy-ions as the physical particles instead of tracking protons with rigidities corresponding to the different ions. Essential physics parameters are therefore modified in hiSixTrack, as described in the following. However, as demonstrated above, these parameters converge to their mono-isotopic counterpart if the tracked particle is identical to the reference particle. 

The new implementation hiSixTrack is similar to use as the standard SixTrack-FLUKA coupling with minor changes required for the simulation set up. In this section, the changes introduced in hiSixTrack are discussed qualitatively while a more detailed technical overview of the implementation is given in the Appendix. 



\subsection{Multi-Isotope Tracking}

To allow for the tracking of multiple different isotopes, additional arrays containing information about the particle mass, charge number and nucleon number are implemented in hiSixTrack. For the definition of the reference ion species, a dedicated block is introduced in the \texttt{fort.3} file. In this block the nuclear charge number, mass number and physical particle mass (in GeV/$c^2$) is given by the user. Combining this information allows for the definition of $\chi$ of each particle, which is also stored in a dedicated array. 

The definition of $\delta$ is changed to obey the multi-isotopic definition given in \eqref{eq:15010701}:
%
\begin{align}
  \delta = \frac{P - P_0}{P_0}\quad \quad  \rightarrow \quad \quad \delta = \frac{P \frac{m_0}{m} - P_0}{P_0} \, .
\end{align}


With the re-definition of $\delta$ and the information on $\chi$ available, the tracking maps implemented in SixTrack are modified to include the isotopic dispersion. The implementation follows the tracking maps derived from the generalized Hamiltonian presented in the previous chapter. For collimation studies the thin lens implementation of SixTrack is used and accordingly all thin-lens tracking maps are modified to allow for the multi-isotopic tracking. Note that instead of the canonic momenta $p_x$ and $p_y$, SixTrack computes the evolution of $x',y'$, such that the corresponding tracking map for a thin-lens quadrupole is modified for hiSixTrack as follows:
%
\begin{align}
  (x')^F = (x')^I + \frac{k \, L}{1+\delta} \quad \quad \rightarrow \quad \quad   (x')^F = (x')^I + \frac{\chi \, k \, L}{1+\delta} \, .
\end{align}

Further details on the implementation are given in \chapref{chap:implement}.


\subsection{Definition of the Heavy-Ion Species}



The algorithm to load the initial distribution in the SixTrack-FLUKA coupling was already prepared, though not completely set up, for the eventual input of the particle species. The general stucture of input therefore includes by default not only the six-dimensional particle coordinates but also information on the particle species. In the sampling of the initial distribution file, the information on $A,Z,m$ is accordinly adjusted.

The subroutine initializing the particle properties in the standard SixTrack-FLUKA coupling ignores this information. For heavy-ion applications this algorithm is adjusted to populate the respective arrays with the isotope information. With this change, hiSixTrack is capable to load an initial bunch of arbitrary ions and store information on the particle type throughout the tracking and send this information from hiSixTrack to FLUKA and back. 

Note that the mass of both reference particle and tracked particle must coincide with their associated masses in FLUKA. If this is not respected, the hiSixTrack-FLUKA coupling refuses their exchange between the codes.







\subsection{Benchmarking of Ion Tracking in hiSixTrack}

The updated tracking algorithm can be benchmarked against heavy-ion tracking in the STIER approach. Taking into account that STIER provides symplectic tracking with multipole contributions to the same order as it is implemented in hiSixTrack, a full agreement of the particle tracks computed in the two approaches can be expected. In the following, different simulation cases are discussed to study the accuracy  at which isotopic and chromatic dispersion are computed in hiSixTrack:
%
\begin{enumerate}
    \item Study of the betatron motion for particles of the reference species with perfect momentum ($\delta = 0$; $\chi=1$),
    \item Tracking of different isotopes with ideal momentum to study the modelling of the isotopic dispersion  ($\delta =0$, $\chi \neq 1$)
    \item Tracking of off-momentum particles of the reference species to study the modelling of the chromatic dispersion ($\delta \neq 0$, $\chi =1$)  
    \item Tracking of off-momentum particles of an unmatched species ($\delta \neq 0$, $\chi \neq 1$)
\end{enumerate}

Especially if the particle tracks are observed over many turns, small numeric imprecisions induced by improper implementation (wrong setting of brackets etc.) of the tracking maps may lead to significant changes of the particle tracks. All simulations assume to the reference isotope to be \lead with a rest mass of $m_0=193.68769\,\text{GeV}/c^2$. The collimators are removed from the simulation, such that the tracking result can not be adulterated by scattering in the collimators. Furthermore, the RF cavities are not included since the momentum kicks for ions and protons deliver different kicks in $\delta$.


\subsubsection{Betatron Motion without Dispersion}

\begin{figure}[t]  
    \centering
    \includegraphics[width=0.9\textwidth]{pictures/16042801.pdf}
    \caption{Tracks of on-momentum particles of the reference species starting from IP1 on an horizontal annular halo at an amplitude of $5.7\,\sigma$, as simulated with hiSixTrack.}  
    \label{pic:16042801}    %/afs/cern.ch/work/p/phermes/private/150629_coupling_ions/hiSix/160316_runI_2011/runI.tracks/run_00001/betatron_benchmark.pdf
\end{figure}



The simulated betatron motion in hiSixTrack is benchmarked against STIER by means of comparing on-momentum particles ($\delta = 0$) of the respective reference species ($\chi=1$) with identical initial conditions in IP1. As an example, the betatron oscillations simulated for ten on-momentum particles, randomly generated at IP1 at the surface of a horizontal annular halo in $x,x'$ at an amplitude of $5.7\,\sigma$, is shown in \figref{pic:16042801}. 

For the benchmarking of the tracking algorithm in hiSixTrack, the simulated tracks of 100 particles with $\chi=1$ and different betatron amplitudes starting in IP1 are compared to the tracks extracted from STIER. The tracking is performed over $10^6$ turns through the magnetic lattice of the LHC, in the 2011 configuration with crossing and seperation bumps switched on in all experimental IRs, except IR1. 

The comparison of the simulated tracks shows, within the precision of the floating point numbers dumped from SixTrack, no difference between the two tracking approaches. This proves the consistency of the tracking maps derived in \chapref{chap:trackingmaps} with the ion-equivalent proton tracking used in STIER.






\subsubsection{Tracking of different Isotopes without chromatic Dispersion}

  \begin{figure}[t]
  \centering
  \includegraphics[width=0.9\textwidth]{pictures/16042802.pdf}
  \caption{Simulated heavy-ion tracks for different on-momentum ($\delta=0$) isotopes starting with the same initial conditions in IP1.}  
  \label{pic:15080501}
  %/home/phermes/Dropbox/PhD/notebooks/graphics/123.pdf
  \end{figure}

To benchmark the isotopic dispersion without the influence of betatron motion and chromatic dispersion, different isotopes to which the magnetic lattice is not matched ($\chi\neq 1$) are simulated to start in IP1 without momentum per mass offset $\delta=0$ and betatron amplitude $x=y=x'=y'=0$. The resulting tracks from hiSixTrack and STIER are compared in \figref{pic:15080501} for the isotopes listed with their $\chi$ values and the resulting $\delta_\text{eff}$ in \tabref{tab:15080501}. 

While the study case is unphysical and without correspondence in the real machine, it is well suited for the study of the isotopic dispersion. In the simulated scenario the particle motion is undisturbed of initial betatron offsets and chromatic dispersion, which can enhance or reduce the effect of the isotopic dispersion on the particle tracks.

The particle tracks of all isotopes except $^{207}$Pb$^{82+}$ were simulated for less than one turn through the LHC, because the strong dispersion leads to very large horizontal amplitudes after only a few magnets. The isotope $^{207}$Pb$^{82+}$ is tracked over $10^6$ turns. Also in this scenario, the tracks simulated by hiSixTrack and STIER are identical, confirming the accuracy of the implementation in hiSixTrack. 


\begin{table}[b]
\centering
\caption{Isotopes used for the benchmarking of the heavy-ion tracking. Values for $\chi$ and $\delta_\text{eff}$ are computed by means of Eqs.~\ref{} with respect to the reference isotope \lead. The masses correspond to the fully stripped ions and were extracted from FLUKA.}
\label{tab:15080501}
\begin{tabular}{cccccc}
\toprule
Element & $A$ & $Z$ & $m$ [GeV/$c^2$] & $\chi$   & $\delta_\text{eff}$ \\ \midrule
H       & \phantom{12}1   & \phantom{1}1   & 0.93827         & 2.51744  & -0.6030              \\
H       & \phantom{12}3   & \phantom{1}1   & 2.80892      & 0.84090  & \phantom{-}0.1892              \\
He      & \phantom{12}4   & \phantom{1}2   & 3.72738         & 1.26740  & -0.2110              \\
Au      & 206 & 79  & 191.833      & 0.97273  & \phantom{-}0.0280             \\
Pb      & 207 & 82  & 192.755      & 1.00484  & -0.0048             \\
Tl      & 208 & 81  & 193.693      & 0.98778 & \phantom{-}0.0124           \\ \bottomrule
\end{tabular}
\end{table}



\subsubsection{Tracking of the Reference Isotope with Momentum Offset}

To confirm the accuracy of the chromatic tracking without isotopic dispersion and betatron offset, the tracking is benchmarked for particles of the reference species ($\chi=0$) with momentum deviations relative to the reference particle ($\delta\neq0$). With each tool, 100 initial particles are simulated starting at IP1 with different momenta and initial amplitudes of $x=y=0$ and $x'=y'=0$. The tracking is carried out for a maximum of $10^6$ turns, which can only be achieved if the chosen momentum offsets are small enough. The tracks simulated with both tools are in full agreement.


\subsubsection{Unmatched Isotope with Momentum Offset}
Figure \ref{pic:16070801} shows the simulated track of a $^{207}$Pb$^{82+}$ ion starting from IP1 in a LHC lattice matched for \lead with $\delta \neq 0$ and an initial betatron offset. This scenario is studied for 200 particles of the species $^{207}$Pb$^{82+}$ with different values of $-5 \cdot 10^{-4} < \delta < 5 \cdot 10^{-4}$ and starting conditions in $x,x',y,'y$ randomly sampled from a Gaussian distribution in IP1. The comparison of the tracking in hiSixTrack and SixTrack with ion-equivalent rigidities shows a full agreement of the simulated tracks. The tracking of this isotope is limited to a maximum of five turns due to the large trasnverse offsets reached. 

\begin{figure}[htbp]
  \centering
  \begin{tikzpicture}
    \node[anchor=south west,inner sep=0] (image) at (0,0) {\includegraphics[width=0.70\linewidth]{pictures/16070802.pdf}};
    %\node [draw,rotate=90,x={(image.south east)},y={(image.north west)}]                   at (0.50,0.50)    {text0};
    %\node [draw,rotate=0 ,x={(image.south east)},y={(image.north west)}]                   at (0.22,0.96)    {text1};
    %\node [draw,rotate=0 ,x={(image.south east)},y={(image.north west)},anchor=west]       at (0.22,0.80)    {text2};
    %\draw[->,color=black,thick,x={(image.south east)},y={(image.north west)}]             (0.42,0.22) -- (0.37,0.23);
  \end{tikzpicture}
  \caption{Horizontal motion of $^{207}$Pb$^{82+}$ with momentum offset $\delta \neq 0$ starting from IP1.}  
  \label{pic:16070801}
  %/afs/cern.ch/work/p/phermes/private/150629_coupling_ions/reference/coupling/testrun7/test_aper.uWy/run_00001/offenergy_iso.pdf
  \end{figure}



\subsubsection{Symplecticity}

\begin{figure}[t]  
    \centering
    \includegraphics[width=0.6\textwidth]{pictures/16021202.pdf}
    \caption{Phase space parameters of an initial beam halo at $5.5\,\sigma$ at the location of the TCP.C6L7.B1 over $10^6$ turns, simulated with hiSixtrack. The volume covered in phase space remains constant.}  
    \label{pic:16021201}
    %/media/phermes/ph3tboffice/ph1tbwd/thesis/plots/160212_hisix_symplecticity/symplecticity.pdf
\end{figure}

With the symplecticity for the individual beam line elements already proven analytically (see \chapref{chap:sympl}), the global implementation in hiSixTrack can be studied for its symplecticity. For this purpose a bunch of on-momentum particles of the reference species starting as an annular halo at IP1 is tracked over $10^6$ turns. Sextupoles are switched off, such that the phase space ellipse should remain constant and not be shifted by non-linear forces. The study case is the LHC configuration of the 2015 heavy-ion run at $6.37\,Z$ TeV and the particle distribution in phase space is compared with itself every turn at the TCP.C6L7.B1.

The simulation result is shown in \figref{pic:16021201} proving that the volume of the phase space ellipse remains constant even after the large number of turns the particles are tracked through the LHC. This finding confirms the global symplecticity of the implementation in hiSixTrack and is in line with the symplecticity check carried out for the transport maps of the individual beam line elements. Together with the tracking benchmarking presented above, this result allows the conclusion that the implementation in hiSixTrack is accurate and trustworthy.  







\section{The hiSixTrack-FLUKA Coupling} \label{chap:hisix_coupling}

With the native heavy-ion tracking established in hiSixTrack, the on-line coupling for fragmentation simulations with FLUKA becomes possible. The algorithms which enable the particle exchange are adapted for heavy-ion applications and additional output data is produced to accuratly account for heavy-ion losses simulated in the framework of the hiSixTrack-FLUKA coupling. In this chapter the essential modifications are qualitatively described. 

\begin{figure}[b]
  \centering
  \begin{tikzpicture}
    \footnotesize
    \node[anchor=south west,inner sep=0] (image) at (0,0) {\includegraphics[width=1.0\linewidth]{pictures/16070602.pdf}};
    \node [x={(image.south east)},y={(image.north west)}]    at (0.89,0.340)    {$^{206}$Pb$^{82+}$};
    \node [x={(image.south east)},y={(image.north west)}]    at (0.93,0.530)    {$^{208}$Pb$^{82+}$};
    %\node [draw,rotate=0 ,x={(image.south east)},y={(image.north west)}]                   at (0.22,0.96)    {text1};
    %\node [draw,rotate=0 ,x={(image.south east)},y={(image.north west)},anchor=west]       at (0.22,0.80)    {text2};
    %\draw[->,color=black,thick,x={(image.south east)},y={(image.north west)}]             (0.42,0.22) -- (0.37,0.23);
  \end{tikzpicture}
  \caption{Simulated fragmentation of four \lead ions at the TCP and the subsequent motion of their residual fragments, as simulated with the hiSixTrack-FLUKA coupling.}  
  \label{pic:16021803}
  %/afs/cern.ch/work/p/phermes/private/150629_coupling_ions/hiSix/160204_ion_runII_2015/fragmentation_tracks/run_00001/fragmentation.pdf
  \end{figure}



\subsection{Heavy Ion Exchange between hiSixTrack and FLUKA}



In the nominal framework of the SixTrack-FLUKA coupling, with a defined particle species (protons), it is sufficient to send the proton energy and coordinates to FLUKA and back, with a transfer of information defining the particle type being not required. For the implementation of the hiSixTrack-FLUKA coupling, the inclusion of the particle species (defined by $A,Z$) and mass $m$ is important to ensure the use the correct physics models in both FLUKA and hiSixTrack. Every particle is identified in SixTrack by its particle ID, which is unique for each initial particle and each residual fragment. The hiSixTrack-FLUKA coupling assigns the new particle ID automatically to a newly created fragment. 

The particles sent back to the tracker are selected by means of their FLUKA particle ID which classifies them into different particle categories. For the heavy-ion coupling, the selection of protons is replaced by heavy ions (describing all fully stripped ions beyond $^{4}$He$^{2+}$), deuterons, tritium, $^{3}$He$^{2+}$ and $^{4}$He$^{2+}$. Protons are by default not sent back to hiSixTrack, but this feature can easily be activated with a minor modification of the corresponding FLUKA subroutine \texttt{fluscw.f}. The multi-isotopic modifications in hiSixTrack can in principle eligible support to the tracking of all other charged particles which are now excluded from the tracking (protons, pions, $\Delta$-baryons, ...). However, they are not included in the tracking, because with their extreme rigidity offset with respect to the main beam they are absorbed very locally, as it was demonstrated for the residual protons tracked in STIER (see \figref{pic:16021501}). In real operation these losses in the warm regions close to the collimators are shadowed by particle showers which are not modelled in the presented framework, such that their inclusion into the simulation would most probably not lead to conclusive findings. 

In addition, the simulations with the hiSixTrack-FLUKA coupling are space- and time-consuming due to the large number of different fragments produced. The total number of particles which can be tracked in one hiSixTrack simulation (the number of different particle IDs) is limited to 2000 due to limitations in working memory in the CERN cluster, imposing an upper boundary for the number of initial particles used per job. In the mentioned configuration a reasonable number of initial particles per job is 140, leaving enough margin for the production of more than 13 residual particles from each initial \lead ion. The sole inclusion of protons requires a significant reduction of the number of initial particles per job: the study with impact parameters between 1 and 10$\,\mu$m in the 2011 configuration showed that a large enough margin constraints the number of initial particles to be below 40 (which is in line with the production yield of 40\% at $b=10\,\mu$m that was found in the STIER simulations, see \tabref{tab:importance}, and considering the effect of fragmentation at subsequent collimators). In conclusion, the selection of heavy ions in the hiSixTrack-FLUKA coupling serves the purpose of reducing the required space and running time without a significant impact on the simulated loss pattern in regions beyond the normal conducting cleaning insertions IR7 and IR3. 

The particle exchange in the hiSixTrack-FLUKA coupling was extensively tested to ensure that the correct particle species and properties are transferred between the codes. The output of both hiSixTrack and FLUKA can be  adjusted to save information about the particle species and their six-dimensional phase space parameters. The tests carried out with 1 million particles demonstrate the accuracy of particle exchange.

Details on the technical implementation in the FLUKA subroutines and the SixTrack code is are given in \chapref{}.


\subsection{Benchmarking of the Fragmentation Simulation}

\begin{figure}[htbp]
  \centering
  \begin{tikzpicture}
    \node[anchor=south west,inner sep=0] (image) at (0,0) {\includegraphics[width=0.7\linewidth]{pictures/16072401.pdf}};
    %\node [draw,rotate=90,x={(image.south east)},y={(image.north west)}]                   at (0.50,0.50)    {text0};
    %\node [draw,rotate=0 ,x={(image.south east)},y={(image.north west)}]                   at (0.22,0.96)    {text1};
    %\node [draw,rotate=0 ,x={(image.south east)},y={(image.north west)},anchor=west]       at (0.22,0.80)    {text2};
    %\draw[->,color=black,thick,x={(image.south east)},y={(image.north west)}]             (0.42,0.22) -- (0.37,0.23);
  \end{tikzpicture}
  \caption{Heavy-Ion spectrum of particles leaving the TCP for an initial \lead beam at $7\,Z\,$TeV as simulated with FLUKA and with the hiSixTrack-FLUKA coupling. }  
  \label{pic:16072401}
  %/media/phermes/local/hisix_results/HLLHC/B1H/analysis/postprocessing/fragmentation_benchmark.pdf
\end{figure}

The accuracy of the fragmentation simulation is studied by a comparison of the heavy-ion spectrum coming out of the primary collimator between FLUKA and the hiSixTrack-FLUKA coupling. The geometry used in FLUKA is identical to the setup used for the STIER simulations. The impacting beam consists of $10^6$ particles of the isotope \lead with a momentum of $7\,Z\,$TeV$/c$ with $b=1\,\mu$m. In the hiSixTrack-FLUKA coupling the primary beam is simulated to impact the TCP at an impact parameter between 0.9$\,\mu$m and $1.1\,\mu$m. In both methods, the nuclear evaporation model of FLUKA is switched on and the material is set to graphite with the density of CFC. A major difference between the simulations is the fact that the beam used in FLUKA is on-momentum, while the particles used in the hiSixTrack-FLUKA have a momentum spread. 

The spectrum of nuclear mass numbers simulated with the two approaches is shown in \figref{pic:16072401}. Both histograms are normalized such that the values integrated over all bins yields one. Note that only the abundance of isotopes is given, so the data is not weighted with $A$. Overall, the two fragment spectra show an excellent agreement, with minor discrepancies for intermediate mass number, which lie within the expected statistical fluctuations. The rate of surviving \lead ions is slightly higher FLUKA, which can be traced back to the spread of impact parameters and energies in the hiSixTrack-FLUKA coupling. The latter leads to a spread of the angle of incidence increasing the mean traversed distance in the collimator material and thus the fragmentation rate.




\subsection{Collimator Losses}

The proton implementation of SixTrack and the SixTrack-FLUKA coupling consider particles to be lost in a collimator if they undergo inelastic nuclear interactions with large momentum transfer. In the heavy-ion SixTrack framework, the underlying physical processes are different, such that collimator losses have to be counted differently. 

The basic idea behind the loss scaling in hiSixTrack is the comparison of the total energy entering each collimator with the energy leaving it. Therefore, the coupling framework is modified to integrate the total ion energy every time the particle bunch is sent to FLUKA and when it is sent back from FLUKA to hiSixTrack. The difference is then considered as the energy lost at this collimator. This information is written to the  dedicated output file \texttt{fort.208}.

Considering that the particles sent back to hiSixTrack are exclusively heavy-ions while all other particles (neutrons, pions, $\Delta$-baryons, ...) are invisible in this approach even though they may have left the collimator carrying significant energies, the collimator losses derived from the mentioned approach are likely to be overestimated. Therfore the FLUKA subroutines are adapted to store information about all particles leaving the collimator which are not sent back to hiSixTrack into the output file \texttt{fort.66}.

In practice the collimator losses at the TCP, taking into account the correction factor, are reduced by approximately 20\%. More information on the implementation and the structure of the \texttt{fort.208} and \texttt{fort.66} files is given in \chapref{}.




\subsection{Simulation Output}
Specific simulation output is required for the study of heavy-ion losses. The hiSixTrack-FLUKA coupling is adapted to provide additional information about the lost isotopes via new output files and by the modification of output files already implemented before.

\begin{itemize}
  \item \texttt{fort.999}: default output file from the online aperture check; modified to save also information on $A,Z,m$ of the impacting particle.
  \item \texttt{fort.208}: stores information on collimator losses. This includes the number of nucleons and total particle energy lost at each collimator, without providing information about the individual particles lost. 
  \item \texttt{fort.209}: relates the collimator losses to individual particles. For every particle lost at a collimator the collimator ID is saved. The collimator losses can so be drawn back to the individual isotopes.
  \item \texttt{fort.822}: Ion fragments produced at the collimator. Every residual ion produced at any collimator is listed with its particle ID, the parent particle ID, the ID of the collimator at which it is produced and its mass, charge and energy.
  \item \texttt{fort.66}: Correction file for collimator losses. 
\end{itemize}




\section{Loss Map Simulation with the hiSixTrack-FLUKA coupling}



To validate the accuracy of full cleaning simulations with hiSixTrack, the measured BLM pattern in the 2011 heavy-ion run is compared to simulation results obatined with hiSixTrack and STIER. \figref{pic:16042202} shows the resulting loss patterns. 


\subsection{Initial Distribution} \label{chap:pha_shift}

Cleaning simulations with hiSixTrack can in principle be initiated similarly to proton simulations with SixTrack. In this approach the initial distribution is sampled as an annular halo at a betatron amplitude in the interval $[N_P, N_P+\epsilon]$, where $\epsilon$ is a small number and determines the thickness of the sampled ellipse in phase space. The latter is related to the maximum impact parameter which can be achieved if the phase space is conserved. In this approach, the particles continue moving in the machine until they hit the primary collimator after some turns. If non-linear elements, such as sextupoles, are present in the machine, the phase space ellipse is slightly shifted and tilted which can enlarge the impact parameter and lead to non-symmetric impacts on the two collimator jaws (see \figref{pic:16070603}). This should not be confused with a lack of symplecticity, because the volume enclosed by the annular halo remains constant. For protons, this behaviour is acceptable because the scattering in the collimator is not strongly dependent on the impact parameter~\cite{Bruce2014a}. As discussed in earlier chapters, for heavy-ions the spectrum of outcoming ion fragments is strongly dependent on the impact parameter, such that in cleaning simulations it must be better controlled than in the proton case.


\begin{figure}[htbp]
  \centering
  \begin{tikzpicture}
    \node[anchor=south west,inner sep=0] (image) at (0,0) {\includegraphics[width=0.7\linewidth]{pictures/16070605.pdf}};
    %\node [draw,rotate=90,x={(image.south east)},y={(image.north west)}]                   at (0.50,0.50)    {text0};
    %\node [draw,rotate=0 ,x={(image.south east)},y={(image.north west)}]                   at (0.22,0.96)    {text1};
    %\node [draw,rotate=0 ,x={(image.south east)},y={(image.north west)},anchor=west]       at (0.22,0.80)    {text2};
    %\draw[->,color=black,thick,x={(image.south east)},y={(image.north west)}]             (0.42,0.22) -- (0.37,0.23);
  \draw[thick, x={(image.south east)},y={(image.north west)},draw=black] (0.2,0.89) rectangle (0.3,0.97);
  \end{tikzpicture}
  \caption{Annular halo of on-momentum particles tracked from the TCP to the TCP.C6L7.B1 over ten turns with sextupoles (blue) and without sextupoles (red). The black horizontal lines indicate the TCP jaw edges.}  
  \label{pic:16070603}
  %/afs/cern.ch/work/p/phermes/private/150629_coupling_ions/hiSix/160204_ion_runII_2015/halo_tracks/run_00001/phase_ellipse.pdf
  \end{figure}

Instead, a preliminary approach to control the impact parameter at the primary collimators is used the cleaning simulations with hiSixTrack. In an initial high statistics simulation over one turn through the LHC, the impact coordinates on the collimator jaws of the respective TCP (depending on the beam and the transverse directin) are determined. Based on this simulation result the initial conditions can be associated to impact parameters and the impacting jaw. Finally, the initial coordinates resulting in impacts at the desired jaw with the chosen impact parameter can be sampled and multiplied to the required number of initial particles. In terms of computing time the initial simulation and the subsequent sampling of the initial distribution file requires less than one hour, which is negligible to a total simulation time for a full hiSixTrack cleaning simulation of 24~h for 6.5 million particles. The loss maps presented in the following sections are generated with this approach.

Future upgrades of hiSixTrack may include input options for pencil beams, which impact directly on a dedicated collimator with a defined impact parameter. This feature, already implemented in the standard collimation version of SixTrack~\cite{bruce_cwg:166}, is not available in the framework of the SixTrack-FLUKA coupling. The simulation is in this case not initiated at IP1 but directly at the TCP with the subsequent framgentation simulation, comparable to the approach used in STIER. It allows precise studies on the behaviour of the cleaning system with given impact parameters.




\subsection{Loss Map Benchmarking} \label{lm:benchmark}


The full cleaning simulation with hiSixTrack is benchmarked against the measured data of the 2011 heavy-ion run and compared also to the STIER simulation result. The hiSixTrack simulation presented is carried out for B1H with 6.5 million initial \lead ions which start at IP1 and are intercepted by the TCP on the first turn in the machine. The impact parameter at the latter is continuously distributed between 1$\mu$m and 10$\,\mu$m. 

The loss maps are shown in the Figs. \ref{pic:16042202} (full LHC), \ref{pic:16070803} (zoom to IR7) and \ref{pic:16042501} (zoom to IR3). The global loss pattern simulated with the hiSixTrack-FLUKA coupling is very similar to the STIER simulation. While the loss peak distribution is quantitatively in good agreement, the amplitude of the losses is different in the two simulations. This can be drawn back to the different impact parameter and residual particles emitted from collimators downstream of the TCP. As expected, they become visible in the hiSixTrack-FLUKA coupling while they are not simulated in STIER. The most remarkable are the losses downstream of the momentum cleaning region IR3. 

The losses in the warm elements of IR3 and IR7 are, as expected, not modelled accurately in either simulation because the secondary particle showers are not included in the simulation. The general shape of the collimator losses in IR7 is different between the two simulation tools. Note in particular the different ratio of losses at the primary collimator and the TCSG collimators immediately downstream. This difference can be drawn back to two effects: first, the TCP losses in STIER are uniquely originating from out-scattered \lead ions surviving one or multiple turns and then impact on the TCP again. The energy lost in the TCP at the initial passage is not included in STIER, with the evident consequence of a significant reduction of simulated losses at the TCP. Secondly, the STIER simulation includes protons, which are mostly lost at the TCSG collimators and therefore change the loss ratio between TCP and TCSG in favour of the TCSG. This assumption is supported by low statistics hiSixTrack simulations including protons. 

One additional loss peak in the arc between IR8 and IR1 (marked as A8 in \figref{pic:16042202}) is simulated in hiSixTrack. Further investigation shows that this additional peak is indeed generated from multiple collimator interactions. A more detailed analysis is shown in the next section.


The direct comparison of the simulated global heavy-ion beam loss pattern shows that the loss distribution is dominated by particles starting from the IR7 TCP. The difference between the resulting loss pattern from STIER and hiSixTrack are mainly visible at regions downstream of the collimators in IR3 and IR2. This finding is in line with the assumption under which the STIER tool was developed and supports the accuracy of the loss maps simulated with it. 

% \begin{figure}[t]  
%     \centering
%     \includegraphics[width=1.0\textwidth]{pictures/16042202.pdf}
%     \caption{B1H betatron loss maps measured in the 2011 heavy-ion run compared to simulations with the hiSixTrack-FLUKA coupling and STIER.}  
%     \label{pic:16042202}
%     %/home/phermes/Dropbox/PhD/pictures/160422_hisix_STIER_2011/LHC_2011.pdf
% \end{figure}




\begin{figure}[t]
  \centering
  \begin{tikzpicture}
    \small
    \node[anchor=south west,inner sep=0] (image) at (0,0) {\includegraphics[width=1.0\linewidth]{pictures/16042202.pdf}};
    %\node [draw,rotate=90,x={(image.south east)},y={(image.north west)}]                   at (0.50,0.50)    {text0};
    %\node [draw,rotate=0 ,x={(image.south east)},y={(image.north west)}]                   at (0.22,0.96)    {text1};
    \node [draw,rotate=0 , fill=white, x={(image.south east)},y={(image.north west)},anchor=west]       at (0.94,0.58)    {A8};
    \draw[->,color=black,thick,x={(image.south east)},y={(image.north west)}]             (0.95,0.55) -- (0.915,0.48);
  \end{tikzpicture}
    \caption{B1H betatron loss maps measured in the 2011 heavy-ion run compared to simulations with the hiSixTrack-FLUKA coupling and STIER.}  
    \label{pic:16042202}
  %/home/phermes/Dropbox/PhD/pictures/160422_hisix_STIER_2011/IR7_2011.pdf
  \end{figure}





\begin{figure}[t]
  \centering
  \begin{tikzpicture}
    \node[anchor=south west,inner sep=0] (image) at (0,0) {\includegraphics[width=1.0\linewidth]{pictures/16070803.pdf}};
    %\node [draw,rotate=90,x={(image.south east)},y={(image.north west)}]                   at (0.50,0.50)    {text0};
    %\node [draw,rotate=0 ,x={(image.south east)},y={(image.north west)}]                   at (0.22,0.96)    {text1};
    %\node [draw,rotate=0 ,x={(image.south east)},y={(image.north west)},anchor=west]       at (0.22,0.80)    {text2};
    %\draw[->,color=black,thick,x={(image.south east)},y={(image.north west)}]             (0.42,0.22) -- (0.37,0.23);
  \end{tikzpicture}
  \caption{B1H betatron loss maps measured in the 2011 heavy-ion run compared to simulations with the hiSixTrack-FLUKA coupling and STIER, zoomed to IR7.}  
  \label{pic:16070803}
  %/home/phermes/Dropbox/PhD/pictures/160422_hisix_STIER_2011/IR7_2011.pdf
  \end{figure}


\begin{figure}[t]  
    \centering
    \includegraphics[width=1.0\textwidth]{pictures/16042601.pdf}
    \caption{Comparison of the measured loss map in the 2011 heavy-ion run to simulations with STIER and hiSixTrack, zoomed to IR3.}  
    \label{pic:16042501}
    %/home/phermes/Dropbox/PhD/pictures/160422_hisix_STIER_2011/IR3_2011.pdf
\end{figure}

The following conclusions are drawn from the benchmarking of the loss maps simulated with hiSixTrack. 

\begin{itemize}
  \item The accuracy of global loss simulations is improved with respect to STIER by the inclusion of scattering and fragmentation at all collimators. This can become of interest for the study of TCLD collimators, where ions scattered out of secondary collimators or the TCLD collimators themselves may lead to losses in the DS region of IR7 and therefore reduce their efficiency. 
  \item The most critical losses at the IR7 DS are simulated similarly as in STIER. This demonstrates that the assumptions used for the set up of the STIER simulation tool are valid and the simulation results obtained with it are accurate within its limitations. 
\end{itemize}




\subsection{Contribution of Secondary Fragmentation}

With the inclusion of secondary fragmentation in hiSixTrack the impact of the latter on the final loss map can be studied. The simulation of the 2011 heavy-ion run can be analyzed for the contribution of secondary ion fragments to the final loss pattern. The result of this analysis is given in terms of the integrated losses $\eta_{sec}^{int}$ of particles generated at collimators different from the IR7 TCP normalized by the total amount of losses $\eta_{tot}^{int}$ integrated for the region of interest. The comparison is shown for the two IR7 DS loss clusters, the four arc cluster A1 to A4 downstream of IR7 in \tabref{tab:secondary} and for all aperture losses in the LHC ring. 
%
\begin{table}[b]
\centering
\caption{Energy fraction of ion fragments generated at secondary collimators in IR7 with respect to the total integrated energy lost in different LHC regions. }
\label{tab:secondary}
\begin{tabular}{cccccccc}
                                \toprule                                             & DS1              & DS2              & A1                 & A2                & A3                & A4              & Global \\ \midrule
%\begin{tabular}[c]{@{}l@{}}Energy fraction\\ of sec. fragments\end{tabular} 
$\eta_{sec}^{int}/\eta^{int}_{tot}$& $5\cdot 10^{-3}$ & $3\cdot 10^{-3}$ & $3\cdot 10^{-2}$ & $1 \cdot 10^{-4}$ & $6 \cdot 10^{-4}$ & $1\cdot10^{-3}$ & $8\,\cdot10^{-3}$  \\ \bottomrule
\end{tabular}
\end{table}
%

The simulation data shows that only minor contributions to all loss peaks arise from these secondary fragmentation processes. The highest contribution in the concrete simulation case is reached at the A1 cluster where 3\% of the losses are caused by these heavy-ion fragments. 

While the losses in the regions mentione above are affected only insignificantly by secondary fragmentation processes, in the arc region between IR7 and IR8 a new loss peak A8 is simulated in hiSixTrack. The STIER simulation does not predict losses at this location. 



On a global scale, approximately 0.8\% of aperture losses arise from these ions. 



In conclusion, ions from fragmentation at secondary collimators contribute only insignificantly to the global loss map, even in the DS clusters downstream of IR7. This finding quantitatively confirms the conclusion based on a comparison of the STIER and hiSixTrack loss maps drawn in the previous chapter. Note that this conclusion might not be valid for other study cases at different particle momenta, with other heavy-ion species or collimator settings.